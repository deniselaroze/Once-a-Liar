\documentclass[12pt]{article}
\usepackage{amsfonts, amsmath, amssymb}
\usepackage{dcolumn}
\usepackage{subfigure, subfloat}
\usepackage{anysize}
\usepackage{verbatim}
\usepackage{multirow}
\usepackage{indentfirst}
\usepackage{rotating}
\usepackage{setspace}
\usepackage{graphicx}
\usepackage{makeidx}
%\usepackage{overcite}
%\usepackage{showidx}
\usepackage{natbib}
\usepackage{booktabs}
\doublespace
\bibpunct[, ]{(}{)}{,}{a}{}{,}
\makeindex
\newcommand{\mr}[1]{\multicolumn{1}{r}{#1}}
\newcommand{\mc}[1]{\multicolumn{1}{c}{#1}}
\newcommand*{\betabf}{\ensuremath{\boldsymbol{\beta}}}  % bold beta
\newcommand{\bff}{\begin{frame}[fragile]}
\newcommand{\R}{\textbf{R }}
\newcommand{\cov}{\text{cov}}
\newcommand{\var}{\text{var}}
\newcommand{\aindex}[1]{\index{#1@\citet{#1}}}
\newcommand{\bhm}{\hat{\beta}}
\newcommand{\bht}{$\hat{\beta}$}
\pagestyle{myheadings}
\newcolumntype{d}[1]{D{.}{.}{#1}}
\marginsize{1.25in}{1.25in}{1.25in}{1.25in}
\newcommand{\tm}{$_{t-1}$}
\title{Why we Cheat: Evidence from Tax Compliance Experiments
\thanks{Working Paper. Centre for Experimental Social Sciences, Nuffield College, University of Oxford. November, 2016.}
}
\author{Raymond M. Duch\\raymond.duch@nuffield.ox.ac.uk \\Nuffield College \\ University of Oxford 
\and Hector Solaz \\h.solaz@bham.ac.uk \\ Department of Economics \\ University of Birmingham}
\newcommand{\hi}{\hangindent=0.25in}
\begin{document}
%\graphicspath{{coalition_reasoning/text/}}
\bibliographystyle{apsr}
\begin{singlespace}
\maketitle
\end{singlespace}
\newpage

\linespread{1.6}

\begin{abstract}

%\par Cheating is costly -- the economic, political and social costs resulting from unethical behavior are enormous.  Reducing the incidence of cheating behavior requires an understanding of who cheats, why they cheat, and under what circumstances they cheat.  

\par We conduct tax compliance experiments in which subjects earn real money, are subject to a tax, and can lie about their earnings.  Performing better on incentivized real effort tasks results in more cheating.  Cheating rises as the tax rate increases; but the higher levels of cheating by those who excelled at the tasks persists in high and low tax treatments.  This correlation persists when earnings are associated with luck; the correlation persists in experimental treatments with more redistributive taxation.  The correlation persists for choices made in other games by the same subjects : High performance types give less in a conventional Dictator Game and also cheat more in a classic die game in which they privately report the results of tossing a die. 

%There may in fact be ``high performance types'' in the population who have a general predilection for cheating (or lying).  


\end{abstract}

\newpage



\section{Why We Cheat?}

\par Individuals in the population, controlling for its costs and benefits, systematically lie or cheat {\citep{Gneezyetal2013,Cappelenetal2013,Hurkensetal2009}.  This essay demonstrates that it is High Performance types who are more likely to cheat.  Our evidence is based primarily on experimental games in which choices resemble the decision to comply with taxation.  We provide important insights into tax cheating in particular but also into cheating behavior more generally.

%\par Cheating on one's taxes represents one area of decision making that has social consequences.  

\par It is generally accepted that cheating is costly -- there is a considerable body of literature documenting the economic and political costs associated with cheating.  An area in which the economic costs of cheating are particularly prominent is taxation.  One of these claims is that tax cheating by the rich contributes to growing economic inequality \citep{Zucman2015,Piketty2014,Atkinson2015}.  This is a prominent issue, of course, because there are significant opportunities for cheating, particularly by the self-employed ``rich'', in virtually all tax regimes.\footnote{The U.S. Internal Revenue Service audits less than two percent of tax returns and the penalties are relatively light \citep{Posner2000}.  Audit rates and penalties are even lower in most other countries.}  

\par Our effort to explain cheating behavior in the population is based on experimental games in which subjects make decisions that resemble the decision to cheat or not to cheat on one's taxes.  The literature on tax compliance focuses on why individuals do not cheat.  \citet{Benabouetal2011} make a persuasive case for how intrinsic motivation and reputation concerns in the population can inform the design of optimal incentives that depart from standard Pigou-Ramsey taxation.  Non-standard preferences play an important role in tax compliance.  As \citet{Luttmeretal2014} point out there is considerable evidence suggesting that intrinsic motivations -- or ``tax morale'' -- contribute to explaining tax compliance. 

\par While there is a recognition that intrinsic motivations play an important role in understanding tax compliance they are typically treated as being homogeneous in the population.  This is probably not the case.  We contend that High Performance types in the population are less motivated by these intrinsic payoffs.  We present evidence indicating that High Performance types are more likely to cheat.  Our evidence is based primarily on their choices in tax compliance experiments but we claim this generalizes to any situation in which there are opportunities to cheat.


%  Because these high ability types typically have higher incomes this suggests that the intrinsic motivations of the rich are significantly less conducive to tax compliance than is the case for the poor.

\section{Conjectures Regarding Cheating}

%\par We propose two factors to explain cheating:  One is simply the economic costs of compliance -- as the tax bill rises individuals will cheat more (and this would be a function of income and tax rate).   A second, less obvious, contributing factor is ability.  





%\paragraph{The Decision to Cheat.} These correlations between tax morale and expressed cheating preferences suggest two considerations that likely condition intrinsic motivations to comply.  

 

\par Our contention is that High Performance types value cheating more than do Low Performance types.  This section states our conjectures along with possible counterfactuals.  We illustrate our conjecture with a tax declaration decision that is made under certainty. Failure to report one�s full income to the tax authorities in this illustration does not provoke any penalties or any other form of social costs, such as ostracism.   Following \citet{Allinghametal1972}, the taxpayer is expected to choose between two main strategies: (1) she may declare her actual income, or (2) she may declare less than her actual income. The utility of the taxpayer decreases with the amount declared because a percentage of it is deducted through taxes. Separately, the individual derives some pay-off from cheating that is simply determined by their Performance type.  

%intrinsic motivation of the individual to fulfil her duty to pay taxes. This can be understood as an intrinsic reward or warm-glow \citep{Andreoni1990,Benabouetal2011}.

\par The actual income of the taxpayer, $i$, is endogenously determined by her performance, $P_{i}$; and a wage component that is performance based, $s$. Only the taxpayer knows this income. Tax is levied at a constant rate, $t$, on declared income, which is determined by what proportion, $c_{i}$, of actual income the taxpayer declares.  We can think of $c_{i}$ as dichotomous (cheat or not cheat) or as some continuum of cheating that varies from 0 (no cheating) to 1 (full compliance).  And we can think of $P_{i}$ as indicating one's ability or productivity such that a higher value represents high productivity.  The payoff from  cheating is then determined by two terms: $\alpha[(c_{i})*(P_{i}*s*t]+\beta [P_{i}*c_{i}]$.

\par The first square bracket captures the standard pay-off from cheating and has a weight of $\alpha$:  For any level of cheating, $(c_{i})$, the payoffs (from cheating) rise with $(P_{i}*s*t)$.  This is the case because ones taxable income, $P_{i}*s$ increases with ones Performance type (more productive employees earn more money).  

\par We conjecture that Performance type has an affect on the cheating decision that is independent of this cost term.  The second term, $[P_{i}*c_{i}]$, captures this effect of tax-payer Performance type on cheating pay-offs.  For any level of cheating, the pay-offs (from cheating) rise with one's Performance type.  The importance of this second term is represented by $\beta$. 

\begin{comment}
\par The taxpayer will now choose $c$ so as to maximize:

%\begin{footnotesize}
%\begin{equation}
%U(x_{i}) = [(\alpha+p_{i}*s)-(t*x_{i})]+\beta^{p_{i}}(t*x_{i})
%\label{eq:one}
%\end{equation}
%where $0\le$$p_{i}$$\le$1 and $0\le\beta$$<1$

%\end{footnotesize}

\begin{equation}
U(c_{i}) = \alpha[(c_{i})*(P_{i}*s*t]+\beta (P_{i}*c_{i})
\label{eq:one}
\end{equation}
%where $0\le$$p_{i}$$\le$1 and $0\le\beta$$<1$
\end{comment}

\par This representation of tax compliance is a significant simplification. Our formulation ignores all possible sources of uncertainty that are commonly associated with tax cheating. Since there are no audits in the experiments reported here, there is no possibility of receiving a penalty for lying about one's income.

% In today�s world characterized by globalization and fast technological advances in financial transactions, taxpayers may availed themselves of tax havens and banking services that allow them to avoid paying taxes, sometimes even in a legal way. Hence, even if we ignore these possible sources of uncertainty, we hope to have retained enough of the structure of the problem to make the analysis worthwhile.


\par Our conjecture regarding High Performance types is captured by the right-hand term $\beta (P_{i}*c_{i})$.  At any level of cheating ($c_{i}>0$), the pay-offs from cheating will be great for High, compared to, Low Performance types.  A large weight, $\beta$, on the interaction term would lend credence to our argument.  But Performance, $P_{i}$, also increases the pay-offs to cheating because it increases one's taxable income, $P_{i}*s$.  The challenge is distinguishing the independent contribution of these two terms to the cheating decision -- our experiments are designed to disentangle these two effects of Performance type on the pay-offs from cheating.

%\par We explore our conjectures regarding cheating in the controlled environment of a lab experiment where we can carefully calibrate relevant treatment effects and minimise confounding factors.  



%  But a number of factors may be at play here.  First, at any tax rate the absolute contributions of the rich are greater than the poor.  Intrinsic satisfaction might simply be negatively correlated with the absolute magnitude of tax contributions.  And as this amount grows, cheating rises.  This could simply be a distributional effect -- I am richer than others and therefore am entitled to cheat.  Wealth somehow empowers individuals to behave unethically.  Or it could be a question of price -- for high earners the ``cost'' of accommodating the intrinsic benefits of compliance become to high and the likelihood of cheating rises.

%\par If there are opportunities for shirking then all (self-interested) taxpayers should exploit them.  Compliance by taxpayers though is dramatically higher than what one would expect given a self-interested model of tax compliance.    Our conjecture though is that self-interest or greed will lead the rich to comply less than the poor.

%\par Under most redistributive tax regimes, the net costs to the rich taxpayer of compliance will be higher than those for poor taxpayers.  And the self-interested poor clearly have an interest in reinforcing a norm of tax compliance while the self-interested rich have precisely the opposite incentive.


%\par Alternatively, differences in intrinsic motivations of the rich and poor may not simply be a function of the magnitude of their tax contributions.


\paragraph{Performance.}  We argue that High Performance types are more likely to cheat.  There are relatively few explicit efforts linking ability to cheating.  An exception is \citet{Gilletal2013} who find, in their online experiments with real effort tasks, that highly productive workers cheat more.  \citet{Ariely2012} finds that creativity is, while intelligence is not, correlated with cheating.

%Our central argument is that high ability types receive significantly lower intrinsic payoffs from tax compliance than do low ability types. 

\par Our conjecture is consistent with a body of literature that links wealth, competitive success and power with greedy and unethical behaviour.  \citet{Piffetal2012} implement seven experiments with convenience samples that support this correlation between wealth and unethical behaviour.\footnote{Although in a subsequent letter to the editor \citet{Francis2012} raised questions about the plausibility of the results (all seven experiments rejecting the null hypothesis) given that the observed Power of the tests in each of these experiments hovered around .5. Though subsequent replications of these experiments by \citet{Duboisetal2015} generated results that are consistent with the \citet{Piffetal2012} results.}  An analysis of shop-lifting behaviour found that incident rates are significantly higher for those in the highest income categories \citep{Blancoetal2008}.  Under this line of reasoning the rich's antipathy towards tax compliance is based on more favorable attitudes toward greed and a predilection for unethical behavior.  But we contend that its not wealth per se that explains unethical behavior; rather ability accounts for the rich's proclivity to engage in anti-social behavior.  

%A case in point is the claim by \citet{Piffetal2012}:

%\begin{quote}
%We predict that, given their abundant resources and increased independence, upper-class individuals should demonstrate greater unethical behavior and that one important reason for this tendency is that upper-class individuals hold more favorable attitudes toward greed. 
%\end{quote}


\par A related literature suggests that competition breeds a sense of entitlement on the part of winners \citep{Major1994,Majoretal1989} that in turn facilitates dishonest behaviour by the winners.  Recent experimental results \citep{Schurretal2016} indicate that winning a competition predicts dishonest behaviour.  These results are consistent with our conjecture that individuals ``who are likely to be winners'' are more likely to cheat.  \citet{Schurretal2016} suggest that dishonest behaviour results from winning a competition in which there are identifiable losers.  We contend, somewhat differently, that there are types in the population that identify themselves as ``winners'' or what we could call High Performance types.   And High Performance types are more likely to cheat.  Since they typically self-identify as High Performance types they do not require explicit competitions, that identify winners and losers, in order to exhibit dishonest behaviour.

%will be   undermines intrinsic motivations.  Individuals perceive themselves as becoming rich because of their abilities and effort.  And successful types resist efforts by authorities to tax their ``well-deserved'' earnings.  Noncompliance here is a function of recognizing your type -- ``I have above-average abilities or talents and therefore I am entitled to cheat at my taxes.''  

%\par Self-interest or simple unethical and greedy predispositions may result in cheating by the rich.  Those with below-average incomes can expect to be net beneficiaries of redistribution, or have less greedy and unethical predispositions, and hence are more likely to comply.  This could represent an important de facto constraint on redistribution because the most important source of tax revenues will be those least likely to be ``tax compliant''. 
  
\par The mechanism may be the sense of entitlement that is associated with identifying as a High Performer type.   Experimental evidence suggests that a sense of entitlement is an important contributing factor to dishonest behaviour \citep{Vincentetal2015,Major1994,Majoretal1989}.  And performing well, or excelling in competition, appears to enhance these feelings of entitlement \citep{Vincentetal2015,Major1994,Majoretal1989}.  Entitlement may be the underlying mechanism that generates anti-social or dishonest behavior; being above average entitles one to be ``above the law.''

 
%\subsection{Conjecture 2: The Price of Intrinsic Payoffs}

%\par These various findings are typically based on attitudes or opinions regarding redistributive policies that are measured using survey instruments.   A less obtrusive measure is simply to observe tax compliance across varying levels of redistributive taxation.       

%Tax compliance may in fact not simply be driven by one's expected net financial situation in regimes with redistributional taxation.  It may also be the case that individuals condition their preferences for redistributive policies on norms regarding fair rates of taxation.  Individuals may express preferences for very aggressive redistributive outcomes, but these preferences are not necessarily correlated with the redistributive tax policies they find acceptable.  Empirical studies examining redistributive preferences have frequently found this to be the case.  Often differences emerge between the rich and the poor in their preferences for redistributive outcomes, and yet the rich and poor often respond similarly to specific tax policies that promote redistribution \citep{Bartels2005}.  And there are various explanations... Other explanations -- status anxiety ... \citep{Kuziemkoetal2013b}.  Poor knowledge of income distribution.... etc.   

\par Finally, recent experimental findings suggest that relatively small incentives lead powerful subjects to make unethical decisions that harm less powerful subjects \citep{Swanneretal2015}.  When put in a position of high power and offered an incentive, the majority of subjects knowingly falsely informed on a confederate.  The mechanism here may be that incentives cause powerful people to focus on their own rewards rather than the well-being of others. 

\par These findings are consistent with our intuition that ability causes cheating.  Individuals who are financially successful or who typically win at competition or have attained powerful positions tend to be High Performance types.  We contend that its this self-perceived ability that leads to unethical behaviour.  And its not a perception that necessarily needs to be primed -- individuals know their performance type.
\par $C_{1}$: High performance reduces the percent of income declared for tax purposes.   

\paragraph{Cost of Compliance.} The expected costs of cheating matter.  The intrinsic payoffs individuals might realise from tax compliance obviously come at an ``extrinsic'' cost.  Individuals in the population, regardless of whether they are of high or low ability, share a similar, presumably downward sloping, demand function for ``intrinsic'' rewards.  This is consistent with findings suggesting that charitable giving increases as its price decreases \citep{Andreonietal2002,Gneezyetal2011} but also experimental evidence that lying is sensitive to the costs associated with telling the truth \citep{Gibsonetal2013}.  As the size of their tax obligation rises, which will be a function of income and tax rates, we should see cheating -- the percentage of income undeclared -- increase. 

%\par  High ability types, because they earn higher incomes, typically face high costs associated with indulging their prosocial instincts.  

\par Our tax treatments are designed to help tease out the relative importance of ability versus costs in determining tax compliance:  We observe low and high ability types at different prevailing tax rates; and for any compliance cost level, we observe how ability levels affect rates of cheating.

\par $C_{2}$: High income reduces the percent of income declared for tax purposes. 

\paragraph{Winners versus Losers.} Cheating behaviour may simply be triggered by discrete events that identify a winner or loser.  Experimental evidence from \citet{Schurretal2016} demonstrates that unethical behaviour by ``winners'' is triggered by competition that clearly identifies winners and losers.  Our experimental design goes to considerable length to ensure we are not confounding ``winning'' with High Performance type.
 
 %High performance types are typically ``winners'' but as we pointed out earlier its not the outcome of any particular competition per se that causes these individuals to cheat.  
 
%The classic Meltzer and Richards narrative suggests that individuals make choices regarding redistributive taxation based on their location in the overall income distribution.  These decisions might simply be motivated by envy/generosity or by calculated self-interest -- either way they imply a recognition of one's income circumstances relative to the prevailing overall income distribution.  From a strictly self-interested perspective, the rich may be inclined to cheat because they recognise that compliance implies a net loss of income or wealth to others.  

%\par Or wealth may simply embolden individuals to engage in unethical behaviour relative to those who are less fortunate -- the ``losers''.  An attitude that is nicely captured by a comment Leona Helmsley is reported to have made: ``We don't pay taxes. Only the little people pay taxes.''  In a similar fashion, \citet{Schurretal2016} argue that unethical behaviour by ``winners'' is triggered by competition that clearly identifies winners and losers.

\par First, arguments about winners and unethical behaviour assume that individuals are informed about their location in the overall income distribution or about their success in competitions.  One cannot ``be'' rich in this sense without perceiving that there are others who are poor, i.e., earning less than they do.  Or one cannot be a ``winner'' in a competition without there being ``losers'' \citep{Schurretal2016}.  Its the fact that one knows one is rich or a winner that gives license to unethically or greedily behavior.  In our experiments, subjects are not provided with this information which helps us isolate the effect of performance type as opposed to the effect of one's ``winner'' versus ``loser'' status.

%\par Again, with respect to tax compliance, this is an income effect that is difficult to distinguish from the affect of ability (the rich will have higher abilities) or the price of intrinsic payoffs (the rich pay higher taxes).  

%\par The argument about ``winning'' and cheating distinctly invokes knowledge of how individuals perform relative to others in the ``competition'', or of how one's earnings compare to other's earnings. 


\par Secondly, in our experiments we observe cheating behaviour over multiple rounds of the same tax compliance game.  We can think of each round as a competition.  And we observe the extent to which our High and Low Performance types exhibit consistent cheating behaviour.  While its true that we expect High Performance types to do well on average, we will observe rounds in which they perform below average.  If winning and losing in particular competitions affects cheating behaviour (as opposed to being affected by ones type) then our subjects' cheating behavior should respond to these deviations from average performance levels.

\par $C_{3}$: ``Winners'' declare lower percent of income for tax purposes. 

\paragraph{Luck Trumps Ability.}  Some argue that tax compliance is conditioned on beliefs regarding returns to labor inputs \citep{AlesinaAngeletos2005}.  If rich taxpayers believe labor markets are efficient and reward ability they are more averse to taxation.  \citet{Frank2016} suggests that the failure to recognize the role of chance in determining success and income my increase one's reluctance to pay taxes.  On the other hand, if opportunities for wealth are perceived to be more weakly correlated with ability and more likely to be determine by luck, status, class or government largess, the rich will be more tax compliant.

\par This argument implies that the impact of ability on cheating should be moderated when luck or status is perceived as an important determinant of success.  There is evidence in the work on tax compliance to suggest that cheating is conditioned on the source of income or wealth \citep{Durhametal2014}.\footnote{Generally, when experiments employ earned income they find a negative relationship between income and compliance  \citep{Almetal2006,Anderhubetal2001,Bradly1987,Beckeretal1987,Trivedietal2006,Cherryetal2005,Chanetal1999,Chanetal1996}.  Evidence from Dictator games suggest that windfall wealth tends to have an overall positive effect on contributions while in Dictator experiments with earned income subjects generally tend to be much less generous \citep{Cherryetal2002,Hoffmanetal1996}.}  

 
 \par We explore this claim by implementing two variations in the baseline compliance experiment: a ``status'' and a ``shock'' variant.  Random assignment in both variants signals to some subjects that part of their earnings results from chance or luck.  If the argument is correct, then cheating should be moderated for those High Performance types who benefit from chance.
 
 \hangindent=1cm $C_{4}$: When income results from luck (as opposed to ability) individuals declare a higher percent of income for tax purposes. 
 
 \paragraph{Fairness of the Tax Regime.} A fifth conjecture is that there are social norms regarding the appropriateness or ``fairness'' of different tax regimes.  Intrinsic motivation for tax compliance may vary according to perceived variations in ``fairness.''  \citet{Besleyetal2015} draw such a conclusion from their study of the UK poll tax.   The 1990 introduction of the Poll Tax in the UK clearly induced a negative shock in the intrinsic motivation to comply with taxation.  Compliance dropped because taxpayers questioned the ``fairness'' of the tax being imposed.  Cheating rises when individuals perceive the tax regime as unfair and vice-versa.  

\par Fairness here refers to the manner in which taxes are raised and redistributed -- the progressivity of the tax regime would be a case in point.  Heterogeneity in intrinsic motivations or compliance could simply reflect variations in perceived fairness of the tax regime.  Highly progressive tax regimes, for example, might create social norms that promote compliance.  Our ``Redistribute'' treatment introduces progressive redistribution of the tax revenues collected in each group.  If progressivity increases the intrinsic benefits of compliance for high ability types then we should see a moderation of the correlation between performance and cheating.

 \hangindent=1cm $C_{5}$: Under redistributive tax regimes individuals declare a higher percent of income for tax purposes. 
  

\paragraph{Generalized Cheating Behaviour.} Our conjecture is that 1) individuals in the population self-identify as High Performance types; and 2) their proclivity to cheat is not restricted to a particular domain (i.e., tax compliance).  We designed treatments that explicitly test these two conjectures.

%Since we are concerned that our general conjecture regarding performance and cheating not be seen as an artefact of these particular tax compliance experiment 

\par Performance on a real effort task (RET) allows us to identify those with high ability.  In order to assess whether subjects self-identify as high versus low performance types we elicit, in an incentivized fashion, subjects' assessments of both how well they expect to perform and, after the fact, their relative performance on the RET compared to other group members.  Our conjecture is that subjects will anticipate their high or low performance on the RET.  Hence subjects recognise their ability without requiring any feedback on other group members' performance.

\par Second, the proclivity to cheat by High Performance types is not restricted to taxes.  Accordingly we have included other opportunities for subjects to cheat.  One of these efforts simply gauges the other-regarding preferences of the subjects.  Prior to the RET, and before they make decisions in the tax compliance module, subjects play a conventional Dictator Game that allows us to compare the generosity of High versus Low Performance types.  Subjects also have the opportunity to cheat at the end of the experiment when they toss a die to determine compensation for completing a questionnaire.  After the tax compliance games, subjects play a version of the \citet{Fischbacheretal2013} die tossing game that again allows us to compare the unethical behaviour of high versus low performance types.  Our expectation is that Higher Performance types will lie more about their die toss than is the case low ability types.

 \hangindent=1cm $C_{6}$: Cheating by high performance types generalizes to decisions other than declared income for tax purposes. 
 
\paragraph{Summary}  Our conjecture is that High Performance types cheat more than Low Performance types.  Other factors might moderate this relationship.  In particular, for both High and Low performance types, we might see cheating increase as the price of honest behavior, tax compliance in our experiments, increases.

%As many have pointed out, tax compliance on self-reported income is a function of intrinsic motivation and concern with reputation \citep{Benabouetal2011b}. 


\par Contextual factors might moderate the relationship between ability and cheating.  High Performance types might cheat less when income is determined by luck or status.  And the intrinsic benefits of compliance by High Performance types may be higher under highly progressive tax regime (again moderating this ability-cheating correlation).  These possibilities are explored in the experimental design.  High Performance types self-identify -- a conjecture also explored in the experimental design.  And we contend that this proclivity to cheat or engage in unethical behaviour by High Performance types is a general phenomenon.  Accordingly we incorporate diverse decision making situations in which subjects have an opportunity to cheat.

\section{Tax Compliance Experiments: Design}

%\par For example, one of our goals is to determine whether, in contexts where chance or status plays an important role in determining wealth, the rich are more willing to contribute to redistribution efforts.  There have been efforts to explore this argument with observational data analysis but they only imprecisely calibrate the hypothesised theoretical effects.  \citet{Giulianoetal2013}, for example, analyse responses to redistribution preference questions from the General Social Survey and World Values Survey arguing that the macro-economic shocks of the Great Recession are perceived as chance events and therefore increase public support for redistribution.  They find that those who are more negatively affected by the shock answer more favourably to questions about government redistribution.  But the theory is really about the redistribution preferences of the rich who have benefited favourably by chance events and whether this makes them more favourable to redistribution.  Its not clear whether a significant negative macro-economic shock provides a good test of the hypothesis regarding the rich's redistribution preferences: Do we expect the rich to perceive themselves as ``lucky'' because they were spared the worst of a very large macro-economic shock?



%We employ these experiments in order to understand individuals' preferences regarding redistribution and about redistributive tax rates in particular.  

%\par This essay reports the results for two treatments designed to provide insight into which of these factors actually shape tax compliance and hence preferences for redistributive taxation.  The experiment consists of four modules. Subjects are paid for all four modules at the end of experiment, and do not receive feedback about earnings until the end of the experiment. Participants receive printed instructions at the beginning of each module, and instructions are read and explained aloud. 

%Table~\ref{tab:treatments} summarises these treatments.  The first set of treatments are for the tax and audit rates.   These are designed to provide insight into our hypothesised ``tax norm.''   By varying the tax rate......  These are implemented in modules 2 and 3....

%\subsection{Our Design}

%\par These survey data are suggestive but limit our ability to identify the causal mechanism underlying tax cheating.  We propose a set of tax compliance experiments that help isolate these causal mechanisms. 

\paragraph{Tax Cheating.} We employ real effort tax compliance experiments in order to understand cheating.\footnote{There is a considerable literature on tax compliance experiments \citep{Slemrod2007,Almetal2015}.  Typically these experiments have explored whether or not compliance is affected by perceived features of the tax system\citep{Almetal1992,Spiceretal1980,Falkinger1995,Cowell1990} or changes in the tax system\citep{Heinemannetal2013}.} Our tax compliance experiments are designed to isolate in as neutral a context as possible the micro-foundations for cheating. It is explicitly a highly simplified tax regime and we make no claims regarding the external validity of the ``stylised'' regime itself.  Individuals reveal their type by performing the RET.  Their type explains subsequent cheating behavior.  Subjects are randomly assigned to different versions of this ``stylised'' tax regime that help isolate the factors causing individuals to cheat.

%\footnote{Its important to point out that the game was not framed in terms of tax payments.  The framing was very neutral -- rather than referring to taxes or penalties we simply referred to deductions from gains.} 

\par This essay reports the results for five treatments designed to identify the factors causing cheating.  Subjects are paid at the end of experiment, and do not receive feedback about earnings until the end of the experiment. Participants receive printed instructions at the beginning of each module, and instructions are read and explained aloud. 

\par The tax treatments consist of ten rounds each. Table~\ref{tab:treatments} summarises the treatments.  Prior to the tax treatments, participants are randomly assigned to groups of four.  In four of the treatments we follow a partner matching such that the composition of each group remains unchanged for the two tax treatment modules.  In a fifth treatment, Sessions 15-20, participants are randomly assigned to groups of four after each of the ten rounds of the two tax modules.

\par Each round is divided in two stages. In the first stage subjects perform a real effort task. This task consist of computing a series of additions in one minute. Their Preliminary Gains depend on how many correct answers they provide, getting a set number of ECUs for each correct answer (in the Baseline Treatment this is 150 ECUs for each correct answer where 300 ECU = \pounds1). 

\par After subjects receive information concerning their Preliminary Gains, they are asked to declare these gains.  A certain percentage or ``tax'' (that depends on the treatment) of these Declared Gains is then deducted from their Preliminary Gains.\footnote{We explicitly avoid framing the game in terms of ``taxes''.  Subjects are told that a deduction (rather than a ``tax'') would be applied to earnings.  And while our neutral framing does not identify the deduction as a tax, we do explicitly tell subjects that there is a probability of verification of income (an ``audit'') that could result in a penalty.  We believe this framing signals to subjects that compliance is encouraged.  In the results reported here, subjects are informed that the probability of such an ``audit'' is zero. And they are told, accurately, that subsequent modules of the session could have non-zero audits.  We only report the results here of the zero audit module.}  These deductions are then divided amongst the members of the group (in most treatments the deductions are divided evenly amongst group members).  Note that in each session the tax rate is consistent and it does not vary.  The tax treatments are the following: 10\%, 20\%, 30\%.\footnote{One of the Redistribute sessions had a tax rate of 40\%.} 

\begin{table}[h!]
\caption{Summary of Tax Compliance Experimental Treatments}\label{tab:treatments}
\begin{center}
\begin{tabular}{p{2cm}p{2cm} p{2cm} p{2cm}p{4cm}}
Session & Participants & Groups & Tax Rate & Treatment  \\ \hline \hline
1 & 24 & 6  & 10\% & Baseline\\
2 & 24 & 6  & 20\% & Baseline \\
3 & 24 & 6  & 30\% & Baseline \\
%4 & 24 & 6  & 40\% & 0\% & 100\% & Equal Salary\\
%5 & 24 & 6 & 50\% & 0\% & 100\%  & Equal Salary\\
%6 & 20 & 5  & 10\% & 30\% & 70\% & Equal Salary \\
%7 & 24 &  6 & 20\% & 30\% & 70\% & Equal Salary\\
%8 & 20 &  5  & 30\% & 30\% & 70\%  & Equal Salary\\
%9 & 24 & 6 & 40\% & 30\% & 70\% & Equal Salary\\
%10 & 24 & 6 & 10\% & 30\% & 70\% & Different Salary \\
%11 & 24 & 6  & 20\% & 30\% & 70\%  & Different Salary\\
%12 & 20 & 5 & 30\% & 30\% & 70\% & Different Salary \\
%13 & 24 &  6  & Endogenous & 0\% & 100\% & Equal Salary \\
%14 & 20 & 5 & Endogenous & 30\% & 70\% & Equal Salary \\
4 & 24 & 6  & 10\% &  Status\\
5 & 12 & 3 & 20\% &  Status \\
6 & 16 &  4  & 20\% & Status \\
7 & 20 & 5 & 30\% & Status \\
8 & 24 & 6  & 10\% & Redistribute\\
9 & 20 & 5 & 20\% & Redistribute \\
10 & 20 &  5  & 30\% &  Redistribute \\
11 & 20 & 5 & 40\% &  Redistribute \\
12 & 16 & 4 & 10\% & Shock \\
13& 20 & 5 & 20\% &  Shock \\
14 & 20 & 5 & 30\% &  Shock \\
15 & 16 & 4 & 10\% &  Baseline Non-fixed \\
16 & 16 & 4 & 10\% &  Baseline Non-fixed \\
17 & 16 & 4 & 10\% &  Baseline Non-fixed \\
18 & 12 & 3 & 10\% &  Baseline Non-fixed \\
19 & 12 & 3 & 20\% &  Baseline Non-fixed \\
20 & 16 & 4 & 30\% &  Baseline Non-fixed \\
\hline \hline
\end{tabular}
\end{center}
\end{table}

\par We implemented five treatments designed to identify conditions under which subjects might vary their degree of tax cheating.  In the first equal salary (\emph{Baseline}) treatment subjects get the same payment for correct answers to the real effort test (10 pence).  This represents the ``ability'' treatment in which salaries are strictly tied to performance.  The second (\emph{Status}) treatment consists of an inequality salary treatment in which two (``Low Status'') subjects get 5 pence per correct answer and two (``High Status'') subjects get 15 pence per correct answer.  Random assignment determines those subjects earning higher returns to effort and hence introduces our notion of ``luck'' or ``status'' into the resulting income distribution. 

\par  In a third (\emph{Redistribute}) treatment  the two participants per group with the lower income (each round) receive 35\% of the pooled deductions, while the two with higher income receive 15\% -- in case of ties on the number of additions computed (income), the division is decided at random.  This represents the treatment in which the redistributive use of the tax revenues is the most aggressive.

\par Our fourth (\emph{Shock}) treatment (Sessions 12-14) randomly assigns half of the subjects to a control treatment that resembles the \emph{Baseline} treatment.  Half of the subjects are randomly assigned to a ``shock'' treatment in which their earnings from the RET are incremented by 150 ECUs (50 pence).  Subjects are not informed about their bonus until after they complete their RET and before they report their income.  Subjects are presented with a breakdown of their earnings -- gains associated with their performance in the RET and the portion associated with the bonus.  These random assignments occur after each round is played.  

\par Our final (\emph{Baseline Non-fixed}) treatment (Sessions 15-20) resembles the Baseline treatment with the exception that subjects were randomly assigned to new groups after each round of play.  %Subjective assessments of their ability were also elicited in Sessions 15-20.  In these sessions we also implemented a game in which subjects report the result of a privately tossed die, 

%\par  \emph{Inequality and Redistribution Treatments.}  We also included treatments that were designed to determine whether subjects conditioned their compliance behaviour on the impact of the tax revenues on redistribution.  Accordingly in the first equal salary treatment subjects get the same payment for correct answers to the real effort test (10 cents).  This represents the least redistributive of the salary treatments.  In the different salary treatment (Inequality) the two low performing subjects get 5 cents per correct answer and the high performing subjects get 15 cents per correct answer.  The distribution of tax revenues in this treatment result in a moderately higher degree of redistribution.  Finally in a third ``Diff MPCR'' treatment (Redistribution) the two participants per group with the lower income (each round) receive 35\% of the pooled deductions, while the two with higher income receive 15\% -- in case of ties on the number of additions computed (income), the division is decided at random.  This represents the treatment in which the redistributive use of the tax revenues is the most aggressive.

%; 4) a tax efficiency treatment in which subjects earn the same per correct answer (treatment 1) but the tax revenues are arbitrarily reduced by 35\% before they are distributed to the group; 5) a similar tax efficiency treatment although in which there is both redistribution (the Diff MPCR treatment) and an arbitrary reduction by 35\% in tax revenues before they are distributed.




%If the audit finds a discrepancy between the Preliminary and Declared gains an extra amount is deducted from the Preliminary Gains. In both modules the amount correspond to 50\% of the observed discrepancy. In addition, the regular deduction applies to the Preliminary Gains and not to the declared amount. Deductions applying to the four group members are then pooled and equally distributed amongst those members. 

%\par \emph{Audit Rate (AR) Treatments.} In each module there is a certain probability that the Declared Gains are compared with the actual Preliminary Gains in order to verify these two amounts correspond. In the second module (in sessions 1 through 5) the probability is 0\%, while this probability changes to 100\% in the third module (in sessions 1 through 5). In sessions 6 through 12, the audit rate is 30\% in the second module and 70\% in the third module.  If the audit finds a discrepancy between the Preliminary and Declared gains an extra amount is deducted from the Preliminary Gains. In both modules the amount correspond to 50\% of the observed discrepancy. In addition, the regular deduction applies to the Preliminary Gains and not to the declared amount. Deductions applying to the four group members are then pooled and equally distributed amongst those members. 


\paragraph{Subjective Ability.} In Sessions 15-20 we elicit incentivised subjective assessments of ability in the RET.  After completing the first set of practice rounds for the RET, but prior to the RET for the first round, we asked subjects to indicate the expected ranking (rank 1 through 4) of their first round performance relative to other members of their group.  They were informed that they would earn 150 ECUs if they were correct.  A similar question was then asked of each subject for two randomly selected rounds (excluding the first round).  In these cases subjects were asked to rank their performance in the RET effort they had just completed.

\paragraph{Selfish Preferences.} We measure the general other-regarding preferences of subjects in the first module of each session with a standard Dictator Game \citep{Engel2011}. Subjects are asked to allocate an endowment of 1000 ECUs between them and another randomly selected participant in the room. Participants are informed that only half of them will receive the endowment, and the ones who receive the endowment will be randomly paired with those who don't. However, before the endowments are distributed and the pairing takes place, they may allocate the endowment between themselves and the other person as they wish if they were to receive the endowment. 

\paragraph{Lying.} In Sessions 15-20, we implemented a version of the \citet{Fischbacheretal2013} die tossing game in order to compensate respondents for completing an attitudinal and demographic questionnaire.  Subjects were asked to toss a die (in total privacy without any means for the experimenter to see the result) and report the result.  Subsequent to their first toss they were given an opportunity to toss the die as many times as they wanted in order to ensure the die was fair.  The number reported from the die toss translated to payoffs at a rate of 100 ECU per unit reported (where 300 ECU = \pounds1) -- so they could earn up to \pounds2. 

\paragraph{Subjects and Earnings.} \par Subjects are informed that there is a certain probability that the Declared Gains are compared with the actual Preliminary Gains in order to verify these two amounts correspond. In the module results presented in this essay the probability of such an audit is 0\%.  At the end of each round participants are informed of their Preliminary and Declared gains; the amount they receive from the deductions in their group; and the earnings in the round. At the end of each module one of the ten rounds is chosen at random, and their earnings are based on their profit for that round.  At the end of the experiment ECU earnings are converted at the exchange rate 300 ECUs = 1\pounds.  While the earnings are prepared participants answer a questionnaire, which consists on an Integrity Test, and a series of socio-demographic questions.

\par All of the sessions were conducted at CESS (Centre for Experimental Social Sciences), a research facility of Nuffield College, at the University of Oxford. Subjects are undergraduate and graduate students from Oxford. Some subjects had participated in previous experiments, but all of them were inexperienced in this particular type of experiment. No subject participated in more than one session of the study. On average, a session lasted around 90 minutes, including instructions and payment of subjects, and the average payment was around 17\pounds. The experiment was computerized using ZTREE \citep{Fischbacher2007}. A copy of the instructions can be found in the Appendix.

\section{Tax Compliance Experiment: Results}

\subsection{Effort}

\par We implemented treatments in which income from the additions RET is entirely determined by performance -- each subject receives 10 pence per correct addition; a treatment in which a randomly assigned status (high/low) determines the subject's income associated with correct additions (5 versus 15 pence); and a treatment in which half the subjects are randomly assigned to a shock treatment whereby their income is incremented by 50 pence.  The mean correct additions is similar across treatments:  12.2 in the baseline; 12.7 in the low status; 13.2 in the high status; and 12.8 in the shock treatment.  There is no indication that effort was conditioned on the tax rates -- the average correct additions for the 10\%, 20\%, and 30\% tax rates, respectively, were 12.7, 11.8, and 12.1.\footnote{Figure~\ref{fig:additions} in the Appendix summarises the performance of the subjects in the different treatments.}

\subsection{Cheating}

\par The outcome variable of interest is the amount of misreported earnings in this tax compliance game.  The audit rate is zero and hence subjects were not penalized for cheating.  Revenue collected from these taxes is (with the exception of one treatment) distributed equally amongst subjects and hence there are no social gains (or losses) associated with compliance.  The equilibrium choice for all subjects is to report zero earnings.  

%An extensive literature on public goods games suggests that subjects will make positive contributions even in settings where the equilibrium choice is to give nothing \citep{Ledyard1995}. 

\par Virtually all of the subjects in the experimental sessions cheated -- 81 percent of the declarations under-reported the subjects' actual earnings.\footnote{This is almost precisely the percentage of individuals evading in the \citet{Dwengeretal2015} German local tax field experiment.  And this suggests that the overwhelming majority of subjects conform to the \citet{Allinghametal1972} theoretical framework.}  Figure~\ref{fig:evaded} summarises the subjects' cheating behaviour.  On average, subjects' choices resemble standard compliance models  \citep{Allinghametal1972}:  Subjects report about 20 percent of total income earned over the 10 sessions with zero audit.\footnote{The average 20 percent compliance results are similar to those obtained by \citet{Almetal2015} in their tax experiments and benchmark closely to the compliance rates they report from their analysis of actual U.S. Internal Revenue Service data made available by the National Research Program initiative.  They find that the average compliance rate for Sole proprietor Section C filers in the NPR data is 31 percent (24 percent in the data weighted to the population distribution).  And in their experiments with student subjects, the compliance rate is 28 percent in the zero audit probability treatment.}  There is variation in cheating.  The left graph in Figure~\ref{fig:evaded} presents the frequency of subjects' average ratio of non-declared to total earnings.  About two-thirds of the subjects are cheating virtually in every round and about 10 percent never cheat.
 
 
%Typically subjects switch from declaring all (or close to all) of their earnings to declaring zero earnings in subsequent rounds.  This is consistent with arguments regarding the fixed costs associated with lying that predict maximal lying once economic types estimate that the benefits of lying outweigh their costs \citep{Kajackaiteetal2015,Gneezyetal2013}.
 
 \begin{figure}[!ht]
\caption{Subjects' Non-declared to Total Earnings}\label{fig:evaded}
\centerline{\subfigure[Frequency Average Non-declared to Total Earnings]{\includegraphics[width=.6\textwidth]{{undeclared2.pdf}}} \\
\subfigure[Frequency: Alway, Sometimes, Never Cheat]{\includegraphics[width=.6\textwidth]{cheat_type2.pdf}}}
%\centerline{\includegraphics[width=\textwidth]{Predictions_normal_probit_LowTax.pdf}}
\end{figure}

\clearpage


%\begin{figure}[p]
%\caption{Frequency of Subjects' Average of Non-declared to Total Earnings}\label{fig:evaded}
%\centerline{\includegraphics[width=\textwidth]{undeclared.pdf}}
%\end{figure}

%\clearpage

\par Subjects earn and report income over ten periods.  We do not expect significant within subject variation with respect to cheating.  Our principal conjecture is that compliance is shaped by one's performance type that does not vary significantly by subject over the ten rounds.  Our expectation then is not unlike those of \citet{Kajackaiteetal2015,Gneezyetal2013} -- subjects, given a particular treatment condition, will either sort into economic types and cheat most of the time or will rarely cheat.

%And secondly it is affected by the cost of compliance that is determined by performance, salary and tax rate.  The latter two variables do no vary across rounds of an experiment session (the exception is the shock treatment in which there is an unexpected bonus). 

\par The right graph in Figure~\ref{fig:evaded}  suggests this is a reasonable characterisation of the subjects' behaviour.  Subjects are categorised as ``Always Cheat'' if in at least 9 out of 10 rounds they reported less than 15 percent of their earned income; as ``Never Cheat'' if in at least 9 out of 10 rounds they reported more than 85 percent of their earned income; and ``Sometimes Cheat'' if they fell between these two extremes.   Two-thirds of the subjects are consistent cheaters using this categorisation strategy.  And about 10 percent are serial non-cheaters.   One-quarter of the subjects vary their cheating behaviour across the 10 rounds.  

%\begin{figure}[p]
%\caption{Frequency of Always, Never, and Sometimes Cheating}\label{fig:cheat}
%\centerline{\includegraphics[width=\textwidth]{cheat_type.pdf}}
%\end{figure}

%\clearpage

\par This result is consistent with other claims that individuals in the population sort into types that are highly prone to cheat or lie (subject to an assessment of costs and benefits) while others rarely cheat or lie \citep{Kajackaiteetal2015,Gneezyetal2013}.  Our conjecture is that one's Performance type determines cheating proclivities.   

\par Figure~\ref{fig:treatment} presents the difference in treatment effects between Low and High Performance types, along with their boot-strapped confidence intervals estimated with clustered standard errors.   We define Low and High Performance types according to whether their average performance was above or below the median performance on the real effort tasks under each treatment (which is either 11 or 12 correct additions depending on the treatment).  The point estimates represent the percent of earnings evaded by the High Performance types subtracted from those for the Low Performance types.  The typical difference across these seven treatments is about -0.20.

\par  Clearly performance type matters -- High Performance types cheat more.  And the results are quite robust to treatment:  High Performance types cheat more regardless of treatment.  In the Baseline treatment, there is a difference of more than -0.20 percent in cheating -- High Performance types do not declare about 85 percent of their income while Low Performance types cheat at a rate of about 65 percent.  Moreover, the confidence intervals calculated using clustered standard errors here do not include zero.  This is consistent with our conjecture that the interaction term $(P_{i}*c_{i})$ weighs heavily in the cheating calculus described earlier.\footnote{Risk preferences could affect cheating behavior.  The tax compliance games resemble a public good game in that each subject's final earnings is affected by tax ``contributions'' by other members of the group.  Given there is uncertainty as to the choices of other group members, risk aversion can reduce the contributions of subjects \citep{Schechter2007,Teyssier2012}.  Risk preferences may be confounded with performance type in our tax compliance games.  We explore the impact of risk preferences on our findings in the Online Appendix and conclude that the performance type effect we estimate is not confounded with risk preferences.}

\begin{figure}[!ht]
\caption{Difference in Mean Non-declared Earnings High-Low Performance Types}\label{fig:treatment}
\centerline{\includegraphics[width=\textwidth]{bootstrap_diff-1.pdf}}
\end{figure}

\clearpage

\paragraph{Cost of Compliance ($\alpha$) versus Performance Type ($\beta$).}  The cheating calculus described earlier is driven both by the cost of compliance (which, for any performance type, is a function of salary and the prevailing tax rate) and whether one is a high or low performance type.  Table~\ref{tab:compare} provides some insight into the relative magnitudes of $\alpha$ and $\beta$.  Comparing the \% Evaded columns for Low versus High Performance types confirms the Figure~\ref{fig:treatment} results: High types consistently cheat more, regardless of the tax rate.  

\begin{table}[h!]
\caption{Tax Rates, Average Cost of Compliance, and Cheating}\label{tab:compare}
\begin{center}
\begin{tabular}{ld{3}d{3}d{3}d{3}d{3}}
 & \mc{Low Ability}&& \mc{High Ability}  & \\
\hline\hline
 & \mc{Cost to Comply}& \mc{\% Evaded} & \mc{Cost to Comply}  & \mc{\% Evaded} \\
\hline\hline 
10\% Tax            & 150  &  65\%    &  205   &  81\%  \\
                                 & &   &    &   \\
20\% Tax        & 260 &  62\%  &  422  &  87\%  \\
                            & &        &  &   \\
30\% Tax        & 398 &  78\%   &  851        &  89\%   \\

\hline\hline
\end{tabular}
\end{center}
\end{table}
  
\par Secondly, the \% Evaded is clearly higher in the 30\% tax rate compared to the 10\% tax rate for both Low and High Ability types.  And, the \% Evaded for the High Ability types in the 30\% tax treatment is higher than the \% Evaded for the Low Ability types in the 30\% tax treatment.  This is consistent with the notion that both higher tax rates and higher performance levels contribute to cheating.  Its reasonable to conclude that the tax component ($t$) of the cheating calculus is affecting the pay-offs from cheating as conjectured.

\par Surprisingly, the entire measure of the cost to comply term ($(P_{i}*s*t)$) in the cheating calculus is at best very weakly correlated cheating.  Within the Low and High ability types there clearly is a positive relationship between cost of complying and the \% Evaded.  But if we compare the 10\% and 20\% treatment average costs to comply figures from the Low Ability types, they are larger than the average cost to comply for High Ability types in the 10\% tax treatment.  Yet the \% Evaded for the High Ability types in the 10\% tax treatment is almost 20\% higher than it is for these two Low Ability tax categories -- in spite of having a lower average cost to comply.  This at least suggests that Performance type might be a stronger driver of cheating than is the cost of complying.

  
\paragraph{Cheating in Context.} Our experimental results suggest that the correlation between ability and cheating is very robust, although there is some evidence context can matter.  In the \emph{Status} sessions, subjects are randomly assigned to either a low or high wage rate.  Again for both treatments we see that cheating is significantly higher for high as opposed to low performance types.  The difference is higher for the those assigned to the low status treatment and with a confidence interval below zero.

%Those in the low status treatment cheat more than those assigned the baseline treatment but the spread between low and high performance types is roughly similar -- 20 percent.   On average, high performance types in the high status treatment have particularly high overall levels of cheating (about 90 percent) and their cheating rates are about 30 percent higher than low performance types in the high status treatment (who have average cheating rates of about 60 percent).  

  
\par In the \emph{Shock} sessions, half the subjects randomly receive a bonus (\emph{Shock}) after completing their RET.  Differences for high versus low performance types is about -0.20 in the \emph{No Shock} condition and roughly -0.10 for subjects in the \emph{Shock} treatment.  In the Redistribute sessions, we see  a moderation in the difference between low and high performance types -- it is just under -.0.10.

\par In our highly stylized tax regime, subjects interact in small groups of four and they retain the same matched partners throughout the 10 rounds reported in this essay.  Features of this decision making resemble a standard public goods game.  Accordingly, some subjects, particularly those who are net beneficiaries from the public good, may comply in an effort to encourage other members to reciprocate, i..e,  contribute generously to the public good in subsequent rounds \citep{Andreoni1995,Fischbacheretal2001,Sonnemansetal1999}.  This could explain higher compliance by low performers in our tax compliance games.  

\par We explore this possibility by implementing six sessions (Sessions  15-20) in which subjects are randomly assigned to a new group after each round of the tax game.  Figure~\ref{fig:treatment} reports the difference between high-low performance types in these three non-fixed sessions.  Cheating by high performance types compared to low performers is even greater in the non-fixed (Sessions 15-20), relative to the, matched baseline treatments (Sessions 1-3).  Its unlikely that the higher levels of cheating by high ability types in our experiment are an artefact of the fixed matching design.  

\paragraph{High Performance Types.}  We conjecture that individuals know their type and behave accordingly.  Cheating by high ability types does not need to be primed necessarily by an external cue signalling their type.  A number of features of our experiments allow us to test this conjecture.

\par Subjects here are not informed of the distribution of outcomes.  Hence the difference in the cheating behaviour we observe is unlikely to result from subjects comparing their outcome in each round to the performance of other members of their group.  Rather subjects are self-aware of their performance type.  One indication of this is that subjects perform quite consistently over the multiple rounds of the addition real effort task.  Recall we categorised subjects into High and Low Performance types based on whether they performed, on average, more or less than 11 correct additions (the overall mean).   Rarely did each subject's performance deviate significantly from their performance ``type'' (either Low Performance types deviating significantly above the mean or High Performance types deviating significantly below the mean).  As an illustration, we define low performance deviations as an outcome in which a Low Performance type performs more than 12 additions.  And high performance deviations occur when a High Performance type only manages less than 10 correct additions.  There are only 78 occurrences of the high performance deviations which represents less than 5 percent of high performer outcomes.  And there are only 68 occurrences of low performance deviations which also represents less than 5 percent of low performer outcomes.  

\par  Cheating behaviour should be consistent within subjects -- it should not fluctuate significantly and we would not expect it to respond to stochastic shocks in performance.  This implies that, for any particular subject, cheating behaviour is not correlated with RET outcomes that deviate significantly from their overall performance levels.  Its not the case, for example, that when a high performance type experiences an unexpectedly poor RET performance her cheating sharply declines.  To test this, we calculated, for occurrences of both low and high performance deviations, the average change in percent of earnings evaded.  In both cases, the average change was not significantly different from zero.\footnote{Figure~\ref{fig:deviate}  in the Online Appendix provides a frequency plot of these deviations in cheating for the low and high performance deviation cases.}   

\par Finally, as part of Sessions 15-20 we elicited, in an incentive compatible fashion, subjective assessment of ability on the RET.  At the very outset of the tax compliance module, after subjects had become familiar with the RET (but before they began playing any of the 10 rounds), we asked subjects how they expected their performance to rank relative to the other three members of their group.   

\par Table~\ref{tab:subjective} presents a cross-tabulation of subject's rank with our categorization of the subject into a low or high ability type.  As expected, subjects who performed above average on the RET are much more likely to have anticipated being either ranked first or second in their group.  Almost 60 percent of the High Performance types anticipated being ranked first and about 35 percent expected to be ranked second.  Very few of the High Performance types anticipated being ranked third or fourth.  On the other hand the Low Performance types were much more likely to anticipate performing poorly: half of the Low Performance types expected to rank third or fourth while only about 20 percent of them expected to rank first.  

\begin{table}[h!]
\caption{Subjective Assessment of Performance on Real Effort Task (Sessions 15-20)}\label{tab:subjective}
\begin{center}
\begin{tabular}{ld{3}d{3}}
 & \mc{Low Ability} & \mc{High Ability} \\
\hline\hline
First Rank                  &  21\%   &  57\%  \\
                                    &    &   \\
Second Rank            &  26\%  &  37\%  \\
                                    &  &   \\
Third Rank           &  37\%        &  5\%   \\
                                    &          &   \\
Fourth Rank           & 15\% & 2\% \\
                                    &          &   \\
Total                  &  84 & 60  \\
\hline\hline
\end{tabular}
\end{center}
\end{table}
  

\par Our conjecture is that High and Low Performance types generally know their type.  Their cheating behavior is not conditioned on a particular outcome (such as winning a competition or ranking first in a tournament).  In our experiments subjects are never told about the performance of other group members -- nevertheless the High Performance types cheat more than the Low Performers.  And cheating behaviour is relatively constant across each round of the tax experiment; cheating does not respond to positive or negative shocks to subjects' typical performance on the RET.  Finally, we explicitly ask subjects to assess their expected performance on the RET and there is a strong correlation between self assessed ability and performance on the RET.  There is strong evidence to suggest that individuals recognise that they are either a Low or High Performance type and this determines how much they cheat in each round of the game. 

\paragraph{Generalized Cheating Behavior.}  We conjecture that high ability types will be more likely to cheat in contexts other than this particular tax game.  To test this we have subjects make decisions in two other games in which they have an opportunity to behave selfislly or unethically.  
 
\par Prior to the tax compliance modules, subjects are randomly paired and one of the pair is randomly given the opportunity to share any portion of an endowment of 1000 ECUs with his or her partner.  In this standard Dictator Game we expect that high ability types (who are identified in a subsequent tax compliance game) would be much less generous than the low ability types.  Analysis of the amounts given in the Dictator Game clearly confirm our conjecture.  The average offer, over all the sessions, by low performance types is 289 ECUs and by high performance types it is 197 ECUs.  And the overall correlation between number of correct additions and offers in the Dictator Game is -0.22.  Even prior to performing the RET, high performance types clearly distinguish themselves as being less generous.

\par We implemented a more direct measure of cheating in Sessions 15-20.  After subjects complete the tax modules they are asked to complete a questionnaire.  Compensation for this effort is determined by their reporting the results of a completely private toss of a die.  Subjects can report truthfully or lie -- there is no way that the experimenter can verify the result of the subject's die toss.  The authors of this game (and subsequently many others) find that a significant percentage of subjects lie -- and this is the case in our experiment.  The extent of lying is inferred by the extent to which the frequency distribution of the die outcomes (1 through 6) deviates from a uniform distribution. 

\par  Figure~\ref{fig:die} presents the distribution of reported results for High and Low Performance types.  The High Performance types are not shy about lying.  About 70 percent of high ability subjects report having tossed a six.  The Low Performance types are somewhat more shy about lying -- although clearly they lie.  About 40 percent report rolling a six -- another 30 percent of low ability types report a five.\footnote{This suggests that over 50 percent of the High Performance types lied about rolling a six compared to around 22 percent of the Low Performance types.  Its also the case that about 15 percent of the Low Performance types lied about rolling a five.  On average about 40 percent of the subjects lied about their die outcome which is somewhat higher than the 0.216 estimated in the  \citet{Abeleretal2016} meta-study of the die game.}  This experiment provides subjects with an anonymous, uncomplicated and non-strategic setting in which they have an opportunity to cheat.  The high level of ``extreme'' cheating by our High Performance types in the die-rolling experiment provides further support for our conjecture about ability and cheating.

\begin{figure}[!ht]
\caption{Die Toss: Reported Results for Low and High Ability Types}\label{fig:die}
\centerline{\includegraphics[width=\textwidth]{die_result2.pdf}}
\end{figure}

\clearpage

\par Performance type and wealth are correlated in the tax compliance game.  Observing more cheating by High Performance types in our tax compliance experiments could simply reflect resistance to compliance as the size of that contribution gets bigger -- suggesting $\beta=0$ and $\alpha \ne 0$ in the cheating calculus.  We address this with the multivariate analysis in the next section.  But our results concerning generalized cheating behavior also speak to this issue.  As we demonstrated above, when our High Performance types unexpectedly perform poorly -- and hence earn less money -- we do not see a change in their cheating behavior.  This suggests that their ``cheating reflex'' is not simply responding to the magnitude of their tax obligation.

\par A second insight from the analysis concerning generalized cheating behavior is that this ``cheating reflex'' by High Performance types is not confined to the tax compliance game.  We observe the cheating behavior of High Performance types in other games where the payoffs from cheating are entirely unrelated to their wealth from the real effort tasks.  High Performance types give less money in a conventional Dictator Game and also are more likely to cheat in a classic die game in which they privately report the results of tossing a die.   Their cheating behavior in these two games, unrelated to the RET, cannot be confounded with a wealth effect.  Hence we are confident that high levels of cheating behavior by High Performance types in the tax compliance games indicates that higher ability is negative correlated with intrinsic payoffs from tax compliance. 
 
\subsection{Multivariate Analysis}

\par Cheating is measured in two fashions in this experiment: 1) simply whether or not the subject correctly reported her income from the real effort task; and 2) the percent of a subject's actual income from the real effort task that was not reported.  Here we estimate logit models with a dichotomous dependent variable coded zero for subjects who reported their actual winnings and coded one for those who reported amounts that deviated from their actual winnings.  We report clustered standard errors to reflect that fact that each subject plays the tax compliance game ten times.  \footnote{Table~\ref{tab:evade2b} in Appendix 1 replicates the analysis in Table~\ref{tab:cheat1} with a dependent variable that measures the percent of a subject's earnings that were not reported.  The results essentially confirm the findings from Table~\ref{tab:cheat1}.}

\par Table~\ref{tab:cheat1} reports logit regression results for the dichotomous measure of cheating.  The first results column in Table~\ref{tab:cheat1} presents the estimated coefficients for all 2,840 decisions taken by 284 subjects in the 14 lab sessions in which participants were randomly matched to fixed partners.  Included in the model is \# of Additions  (number of correct additions) which is our measure of performance and Compliance Cost (Tax Rate X Actual Earnings).  The significant coefficient on performance suggests that across all treatment contexts High Performance types are much more likely to cheat than Low Performance types.  Compliance Cost, as was suggested by Table~\ref{tab:compare}, matters less than anticipated -- the coefficient is small and statistically insignificant.  Cheating is correlated with one's performance type rather than with the actual cost of compliance.

\par In the second column of Table~\ref{tab:cheat1} we report results for a fully-specified logit regression model that includes dummy variables for each of the treatment sessions and treatments interacted with the performance variable.  Of particular interest are these interaction terms -- two of them are reasonably large.  For those randomly assigned to low wages in the status treatment, performance seems to matter disproportionately more than in the other treatments.  And there is evidence here that the Redistribution treatment significantly moderated the correlation between performance (\# of Additions) and cheating.  

\par The subsequent columns of Table~\ref{tab:cheat1} present logit regression results separately for each of the treatment sessions.  For the Baseline treatment there is a positive relationship between performance and cheating (although the estimate is imprecisely estimated) and cheating increases with the Compliance Cost.  

%\par Results are reported for the Baseline treatment in which all subjects earn the same amount for each correct addition (10 pence).  This Baseline treatment represents a context in which the labor market rewards ability and there are no structural factors that cause some individuals to earn more for their (equal) ability than others.  There is a positive relationship between Performance and cheating although the estimate is imprecisely estimated.  Consistent with our second conjecture, cheating increases with the cost of tax compliance -- and this is statistically significant.  The strong correlation between Performance and Cost of Compliance and the relatively small N result in large clustered standard errors here and hence a relatively  imprecise estimate of the Performance and cheating relationship.

\par In the standalone Status Model the correlation between performance (\# of Additions) and cheating is positive and precisely estimated.   For those with low status in these sessions, there is a strong correlation between performance and cheating -- performing well in the ``unfair'' treatment results in particularly high levels of cheating.  The coefficient on the High Salary X Additions interaction term suggests that the correlation between performance and cheating for those assigned to the high status condition, while still positive, is moderated relative to the low status subjects.  Cost of compliance is statistically insignificant in this model.


%\par We conjectured that when an element of luck or status determines earnings, the high ability types might moderate their cheating behaviour.  The Status Model in Table~\ref{tab:cheat1} presents the results for ``status'' treatments in which subjects were randomly assigned to high and low wages.  The model includes a High Salary dummy term and its interaction with Performance.  The Status Model results provide even stronger support for our central conjecture -- the correlation between Performance and cheating is positive and precisely estimated. 

%\par The High Salary interaction terms in the Status Model suggests, though, that this correlation is different for those assigned to the low, as opposed to high, wage treatment.  For those with low status, there is a  very strong correlation between performance and cheating -- performing well in the ``unfair'' treatment results in high levels of cheating.  The coefficient on the High Salary X Additions interaction term suggests that the correlation between performance and cheating for those assigned to the high status condition, while still positive, is moderated relative to the low status subjects.

%In the Baseline treatment we expect performance to be correlated with cheating -- subjects with more correct additions should cheat more.    Nevertheless, even in the high tax condition, this positive relationship between Performance and cheating persists.

%Figure~\ref{fig:status1} presents predicted cheating probabilities based on the model results.  

%For the Baseline, Status, Shock and Redistribute treatments, the graphs plot the predicted probability of cheating against performance in the RET.  Each graph includes the Baseline plot as a benchmark.  The graphs on the left are for the tax rate of 10 percent and on the right for the tax rate of 30 percent.  In all treatments there are statistically significant positive relationships between performance and cheating. 



%\par We conjectured that when an element of luck or status determines earnings, the high ability types might moderate their cheating behaviour.  The Status Model in Table~\ref{tab:cheat1} and the first row of graphs in Figure~\ref{fig:status1} address this hypothesis -- they present the results for ``status'' treatments in which subjects were randomly assigned to high and low wages.  The model includes a High Salary dummy term and its interaction with Performance.  The Status Model results are similar to the baseline case with respect to the Performance variable -- there is a significant and positive correlation between performance and cheating.  

%\par The High Salary interaction terms in the second Status Model suggests, though, that this correlation is different for those assigned to the low, as opposed to high, wage treatment.  For those with low status, there is a  very strong correlation between performance and cheating -- performing well in the ``unfair'' treatment results in high levels of cheating.  The coefficient on the High Salary X Additions interaction term suggests that the correlation between performance and cheating for those assigned to the high status condition, while still positive, is moderated relative to the low status subjects.  But regardless of treatments, its clear from Figure~\ref{fig:status1} that the high ability types cheat more than those with low abilities.

%\par The negative coefficient on the cost of compliance variable suggests that in the status treatment cheating declines as the cost of compliance increases.  This may lend credence to the status conjecture -- to some extent those who realize ``windfall'' wealth, independent of their performance, cheat less suggesting the are willing to incur the costs of compliance.  The negative coefficient on the High Salary X Additions interaction term and the negative coefficient on cost of compliance provide some support for the status conjecture.  But what is interesting here is that the positive correlation between Performance and cheating persists, in spite of the status treatment condition, and for both low and high status subjects.

\begin{comment}
\begin{sidewaystable}[!htbp]  
\caption{Probit model of cheating regressed on performance}
  \label{tab:cheat1} 
\begin{tabular}{@{\extracolsep{3pt}}lcccccc} 
\\[-1.8ex]\hline 
\hline \\[-1.8ex] 
 & \multicolumn{6}{c}{\textit{Dependent variable:}} \\ 
\cline{2-7} 
\\[-1.8ex] & Baseline & Status & Status & Shock & Shock & Redistribute\\ 
\hline \\[-1.8ex] 
 \# of Additions & 0.036$^{**}$ & 0.283$^{***}$ & 0.400$^{***}$ & 0.113$^{***}$ & 0.095$^{***}$ & 0.060$^{***}$ \\ 
  & (0.018) & (0.029) & (0.054) & (0.024) & (0.031) & (0.018) \\ 
  & & & & & & \\ 
 Cost of Compliance & 0.003$^{***}$ & $-$0.002$^{***}$ & $-$0.001$^{***}$ & 0.001$^{***}$ & 0.002$^{***}$ & $-$0.0004 \\ 
  & (0.0004) & (0.0004) & (0.0005) & (0.0004) & (0.0005) & (0.0003) \\ 
  & & & & & & \\ 
 High Salary &  &  & 1.606$^{***}$ &  &  &  \\ 
  &  &  & (0.581) &  &  &  \\ 
  & & & & & & \\ 
 High SalaryXAdditions &  &  & $-$0.190$^{***}$ &  &  &  \\ 
  &  &  & (0.059) &  &  &  \\ 
  & & & & & & \\ 
 Receive Shock &  &  &  &  & $-$0.559 &  \\ 
  &  &  &  &  & (0.568) &  \\ 
  & & & & & & \\ 
 ShockXAdditions &  &  &  &  & 0.021 &  \\ 
  &  &  &  &  & (0.048) &  \\ 
  & & & & & & \\ 
 Constant & $-$0.428$^{**}$ & $-$1.315$^{***}$ & $-$2.487$^{***}$ & $-$0.845$^{***}$ & $-$0.661$^{*}$ & 0.531$^{***}$ \\ 
  & (0.186) & (0.248) & (0.502) & (0.279) & (0.356) & (0.170) \\ 
  & & & & & & \\ 
\hline \\[-1.8ex] 
Observations & 720 & 720 & 720 & 560 & 560 & 840 \\ 
Log Likelihood & $-$319.530 & $-$219.205 & $-$212.837 & $-$218.695 & $-$216.828 & $-$343.606 \\ 
Akaike Inf. Crit. & 645.061 & 444.411 & 435.674 & 443.390 & 443.655 & 693.212 \\ 
\hline 
\hline \\[-1.8ex] 
\textit{Note:}  & \multicolumn{6}{r}{$^{*}$p$<$0.1; $^{**}$p$<$0.05; $^{***}$p$<$0.01} \\ 
\end{tabular} 
\end{sidewaystable} 

\clearpage
\end{comment}

\begin{comment}

\begin{sidewaystable}[!htbp]  
%\begin{table}[!htbp] \centering 
\caption{Logit model of cheating regressed on performance}
  \label{tab:cheat1}  
\begin{tabular}{@{\extracolsep{3pt}}lcccccc} 
\\[-1.8ex]\hline 
\hline \\[-1.8ex] 
 & \multicolumn{6}{c}{\textit{Dependent variable:}} \\ 
\cline{2-7} 
\\[-1.8ex] & Baseline & Status & Status & Shock & Shock & Redistribute\\ 
\hline \\[-1.8ex] 
 \# of Additions & 0.056 & 0.511$^{***}$ & 0.769$^{***}$ & 0.202$^{***}$ & 0.164$^{*}$ & 0.111 \\ 
  & (0.077) & (0.095) & (0.187) & (0.078) & (0.087) & (0.074) \\ 
  & & & & & & \\ 
 Cost of Compliance & 0.005$^{**}$ & $-$0.004$^{**}$ & $-$0.003 & 0.003$^{**}$ & 0.004$^{*}$ & $-$0.001 \\ 
  & (0.002) & (0.002) & (0.002) & (0.001) & (0.002) & (0.002) \\ 
  & & & & & & \\ 
 High Salary &  &  & 3.356 &  &  &  \\ 
  &  &  & (2.054) &  &  &  \\ 
  & & & & & & \\ 
 High Salary X Additions &  &  & $-$0.401$^{**}$ &  &  &  \\ 
  &  &  & (0.203) &  &  &  \\ 
  & & & & & & \\ 
 Receive Shock &  &  &  &  & $-$1.233 &  \\ 
  &  &  &  &  & (1.222) &  \\ 
  & & & & & & \\ 
 Receive Shock X Additions &  &  &  &  & 0.057 &  \\ 
  &  &  &  &  & (0.083) &  \\ 
  & & & & & & \\ 
 Constant & $-$0.757 & $-$2.415$^{**}$ & $-$4.894$^{***}$ & $-$1.650 & $-$1.279 & 0.794 \\ 
  & (0.623) & (0.941) & (1.799) & (1.071) & (1.077) & (0.739) \\ 
  & & & & & & \\ 
\hline \\[-1.8ex] 
Observations & 720 & 720 & 720 & 560 & 560 & 840 \\ 
R$^{2}$ & 0.175 & 0.325 & 0.355 & 0.173 & 0.186 & 0.026 \\ 
%$\chi^{2}$ & 84.990$^{***}$ (df = 2) & 143.531$^{***}$ (df = 2) & 158.400$^{***}$ (df = 4) & 60.165$^{***}$ (df = 2) & 64.787$^{***}$ (df = 4) & 12.439$^{***}$ (df = 2) \\ 
\hline 
\hline \\[-1.8ex] 
\textit{Note:}  & \multicolumn{6}{r}{$^{*}$p$<$0.1; $^{**}$p$<$0.05; $^{***}$p$<$0.01} \\ 
\end{tabular} 
\end{sidewaystable} 
%\end{table} 

\clearpage
\end{comment}

\begin{comment}

% Table created by stargazer v.5.2 by Marek Hlavac, Harvard University. E-mail: hlavac at fas.harvard.edu
% Date and time: jue, oct 13, 2016 - 12:09:06
\begin{footnotesize}
\begin{table}[!htbp] \centering 
\caption{Logit model of cheating regressed on performance}
  \label{tab:cheat1} 
\begin{tabular}{@{\extracolsep{3pt}}lcccccc} 
\\[-1.8ex]\hline 
\hline \\[-1.8ex] 
 & \multicolumn{6}{c}{\textit{Estimated Logit Models}} \\ 
\cline{2-7} 
\\[-1.8ex] & All & All Full & Baseline & Status & Shock & Redistribute\\ 
\hline \\[-1.8ex] 
 Additions & 0.181$^{***}$ & 0.144$^{**}$ & 0.056 & 0.769$^{***}$ & 0.164$^{*}$ & 0.111 \\ 
  & (0.040) & (0.072) & (0.077) & (0.187) & (0.087) & (0.074) \\ 
%  & & & & & & \\ 
 Cost of Compliance & 0.0005 & 0.001 & 0.005$^{**}$ & $-$0.003 & 0.004$^{*}$ & $-$0.001 \\ 
  & (0.001) & (0.001) & (0.002) & (0.002) & (0.002) & (0.002) \\ 
%  & & & & & & \\ 
 High Status &  & $-$1.087 &  &  &  &  \\ 
  &  & (1.207) &  &  &  &  \\ 
 % & & & & & & \\ 
 Low Status &  & $-$4.349$^{**}$ &  &  &  &  \\ 
  &  & (1.833) &  &  &  &  \\ 
%  & & & & & & \\ 
 No Shock &  & $-$0.594 &  &  &  &  \\ 
  &  & (1.242) &  &  &  &  \\ 
 % & & & & & & \\ 
 Receive Shock &  & $-$1.149 &  &  &  &  \\ 
  &  & (1.311) &  &  &  &  \\ 
  %& & & & & & \\ 
 Redistribute &  & 1.330 &  &  &  &  \\ 
  &  & (1.039) &  &  &  &  \\ 
  %& & & & & & \\ 
 Additions X High Status &  & 0.089 &  &  &  &  \\ 
  &  & (0.097) &  &  &  &  \\ 
  %& & & & & & \\ 
  Additions X Low Status &  & 0.552$^{***}$ &  &  &  &  \\ 
  &  & (0.187) &  &  &  &  \\ 
 % & & & & & & \\ 
 Additions X No Shock &  & 0.064 &  &  &  &  \\ 
  &  & (0.103) &  &  &  &  \\ 
 % & & & & & & \\ 
 Additions X Receive Shock &  & 0.117 &  &  &  &  \\ 
  &  & (0.109) &  &  &  &  \\ 
 % & & & & & & \\ 
 Additions X Redistribute &  & $-$0.090 &  &  &  &  \\ 
  &  & (0.092) &  &  &  &  \\ 
 % & & & & & & \\ 
 High Status &  &  &  & 3.356 &  &  \\ 
  &  &  &  & (2.054) &  &  \\ 
 % & & & & & & \\ 
 Additions X High Status &  &  &  & $-$0.401$^{**}$ &  &  \\ 
  &  &  &  & (0.203) &  &  \\ 
 % & & & & & & \\ 
 Receive Shock &  &  &  &  & $-$1.233 &  \\ 
  &  &  &  &  & (1.222) &  \\ 
 % & & & & & & \\ 
 Additions X Receive Shock &  &  &  &  & 0.057 &  \\ 
  &  &  &  &  & (0.083) &  \\ 
 % & & & & & & \\ 
 Constant & $-$0.556 & $-$0.540 & $-$0.757 & $-$4.894$^{***}$ & $-$1.279 & 0.794 \\ 
  & (0.426) & (0.683) & (0.623) & (1.799) & (1.077) & (0.739) \\ 
 % & & & & & & \\ 
\hline \\[-1.8ex] 
Observations & 2,840 & 2,840 & 720 & 720 & 560 & 840 \\ 
R$^{2}$ & 0.099 & 0.151 & 0.175 & 0.355 & 0.186 & 0.026 \\ 
%$\chi^{2}$ & 170.461$^{***}$ (df = 2) & 263.833$^{***}$ (df = 12) & 84.990$^{***}$ (df = 2) & 158.400$^{***}$ (df = 4) & 64.787$^{***}$ (df = 4) & 12.439$^{***}$ (df = 2) \\ 
\hline 
\hline \\[-1.8ex] 
\textit{Note:}  & \multicolumn{6}{r}{$^{*}$p$<$0.1; $^{**}$p$<$0.05; $^{***}$p$<$0.01} \\ 
\end{tabular} 
\end{table}
\end{footnotesize}

\clearpage
\end{comment}

\begin{comment}

% Table created by stargazer v.5.2 by Marek Hlavac, Harvard University. E-mail: hlavac at fas.harvard.edu
% Date and time: lun, oct 17, 2016 - 15:34:51
\begin{table}[!htbp] \centering \footnotesize
\caption{Logit model of cheating regressed on performance}
  \label{tab:cheat1} 
\begin{tabular}{@{\extracolsep{5pt}}lccccccc} 
\\[-1.8ex]\hline 
\hline \\[-1.8ex] 
% & \multicolumn{7}{c}{\textit{Dependent variable:}} \\ 
%\cline{2-8} 
%\\[-1.8ex] & (1) & (2) & (3) & (4) & (5) & (6) & (7)\\ 
\\[-1.8ex] & Full & Full & Baseline & Status & Shock & Redistribute  & Non-Fixed\\ 
\hline \\[-1.8ex] 
Additions & 0.181$^{***}$ & 0.144$^{**}$ & 0.056 & 0.769$^{***}$ & 0.164$^{*}$ & 0.111 & 0.157$^{**}$ \\ 
  & (0.040) & (0.072) & (0.077) & (0.187) & (0.087) & (0.074) & (0.077) \\ 
Compliance  & 0.0005 & 0.001 & 0.005$^{**}$ & $-$0.003 & 0.004$^{*}$ & $-$0.001 & $-$0.001 \\ 
Cost  & (0.001) & (0.001) & (0.002) & (0.002) & (0.002) & (0.002) & (0.002) \\ 
 High Status &  & $-$1.087 &  &  &  &  &  \\ 
  &  & (1.207) &  &  &  &  &  \\ 
 Low Status &  & $-$4.349$^{**}$ &  &  &  &  &  \\ 
  &  & (1.833) &  &  &  &  &  \\ 
 No Shock &  & $-$0.594 &  &  &  &  &  \\ 
  &  & (1.242) &  &  &  &  &  \\ 
 Receive Shock &  & $-$1.149 &  &  &  &  &  \\ 
  &  & (1.311) &  &  &  &  &  \\ 
 Redistribute &  & 1.330 &  &  &  &  &  \\ 
  &  & (1.039) &  &  &  &  &  \\ 
 Additions X  &  & 0.089 &  &  &  &  &  \\ 
 High Status &  & (0.097) &  &  &  &  &  \\ 
  Additions X  &  & 0.552$^{***}$ &  &  &  &  &  \\ 
  Low Status &  & (0.187) &  &  &  &  &  \\ 
  Additions X &  & 0.064 &  &  &  &  &  \\ 
   No Shock &  & (0.103) &  &  &  &  &  \\ 
 Additions X  &  & 0.117 &  &  &  &  &  \\ 
  Receive Shock &  & (0.109) &  &  &  &  &  \\ 
 Additions X  &  & $-$0.090 &  &  &  &  &  \\ 
  Redistribute &  & (0.092) &  &  &  &  &  \\ 
 High Status &  &  &  & 3.356 &  &  &  \\ 
  &  &  &  & (2.054) &  &  &  \\ 
 Additions X  &  &  &  & $-$0.401$^{**}$ &  &  &  \\ 
  High Status &  &  &  & (0.203) &  &  &  \\ 
 Receive Shock &  &  &  &  & $-$1.233 &  &  \\ 
  &  &  &  &  & (1.222) &  &  \\ 
 Additions X &  &  &  &  & 0.057 &  &  \\ 
  Receive Shock &  &  &  &  & (0.083) &  &  \\ 
 Constant & $-$0.556 & $-$0.540 & $-$0.757 & $-$4.894$^{***}$ & $-$1.279 & 0.794 & $-$0.269 \\ 
  & (0.426) & (0.683) & (0.623) & (1.799) & (1.077) & (0.739) & (0.648) \\ 
\hline \\[-1.8ex] 
Observations & 2,840 & 2,840 & 720 & 720 & 560 & 840 & 880 \\ 
R$^{2}$ & 0.099 & 0.151 & 0.175 & 0.355 & 0.186 & 0.026 & 0.064 \\ 
%$\chi^{2}$ & 170.461$^{***}$ (df = 2) & 263.833$^{***}$ (df = 12) & 84.990$^{***}$ (df = 2) & 158.400$^{***}$ (df = 4) & 64.787$^{***}$ (df = 4) & 12.439$^{***}$ (df = 2) & 38.119$^{***}$ (df = 2) \\ 
\hline 
\hline \\[-1.8ex] 
\textit{Note:}  & \multicolumn{7}{r}{$^{*}$p$<$0.1; $^{**}$p$<$0.05; $^{***}$p$<$0.01} \\ 
\end{tabular} 
\end{table}
\end{comment}

% Table created by stargazer v.5.2 by Marek Hlavac, Harvard University. E-mail: hlavac at fas.harvard.edu
% Date and time: mar, nov 08, 2016 - 12:59:43

\begin{comment}

\begin{table}[!htbp] \centering \footnotesize
\caption{Logit model of cheating regressed on performance}
  \label{tab:cheat1} 
  \begin{tabular}{@{\extracolsep{5pt}}lccccccc} 
\\[-1.8ex]\hline 
\hline \\[-1.8ex] 
\\[-1.8ex] & Full & Full & Baseline & Status & Shock & Redistribute  & Non-Fixed\\  
\hline \\[-1.8ex] 
 \# of Additions & 0.181$^{}$ & 0.144$^{}$ & 0.056 & 0.769$^{}$ & 0.164$^{}$ & 0.111 & 0.157$^{}$ \\ 
  & (0.040) & (0.072) & (0.077) & (0.187) & (0.087) & (0.074) & (0.077) \\ 
 Compliance & 0.0005 & 0.001 & 0.005$^{}$ & $-$0.003 & 0.004$^{}$ & $-$0.001 & $-$0.001 \\ 
 Cost & (0.001) & (0.001) & (0.002) & (0.002) & (0.002) & (0.002) & (0.002) \\ 
 High Status &  & $-$1.087 &  &  &  &  &  \\ 
  &  & (1.207) &  &  &  &  &  \\ 
 Low Status &  & $-$4.349$^{}$ &  &  &  &  &  \\ 
  &  & (1.833) &  &  &  &  &  \\ 
 No Shock &  & $-$0.594 &  &  &  &  &  \\ 
  &  & (1.242) &  &  &  &  &  \\ 
 Receive Shock &  & $-$1.149 &  &  &  &  &  \\ 
  &  & (1.311) &  &  &  &  &  \\ 
 Redistribute &  & 1.330 &  &  &  &  &  \\ 
  &  & (1.039) &  &  &  &  &  \\ 
 \# of Additions X &  & 0.089 &  &  &  &  &  \\ 
   High Status&  & (0.097) &  &  &  &  &  \\ 
  \# of Additions X  &  & 0.552$^{}$ &  &  &  &  &  \\ 
  Low Status&  & (0.187) &  &  &  &  &  \\ 
  \# of Additions X  &  & 0.064 &  &  &  &  &  \\ 
  No Shock &  & (0.103) &  &  &  &  &  \\ 
 \# of Additions X  &  & 0.117 &  &  &  &  &  \\ 
  Receive Shock &  & (0.109) &  &  &  &  &  \\ 
 \# of Additions X  &  & $-$0.090 &  &  &  &  &  \\ 
 Redistribute &  & (0.092) &  &  &  &  &  \\ 
 High Status &  &  &  & 3.356 &  &  &  \\ 
  &  &  &  & (2.054) &  &  &  \\ 
 \# of Additions X  &  &  &  & $-$0.401$^{}$ &  &  &  \\ 
 High Status &  &  &  & (0.203) &  &  &  \\ 
 Receive Shock &  &  &  &  & $-$1.233 &  &  \\ 
  &  &  &  &  & (1.222) &  &  \\ 
 \# of Additions X  &  &  &  &  & 0.057 &  &  \\ 
 Receive Shock &  &  &  &  & (0.083) &  &  \\ 
 Constant & $-$0.556 & $-$0.540 & $-$0.757 & $-$4.894$^{}$ & $-$1.279 & 0.794 & $-$0.269 \\ 
  & (0.426) & (0.683) & (0.623) & (1.799) & (1.077) & (0.739) & (0.648) \\ 
\hline \\[-1.8ex] 
AIC & 2354.64 & 2281.27 & 647.1 & 435.47 & 442.26 & 693.22 & 932.68 \\ 
Observations & 2,840 & 2,840 & 720 & 720 & 560 & 840 & 880 \\ 
\hline 
\hline \\[-1.8ex] 
%\textit{Note:}  & \multicolumn{7}{r}{$^{*}$p$<$0.1; $^{**}$p$<$0.05; $^{***}$p$<$0.01} \\ 
\end{tabular} 
\end{table} 

\end{comment}

\clearpage

% Table created by stargazer v.5.2 by Marek Hlavac, Harvard University. E-mail: hlavac at fas.harvard.edu
% Date and time: mar, nov 08, 2016 - 12:59:43

\begin{table}[!htbp] \centering \footnotesize
\caption{Logit model of cheating regressed on performance}
  \label{tab:cheat1} 
\begin{tabular}{@{\extracolsep{5pt}}lccccccc} 
\\[-1.8ex]\hline 
\hline \\[-1.8ex] 
\\[-1.8ex] & Full & Full & Baseline & Status & Shock & Redistribute  & Non-Fixed\\  
\hline \\[-1.8ex] 
 \# of Additions & 0.181$^{}$ & 0.144$^{}$ & 0.056 & 0.769$^{}$ & 0.164$^{}$ & 0.111 & 0.157$^{}$ \\ 
  & (0.040) & (0.072) & (0.077) & (0.187) & (0.087) & (0.074) & (0.077) \\ 
 Compliance & 0.0005 & 0.001 & 0.005$^{}$ & $-$0.003 & 0.004$^{}$ & $-$0.001 & $-$0.001 \\ 
 Cost & (0.001) & (0.001) & (0.002) & (0.002) & (0.002) & (0.002) & (0.002) \\ 
 High Status &  & $-$1.087 &  &  &  &  &  \\ 
  &  & (1.207) &  &  &  &  &  \\ 
 Low Status &  & $-$4.349$^{}$ &  &  &  &  &  \\ 
  &  & (1.833) &  &  &  &  &  \\ 
 No Shock &  & $-$0.594 &  &  &  &  &  \\ 
  &  & (1.242) &  &  &  &  &  \\ 
 Receive Shock &  & $-$1.149 &  &  &  &  &  \\ 
  &  & (1.311) &  &  &  &  &  \\ 
 Redistribute &  & 1.330 &  &  &  &  &  \\ 
  &  & (1.039) &  &  &  &  &  \\ 
 \# of Additions X &  & 0.089 &  &  &  &  &  \\ 
   High Status&  & (0.097) &  &  &  &  &  \\ 
  \# of Additions X  &  & 0.552$^{}$ &  &  &  &  &  \\ 
  Low Status&  & (0.187) &  &  &  &  &  \\ 
  \# of Additions X  &  & 0.064 &  &  &  &  &  \\ 
  No Shock &  & (0.103) &  &  &  &  &  \\ 
 \# of Additions X  &  & 0.117 &  &  &  &  &  \\ 
  Receive Shock &  & (0.109) &  &  &  &  &  \\ 
 \# of Additions X  &  & $-$0.090 &  &  &  &  &  \\ 
 Redistribute &  & (0.092) &  &  &  &  &  \\ 
 High Status &  &  &  & 3.356 &  &  &  \\ 
  &  &  &  & (2.054) &  &  &  \\ 
 \# of Additions X  &  &  &  & $-$0.401$^{}$ &  &  &  \\ 
 High Status &  &  &  & (0.203) &  &  &  \\ 
 Receive Shock &  &  &  &  & $-$1.233 &  &  \\ 
  &  &  &  &  & (1.222) &  &  \\ 
 \# of Additions X  &  &  &  &  & 0.057 &  &  \\ 
 Receive Shock &  &  &  &  & (0.083) &  &  \\ 
 Constant & $-$0.556 & $-$0.540 & $-$0.757 & $-$4.894$^{}$ & $-$1.279 & 0.794 & $-$0.269 \\ 
  & (0.426) & (0.683) & (0.623) & (1.799) & (1.077) & (0.739) & (0.648) \\ 
\hline \\[-1.8ex] 
AIC & 2354.64 & 2281.27 & 647.1 & 435.47 & 442.26 & 693.22 & 932.68 \\ 
Observations & 2,840 & 2,840 & 720 & 720 & 560 & 840 & 880 \\ 
\hline 
\hline \\[-1.8ex] 
%\textit{Note:}  & \multicolumn{7}{r}{$^{*}$p$<$0.1; $^{**}$p$<$0.05; $^{***}$p$<$0.01} \\ 
\end{tabular} 
\end{table} 

\clearpage

%\par Our Shock treatments are designed to exaggerate, or strengthen, the ``windfall'' framing of the bonus earnings given to randomly selected subjects.  Subjects randomly assigned to the shock treatment receive an extra bonus, after performing the real effort task (RET), that is clearly differentiated from their earnings associated with performance on the real effort task. This framing of a windfall bonus might induce less cheating on the part of those who are the lucky beneficiaries.  

\par For the Shock version of the tax compliance experiment, the Performance coefficient is positive and statistically significant.  The Receive Shock X Additions interaction term is positive but the magnitude of the interaction coefficient is relatively small and it is not precisely estimated.  The ``Redistribute'' Model in Table~\ref{tab:cheat1} presents the impact of a treatment in which each group's tax revenues are distributed unequally with a higher percentage going to the two subjects that performed most poorly.  While the coefficients on \# of Additions and Compliance Cost remain positive they are not statistically significant. 

%\par We conjectured that cheating behaviour might be conditioned on how revenues from the tax compliance experiment were being distributed.  


%The expectation here is that high performance types might be more generous when the tax regime is more ``fair'' and redistributive.  There is some evidence to this effect in the Redistribute Model results.  While the coefficient on Additions remains positive its quite imprecisely estimated and the Cost of Compliance is also not statistically significant.  



\paragraph{Learning.} We conjecture that individuals know their Performance type -- this is not something that they learn as part of the game.  There are a number of features of the experimental design and multivariate estimation that allow us to assess the extent to which this is in fact the case.  First, with respect to the experimental design.  Subjects are never informed about the distribution of performance types or of the RET income distribution.  Hence, the robust correlation we observe between performance and cheating suggests that subjects are very aware of their ``Performance type''.  Any notion that cheating is emboldened simply by being rich, (i.e., by one's wealth relative to others in the population) or by having won a competition (in which there are identifiable losers), is clearly challenged by these experimental results.  Subjects are never explicitly provided with information regarding their relative ``wealth'' or ``success'' in the repeated game.

\par In spite of never being told about their relative performance, subjects might learn this over the repeated play of the same game. Subjects play the tax compliance game over 10 periods which might result in some learning about one's performance levels; in particular, subjects observe how much they earn and also their revenues from tax redistribution.  But learning unlikely accounts for the Performance effect we observe on cheating.  To control for learning, we estimated all of the models in Table~\ref{tab:cheat1} for only the first period of each of the different treatment sessions.  We present the results in Table~\ref{tab:round1} in the Online Appendix.  For the most part we find that Performance remains strongly, and positively, correlated with cheating.  Its unlikely that our results in Table~\ref{tab:cheat1} are an artefact of learning.
 
\par  As we pointed out earlier, the structure of these tax compliance games resemble a standard public goods game.  Recall that subjects are randomly matched to partners and they keep these partners over the 10 iterations of the game.  As a result, strategic considerations could lead subjects, particularly the Low Performers, to comply with prevailing tax rates in order to encourage others, in particular High Performers, to cooperate, i.e., report their actual earnings, in subsequent rounds of the game.  There is also good evidence to suggest that in these public goods games, subjects become particularly uncooperative as they play the final rounds of the game.  One could imagine that our Performance type effects could be confounded with strategic behaviour by subjects playing our repeated tax compliance game.   These concerns primarily result from our decision to have a fixed partner matching design.  Accordingly, we implement a series of sessions in which subjects were randomly re-matched to a new group after each iteration of the tax compliance game.  We present, in the last column, results for the these Non-fixed matching sessions (i.e., the Non-Fixed model).  Again consistent with the results in Table~\ref{tab:cheat1}, the Performance coefficient is positive and precisely estimated while the Cost of Compliance coefficient is not significant.  There is no evidence that our results are an artefact of strategic reasoning and learning associated with the fixed partner matching design.


\paragraph{Predicted Rates of Cheating}

%\par The Status treatment also confirms our conjecture regarding Performance and cheating.  In both the High and Low Status treatments, the relationship is positive with high performance types cheating at, statistically significant, higher levels than the low performance types.  The difference from the Baseline treatment is that in the Status treatment cheating does not seem to respond to the cost of compliance.  In the Low Status treatment, high performance types are very aggressive cheaters regardless of whether they are in the low or high cost condition.  High performance types are less aggressive cheater if they are lucky enough to be in the High Status treatment -- in fact they seem to be somewhat more compliant in the high cost, as opposed to, low cost treatment.

\par The estimated coefficients in Table~\ref{tab:cheat1} constitute strong support for our conjecture that High Performance types will cheat more than Low Performance types.  Surprisingly, Compliance Cost had a much weaker affect on the decision to cheat.  And for the most part there was not much contextual variation in this performance effect on cheating.  Figure~\ref{fig:status1} summarizes the model estimates from Figure~\ref{fig:status1} by presenting the predicted probabilities of cheating for particular performance levels and for the 10\% and 30\% tax regimes.   We generate predicted probabilities for three performance levels for which we have large numbers of observed values in the actual data: 7, 12, and 17 correct additions.

\par Figure~\ref{fig:status1} confirms our conclusions from the multivariate discussion.  First, cost of compliance does not appear to have a particularly strong affect on cheat -- the differences between the 10\% and 30\% tax regimes are not large.  On the other hand performance clearly matters -- we see a reasonably strong positive correlation between performance and the probability of cheating.  And in most cases, predicted cheating at the 17 correct addition level is significantly higher than predicted cheating for those correctly adding 7 pairs of two-digit numbers. The one exception is the Redistribution treatment in which there is no discernable difference between High and Low Performance types.  But this appears to result from a rise in cheating by Low Performance types under redistribution rather than a drop in cheating by the High Performers.

\begin{figure}[p]
\caption{Predicted Probability of Cheating for Treatments: 10\% \& 30\% Tax}\label{fig:status1}
\centerline{\includegraphics[width=1.25\textwidth]{fig/treatment_v_tax-2.pdf}}
\end{figure}

\clearpage



\section{Discussion}

\par The goal of this essay is to understand why we cheat.  Tax compliance experiments are the means by which we identify the causal mechanisms associated with cheating. Our results are consistent with recent findings indicating heterogeneity in lying and cheating {\citep{Gneezyetal2013,Cappelenetal2013,Hurkensetal2009}.  Subjects in our experiments who perform better on real effort tasks cheat more in this tax compliance game.  This supports our central claim, Conjecture 1, that ability triggers cheating.  

%\par As is the case with much of this literature we also find that cheating responds to its costs and benefits.  But we also find that particular types in the population have a higher proclivity to cheat than others -- specifically high performance types are more likely to cheat than low ability types.

%\par  We gain insight into the robustness of this causal claim from the experimental design.  

\par The cost of complying with taxation should affect cheating -- our Conjecture 2.   We observe cheating by Low and High Performance types in different tax treatments and with different earnings.  Surprisingly, there is only weak evidence that cheating rises when the price of tax compliance increases.  Performance type is a much better predictor of the likelihood of cheating.

%  Nevertheless, we observe that high performance types are consistently cheating more than low performance types regardless of the costs associated with compliance.  The results suggest that its not simply wealth or earnings per se that trigger cheating but rather ability.

%\par Experimental treatments were implemented to explore whether High Performance types moderate their cheating under certain conditions. 

\par Conjecture 4 suggests that High Performers cheat less in contexts where they perceive luck or status to play an important role in determining their success.  When our subjects' earnings are associated with luck, there is no signifiant moderation in the cheating gap between high and low performance types.  For those assigned to the high wage treatment, the correlation between Performance and tax cheating is essentially the same as in the treatment in which all wages were the same.  For those assigned to the low wage treatment we actually see a significant increase in the positive correlation between Performance and cheating.  In another treatment, half the subjects were randomly assigned to receive a lucky bonus after they competed their real effort task.  The correlation between Performance and cheating remained positive and significant for both those receiving and not receiving the random bonus payment.
 
\par Context was not irrelevant in our experimental sessions.  Implementing a more redistributive tax regime eliminated the difference in cheating between High and Low Performance types -- our Conjecture 5.  But this results because Low Performers cheat more rather than because High Performance types cheat less!  

\par Conjecture 6 suggests that this predilection for cheating by High Performance types is not confined to tax compliance but rather reflects a general behavioral trait.  Results from choices made in other games by the same subject confirm this is the case: High Performance types give less money in a conventional Dictator Game and also are more likely to cheat in a classic die game in which they privately report the results of tossing a die.

\paragraph{Implications} Cheating is costly. Annual bribes in the world are estimated to exceed \$1 Trillion and these negatively affect provision of public services and increase inequality \citep{IMF2016}.  There is growing evidence of the economic costs associated with cheating -- for example, \citet{Balafoutasetal2015} on the economic efficiency costs associated with tax evasion in credence goods.  It comes as no surprise that with globalization and technological advances in financial transactions, the rich availed themselves of tax havens and banking services that allow them to avoid paying taxes.  A growing portion of the wealth of the richest Americans is held in off-shore banking havens \citep{Zucman2015}. 

\par Reducing cheating likely improves the functioning of markets and of our political institutions.  The public and private sectors invest considerable resources aimed at reducing cheating.  Achieving this goal, though, requires a better understanding of why, and under what circumstances, individuals cheat.  

\par Our findings suggest why poorly-specified models of the decision to cheat can generate unintended, if not contradictory, policy outcomes.   Most accounts of cheating ignore its heterogeneity in the population.  We find though that High Performance types have much higher proclivities for cheating than do Low Performers.  Policies designed to reduce cheating, but that ignore this heterogeneity, could have no effect, or possibly result in perverse outcomes.  In fact, maybe there are circumstances in which we want to facilitate cheating!

\par Efforts to increase tax compliance are a case in point.  A popular recent strategy, building on experimental evidence, is to promote timely tax compliance with framing strategies that appeal to intrinsic motivations or reputational concerns  \citep{Blumenthaletal2001,Fellneretal2013,Castroetal2015}.  Many of these efforts (and the findings on which they are based) ignore the Low/High Performance heterogeneity in the taxpayer population.  Our results though suggest that ignoring heterogeneity in the case of these policies might generate higher tax revenues but at a cost -- higher levels of post-tax income inequality.   Its not simply that High Performance types (who are often rich) cheat more at their taxes.  But its also the case that Low Performance types, typically the poor, respond more positively to these framing appeals than the High Performance types, i.e, the rich.\footnote{The asymmetry we find in our experiments resembles the heterogeneous effect that \citet{Dwengeretal2015} report for their ``compliance'' rewards treatments that consisted of either a social or private recognition for compliance with the German local church tax.  They find that it has a positive effect on subjects who are baseline ``donor'' types (those who overpay in pre-treatment) and a negative effect on subjects who are baseline ``evader'' types (those who underpay in pre-treatment).} 

\par These results also concern the design of employment contracts.  Our results are consistent with \citet{Gilletal2013} who find that subjects who are more productive in their real effort tasks also cheat more.  It is true that employee theft represents a serious cost to the economy.  On the other hand, its not clear that simply adopting policies that eliminate the possibility of cheating would be optimal.  In a competitive labor market, firms who want to attract High Performance types may need to create, or at least tolerate, opportunities for cheating.  Employment contracts that make it impossible, or extremely costly, to cheat might not be in the best interests of competitive firms.

\par A related issue concerns the reduction of corrupt practices in the public sector.  Evidence of corruption in the public sector is voluminous.  An influential literature inspired by \citet{Beckeretal1974}, in particular, identifies raising wages as one strategy for reducing corruption in developing countries where public sector salaries are very low \citep{Akerloffetal1990}.  These insights have had a significant impact on policies designed to reduce public sector corruption \citep{Olkenetal2012}.  Our strong correlation between performance and cheating proclivities suggests that simply raising wages might result, at least in the short term, in perverse outcomes. The higher salaries (along with the opportunities to cheat) could entice High Performance types into public sector employment which could raise the level of corrupt activity.  This might explain why some efforts at reducing public sector corruption by increasing wages in low-income developing countries have disappointed and, in some instances, had the opposite outcome.\footnote{A case in point is the experience the Ghana government had with their civil service salary reform.  Raising the salaries of enforcement officers resulted in higher levels of transport-related bribes \citep{Foltzetal2016}!}

\par Cheating results in significant economic costs both in the public and private sectors.  Cheating proclivities are strongly correlated with ability or performance.  This has consequences for how we model cheating behavior and for policies designed to reduce its occurrence in the public and private sectors. 

\begin{comment}
Our results suggest that it will be extremely difficult to exploit intrinsic motivation or reputational concerns to reduce such behaviour on the part of high performance types who are disproportionately represented amongst the rich.  We explored a number of different contextual factors that we thought might appeal to the intrinsic motivation of those with high abilities.  None of these eliminated the strong correlation between performance and cheating.

\par The experimental results suggest the p term.....  In the case of tax compliance.... Of particular concern for redistributive taxation is our experimental finding indicating that those with high abilities are more likely to exhibit weak intrinsic motivations.  Given that the rich are often high performance types this does not bode well for efforts to reduce inequality through redistributive taxation.   

\par With respect to corrupt practices by government employees.....

\par Corrupt business practices....   Another implication of these findings is that High and Low Performance types will sort according to the opportunities for cheating.  Holding renumeration constant, this suggests that firms or sectors of the economy that offer more opportunities for cheating may be more attractive to High Performance types.  These cheating opportunities might come in the form of arrangements for rewarding productivity... Its not simply the case, as many have pointed out, that high productivity types opt into/prefer performance based/piece rate compensation packages because they are advantaged compared to hourly/salary but it may be that high productivity types are attracted by the opportunities for misreporting their performance (David Gill hints at this.... ).   Eliminating cheating in a firm might not be an optimal strategy at least from the perspective of its human capital.

vary -- there are different cheating ``types'' in the population.  Efforts to manage cheating must be tailored to these cheating ``types'' in the population.  Our contribution is to identify one characteristic that prominently singles out cheaters -- high performance types.



\par These results provide an important insight into tax compliance.  Individuals who are successful because of ability or effort are going to cheat -- data on tax morale and actual tax avoidance indicate this is the case; as do our experimental results.  It comes as no surprise then that with globalization and technological advances in financial transactions, the rich availed themselves of tax havens and banking services that allow them to avoid paying taxes.  As \citet{Zucman2015} points out, a growing portion of the wealth of the richest Americans is held in off-shore banking havens. 

\end{comment}


 





\newpage
%\begin{singlespace}
\bibliography{dave}
%\end{singlespace}

\end{document}

\newpage

\begin{center}

\section*{FOR ONLINE PUBLICATION: APPENDIX}
\newpage

\end{center}

\subsection*{Appendix 1: Subjects' Performance on Real Effort Task}
\begin{figure}[p]
\caption{Real Effort Tasks: Correct Additions in Baseline, Status and Shock Treatments}\label{fig:additions}
\centerline{\includegraphics[width=\textwidth]{addition2.pdf}}
\end{figure}

\clearpage


\subsection*{Appendix 2: Regression on Percent of Earnings not Reported}


\par Table~\ref{tab:evade2b} replicates the analysis in Table~\ref{tab:cheat1} with a dependent variable that measures the percent of a subject's earnings that were not reported.  The results essentially confirm the findings from Table~\ref{tab:cheat1}.  Our principal conjecture from the main text is confirmed in all treatments: there is a significant positive correlation between earnings and cheating.

\begin{comment}

% Table created by stargazer v.5.1 by Marek Hlavac, Harvard University. E-mail: hlavac at fas.harvard.edu
% Date and time: Tue, Oct 06, 2015 - 11:08:09
\begin{sidewaystable}[!htbp] \centering 
  \caption{Percent Evaded Regressed on Performance} \label{tab:evade2b}
\begin{tabular}{@{\extracolsep{5pt}}lcccccc} 
\\[-1.8ex]\hline 
\hline \\[-1.8ex] 
 & \multicolumn{5}{c}{\textit{Dependent Variable: Percent Evaded}} \\ 
\cline{2-7} 
\\[-1.8ex] & Baseline & Status & Status & Shock & Shock & Redistribute \\ 
\hline \\[-1.8ex] 
 \# of Additions & 0.025 & 0.107$^{***}$ & 0.114$^{***}$ & 0.053$^{***}$ & 0.049$^{**}$ & 0.056$^{***}$ \\ 
  & (0.016) & (0.015) & (0.020) & (0.017) & (0.023) & (0.013) \\ 
  & & & & & & \\ 
 Cost of Compliance & 0.002$^{***}$ & $-$0.001$^{***}$ & $-$0.001$^{**}$ & 0.001$^{**}$ & 0.001$^{**}$ & $-$0.0003 \\ 
  & (0.0003) & (0.0002) & (0.0003) & (0.0003) & (0.0003) & (0.0002) \\ 
  & & & & & & \\ 
 High Salary &  &  & 0.119 &  &  &  \\ 
  &  &  & (0.337) &  &  &  \\ 
  & & & & & & \\ 
 High Salary X Additions &  &  & $-$0.018 &  &  &  \\ 
  &  &  & (0.026) &  &  &  \\ 
  & & & & & & \\ 
 Receive Shock &  &  &  &  & $-$0.096 &  \\ 
  &  &  &  &  & (0.417) &  \\ 
  & & & & & & \\ 
 Receive Shock X Additions &  &  &  &  & $-$0.008 &  \\ 
  &  &  &  &  & (0.031) &  \\ 
  & & & & & & \\ 
 Constant & $-$0.265 & $-$0.189 & $-$0.279 & $-$0.128 & $-$0.106 & 0.262$^{*}$ \\ 
  & (0.163) & (0.166) & (0.253) & (0.207) & (0.286) & (0.139) \\ 
  & & & & & & \\ 
\hline \\[-1.8ex]
$\phi$ & 0.409 & 0.466 & 0.467 & 0.429 & 0.431 & 0.434\\
& (0.017) & (0.021) & (0.021) & (0.021) & (0.021) & (0.018)\\
 \hline \\[-1.8ex]
Observations & 720 & 720 & 720 & 560 & 560 & 840 \\ 
Log Likelihood & $-$319.530 & $-$219.205 & $-$212.837 & $-$218.695 & $-$216.828 & $-$343.606 \\ 
Akaike Inf. Crit. & 645.061 & 444.411 & 435.674 & 443.390 & 443.655 & 693.212 \\ 
\hline 
\hline \\[-1.8ex] 
\textit{Note:}  & \multicolumn{5}{r}{$^{*}$p$<$0.1; $^{**}$p$<$0.05; $^{***}$p$<$0.01} \\ 
\end{tabular} 
\end{sidewaystable} 

\clearpage
\end{comment}


\begin{comment}

% Table created by stargazer v.5.2 by Marek Hlavac, Harvard University. E-mail: hlavac at fas.harvard.edu
% Date and time: lun, oct 17, 2016 - 16:37:07

\begin{table}[!htbp] \centering \footnotesize
  \caption{Percent Evaded Regressed on Performance} \label{tab:evade2b}
\begin{tabular}{@{\extracolsep{5pt}}lcccccc} 
\\[-1.8ex]\hline 
\hline \\[-1.8ex] 
% & \multicolumn{6}{c}{\textit{Dependent variable:}} \\ 
%\cline{2-7} 
%\\[-1.8ex] & (A.1) & (A.2) & (A.3) & (A.4) & (A.5) & (A.6)\\ 
\\[-1.8ex] & Full & Full & Baseline & Status & Shock & Redistribute  \\  
\hline \\[-1.8ex] 
 \# of Additions & 0.061$^{***}$ & 0.067$^{***}$ & 0.025 & 0.114$^{***}$ & 0.049$^{**}$ & 0.056$^{***}$ \\ 
  & (0.007) & (0.013) & (0.016) & (0.020) & (0.023) & (0.013) \\ 
 Cost of Compliance & 0.0001 & 0.0002 & 0.002$^{***}$ & $-$0.001$^{**}$ & 0.001$^{**}$ & $-$0.0003 \\ 
  & (0.0001) & (0.0001) & (0.0003) & (0.0003) & (0.0003) & (0.0002) \\ 
 High Status &  & 0.063 &  &  &  &  \\ 
  &  & (0.276) &  &  &  &  \\ 
 Low Status &  & $-$0.055 &  &  &  &  \\ 
  &  & (0.302) &  &  &  &  \\ 
 No Shock &  & 0.134 &  &  &  &  \\ 
  &  & (0.329) &  &  &  &  \\ 
 Receive Shock &  & 0.189 &  &  &  &  \\ 
  &  & (0.338) &  &  &  &  \\ 
 Redistribute &  & 0.483$^{**}$ &  &  &  &  \\ 
  &  & (0.213) &  &  &  &  \\ 
 \# of Additions X  &  & $-$0.005 &  &  &  &  \\ 
  High Status &  & (0.021) &  &  &  &  \\ 
  \# of Additions X  &  & 0.027 &  &  &  &  \\ 
  Low Status &  & (0.024) &  &  &  &  \\ 
  \# of Additions X &  & $-$0.003 &  &  &  &  \\ 
   No Shock &  & (0.025) &  &  &  &  \\ 
 \# of Additions X &  & $-$0.008 &  &  &  &  \\ 
 Receive Shock  &  & (0.026) &  &  &  &  \\ 
 \# of Additions X  &  & $-$0.029$^{*}$ &  &  &  &  \\ 
 Redistribute &  & (0.017) &  &  &  &  \\ 
 High Status &  &  &  & 0.119 &  &  \\ 
  &  &  &  & (0.337) &  &  \\ 
 \# of Additions X  &  &  &  & $-$0.018 &  &  \\ 
 High Status &  &  &  & (0.026) &  &  \\ 
 Receive Shock &  &  &  &  & $-$0.096 &  \\ 
  &  &  &  &  & (0.417) &  \\ 
 \# of Additions X  &  &  &  &  & $-$0.008 &  \\ 
  Receive Shock &  &  &  &  & (0.031) &  \\ 
 Constant & $-$0.034 & $-$0.239 & $-$0.265 & $-$0.279 & $-$0.106 & 0.262$^{*}$ \\ 
  & (0.082) & (0.162) & (0.163) & (0.253) & (0.286) & (0.139) \\ 
\hline \\[-1.8ex] 
Observations & 2,840 & 2,840 & 720 & 720 & 560 & 840 \\ 
R$^{2}$ & 0.068 & 0.083 & 0.140 & 0.154 & 0.091 & 0.045 \\ 
Log Likelihood & 9,244.081 & 9,253.867 & 2,263.922 & 2,449.045 & 1,853.218 & 2,709.413 \\ 
\hline 
\hline \\[-1.8ex] 
\textit{Note:}  & \multicolumn{6}{r}{$^{*}$p$<$0.1; $^{**}$p$<$0.05; $^{***}$p$<$0.01} \\ 
\end{tabular} 
\end{table} 
\clearpage

\end{comment}

\begin{comment}

% Table created by stargazer v.5.2 by Marek Hlavac, Harvard University. E-mail: hlavac at fas.harvard.edu
% Date and time: mi?, oct 26, 2016 - 8:47:20
\begin{table}[!htbp] \centering \footnotesize
  \caption{Percent Evaded Regressed on Performance}
   \label{tab:evade2b}
\begin{tabular}{@{\extracolsep{5pt}}lccccccc} 
\\[-1.8ex]\hline 
\hline \\[-1.8ex] 
% & \multicolumn{7}{c}{\textit{Dependent variable:}} \\ 
%\cline{2-8} 
%\\[-1.8ex] & (1) & (2) & (3) & (4) & (5) & (6) & (7)\\ 
\\[-1.8ex] & Full & Full & Baseline & Status & Shock & Redistribute  & Non-Fixed\\  
\hline \\[-1.8ex] 
 \# of Additions & 0.061$^{***}$ & 0.067$^{***}$ & 0.025 & 0.114$^{***}$ & 0.049$^{**}$ & 0.056$^{***}$ & 0.091$^{***}$ \\ 
  & (0.007) & (0.013) & (0.016) & (0.020) & (0.023) & (0.013) & (0.013) \\ 
 Compliance & 0.0001 & 0.0002 & 0.002$^{***}$ & $-$0.001$^{**}$ & 0.001$^{**}$ & $-$0.0003 & $-$0.0002 \\ 
 Cost & (0.0001) & (0.0001) & (0.0003) & (0.0003) & (0.0003) & (0.0002) & (0.0003) \\ 
 High Status &  & 0.063 &  &  &  &  &  \\ 
  &  & (0.276) &  &  &  &  &  \\ 
 Low Status &  & $-$0.055 &  &  &  &  &  \\ 
  &  & (0.302) &  &  &  &  &  \\ 
 No Shock &  & 0.134 &  &  &  &  &  \\ 
  &  & (0.329) &  &  &  &  &  \\ 
 Receive Shock &  & 0.189 &  &  &  &  &  \\ 
  &  & (0.338) &  &  &  &  &  \\ 
 Redistribute &  & 0.483$^{**}$ &  &  &  &  &  \\ 
  &  & (0.213) &  &  &  &  &  \\ 
 \# of Additions  &  & $-$0.005 &  &  &  &  &  \\ 
 X High Status &  & (0.021) &  &  &  &  &  \\ 
  \# of Additions  &  & 0.027 &  &  &  &  &  \\ 
 X Low Status &  & (0.024) &  &  &  &  &  \\ 
  \# of Additions  &  & $-$0.003 &  &  &  &  &  \\ 
  X No Shock &  & (0.025) &  &  &  &  &  \\ 
 \# of Additions  &  & $-$0.008 &  &  &  &  &  \\ 
  X Receive Shock &  & (0.026) &  &  &  &  &  \\ 
 \# of Additions  &  & $-$0.029$^{*}$ &  &  &  &  &  \\ 
 X Redistribute &  & (0.017) &  &  &  &  &  \\ 
 High Status &  &  &  & 0.119 &  &  &  \\ 
  &  &  &  & (0.337) &  &  &  \\ 
 # of Additions  &  &  &  & $-$0.018 &  &  &  \\ 
 X High Status &  &  &  & (0.026) &  &  &  \\ 
 Receive Shock &  &  &  &  & $-$0.096 &  &  \\ 
  &  &  &  &  & (0.417) &  &  \\ 
 # of Additions  &  &  &  &  & $-$0.008 &  &  \\ 
 X Receive Shock &  &  &  &  & (0.031) &  &  \\ 
 Constant & $-$0.034 & $-$0.239 & $-$0.265 & $-$0.279 & $-$0.106 & 0.262$^{*}$ & $-$0.577$^{***}$ \\ 
  & (0.082) & (0.162) & (0.163) & (0.253) & (0.286) & (0.139) & (0.137) \\ 
  & & & & & & & \\ 
\hline \\[-1.8ex] 
Observations & 2,840 & 2,840 & 720 & 720 & 560 & 840 & 880 \\ 
R$^{2}$ & 0.068 & 0.083 & 0.140 & 0.154 & 0.091 & 0.045 & 0.104 \\ 
Log Likelihood & 9,244.081 & 9,253.867 & 2,263.922 & 2,449.045 & 1,853.218 & 2,709.413 & 2,538.240 \\ 
\hline 
\hline \\[-1.8ex] 
\textit{Note:}  & \multicolumn{7}{r}{$^{*}$p$<$0.1; $^{**}$p$<$0.05; $^{***}$p$<$0.01} \\ 
\end{tabular} 
\end{table} 

\end{comment}

% Table created by stargazer v.5.2 by Marek Hlavac, Harvard University. E-mail: hlavac at fas.harvard.edu
% Date and time: mi?, nov 09, 2016 - 15:45:32
\begin{table}[!htbp] \centering \footnotesize
  \caption{Percent Evaded Regressed on Performance}
   \label{tab:evade2b}
\begin{tabular}{@{\extracolsep{5pt}}lccccccc} 
\\[-1.8ex]\hline 
\hline \\[-1.8ex] 
\\[-1.8ex] & Full & Full & Baseline & Status & Shock & Redistribute  & Non-Fixed\\  
\hline \\[-1.8ex] 
 \# of Additions & 0.061$^{}$ & 0.067$^{}$ & 0.025 & 0.114$^{}$ & 0.049$^{}$ & 0.056$^{}$ & 0.091$^{}$ \\ 
  & (0.007) & (0.013) & (0.016) & (0.020) & (0.023) & (0.013) & (0.013) \\ 
 Compliance & 0.0001 & 0.0002 & 0.002$^{}$ & $-$0.001$^{}$ & 0.001$^{}$ & $-$0.0003 & $-$0.0002 \\ 
 Cost & (0.0001) & (0.0001) & (0.0003) & (0.0003) & (0.0003) & (0.0002) & (0.0003) \\ 
 High Status &  & 0.063 &  &  &  &  &  \\ 
  &  & (0.276) &  &  &  &  &  \\ 
 Low Status &  & $-$0.055 &  &  &  &  &  \\ 
  &  & (0.302) &  &  &  &  &  \\ 
 No Shock &  & 0.134 &  &  &  &  &  \\ 
  &  & (0.329) &  &  &  &  &  \\ 
 Receive Shock &  & 0.189 &  &  &  &  &  \\ 
  &  & (0.338) &  &  &  &  &  \\ 
 Redistribute &  & 0.483$^{}$ &  &  &  &  &  \\ 
  &  & (0.213) &  &  &  &  &  \\ 
 \# of Additions X  &  & $-$0.005 &  &  &  &  &  \\ 
 High Status &  & (0.021) &  &  &  &  &  \\ 
 \# of Additions X  &  & 0.027 &  &  &  &  &  \\ 
 Low Status &  & (0.024) &  &  &  &  &  \\ 
  \# of Additions X  &  & $-$0.003 &  &  &  &  &  \\ 
 No Shock &  & (0.025) &  &  &  &  &  \\ 
 \# of Additions X  &  & $-$0.008 &  &  &  &  &  \\ 
 Receive Shock &  & (0.026) &  &  &  &  &  \\ 
 \# of Additions X  &  & $-$0.029$^{}$ &  &  &  &  &  \\ 
 Redistribute &  & (0.017) &  &  &  &  &  \\ 
 High Status &  &  &  & 0.119 &  &  &  \\ 
  &  &  &  & (0.337) &  &  &  \\ 
 \# of Additions X  &  &  &  & $-$0.018 &  &  &  \\ 
 High Status &  &  &  & (0.026) &  &  &  \\ 
 Receive Shock &  &  &  &  & $-$0.096 &  &  \\ 
  &  &  &  &  & (0.417) &  &  \\ 
 \# of Additions X  &  &  &  &  & $-$0.008 &  &  \\ 
 Receive Shock &  &  &  &  & (0.031) &  &  \\ 
 Constant & $-$0.034 & $-$0.239 & $-$0.265 & $-$0.279 & $-$0.106 & 0.262$^{}$ & $-$0.577$^{}$ \\ 
  & (0.082) & (0.162) & (0.163) & (0.253) & (0.286) & (0.139) & (0.137) \\ 
 \hline \\[-1.8ex] 
AIC & -18480.16 & -18479.73 & -4519.84 & -4886.09 & -3694.44 & -5410.83 & -5068.48 \\ 
Observations & 2,840 & 2,840 & 720 & 720 & 560 & 840 & 880 \\ 
\hline 
\hline \\[-1.8ex] 
\end{tabular} 
\end{table} 

\clearpage

\subsection*{Appendix 3: Logit regression using first round of each session}

\par A conservative test of our argument regarding ability and cheating is to estimate the logit regression models only employing observations from the first round of each of the tax compliance experimental sessions.  Table~\ref{tab:round1} presents the results.  This essentially replicates Table~\ref{tab:cheat1} from the main text -- the only difference is that the estimates are based only on the decisions taken by the subjects in the first round of each session.  Having fewer observations results in less precise estimates but also reduces the amount of information available to distinguish between the highly correlated Addition and Cost of Compliance variables.  \\

\par Nevertheless, the results in Table~\ref{tab:round1} essentially confirm our argument that ability plays an important role in explaining cheating behaviour.  In most of the models, performance is positively, and significantly, correlated with cheating.  The one, important, exception is in the Baseline model in which the Addition variable is insignificant.  When we estimate the Baseline model without the Cost of Compliance control the coefficient on Addition is positive and weakly significant at the 0.1 level.  On balance though its clear that, controlling for the Cost of Compliance, ability is positively correlated with cheating.  

\par It might be the case that high ability types only begin to cheat aggressively because they observe that they are net contributors to the public good.  However, it does not seem to be the case that this correlation is an artefact of learning over the course of the 10 sessions each subject plays.  The correlation between ability and cheating seems to emerge at the outset of the game.

% Table created by stargazer v.5.2 by Marek Hlavac, Harvard University. E-mail: hlavac at fas.harvard.edu
% Date and time: lun, oct 17, 2016 - 16:36:54

% Table created by stargazer v.5.2 by Marek Hlavac, Harvard University. E-mail: hlavac at fas.harvard.edu
% Date and time: mar, nov 08, 2016 - 15:45:09

% Table created by stargazer v.5.2 by Marek Hlavac, Harvard University. E-mail: hlavac at fas.harvard.edu
% Date and time: mar, nov 08, 2016 - 15:45:09

\begin{table}[!htbp] \centering \footnotesize
  \caption{Logit regression using first round of each session} 
  \label{tab:round1} 
\begin{tabular}{@{\extracolsep{5pt}}lccccccc} 
\\[-1.8ex]\hline 
\hline \\[-1.8ex] 
\\[-1.8ex] & Full & Full & Baseline & Status & Shock & Redistribute  & Non-Fixed\\  
\hline \\[-1.8ex] 
 \# of Additions & 0.167$^{}$ & 0.105 & $-$0.004 & 0.717$^{}$ & 0.206 & 0.199$^{}$ & 0.167$^{}$ \\ 
  & (0.057) & (0.089) & (0.088) & (0.192) & (0.187) & (0.098) & (0.092) \\ 
 Compliance & 0.001 & 0.001 & 0.007$^{}$ & $-$0.001 & 0.003 & $-$0.001 & 0.001 \\ 
 Cost & (0.001) & (0.001) & (0.003) & (0.002) & (0.003) & (0.002) & (0.002) \\ 
  High Status &  & $-$0.423 &  &  &  &  &  \\ 
  &  & (1.455) &  &  &  &  &  \\ 
 Low Status &  & $-$4.751$^{}$ &  &  &  &  &  \\ 
  &  & (1.883) &  &  &  &  &  \\ 
 No Shock &  & $-$1.525 &  &  &  &  &  \\ 
  &  & (1.988) &  &  &  &  &  \\ 
 Receive Shock &  & $-$2.307 &  &  &  &  &  \\ 
  &  & (2.456) &  &  &  &  &  \\ 
 Redistribute &  & $-$0.102 &  &  &  &  &  \\ 
  &  & (1.235) &  &  &  &  &  \\ 
 \# of Additions X  &  & $-$0.009 &  &  &  &  &  \\ 
 High Status &  & (0.124) &  &  &  &  &  \\ 
  \# of Additions X  &  & 0.586$^{}$ &  &  &  &  &  \\ 
 Low Status &  & (0.203) &  &  &  &  &  \\ 
  \# of Additions X &  & 0.122 &  &  &  &  &  \\ 
 No Shock  &  & (0.193) &  &  &  &  &  \\ 
 \# of Additions X  &  & 0.165 &  &  &  &  &  \\ 
 Receive Shock &  & (0.199) &  &  &  &  &  \\ 
 \# of Additions X  &  & 0.031 &  &  &  &  &  \\ 
 Redistribute &  & (0.116) &  &  &  &  &  \\ 
 High Status &  &  &  & 4.377$^{}$ &  &  &  \\ 
  &  &  &  & (2.005) &  &  &  \\ 
 \# of Additions X  &  &  &  & $-$0.562$^{}$ &  &  &  \\ 
 High Status &  &  &  & (0.213) &  &  &  \\ 
 Receive Shock &  &  &  &  & $-$1.214 &  &  \\ 
  &  &  &  &  & (2.984) &  &  \\ 
 \# of Additions X  &  &  &  &  & 0.059 &  &  \\ 
 Receive Shock &  &  &  &  & (0.255) &  &  \\ 
 Constant & $-$0.758 & $-$0.367 & $-$0.806 & $-$4.991$^{}$ & $-$1.883 & $-$0.381 & $-$1.113 \\ 
  & (0.493) & (0.869) & (0.802) & (1.668) & (1.806) & (0.864) & (0.752) \\ 
 \hline \\[-1.8ex] 
AIC & 285.25 & 293.31 & 72.94 & 69.55 & 65.52 & 83.36 & 104 \\ 
Observations & 284 & 284 & 72 & 72 & 56 & 84 & 88 \\ 
\hline 
\hline \\[-1.8ex] 
\end{tabular} 
\end{table} 

\begin{comment}

\begin{table}[!htbp] \centering \footnotesize
  \caption{Regression using first round of each session} 
  \label{tab:round1} 
\begin{tabular}{@{\extracolsep{5pt}}lccccccc} 
\\[-1.8ex]\hline 
\hline \\[-1.8ex] 
\\[-1.8ex] & Full & Full & Baseline & Status & Shock & Redistribute  & Non-Fixed\\  
\hline \\[-1.8ex] 
 \# of Additions & 0.167$^{}$ & 0.105 & $-$0.004 & 0.717$^{}$ & 0.206 & 0.199$^{}$ & 0.167$^{}$ \\ 
  & (0.057) & (0.089) & (0.088) & (0.192) & (0.187) & (0.098) & (0.092) \\ 
 Compliance & 0.001 & 0.001 & 0.007$^{}$ & $-$0.001 & 0.003 & $-$0.001 & 0.001 \\ 
 Cost & (0.001) & (0.001) & (0.003) & (0.002) & (0.003) & (0.002) & (0.002) \\ 
  High Status &  & $-$0.423 &  &  &  &  &  \\ 
  &  & (1.455) &  &  &  &  &  \\ 
 Low Status &  & $-$4.751$^{}$ &  &  &  &  &  \\ 
  &  & (1.883) &  &  &  &  &  \\ 
 No Shock &  & $-$1.525 &  &  &  &  &  \\ 
  &  & (1.988) &  &  &  &  &  \\ 
 Receive Shock &  & $-$2.307 &  &  &  &  &  \\ 
  &  & (2.456) &  &  &  &  &  \\ 
 Redistribute &  & $-$0.102 &  &  &  &  &  \\ 
  &  & (1.235) &  &  &  &  &  \\ 
 \# of Additions X  &  & $-$0.009 &  &  &  &  &  \\ 
 High Status &  & (0.124) &  &  &  &  &  \\ 
  \# of Additions X  &  & 0.586$^{}$ &  &  &  &  &  \\ 
 Low Status &  & (0.203) &  &  &  &  &  \\ 
  \# of Additions X &  & 0.122 &  &  &  &  &  \\ 
 No Shock  &  & (0.193) &  &  &  &  &  \\ 
 \# of Additions X  &  & 0.165 &  &  &  &  &  \\ 
 Receive Shock &  & (0.199) &  &  &  &  &  \\ 
 \# of Additions X  &  & 0.031 &  &  &  &  &  \\ 
 Redistribute &  & (0.116) &  &  &  &  &  \\ 
 High Status &  &  &  & 4.377$^{}$ &  &  &  \\ 
  &  &  &  & (2.005) &  &  &  \\ 
 \# of Additions X  &  &  &  & $-$0.562$^{}$ &  &  &  \\ 
 High Status &  &  &  & (0.213) &  &  &  \\ 
 Receive Shock &  &  &  &  & $-$1.214 &  &  \\ 
  &  &  &  &  & (2.984) &  &  \\ 
 \# of Additions X  &  &  &  &  & 0.059 &  &  \\ 
 Receive Shock &  &  &  &  & (0.255) &  &  \\ 
 Constant & $-$0.758 & $-$0.367 & $-$0.806 & $-$4.991$^{}$ & $-$1.883 & $-$0.381 & $-$1.113 \\ 
  & (0.493) & (0.869) & (0.802) & (1.668) & (1.806) & (0.864) & (0.752) \\ 
 \hline \\[-1.8ex] 
AIC & 285.25 & 293.31 & 72.94 & 69.55 & 65.52 & 83.36 & 104 \\ 
Observations & 284 & 284 & 72 & 72 & 56 & 84 & 88 \\ 
\hline 
\hline \\[-1.8ex] 
\end{tabular} 
\end{table} 

\end{comment}

\begin{comment}

\begin{table}[!htbp] \centering\footnotesize
  \caption{Logit regression using first round of each session} 
  \label{tab:round1} 
\begin{tabular}{@{\extracolsep{5pt}}lccccccc} 
\\[-1.8ex]\hline 
\hline \\[-1.8ex] 
% & \multicolumn{7}{c}{\textit{Dependent variable:}} \\ 
%\cline{2-8} 
%\\[-1.8ex] & (A.7) & (A.8) & (A.9) & (A.10) & (A.11) & (A.12) & (A.13)\\ 
\\[-1.8ex] & Full & Full & Baseline & Status & Shock & Redistribute  & Non-Fixed\\  
\hline \\[-1.8ex] 
 Additions & 0.167$^{***}$ & 0.105 & $-$0.004 & 0.717$^{***}$ & 0.206 & 0.199$^{**}$ & 0.167$^{*}$ \\ 
  & (0.057) & (0.087) & (0.088) & (0.192) & (0.187) & (0.098) & (0.092) \\ 
Compliance  & 0.001 & 0.001 & 0.007$^{**}$ & $-$0.001 & 0.003 & $-$0.001 & 0.001 \\ 
Cost  & (0.001) & (0.001) & (0.003) & (0.002) & (0.003) & (0.002) & (0.002) \\ 
 High Status &  & $-$0.423 &  &  &  &  &  \\ 
  &  & (1.493) &  &  &  &  &  \\ 
 Low Status &  & $-$4.751 &  &  &  &  &  \\ 
  &  & (3.076) &  &  &  &  &  \\ 
 No Shock &  & $-$1.525 &  &  &  &  &  \\ 
  &  & (1.896) &  &  &  &  &  \\ 
 Receive Shock &  & $-$2.307 &  &  &  &  &  \\ 
  &  & (2.660) &  &  &  &  &  \\ 
 Redistribute &  & $-$0.102 &  &  &  &  &  \\ 
  &  & (1.303) &  &  &  &  &  \\ 
 Additions X  &  & $-$0.009 &  &  &  &  &  \\ 
  High Status &  & (0.132) &  &  &  &  &  \\ 
  Additions X  &  & 0.586$^{*}$ &  &  &  &  &  \\ 
  Low Status &  & (0.328) &  &  &  &  &  \\ 
  Additions X  &  & 0.122 &  &  &  &  &  \\ 
  No Shock &  & (0.175) &  &  &  &  &  \\ 
 Additions X  &  & 0.165 &  &  &  &  &  \\ 
  Receive Shock &  & (0.227) &  &  &  &  &  \\ 
 Additions X  &  & 0.031 &  &  &  &  &  \\ 
  Redistribute &  & (0.126) &  &  &  &  &  \\ 
 High Status &  &  &  & 4.377$^{**}$ &  &  &  \\ 
  &  &  &  & (2.005) &  &  &  \\ 
 Additions X  &  &  &  & $-$0.562$^{***}$ &  &  &  \\ 
 High Status &  &  &  & (0.213) &  &  &  \\ 
 Receive Shock &  &  &  &  & $-$1.214 &  &  \\ 
  &  &  &  &  & (2.984) &  &  \\ 
 Additions X  &  &  &  &  & 0.059 &  &  \\ 
 Receive Shock &  &  &  &  & (0.255) &  &  \\ 
 Constant & $-$0.758 & $-$0.367 & $-$0.806 & $-$4.991$^{***}$ & $-$1.883 & $-$0.381 & $-$1.113 \\ 
  & (0.493) & (0.880) & (0.802) & (1.668) & (1.806) & (0.864) & (0.752) \\ 
\hline \\[-1.8ex] 
Observations & 284 & 284 & 72 & 72 & 56 & 84 & 88 \\ 
R$^{2}$ & 0.099 & 0.158 & 0.227 & 0.278 & 0.185 & 0.083 & 0.136 \\ 
%$\chi^{2}$ & 18.809$^{***}$ (df = 2) & 30.752$^{***}$ (df = 12) & 11.762$^{***}$ (df = 2) & 14.141$^{***}$ (df = 4) & 7.460 (df = 4) & 4.443 (df = 2) & 8.830$^{**}$ (df = 2) \\ 
\hline 
\hline \\[-1.8ex] 
\textit{Note:}  & \multicolumn{7}{r}{$^{*}$p$<$0.1; $^{**}$p$<$0.05; $^{***}$p$<$0.01} \\ 
\end{tabular} 
\end{table} 
\clearpage

\end{comment}

\begin{comment}

% Table created by stargazer v.5.1 by Marek Hlavac, Harvard University. E-mail: hlavac at fas.harvard.edu
% Date and time: Wed, Nov 11, 2015 - 18:40:45
\begin{sidewaystable}[!htbp] \centering 
  \caption{Regression using first round of each session} 
  \label{tab:round1} 
\begin{tabular}{@{\extracolsep{5pt}}lcccccc} 
\\[-1.8ex]\hline 
\hline \\[-1.8ex] 
 & \multicolumn{6}{c}{\textit{Dependent variable:}} \\ 
\cline{2-7} 
\\[-1.8ex] & Baseline & Status & Status & Shock & Shock & Redistribute \\ 
\hline \\[-1.8ex] 
 \# of Additions & 0.004 & 0.173$^{***}$ & 0.419$^{**}$ & 0.134$^{*}$ & 0.112 & 0.117$^{*}$ \\ 
  & (0.053) & (0.066) & (0.170) & (0.073) & (0.090) & (0.062) \\ 
  & & & & & & \\ 
 Cost of Compliance & 0.004$^{***}$ & $-$0.001 & $-$0.0002 & 0.001 & 0.001 & $-$0.0004 \\ 
  & (0.002) & (0.001) & (0.001) & (0.001) & (0.001) & (0.001) \\ 
  & & & & & & \\ 
 High Salary &  &  & 2.583 &  &  &  \\ 
  &  &  & (1.747) &  &  &  \\ 
  & & & & & & \\ 
 High Salary X Additions &  &  & $-$0.329$^{*}$ &  &  &  \\ 
  &  &  & (0.181) &  &  &  \\ 
  & & & & & & \\ 
 Receive Shock &  &  &  &  & $-$0.959 &  \\ 
  &  &  &  &  & (1.870) &  \\ 
  & & & & & & \\ 
 Receive Shock X Additions &  &  &  &  & 0.058 &  \\ 
  &  &  &  &  & (0.153) &  \\ 
  & & & & & & \\ 
 Constant & $-$0.473 & $-$0.628 & $-$2.963$^{*}$ & $-$1.201 & $-$0.997 & $-$0.204 \\ 
  & (0.535) & (0.607) & (1.599) & (0.800) & (0.960) & (0.546) \\ 
  & & & & & & \\ 
\hline \\[-1.8ex] 
Observations & 72 & 72 & 72 & 56 & 56 & 84 \\ 
Log Likelihood & $-$33.437 & $-$32.642 & $-$29.654 & $-$27.982 & $-$27.760 & $-$38.575 \\ 
Akaike Inf. Crit. & 72.874 & 71.284 & 69.308 & 61.965 & 65.520 & 83.151 \\ 
\hline 
\hline \\[-1.8ex] 
\textit{Note:}  & \multicolumn{6}{r}{$^{*}$p$<$0.1; $^{**}$p$<$0.05; $^{***}$p$<$0.01} \\ 
\end{tabular} 
\end{sidewaystable} 
\end{comment}

\newpage

\subsection*{Appendix 4: Average Cheating Rates over Each Round within Session}

\par Subjects within a particular group earn and report income over ten periods.  For each of these groups, Figure~\ref{fig:period} presents the average cheating rates for each of the ten periods.  Average cheating in each group is for the most part quite constant.  And to the extent that there is a trend in cheating, it is consistent with trends we typically find in public goods games whereby contributions fall as the subjects approach the final period of play  \citep{Levittetal2007}.  Of the 71 groups, there are around ten groups for which average cheating rates rise significantly by period ten.  Hence there is some evidence here to suggest that any obligation subjects feel toward compliance erodes significantly as the end of the game approaches -- in a number of cases reaching levels predicted by equilibrium reasoning.

\begin{figure}[p]
\caption{Ratio of Non-declared to Total Earnings Averaged over Subjects}\label{fig:period}
\centerline{\includegraphics[width=1.5\textwidth]{percevad_period_treatment.pdf}}
\end{figure}
\clearpage

\subsection*{Appendix 5: Cheating Behaviour for Low and High Performance Types When Performance Deviates Significantly from their Average Performance}

\par Our contention is that the RETs identify performance types in the population -- individuals who recognise themselves as having ability versus not having high ability levels.  The expectation we note in the text is that performance is likely to be very consistent within subjects.  As an illustration, we define low performance deviations as an outcome in which a low performance type performs more than 12 additions.  And high performance deviations occur when a high performance type only manages less than 10 correct additions.  There are only 78 occurrences of the high performance deviations which represents less than 5 percent of high performer outcomes.  And there are only 68 occurrences of low performance deviations which also represents less than 5 percent of low performer outcomes.  

\par  Our contention is that one's performance type explains cheating behaviour.  Hence cheating behaviour should be very consistent within subjects -- it should not fluctuate significantly and we would not expect it to respond to stochastic shocks in performance.  This implies that, for any particular subject, cheating behaviour is not correlated with RET outcomes that deviate significantly from their overall performance levels.  Its not the case, for example, that when a high performance type experiences a negative shock her cheating sharply declines.  For both the low and high performance deviations we calculated the average change in percent of earnings evaded.  Figure~\ref{fig:deviate} provides a frequency plot of these deviations in cheating for the low and high performance deviation cases.  As is clear in Figure~\ref{fig:deviate}, for both cases, the average change was not significantly different from zero. 


 \begin{figure}[p]
\caption{Deviations in Performance and Cheating for Low and High Performance Types}\label{fig:deviate}
\centerline{\subfigure[High Performance Types]{\includegraphics[width=.6\textwidth]{{hist_high2.pdf}}} \\
\subfigure[Low Performance Types]{\includegraphics[width=.6\textwidth]{hist_low2.pdf}}}
%\centerline{\includegraphics[width=\textwidth]{Predictions_normal_probit_LowTax.pdf}}
\end{figure}

\clearpage


\newpage

\subsection*{Appendix 6: Risk Aversion}

\par Risk preferences could affect cheating behavior.  The tax compliance games resemble a public good game in that each subject's final earnings is affected by tax ``contributions'' by other members of the group.  Evidence suggests that the risk preferences of subjects playing public goods games affects contributions \citep{Schechter2007,Teyssier2012}.  Given there is uncertainty as to the choices of other group members, risk aversion can reduce the contributions of subjects.  Risk preferences may be confounded with performance type in our tax compliance games.  Those who perform well and end up with relatively high earnings may be disproportionately risk averse.  Complying with taxes in these circumstances is risky because high performers, in particular, stand to lose relatively large amounts of money if they cooperate and other group members do not.  

\par The performance type effect we estimate is not confounded with risk preferences.  The fourth and last module of the experiment consists of a lottery-choice test consisting of ten pairs, which is based in the low-payoff treatment studied in \citep{Holtetal2002}. The lottery choices (shown in  Table~\ref{tab:lottery}) are structured so that the crossover point to the high-risk lottery can be used to infer the degree of risk aversion. Subjects indicate their preferences, choosing Option A or Option B, for each of the ten paired lottery choices, and they know one of these choices would be selected at random ex post and played to determine the earnings for the option selected. Our measure of risk aversion is simply the sum of the safe choices by each subject. 

\begin{table}[h]

\caption{Lottery Choices}\label{tab:lottery}
\begin{tabular}{c|p{7.5cm}p{7.5cm}}

& Option A & Option B \\

\hline

1 &10\% of 2.00\pounds, 90\% of 1.60\pounds &  10\% of 3.85\pounds, Bs. 90\% of 0.10\pounds\\

2 &20\% of 2.00\pounds, 80\% of 1.60\pounds &  20\% of 3.85\pounds, Bs. 80\% of 0.10\pounds\\

3 &30\% of 2.00\pounds, 70\% of 1.60\pounds &  30\% of 3.85\pounds, Bs. 70\% of 0.10\pounds\\

4 &40\% of 2.00\pounds, 60\% of 1.60\pounds &  40\% of 3.85\pounds, Bs. 60\% of 0.10\pounds\\

5 &50\% of 2.00\pounds, 50\% of 1.60\pounds &  50\% of 3.85\pounds, Bs. 50\% of 0.10\pounds\\

6 &60\% of 2.00\pounds, 40\% of 1.60\pounds &  60\% of 3.85\pounds, Bs. 40\% of 0.10\pounds\\

7 &70\% of 2.00\pounds, 30\% of 1.60\pounds &  70\% of 3.85\pounds, Bs. 30\% of 0.10\pounds\\

8 &80\% of 2.00\pounds, 20\% of 1.60\pounds &  80\% of 3.85\pounds, Bs. 20\% of 0.10\pounds\\

9 &90\% of 2.00\pounds, 10\% of 1.60\pounds &  90\% of 3.85\pounds, Bs. 10\% of 0.10\pounds\\

10 &100\% of 2.00\pounds, 0\% of 1.60\pounds &  100\% of 3.85\pounds, Bs. 0\% of 0.10\pounds\\


\bottomrule

\end{tabular}
\label{lottery}
\end{table}
\clearpage

\par The overall correlation between performance (number of correct additions) and the sum of safe lottery choices is -0.03.  Hence we find little evidence that high performance types exhibit high degrees of risk aversion (or vice versa for that matter).  We also estimate the treatment effects controlling for low, medium, and high levels of risk aversion.  

\par Results for the safe choice measure are presented in Figure~\ref{fig:risk_dist_preliminary}.  This represents the distribution of subjects from all 20 sessions we report on here.  We create risk preference categories in order to assess whether controlling for risk preferences significantly affects the high versus low performance effects we present in the text.  Subjects are categorised into three risk categories based on the sum of the safe choices they make: risk seeking (a sum less than 4); risk neutral (a sum between 4 and 6); risk averse (a sum greater than 6).  

\begin{figure}[p]
\caption{Distribution of Sum of Safe Choices in Lottery Game}\label{fig:risk_dist_preliminary}
\centerline{\includegraphics[width=\textwidth]{risk_dist_preliminary.pdf}}
\end{figure}

\clearpage

%\par A version of Figure~\ref{fig:treatment} from the main text, controlling for risk aversion, is presented in Figure~\ref{fig:risk_pref}.  Within all of the risk aversion categories, high performance types consistently cheat more than low performance types; and for the most part the differences in means are statistically significant.\footnote{There is one exception in which low and high performance types have essentially the same average rates of cheating.  The cells sizes get relatively small which accounts for some of the imprecise estimates.  In addition there are empty cells -- these are typically for the risk seeking category.}  

\par In the text, Figure~\ref{fig:treatment} presents the difference in reported earnings for high versus low performance types -- these differences are presented for each of the different treatment sessions.  Again, for each of the treatment sessions, Figure~\ref{fig:risk_pref} compares the high versus low performance type differences within each of the three risk categories.  Results for the baseline treatment session indicate that the relationship between performance type and cheating for the most part persists when we control for risk preferences.  For risk seeking and risk neutral subjects there is a strong positive relationship between performance type and cheating.  Amongst the most risk averse subjects, though, cheating rates are similar for both high and low performance types.  For the most risk averse subjects, regardless of performance type, it would seem that uncertainty regarding the choices of other subjects results in very high levels of cheating. 


\begin{figure}[p]
\caption{Percent Evaded: High and Low Performance Types and Risk}\label{fig:risk_pref}
\centerline{\includegraphics[width=1.3\textwidth]{perevaded_treatment.pdf}}
\end{figure}

\clearpage



\par For subjects in the ``status'' treatment sessions, we also find that our estimated relationship between performance type and cheating is robust to the introduction of controls for risk preferences.  This holds for risk neutral and risk averse subjects in both the low and high ``status'' treatments.  In this case there are insufficient observations in the risk seeking categories to make any comparisons.  We see a similar pattern for subjects in the ``shock'' treatment sessions -- there is a positive relationship between performance type and cheating for risk neutral and risk averse subjects in both the control and ``shock'' treatments.    

\par Figure~\ref{fig:risk_pref} presents the results when we introduce the risk preference controls for those in the ``redistribute'' treatment sessions.  Here we find a weaker, although still positive, relationship between performance and cheating for risk neutral and risk averse subjects.  Finally, the results for the non-fixed sessions broken down in  Figure~\ref{fig:risk_pref} by risk category.  Again, the differences between High and Low Performance types persist in all three risk categories although they are more moderate in the risk seeking category.

\par Table~\ref{tab:cheat1} in the main text presented the multivariate logit regression results for each of the treatment session models.  Table~\ref{tab:cheat1safe} re-estimates these models and includes in the estimation the safe choice risk averse variable described in Figure~\ref{fig:risk_dist_preliminary}.  


\begin{comment}

\begin{sidewaystable}[!htbp]  
\caption{Probit model of cheating regressed on performance (with safe choice)}
  \label{tab:cheat1safe} 
\begin{tabular}{@{\extracolsep{2pt}}lcccccc} 
\\[-1.8ex]\hline 
\hline \\[-1.8ex] 
 & \multicolumn{6}{c}{\textit{Dependent variable:}} \\ 
\cline{2-7} 
\\[-1.8ex] & Baseline & Status & Status & Shock & Shock & Redistribute\\ 
\hline \\[-1.8ex] 
\# of Additions & 0.036$^{**}$ & 0.254$^{***}$ & 0.377$^{***}$ & 0.118$^{***}$ & 0.101$^{***}$ & 0.039$^{**}$ \\ 
  & (0.018) & (0.029) & (0.055) & (0.024) & (0.032) & (0.018) \\ 
  & & & & & & \\ 
 Cost of Compliance & 0.003$^{***}$ & $-$0.002$^{***}$ & $-$0.001$^{**}$ & 0.001$^{***}$ & 0.002$^{***}$ & $-$0.0003 \\ 
  & (0.0004) & (0.0004) & (0.0005) & (0.0004) & (0.0005) & (0.0003) \\ 
  & & & & & & \\ 
 High Salary &  &  & 1.713$^{***}$ &  &  &  \\ 
  &  &  & (0.596) &  &  &  \\ 
  & & & & & & \\ 
 High Salary X Additions &  &  & $-$0.196$^{***}$ &  &  &  \\ 
  &  &  & (0.060) &  &  &  \\ 
  & & & & & & \\ 
 Receive Shock &  &  &  &  & $-$0.456 &  \\ 
  &  &  &  &  & (0.576) &  \\ 
  & & & & & & \\ 
 Receive Shock X Additions &  &  &  &  & 0.016 &  \\ 
  &  &  &  &  & (0.048) &  \\ 
  & & & & & & \\ 
 Safe Choice & 0.027 & $-$0.107$^{***}$ & $-$0.111$^{***}$ & $-$0.083 & $-$0.064 & 0.182$^{***}$ \\ 
  & (0.031) & (0.039) & (0.040) & (0.051) & (0.053) & (0.030) \\ 
  & & & & & & \\ 
 Constant & $-$0.586$^{**}$ & $-$0.412 & $-$1.636$^{***}$ & $-$0.346 & $-$0.307 & $-$0.176 \\ 
  & (0.262) & (0.401) & (0.592) & (0.405) & (0.454) & (0.202) \\ 
  & & & & & & \\ 
\hline \\[-1.8ex] 
Observations & 720 & 720 & 720 & 560 & 560 & 840 \\ 
Log Likelihood & $-$319.151 & $-$215.300 & $-$208.838 & $-$217.409 & $-$216.127 & $-$323.924 \\ 
Akaike Inf. Crit. & 646 & 438 & 429 & 442 & 444 & 655 \\ 
\hline 
\hline \\[-1.8ex] 
\textit{Note:}  & \multicolumn{6}{r}{$^{*}$p$<$0.1; $^{**}$p$<$0.05; $^{***}$p$<$0.01} \\ 
\end{tabular} 
\end{sidewaystable}
\end{comment}

% Table created by stargazer v.5.2 by Marek Hlavac, Harvard University. E-mail: hlavac at fas.harvard.edu
% Date and time: lun, oct 17, 2016 - 16:37:22
\begin{comment}

\begin{table}[!htbp] \centering \footnotesize
\caption{Logit model of cheating regressed on performance (with safe choice)}
  \label{tab:cheat1safe}  
\begin{tabular}{@{\extracolsep{5pt}}lccccccc} 
\\[-1.8ex]\hline 
\hline \\[-1.8ex] 
% & \multicolumn{7}{c}{\textit{Dependent variable:}} \\ 
%\cline{2-8} 
%\\[-1.8ex] & (1) & (2) & (3) & (4) & (5) & (6) & (7)\\
\\[-1.8ex] & Full & Full & Baseline & Status & Shock & Redistribute  & Non-Fixed\\  
\hline \\[-1.8ex] 
 Additions & 0.181$^{***}$ & 0.144$^{**}$ & 0.057 & 0.726$^{***}$ & 0.181$^{*}$ & 0.067 & 0.149$^{*}$ \\ 
  & (0.041) & (0.072) & (0.077) & (0.190) & (0.094) & (0.064) & (0.078) \\ 
Compliance  & 0.0005 & 0.001 & 0.005$^{**}$ & $-$0.002 & 0.003 & $-$0.001 & $-$0.0004 \\ 
Cost  & (0.001) & (0.001) & (0.002) & (0.002) & (0.002) & (0.001) & (0.002) \\ 
 High Status &  & $-$1.269 &  &  &  &  &  \\ 
  &  & (1.246) &  &  &  &  &  \\ 
 Low Status &  & $-$4.468$^{**}$ &  &  &  &  &  \\ 
  &  & (1.832) &  &  &  &  &  \\ 
 No Shock &  & $-$0.572 &  &  &  &  &  \\ 
  &  & (1.249) &  &  &  &  &  \\ 
 Receive Shock &  & $-$1.186 &  &  &  &  &  \\ 
  &  & (1.303) &  &  &  &  &  \\ 
 Redistribute &  & 1.419 &  &  &  &  &  \\ 
  &  & (1.016) &  &  &  &  &  \\ 
 High Status &  &  &  & 3.615 &  &  &  \\ 
  &  &  &  & (2.234) &  &  &  \\ 
 Additions X&  &  &  & $-$0.411$^{**}$ &  &  &  \\ 
  High Status  &  &  &  & (0.209) &  &  &  \\ 
 Receive Shock &  &  &  &  & $-$0.998 &  &  \\ 
  &  &  &  &  & (1.278) &  &  \\ 
 Additions X  &  &  &  &  & 0.044 &  &  \\ 
 Receive Shock &  &  &  &  & (0.086) &  &  \\ 
 Safe choices & $-$0.005 & 0.053 & 0.042 & $-$0.202 & $-$0.138 & 0.333$^{**}$ & 0.182$^{*}$ \\ 
  & (0.088) & (0.093) & (0.140) & (0.229) & (0.245) & (0.147) & (0.100) \\ 
 Additions X &  & 0.100 &  &  &  &  &  \\ 
  High Status  &  & (0.100) &  &  &  &  &  \\ 
  Additions X  &  & 0.562$^{***}$ &  &  &  &  &  \\ 
  Low Status &  & (0.189) &  &  &  &  &  \\ 
  Additions X &  & 0.061 &  &  &  &  &  \\ 
  No Shock &  & (0.103) &  &  &  &  &  \\ 
 Additions X &  & 0.118 &  &  &  &  &  \\ 
 Receive Shock  &  & (0.108) &  &  &  &  &  \\ 
 Additions X  &  & $-$0.097 &  &  &  &  &  \\ 
  Redistribute &  & (0.091) &  &  &  &  &  \\ 
 Constant & $-$0.531 & $-$0.844 & $-$1.005 & $-$3.372 & $-$0.551 & $-$0.405 & $-$1.333$^{*}$ \\ 
  & (0.640) & (0.889) & (1.075) & (2.452) & (1.641) & (0.796) & (0.796) \\ 
\hline \\[-1.8ex] 
Observations & 2,840 & 2,840 & 720 & 720 & 560 & 840 & 880 \\ 
R$^{2}$ & 0.099 & 0.152 & 0.177 & 0.371 & 0.191 & 0.108 & 0.096 \\ 
%$\chi^{2}$ & 170.487$^{***}$ (df = 3) & 266.817$^{***}$ (df = 13) & 85.621$^{***}$ (df = 3) & 166.015$^{***}$ (df = 5) & 66.742$^{***}$ (df = 5) & 52.807$^{***}$ (df = 3) & 58.046$^{***}$ (df = 3) \\ 
\hline 
\hline \\[-1.8ex] 
\textit{Note:}  & \multicolumn{7}{r}{$^{*}$p$<$0.1; $^{**}$p$<$0.05; $^{***}$p$<$0.01} \\ 
\end{tabular} 
\end{table} 
\end{comment}

% Table created by stargazer v.5.2 by Marek Hlavac, Harvard University. E-mail: hlavac at fas.harvard.edu
% Date and time: mar, nov 08, 2016 - 15:47:42
\begin{table}[!htbp] \centering \footnotesize
\caption{Logit model of cheating regressed on performance (with safe choice)}
  \label{tab:cheat1safe}  
\begin{tabular}{@{\extracolsep{5pt}}lccccccc} 
\\[-1.8ex]\hline 
\hline \\[-1.8ex] 
\\[-1.8ex] & Full & Full & Baseline & Status & Shock & Redistribute  & Non-Fixed\\  
\hline \\[-1.8ex] 
 \# of Additions & 0.181$^{}$ & 0.144$^{}$ & 0.057 & 0.726$^{}$ & 0.181$^{}$ & 0.067 & 0.149$^{}$ \\ 
  & (0.041) & (0.072) & (0.077) & (0.190) & (0.094) & (0.064) & (0.078) \\ 
  & & & & & & & \\ 
 Compliance & 0.0005 & 0.001 & 0.005$^{}$ & $-$0.002 & 0.003 & $-$0.001 & $-$0.0004 \\ 
 Cost & (0.001) & (0.001) & (0.002) & (0.002) & (0.002) & (0.001) & (0.002) \\ 
 High Status &  & $-$1.269 &  &  &  &  &  \\ 
  &  & (1.246) &  &  &  &  &  \\ 
 Low Status &  & $-$4.468$^{}$ &  &  &  &  &  \\ 
  &  & (1.832) &  &  &  &  &  \\ 
 No Shock &  & $-$0.572 &  &  &  &  &  \\ 
  &  & (1.249) &  &  &  &  &  \\ 
 Receive Shock &  & $-$1.186 &  &  &  &  &  \\ 
  &  & (1.303) &  &  &  &  &  \\ 
 Redistribute &  & 1.419 &  &  &  &  &  \\ 
  &  & (1.016) &  &  &  &  &  \\ 
 High Status &  &  &  & 3.615 &  &  &  \\ 
  &  &  &  & (2.234) &  &  &  \\ 
 \# of Additions X  &  &  &  & $-$0.411$^{}$ &  &  &  \\ 
 High Status &  &  &  & (0.209) &  &  &  \\ 
 Receive Shock &  &  &  &  & $-$0.998 &  &  \\ 
  &  &  &  &  & (1.278) &  &  \\ 
 \# of Additions X &  &  &  &  & 0.044 &  &  \\ 
 Receive Shock &  &  &  &  & (0.086) &  &  \\ 
 Safe choices & $-$0.005 & 0.053 & 0.042 & $-$0.202 & $-$0.138 & 0.333$^{}$ & 0.182$^{}$ \\ 
  & (0.088) & (0.093) & (0.140) & (0.229) & (0.245) & (0.147) & (0.100) \\ 
 \# of Additions X  &  & 0.100 &  &  &  &  &  \\ 
 High Status &  & (0.100) &  &  &  &  &  \\ 
  \# of Additions X  &  & 0.562$^{}$ &  &  &  &  &  \\ 
  Low Status &  & (0.189) &  &  &  &  &  \\ 
  \# of Additions X  &  & 0.061 &  &  &  &  &  \\ 
  No Shock &  & (0.103) &  &  &  &  &  \\ 
 \# of Additions X  &  & 0.118 &  &  &  &  &  \\ 
 Receive Shock &  & (0.108) &  &  &  &  &  \\ 
 \# of Additions X  &  & $-$0.097 &  &  &  &  &  \\ 
 Redistribute &  & (0.091) &  &  &  &  &  \\ 
 Constant & $-$0.531 & $-$0.844 & $-$1.005 & $-$3.372 & $-$0.551 & $-$0.405 & $-$1.333$^{}$ \\ 
  & (0.640) & (0.889) & (1.075) & (2.452) & (1.641) & (0.796) & (0.796) \\ 
 \hline \\[-1.8ex] 
AIC & 2356.61 & 2280.28 & 648.47 & 429.86 & 442.31 & 654.85 & 914.75 \\ 
Observations & 2,840 & 2,840 & 720 & 720 & 560 & 840 & 880 \\ 
\hline 
\hline \\[-1.8ex] 
\end{tabular} 
\end{table} 


\clearpage
\newpage

\subsection*{Appendix 7: Instructions for Treatment Sessions}

\begin{figure}[p]
\caption{Instructions for Baseline Treatment}\label{fig:instruct_one}
\centerline{\includegraphics[width=\textwidth]{instruct_one.pdf}}
\end{figure}

\clearpage

\begin{figure}[p]
\caption{Instructions for Status Treatment}\label{fig:instruct_two}
\centerline{\includegraphics[width=\textwidth]{instruct_two.pdf}}
\end{figure}

\clearpage

\begin{figure}[p]
\caption{Instructions for Redistribute Treatment}\label{fig:instruct_three}
\centerline{\includegraphics[width=\textwidth]{instruct_three.pdf}}
\end{figure}

\clearpage

\begin{figure}[p]
\caption{Instructions for Shock Treatment}\label{fig:instruct_three}
\centerline{\includegraphics[width=\textwidth]{instruct_four.pdf}}
\end{figure}

\clearpage


\newpage

\subsection*{Appendix 8: Screen Shots from Real Effort Tasks}

\begin{figure}[p]
\caption{Screen Shot One from Real Effort Task}\label{fig:screen_one}
\centerline{\includegraphics[width=\textwidth]{screen_one.pdf}}
\end{figure}

\clearpage

\begin{figure}[p]
\caption{Screen Shot Two from Real Effort Task}\label{fig:screen_two}
\centerline{\includegraphics[width=\textwidth]{screen_two.pdf}}
\end{figure}

\clearpage

\begin{figure}[p]
\caption{Screen Shot Three from Real Effort Task}\label{fig:screen_three}
\centerline{\includegraphics[width=\textwidth]{screen_three.pdf}}
\end{figure}

\clearpage

\begin{figure}[p]
\caption{Screen Shot Four from Real Effort Task}\label{fig:screen_four}
\centerline{\includegraphics[width=\textwidth]{screen_four.pdf}}
\end{figure}

\clearpage

\begin{figure}[p]
\caption{Screen Shot Five from Real Effort Task}\label{fig:screen_five}
\centerline{\includegraphics[width=\textwidth]{screen_five.pdf}}
\end{figure}


\end{document}


\section{Introduction}

\par We use real effort tax compliance experiments in order to recover preferences over tax rates.   There is a considerable literature on tax compliance experiments \citep{Slemrod2007}.  We employ these experiments in order to recover individuals' preferences regarding redistribution and about redistributive tax rates in particular.  There is experimental evidence suggesting that tax compliance depends on features of the tax system  \citep{Almetal1992,Spiceretal1980}. Some of these experimental results suggest that tax compliance is conditioned on the perceived equity of the tax system  \citep{Falkinger1995} although some have not found such a relationship  \citep{Cowell1990}. Typically these experiments have explored whether or not compliance is affected by perceived features of the tax system. 

\par We design the tax compliance lab experiments so that the tax system treatments vary in such a fashion that they allow us to recover preferences over tax rates and over the redistributive features of the tax system.  By administering a large number of these tax compliance experiments that vary primarily in terms of the  redistributive features of the tax  system we generate  quite  precise estimates  of preferences for redistribution and  redistributive taxes.   The idea here is that tax compliance and work effort can be behavioural indicators of preferences for redistributive features of the tax system.

%  Ultimately we expect to implement three different major treatments.  These treatments are associated with efforts to recover preferences over redistribution or redistributive tax rates.   In Treatment Number 1 subjects perform a real effort task and their earnings are taxed  -- the treatment consists of varying the rates of taxation. Treatment Number 2 varies the existing levels of ``endowment'' inequity to see how this affects work effort and tax compliance.  Again here the idea is to administer large numbers of treatments so that we are able to calibrate in a reasonably exact fashion preferences over redistribution. Finally, Treatment Number 3 allows subjects to endogenously select tax rates -- again with sufficient numbers of sessions my expectation is that we can gain considerable insight into preferences over redistributive tax rates.

\section{The Redistributive Preferences Puzzle}

\par There is strong evidence that economic inequality has risen considerably, or is rising, in a number of highly developed democratic countries.  Recent studies suggest that economic inequality has risen considerably \citep{Alvaredoetal2013} but particularly in the United States over the past 30 years \citep{Gordonetal2008,Pikettyetal2003}.  There is some evidence at the macro level that this rising inequality has had political consequences -- specifically an increasingly polarized electorate \citep{McCartyetal2006}.  But the micro-level evidence that the American public would accept redistributive policies to address this inequality is weak at best \citep{Bartels2005,Bartels2008}.  More recently, global financial shocks have resulted in serious social and economic dislocation and have clearly sharpened levels of economic inequality in many countries.  A number of democratically elected governments are under pressure to implement fiscal measures that have, in many cases, dramatic redistributive effects.  Yet the micro-level evidence suggests that poorer voters, who have been particularly disadvantaged by these shocks, are not demanding aggressive redistributive policies to address the resulting inequities \citep{Duchetal2013}.

\par As many scholars have pointed out, public acceptance of or acquiescence in rising levels of inequality in developed democracies is puzzling.  Most of the political economic models assume instrumental rationality will result in heterogeneous preferences.  The rich are presumed to favour less redistribution and the poor are expected to want more.  They are expected to make voting decisions based on these preferences which in turn, given their relative numbers, result in redistributive policies that are responsive to voter expectations.  One would not expect democratic processes to tolerate policies that result in rising levels of inequality over extended periods of time.

\par The Meltzer-Richard model is one of the classic expressions of this argument.  It posits that income inequality promotes redistribution via the preferences of the median voter. The model has had a profound impact on much of the recent comparative literature on the political economy of redistribution  \citep{Meltzeretal1981}.  Redistribution in the Meltzer-Richard model results because all citizens benefit from an assumed universal flat-rate which is financed by a linear income tax \citep{Meltzeretal1981, Romer1975}. All citizens effectively have mean income when there is 100\% taxation. It then follows that any individual with market incomes below the mean income should favour 100\% taxation.  But of course taxation can generate disincentive effects that reduce the mean income. These disincentive effects give rise to middle-income earners for whom the deadweight costs of taxation are higher than the benefits or transfers provided by the government, in spite of the fact that their market income falls below the mean income.  \citet{Meltzeretal1981} argue that the amount of redistribution preferred by the median voter will be a function of the distance between his income and the mean income (holding constant the deadweight cost of taxation).  If inequality increases while the mean income remains constant, then the median voter will become more supportive of redistribution (assuming all citizens exercise their right to vote).   And finally we should expect greater income inequality to be associated with more redistribution if electoral competition generates government policies which reflect the preferences of the median voter.

\par While these hypotheses have been highly influential, redistributive patterns have not necessarily followed those predicted by the Meltzer-Richard model.  For example, why do left parties lose elections in high-income inequality countries?  In democracies, we would expect that citizens turn to redistributive parties in contexts of high-income inequality.  The logic is simple: if the rich are relatively richer (i.e. income inequality is higher), the median voter should demand more redistribution and therefore should support redistributive parties \citep{Meltzeretal1981}. It has been frequently pointed out that, contrary to the Meltzer-Richard model, countries experiencing more unequal distributions of market income typically re-distribute less than countries with less unequal distributions of market income \citep{Pontussonetal2010}. In fact in countries with relatively high earnings inequality, left parties get lower vote and seat shares, and they are more seldom elected to govern. A variety of explanations have been proposed, including models in which the distribution of market income and redistributive policy are jointly determined by other variables, such as government partisanship, union power, strategic electoral behaviour of political parties, and electoral rules \citep{Bradleyetal2003,Iversenetal2009,Pontussonetal2010}

\par Efforts to measure individual attitudes about redistribution and redistributive policies directly have not, for the most part, produced results consistent with the classical model and hence the puzzle persists.  This is particularly the case with respect to voters' preferences regarding redistributive policies.  There might be heterogeneity in abstract preferences for redistribution -- such as the poor think that rich should pay more.  Based on the European Social Survey, \citep{Leon2012}, for example, finds considerable agreement with statements calling for more redistribution and there is the kind of heterogeneity in attitudes we expect -- the poor are more demanding of such redistribution and the rich less so.   
But with respect to preferences for specific redistributive policies there appears to be a distinct absence of heterogeneity -- often the poor and the rich have similar expressed preferences.  Or the poor are more resistant -- see \citet{Kuziemkoetal2011} who speculate that this resistance might result from ``last place aversion''.

\par Confounding factors likely play a role here.  In particular, individuals might simply be poorly informed (for a range of reasons) about the distribution of income and their relatively location in the income distribution.  A number of recent studies document the discrepancies between the objective state of income inequality and individual perceptions of the distribution of incomes. \citet{Nortonetal2011} demonstrate that for the U.S. case there are significant discrepancies between actual and perceived levels of inequality.  Since much of the evidence is based on survey data it is difficult to establish the importance of these confounding factors in explaining redistribution preferences \citep{Alesinaetal2009}.  \citet{Crucesetal2013} attempt to address this problem by embedding experimental treatments within a household survey and, while there is some evidence that exaggerating one�s income distribution rank depresses support for redistributive measures, the effects are weak.

\par The essay reports on a project that attempts to address this puzzle.  The Meltzer Richards model and the various permutations we have seen are premised on a particular decision-making heuristic: individuals locate themselves on the income distribution and then, based on self-interest, make a calculation as to which redistributive measures (for the most part taxation) would make them better off.  Any deviation is then attributed to poor heuristics, misinformation, last place aversion, altruism, etc.  Without necessarily dismissing these factors as inputs into the voter's utility function, we propose an additional heuristic or what you might call a reflex on the part of voters.  We argue that individuals deploy a very simple heuristics when they evaluate redistributive policies.  Essentially they employ a fairness principle divorced from redistributive considerations: ``What is a reasonable `tax' for all citizens?''  Our conjecture is that: 1) this basic reflex ignores the redistributive consequences of the redistributive measure; and 2) the heuristic is similar across income categories.


\section{A Tax Norm and other Hypothesized Explanations}

\par Our project aims to provide a better understanding of the role played by public preferences in explaining policies that promote or tolerate rising levels of inequality.  The literature on redistribution makes clear that there is a multiplicity of factors that very plausibly enter into the individual?s utility function for redistributive taxation.  Any one, or a combination of these, might explain why public attitudes seem inconsistent with the classic Meltzer-Richard hypotheses.  We begin by defining these critical hypotheses, a number of which are clearly prominent in the existing literature.  Experimental treatments incorporating the most plausible explanations will enable us to draw defensible conclusions about public preferences for redistributive taxation.  

\par The conventional wisdom expressed in Meltzer-Richards?s classic theories of redistribution may be borne out if preferences for redistributional taxation prove to be strongly related to the individual?s income relative to average income in the population.  However, we hypothesise that a simple heuristic may enter into the policy utility function: specifically, that individuals condition their preferences for redistributive policies on norms regarding fair rates of taxation.  Individuals may express preferences for very aggressive redistributive outcomes, but these preferences are not necessarily correlated with the redistributive policies they find acceptable.  Empirical studies examining redistributive preferences have frequently found this to be the case.  Often differences emerge between the rich and the poor in their preferences for redistributive outcomes, and yet the rich and poor often respond similarly to specific policies that promote redistribution \citep{Bartels2005}.  Our explanation is that the rich and poor share very similar preferences regarding redistributive policies in spite of having preferences, possibly even strong ones, regarding redistributive outcomes.

\par Our conjecture rests upon preliminary empirical evidence drawn from experimental vignettes administered to large-N samples of the UK population.  UK respondents were provided descriptions of different household configurations (for example, single versus married with one child) and incomes (the incomes varied between \pounds8,000 and \pounds80,000).  Respondents were asked to indicate what they considered to be the appropriate taxes each of these households should pay.  \citet{DuchRueda2013} report that the preferred rates of taxation were relatively low ? on average about 25 per cent.  The study found that average tax rates varied relatively little over the households earning a low of \pounds8,000 to a high of \pounds80,000.  There was also relatively little variation across types of households, and the income of the respondents had no significant effect on the taxes rates they reported.   Further lab experiments again suggest that the there is taxation norm:  \citet{Kaietal2011} find average preferred tax rates, in their mobility treatment, that range between 23 and 31 percent while they are at 20 percent when subjects do not know their income condition.

\par We conjecture that there is a taxation norm which significantly constrains individuals? preferences for redistributional taxation, and as a consequence, preferences for redistributional taxation are relatively homogeneous in any particular population and do not vary across cultural contexts.  Individuals deploy a very simple heuristics when they evaluate redistributive taxation, essentially employing a fairness principle divorced from redistributive considerations: ?What is a reasonable `tax? for all citizens??  We hypothesise that: 1) this basic reflex ignores the redistributive consequences of taxation; and 2) the heuristic is similar across income categories.

\par The fairness principle could generate a quite different outcome.  There is an extensive experimental literature documenting the prevalence of other-regarding preferences in the population \citep{Camerer2003,Fehretal2000,Fehretal2004,Falketal2006}.  Preferences for redistributional taxation may be determined by these other-regarding preferences.  Notions of fairness may not refer to the proportion of income being taxed by the state but rather to outcomes.  Other-regarding attitudes might lead individuals to prefer a ?fair? distribution of income after taxes \citep{Ackertetal2007}.  The important point here though, is that there is no other-regarding preference uniformly distributed over the population.  There are other-regarding types and there are non-other-regarding types.   To the extent that other-regarding preferences are uncorrelated with income (or even positively correlated) then this variable would probably confound any simple Meltzer-Richards relationship between income and redistribution preferences.  Additionally, or possibly interactively, redistributive preferences may be conditioned on the perceived levels of inequality in any particular context ? as is argued by \citet{Piketty1995,Benabouetal2006,Fong2003,Alesinaetal2005}, among others.

\par There are risks of job loss, income loss, downward mobility, and such, in any market economy.  An important body of literature argues that anticipation of falling income or risk aversion (possibly induced by experiences with unemployment) can shape one?s preferences for redistributional taxation.   Individuals who anticipate job loss or are risk averse might appreciate the insurance provided by redistributional policies, such as unemployment insurance provided by the government.   There is evidence based on survey data to this effect, particularly for the US \citep{Alesinaetal2005,Benabouetal2001,Piketty1995}.  Experimental evidence also suggests that risk aversion helps explain why some subjects favour more egalitarian distributions \citep{Frignanietal2012,Horisch2010}.

\par A utilitarian perspective on redistribution recognizes the societal benefits of transferring income to the poor, for whom the marginal utility of consumption is high.  Nevertheless, utilitarians recognize the costs of redistributive taxation (Mankiw 2013).  These include the deadweight costs associated with government administration of these redistributive efforts and the lost revenues associated with the work disincentives for high income taxpayers.  This suggests two conditions that may shape preferences for redistributional taxation.  Preferences for redistributional taxation may be conditioned on efficient outcomes or on the deadweight loss associated with the provision of collective goods. Outcomes ? specifically, how income is redistributed as a result of taxation ? may also determine preferences for redistributional taxation. 
 
\par Cultural norms shape perceptions of acceptable levels of inequality. Alesina and Glasser (2004) argue that the US and European historical experiences have produced very different conceptions of acceptable levels of inequality.  Luttmer and Singhal (2011) argue that redistributive preferences may be culturally determined making them quite stable over time but varying across country.  Preferences for redistribution may be reciprocal or conditional on the cooperative or altruistic behaviour of others in the population (Leon 2012).  Some studies have demonstrated that preferences for redistributive policies are very sensitive to the framing of these initiatives (Bartels 2003; Huber and Paris 2013).

\par There are various cultural scenarios with respect to redistributive preferences.  We conjecture that the taxation norm is common across cultures.  One can imagine that culture shapes the taxation norm, so it might be homogeneous in any particular population yet vary across cultural contexts.  Culture may also be a conditioning variable for the hypothesized causal effects associated with heterogeneity in the population.  Culture could condition the effect of other-regardingness on redistributional policy preferences.  Other-regarding preferences might matter in some contexts but not in others.  For example, this might reflect cultural variations in norms regarding inequality.  Culture might also condition the importance of insurance in shaping preferences for redistributional taxation.

\subsection{Determining the Relative Importance of Different Hypothesized Effects}

\par Each of the arguments described above is certainly plausible and all are buttressed by support from empirical -- typically observational -- data.  Our challenge is to identify research strategies which help us assess the relative importance of each hypothesized effect in the typical individual's utility function for redistributive taxation.  Understanding the relative importance of these effects has significant implications for public policy.  If there exists a widely internalized tax norm, then efforts to promote redistributive taxation on the basis of an assumed relationship between a voter's income and preferences for tax rates, in the Meltzer-Richards tradition, would be futile.  Similarly, although for different reasons, such an appeal would be pointless if other-regarding preferences shape preferences for redistributive taxation.

\par This project's ultimate objective is to tease out the relative importance of different factors that shape redistributive preferences.  Experiments represent a promising method for accomplishing this goal.  The tax compliance lab experiments that we have conducted over the past year have made important advances in the design of protocols for recovering redistributive tax preferences.  We have also made progress in assessing the relative importance of the factors shaping redistributive preferences, with some evidence for the taxation norm argument and ? perhaps  more surprising ? for the notion that redistributive tax preferences are very much shaped by other-regarding preferences.  Accordingly, the project, focuses on preferences for redistributional taxation, addressing the following hypotheses:

\begin{itemize}

\item There is a taxation norm that significantly constrains individuals' preferences for redistributional taxation, and as a result, preferences for redistributional taxation are relatively homogeneous in any particular population and do not vary across cultural contexts.
\item Preferences for redistributional taxation conform to classic theories of redistribution and hence are strongly related to the individual's income relative to average income in the population.
\item Preferences for redistributional taxation are determined by other-regarding preferences.
\item Insurance considerations shape preferences for redistributional taxation.
\item Culture shapes the ``taxation norm'' and hence while it might be homogeneous in any particular population there is variation in levels of the ``taxation norm'' across cultural contexts.
\item Culture conditions the effect of other-regardingness on redistributional policy preferences.  For example, this might reflect cultural variations in norms regarding inequality.
\item Culture conditions the importance of insurance in shaping preferences for redistributional taxation.
\item Preferences for redistributional taxation are conditioned on efficient outcomes or on the deadweight loss associated with the provision of collective goods. 
 \end{itemize}


%\section{The Model}

%\par There is widespread acceptance that a voter�s utility for redistributive policies is arrived at based on a decision-making shortcut, or heuristic.  Voters do not calculate the precise implications of current marginal tax rates on their wellbeing net of government redistributive measures.  

%\par Much of the literature concedes that they do not engage in such calculations.  Instead, individuals take decision-making short cuts.  But virtually all of the literature presumes that individuals employ some heuristic designed to assess the implications of redistributive policies for income distribution.  In particular, individuals are presumed to be concerned with whether redistributive policies would improve, or worsen, their net financial situation.  Again within this general paradigm, a variety of considerations have been identified that might moderate a strong relationship between income and redistributive preferences.  The prospects of social mobility  \citep{Benabouetal2001} might reduce the preferences for redistributive taxation by individuals earning below the median income.  Cultural norms differ with respect to what is an acceptable level of inequality -- \citet{Alesinaetal2004} make the case that the U.S. and European historical experiences have resulted in very different norms regarding what is acceptable levels of inequality.   \citet{Luttmeretal2011} argue that redistributive preferences may be culturally determined and hence are quite stable over time but of course vary across country.  Preferences for redistribution may be reciprocal or conditional on the cooperative or altruistic behaviour of others in the population \citep{Leon2012}.  And some have demonstrated that preferences for redistributive policies are very sensitive to the framing of these initiatives \citep{Bartels2005,Huberetal2013}.  

%\par Effectively all of these arguments retain the basic premise of classic models, such as Meltzer Richards, which presume that individuals condition their preferences for redistribution on a calculation based on the existing distribution of income and where the particular individual is located on this distribution.  They may be misinformed about the distribution and where they are located on it.  Or the conclusions they draw about which types of distribution require remedial policies might vary by cultures.  Or reciprocity might vary in magnitude by contexts that could also condition the redistributive preference.  Or there is framing.  Hence the puzzling persistence of inequities -- and the absence of more aggressive redistributive policies -- in many democratic contexts are presumed to result from individual calculations regarding the distribution of income and their relative location on this distribution.

%\par We suggest a simpler heuristic may enter into the policy utility function -- specifically that individuals condition their preferences for redistributive policies on norms regarding fair rates of taxation.  Individuals may express preferences for very aggressive redistributive outcomes, but these preferences are not necessarily correlated with the redistributive policies they find acceptable.  This is a frequent finding in the empirical literature which examines redistributive preferences.  It is often the case that there are differences between the rich and the poor with respect to preferences regarding redistributive outcomes.  On the other hand, it is also the case that the rich and poor often do not differ with respect to specific policies that promote redistribution (Bartels 2005).  Our explanation is that the rich and poor share very similar preferences regarding redistributive policies in spite of having preferences, possibly even strong ones, regarding redistributive outcomes.

%\par Preliminary evidence based on experimental vignettes administered to large-N samples of the U.K. population suggests that this is particularly the case with respect to tax rates.  U.K. respondents in an online election survey were provided descriptions of different household configurations (for example, single versus married with one child) and incomes (the incomes varied between \pounds 8,000 and \pounds 80,000).  Respondents were then asked to indicate what they considered to be the appropriate taxes each of these households should pay.  Duch and Rueda (2013) report that the preferred rates of taxation were relatively low � on average about 25 per cent.  Of particular interest for this project was their finding that these average tax rates varied relatively little over the households earning a low of \pounds 8,000 to a high of \pounds 80,000; there was also relatively little variation across types of households; and finally the income of the respondents had no significant effect on the taxes rates they reported.

%\par This is suggestive because ultimately redistributive policies imply a tax.  Our conjecture here is that some of the utility that individuals derive from taxation is unrelated to redistribution (either from a self-interested or societal perspective).  Individuals have an internalised norm of what constitutes an appropriate level of taxation -- a norm that is independent of the utility individuals might derive from redistributive policy outcomes.  If this is the case, then utility functions for redistribution which ignore this norm will obviously be underspecified.  It is possible of course that the weight for this element of the utility function is relatively small, in which case ignoring it probably is of little significance.  Alternatively, it may be large, and accounting for it might help us resolve some of the puzzles regarding redistributive policies and outcomes described earlier.

%\par The project is designed to calibrate the importance of this �taxation norm� in the individual�s utility function for redistribution.  We develop experimental treatments in a fashion that enables us to isolate the importance of this particular consideration from the other factors that likely shape individuals' preferences for redistributive policies.  Accordingly, the project, focuses on preferences for redistributional taxation, addressing the following hypotheses:

%\begin{itemize}

%\item There is a taxation norm that significantly constrains individuals' preferences for redistributional taxation, and as a result, preferences for redistributional taxation are relatively homogeneous in any particular population and do not vary across cultural contexts.
%\item Preferences for redistributional taxation conform to classic theories of redistribution and hence are strongly related to the individual's income relative to average income in the population.
%\item Preferences for redistributional taxation are determined by other-regarding preferences.
%\item Insurance considerations shape preferences for redistributional taxation.
%\item Culture shapes the ``taxation norm'' and hence while it might be homogeneous in any particular population there is variation in levels of the ``taxation norm'' across cultural contexts.
%\item Culture conditions the effect of other-regardingness on redistributional policy preferences.  For example, this might reflect cultural variations in norms regarding inequality.
%\item Culture conditions the importance of insurance in shaping preferences for redistributional taxation.
%\item Preferences for redistributional taxation are conditioned on efficient outcomes or on the deadweight loss associated with the provision of collective goods. 
 %\end{itemize}


\section{Experimental Design for Recovering Redistributional Preferences}

\par This essay reports the results for six treatments designed to provide insight into which of these factors actually shape tax compliance and hence preferences for redistributive taxation.  The experiment consists of four modules. Subjects are paid for all four modules at the end of experiment, and do not receive feedback about earnings until the end of the experiment. Participants receive printed instructions at the beginning of each module, and instructions are read and explained aloud. 

%Table~\ref{tab:treatments} summarises these treatments.  The first set of treatments are for the tax and audit rates.   These are designed to provide insight into our hypothesised ``tax norm.''   By varying the tax rate......  These are implemented in modules 2 and 3....


\par The second and third module consist of ten rounds each. Table~\ref{tab:treatments} summarises the treatments that are implemented in these two modules of the experiment. At the beginning of the second module participants are randomly assigned to groups of four and we follow a partner matching. Thus, the composition of each group remains unchanged for the two modules. Each round of these two modules is divided in two stages. In the first module subjects perform a real effort task. This task consist of computing a series of additions in one minute. Their Preliminary Gains depend on how many correct answers they provide, getting 150 ECUs for each correct answer. 

\par \emph{Tax Rate Treatments.} In the second module, once subjects have received information concerning their Preliminary Gains, participants are asked to declare these gains. A certain percentage or ``tax'' (that depends on the treatment) of these Declared Gains is then deducted from their Preliminary Gains.  We conduct a total of twenty-two different sessions that are summarised in Table~\ref{tab:treatments}.  Note that in each session the tax rate is consistent and it does not vary from the second to the third module.  The tax treatments are the following: 10\%, 20\%, 30\%, 40\%, and 50\%.  We also have an endogenously determined tax -- one that is determined by majority vote.

\begin{table}[h!]
\caption{Summary of Tax Compliance Experimental Treatments}\label{tab:treatments}
\begin{center}
\begin{tabular}{p{2cm}p{2cm} p{2cm} p{2cm}p{2cm}p{2cm}p{3cm}}
Session & Participants & Groups & Tax Rate & AR Block 1  & AR Block 2 & Treatment  \\ \hline \hline
1 & 24 & 6  & 10\% & 0\% & 100\% & Equal Salary\\
2 & 24 & 6  & 20\% & 0\% & 100\% & Equal Salary \\
3 & 24 & 6  & 30\% & 0\% & 100\% & Equal Salary \\
4 & 24 & 6  & 40\% & 0\% & 100\% & Equal Salary\\
5 & 24 & 6 & 50\% & 0\% & 100\%  & Equal Salary\\
6 & 20 & 5  & 10\% & 30\% & 70\% & Equal Salary \\
7 & 24 &  6 & 20\% & 30\% & 70\% & Equal Salary\\
8 & 20 &  5  & 30\% & 30\% & 70\%  & Equal Salary\\
9 & 24 & 6 & 40\% & 30\% & 70\% & Equal Salary\\
10 & 24 & 6 & 10\% & 30\% & 70\% & Different Salary \\
11 & 24 & 6  & 20\% & 30\% & 70\%  & Different Salary\\
12 & 20 & 5 & 30\% & 30\% & 70\% & Different Salary \\
13 & 24 &  6  & Endogenous & 0\% & 100\% & Equal Salary \\
14 & 20 & 5 & Endogenous & 30\% & 70\% & Equal Salary \\
15 & 24 & 6  & 10\% & 0\% & 100\%  & Different Salary\\
16 & 12 & 3 & 20\% & 0\% & 100\% & Different Salary \\
17 & 16 &  4  & 20\% & 0\% & 100\% & Different Salary \\
18 & 20 & 5 & 30\% & 0\% & 100\% & Different Salary \\
19 & 24 & 6  & 10\% & 0\% & 100\%  & Different MPCR\\
20 & 20 & 5 & 20\% & 0\% & 100\% & Different MPCR \\
21 & 20 &  5  & 30\% & 0\% & 100\% & Different MPCR \\
22 & 20 & 5 & 40\% & 0\% & 100\% & Different MPCR \\
\hline \hline
\end{tabular}
\end{center}
\end{table}


\par  \emph{Inequality and Redistribution Treatments.}  We also included treatments that were designed to determine whether subjects conditioned their compliance behaviour on the impact of the tax revenues on redistribution.  Accordingly in the first equal salary treatment subjects get the same payment for correct answers to the real effort test (10 pence).  This represents the least redistributive of the salary treatments.  In the different salary treatment (Inequality) the two low performing subjects get 5 pence per correct answer and the high performing subjects get 15 pence per correct answer.  The distribution of tax revenues in this treatment result in a moderately higher degree of redistribution.  Finally in a third ``Diff MPCR'' treatment (Redistribution) the two participants per group with the lower income (each round) receive 35\% of the pooled deductions, while the two with higher income receive 15\% -- in case of ties on the number of additions computed (income), the division is decided at random.  This represents the treatment in which the redistributive use of the tax revenues is the most aggressive.

%; 4) a tax efficiency treatment in which subjects earn the same per correct answer (treatment 1) but the tax revenues are arbitrarily reduced by 35\% before they are distributed to the group; 5) a similar tax efficiency treatment although in which there is both redistribution (the Diff MPCR treatment) and an arbitrary reduction by 35\% in tax revenues before they are distributed.


\par \emph{Audit Rate (AR) Treatments.} In each module there is a certain probability that the Declared Gains are compared with the actual Preliminary Gains in order to verify these two amounts correspond. In the second module (in sessions 1 through 5) the probability is 0\%, while this probability changes to 100\% in the third module (in sessions 1 through 5). In sessions 6 through 12, the audit rate is 30\% in the second module and 70\% in the third module.  If the audit finds a discrepancy between the Preliminary and Declared gains an extra amount is deducted from the Preliminary Gains. In both modules the amount correspond to 50\% of the observed discrepancy. In addition, the regular deduction applies to the Preliminary Gains and not to the declared amount. Deductions applying to the four group members are then pooled and equally distributed amongst those members. 

\par At the end of each round participants are informed of their Preliminary and Declared gains; whether these two amounts have been audited; the amount they receive from the deductions in their group; and the earnings in the round. At the end of each module one round is chosen at random, and their earnings are based on their profit for that round. Participants are only informed of their earnings for each module at the end of the experiment.  



\par \emph{Dictator Game.} In order to evaluate arguments regarding other-regarding preferences and attitudes about redistributive taxation we included in the first module a Dictator Game. Subjects are asked to allocate an endowment of 1000 ECUs between them and another randomly selected participant in the room. Participants are informed that only half of them will receive the endowment, and the ones who receive the endowment will be randomly paired with those who don't. However, before the endowments are distributed and the pairing takes place, they may allocate the endowment between themselves and the other person as they wish if they were to receive the endowment. 

\par \emph{Risk Aversion.}  Concern about job or status security is hypothesized to shape redistribution preferences.  Risk averse subjects should be most enthusiastic about redistributive taxation.  The fourth and last module of the experiment consists of a lottery-choice test consisting of ten pairs, which is based in the low-payoff treatment studied in \citep{Holtetal2002}. The lottery choices (shown in  Table~\ref{tab:lottery}) are structured so that the crossover point to the high-risk lottery can be used to infer the degree of risk aversion. Subjects indicate their preferences, choosing Option A or Option B, for each of the ten paired lottery choices, and they know one of these choices would be selected at random ex post and played to determine the earnings for the option selected. 

\begin{table}[h]

\caption{Lottery Choices}\label{tab:lottery}
\begin{tabular}{c|p{7.5cm}p{7.5cm}}

& Option A & Option B \\

\hline

1 &10\% of 2.00\pounds, 90\% of 1.60\pounds &  10\% of 3.85\pounds, Bs. 90\% of 0.10\pounds\\

2 &20\% of 2.00\pounds, 80\% of 1.60\pounds &  20\% of 3.85\pounds, Bs. 80\% of 0.10\pounds\\

3 &30\% of 2.00\pounds, 70\% of 1.60\pounds &  30\% of 3.85\pounds, Bs. 70\% of 0.10\pounds\\

4 &40\% of 2.00\pounds, 60\% of 1.60\pounds &  40\% of 3.85\pounds, Bs. 60\% of 0.10\pounds\\

5 &50\% of 2.00\pounds, 50\% of 1.60\pounds &  50\% of 3.85\pounds, Bs. 50\% of 0.10\pounds\\

6 &60\% of 2.00\pounds, 40\% of 1.60\pounds &  60\% of 3.85\pounds, Bs. 40\% of 0.10\pounds\\

7 &70\% of 2.00\pounds, 30\% of 1.60\pounds &  70\% of 3.85\pounds, Bs. 30\% of 0.10\pounds\\

8 &80\% of 2.00\pounds, 20\% of 1.60\pounds &  80\% of 3.85\pounds, Bs. 20\% of 0.10\pounds\\

9 &90\% of 2.00\pounds, 10\% of 1.60\pounds &  90\% of 3.85\pounds, Bs. 10\% of 0.10\pounds\\

10 &100\% of 2.00\pounds, 0\% of 1.60\pounds &  100\% of 3.85\pounds, Bs. 0\% of 0.10\pounds\\


\bottomrule

\end{tabular}
\label{lottery}

\end{table}

\par At the end of the experiment their earnings in ECUs are converted to sterling at the exchange rate 300ECUs = 1\pounds.  While the earnings are computed and payoffs prepared participants are asked to answer a questionnaire, which consists on an Integrity Test, and a series of socio-demographic questions.

\par All of the sessions were conducted at CESS (Center for Experimental Social Sciences), a research facility of Nuffield College, at the University of Oxford. Either 20 or 24 subjects participated in each session. Subjects were recruited from undergraduate and graduate courses of that university. Some of the subjects had participated in previous experiments, but all of them were inexperienced in this particular type of experiment. No subject participated in more than one session of the study. On average, a session lasted around 90 minutes, including instructions and payment of subjects, and the average payment was around 17\pounds. The experiment was computerized using ZTREE \citep{Fischbacher2007}. A copy of the instructions can be found in the Appendix.

%Get Integrity Test reference from Essex


\section{Results}


\par \emph{Tax Norm.} Figure~\ref{fig:report} summaries the annual earnings reported over the two principal treatments -- tax and audit rates.  Audit rates clearly matter in the fashion we would expect.  When the audit rate is zero subjects report a small fraction of their income.  At a 30 percent audit rate subjects are reporting, on average, more than 50 percent of their income.  In the 70 and 100 percent audit treatment subjects report almost all of their income.  

\par Of particular interest for our conjecture are the cases in which there is no audit.  Recall that one of our goals in this project is to identify a ``tax norm,'' that is a broad agreement in the population on what constitutes an appropriate level of redistributive taxation.   Of particular interest to us is the threshold tax rate that clearly deters reporting of income.  Results for both the zero and 30 percent audit rates are suggestive although hardly definitive.  In both cases average reported earnings are highest when the tax rates is 10 percent: almost 40 percent with zero audit and about 70 percent with 30 percent audit.  And in both cases we see steep declines when the tax rate rises from 20 to 30 percent.  

%  Note that at the two highest levels of taxation (30 and 50 percent) there is about 90 percent tax evasion. At 10 percent taxation tax evasion drops quite dramatically -- to about 60 percent in early rounds and rising to 70 percent in later rounds.  This suggests that our subjects consider levels of appropriate taxation to range between 10 and 30 percent -- note this is very consistent with the Rivers et al finding that acceptable levels of taxation are in the range of 25 percent.
 

\begin{figure}[p]
\caption{Average Earnings Reported by Tax Rate}\label{fig:report}
\centerline{\includegraphics[width=\textwidth]{homo_ttl.pdf}}
\end{figure}

\clearpage



%\begin{figure}[p]
%\caption{Misreporting of Earnings by Tax Rate}\label{fig:evasion}
%\centerline{\includegraphics[width=\textwidth]{evasion.pdf}}
%\end{figure}

%\par \emph{Income.} One of the puzzles in the empirical work on redistribution is the evidence that voters' redistributive preferences are weakly correlated with their income and with prevailing levels of inequality.

%  At first glance, this weak relationship is replicated in the experiment.   Figure~\ref{fig:cheat_effort} presents the levels of misreporting of earnings for different levels of earnings in the real effort tasks.  In the first 10 sessions in which there was no auditing of earnings again we see that misreporting is high -- but most interestingly here is that misreporting is not related to real task earnings.  For each of the three tax rates we can see that levels of misreporting is similar for high and low earners.


%\begin{figure}[p]
%\caption{Misreporting of Earnings by Real Effort Earnings}\label{fig:cheat_effort}
%\centerline{\includegraphics[width=\textwidth]{cheat_effort.pdf}}
%\end{figure}

%\clearpage


\par \emph{Income.} One of the puzzles in the empirical work on redistribution is the evidence that voters' redistributive preferences are weakly correlated with their income and with prevailing levels of inequality.  As Figure~\ref{fig:homo_earn} indicates when we control for audit rates there is a relationship between earnings and tax compliance.   The zero audit rate is most revealing.  Here we see that compliance drops as the tax rate increases from 20 to 30 percent -- again with the 40 percent tax rate anomaly.  But for the high earners the effect of taxes on compliance is much more dramatic -- there is about 40 percent compliance at the 10 percent tax rate but this declines to almost zero for tax rates greater than 10 percent.  We also see differences between the low and high earners in the 30 percent audit rate treatment although they are much more moderate.  Tax rates have very little impact on compliance for the low earners but amongst high earners compliance declines with rising tax rates.

%  For both the zero and 30 percent audit rate treatments we see that low and high wage earners respond quite differently to the tax rates.   In both cases, low wage earners seem to be entirely insensitive to tax rates -- they report similar average earnings regardless of tax rate.  High wage earners, on the other hand, are very sensitive to tax rates -- as tax rates rise average reported income declines significantly.   This is particularly evident in the case of the zero audit treatment.  In this case, for the high earners, tax compliance is relatively high at a tax rate of 10 percent (just under 40 percent) but it drops off dramatically for the higher tax rates -- effectively no compliance.

%\begin{figure}[p]
%\caption{Misreporting of Earnings by Real Effort Earnings}\label{fig:homo_earn}
%\centerline{\includegraphics[width=\textwidth]{homo_earn.pdf}}
%\end{figure}

%\begin{figure}
%\caption{Earnings Reported by High and Low Earnings}\label{fig:homo_earn}
%\centering{\subfigure[Zero Audit]{\includegraphics[width=.55\textwidth]{homo_earn.pdf}} \qquad
%\subfigure[30 Percent Audit]{\includegraphics[width=.55\textwidth]{homo_earn_30.pdf}} \\
%\subfigure[Zero Audit]{\includegraphics[width=.55\textwidth]{homo_earn.pdf}}
%\subfigure[30 Percent Audit]{\includegraphics[width=.55\textwidth]{homo_earn_30.pdf}}}
%\end{figure}

%\clearpage

\begin{figure}[p]
\caption{Earnings Reported by High and Low Earnings}\label{fig:homo_earn}
\centerline{\subfigure[Zero Audit]{\includegraphics[width=.55\textwidth]{homo_earn.pdf}}
\subfigure[30 Percent Audit]{\includegraphics[width=.55\textwidth]{homo_earn_30.pdf}}}
\end{figure}

\clearpage

\par \emph{Inequality Treatment.} In what we are calling the equality treatments, subjects earn similar ``wages'' for their real effort tasks and ``tax'' revenues are distributed equally amongst the subjects.  Tax compliance behaviour in these treatments is consistent with our argument regarding a ``tax norm`` in the neighbourhood of 20-30 percent.  It is frequently argued though that preferences for redistributive taxation are conditioned on whether tax expenditures reduce inequality via redistribution.  If individuals value equality or have preferences for ``social insurance'' then one would expect their tolerance for tax rates to rise when these taxes result in a reduction in income inequality.  Compliance should be conditioned on the redistributive effects of tax revenues raised.

% And it would seem that it is high wage earners who most clearly are sensitive to this ``tax norm`` threshold.  One of the prominent conjectures of interest in this project is that inequity aversion drives preferences for redistributive policy preferences.  Underlying this inequity aversion might be a variety of hypothesized motivations -- insurance against the effects of downward mobility, other-regarding preferences, etc. 

\par  To explore this notion that redistributive preferences are conditioned on existing levels of inequality we developed a second set of treatments that induce greater levels of inequity into the resulting wages earned in each of the groups.  Two members are randomly allocated to a low wage category (they earn 100 ECU for each correct answer for their real task efforts) and two others are randomly allocated to a high wage category (they earn 200 ECUs for each correct answer).  The distribution of earnings in each group will now be much more inequitable.  If concern with prevailing levels of inequity shapes preferences for redistributive taxes we would expect that the average earnings reported by subjects would be higher in this treatment.  
 
\par We conducted inequality treatment sessions for the three tax rates (10\%, 20\%, and 30\%).  Each of these sessions included a 0\%, 30\%, 70\% and 100\% audit treatment.  The overall findings, summarised in  Figure~\ref{fig:hetero_ttl}, suggest that redistributive tax preferences may in fact be conditioned on prevailing levels of inequity.  In the earlier tax treatments that included more equitable wages, average reported wages did respond quite distinctly to rising tax rates -- particularly in the zero and 30\% audit treatments.    In the results reported in Figure~\ref{fig:hetero_ttl}, where the distribution of wage earnings is more inequitable, average reported wages do not seem to vary much over tax rates.  It may be the case that individuals are more willing to pay higher rates of taxation (in our case report higher wage earnings) when levels of inequality are high.  

\begin{figure}[p]
\caption{Average Earnings Reported in Inequality Treatment}\label{fig:hetero_ttl}
\centerline{\includegraphics[width=\textwidth]{hetero_ttl.pdf}}
\end{figure}

\clearpage

%\par It might be the case, of course, that it is low wage earners who are primarily Inequity aversion, of course, may not be particularly surprising amongst low wage earners where, presumably, it gets confounded with self-interest.  

\par The compliance behaviour of high earners though is of particular interest here because their decision to comply can contribute to the redistributive affect of the taxes revenues raised in each round.  Figure~\ref{fig:homo_hetero_zero} compares high and low wage earners for the equality and inequality wage treatments with an audit rate of zero.  The graph on the left of Figure~\ref{fig:homo_hetero_zero} suggests that subjects earning high wages in the treatment with equal wage rates are resistant to tax rates that rise much above 10 percent.    A different picture emerges in the graph on the right hand side of Figure~\ref{fig:homo_hetero_zero}  which presents the results for the unequal wage treatment, i.e., the treatment with a resulting distribution of wages that is much more unequal.  In the inequity treatment the average wage reporting behaviour of high wage earning subjects does not vary much across tax rates (again 40\% is an outlier here) -- they are not responding to higher tax rates by reducing their reported earnings which was the case for the equal wage treatment.

%  Moreover, there is some limited evidence that high income earners report a higher percentage of their average earnings than is the case for the lower wage earners.


\begin{figure}[p]
\caption{Average Earnings Reported in Equality and Inequality Treatments (Zero Audit Rate)}\label{fig:homo_hetero_zero}
\centerline{\includegraphics[width=\textwidth]{homo_hetero_zero.pdf}}
\end{figure}

\clearpage


\par  \emph{Redistributive Treatment.} One of the explanations put forward to explain the puzzle of the median voter's acquiescence to rising levels of inequality (in the U.S. at any rate) is that they do not perceive clearly the redistributive effects of taxation.  One implication of this argument is that individuals would prefer a tax system that redistributes (from high earners to low earners) over one that does not redistribute.  The conjecture implies that when subjects are made aware of the redistributive effects of taxes collected on their earnings they will be more likely to comply and report higher percentages of their actual earnings \citep{Kuziemkoetal2013}.

\par We implemented a treatment designed to be more explicitly redistributive. In sessions 19 to 22 (Diff MPCR) the two participants per group with the lower income (each round) receive 35\% of the pooled deductions, while the two with higher income receive 15\%. In case of ties  on the number of additions computed (income) it?s decided at random.  Figure~\ref{fig:homo_hetero_zero_2} compares high and low wage earners for the redistributive treatment with the equality and inequality wage treatments described in Figure~\ref{fig:homo_hetero_zero}, again for the audit rate of zero.  For the high earners we see a pattern similar to the one for the inequality treatment -- compliance is not sensitive to tax rate.  In fact there is some evidence of higher compliance for the higher tax rates.


\begin{figure}[p]
\caption{Average Earnings Reported in Equality, Inequality and Redistributive Treatments (Zero Audit Rate)}\label{fig:homo_hetero_zero_2}
\centerline{\includegraphics[width=\textwidth]{homo_hetero_zero_2.pdf}}
\end{figure}

\clearpage



\par We implement two treatments designed to increase the redistributive effects of the tax revenues raised in each round of the tax compliance games.  The results suggest that compliance by subjects, particularly high earners, is less sensitive to tax rates when the taxes raised are more likely to result in reductions in earnings inequality.  Hence there is some evidence here that preferences for redistributive taxation are conditioned on prevailing levels of inequality.  If individuals believe there is income inequality that will be addressed by tax revenues, they are inclined to tolerate high levels of tax rates.

%\par One of the puzzles noted at the outset of this essay is that redistributive preferences are not correlated with income as we would expect.  An explanation is that inequity aversion amongst high income earners generates support for redistributive taxation that would not be predicted by a classic model of redistribution.  And some have even speculated that the genesis for this is that high levels of income inequity represents a threat to high income earners (crime, threats to private property, etc.) and hence reducing income inequality is a valued public good for this segment of the voter population.  Our results do not speak to the genesis of this phenomenon but they do suggest there is some evidence that rising income inequalities might generate a greater willingness to pay high taxes.

\par \emph{Other-regarding Preferences.} Our interpretation of the results in Figure~\ref{fig:report} is that reporting compliance at different levels of taxation in the zero audit treatment provides some insight into generally accepted levels of redistributive taxation.  Compliance seems to differ rather dramatically between a 10 and 30 percent level of taxation.  An alternative argument that we explored earlier is that any expression of redistributive preference is contingent upon some underlying other-regarding preference.  Hence reporting compliance under any tax regime will only occur primarily for other-regarding types in the population.  In order to explore this possibility subjects in our experiment also played a Dictator Game in which they were allocated a sum of 300 ECUs (1 \pounds) and then were given the opportunity to share any amount of this with a randomly selected recipient from the other subjects.

\par Figure~\ref{fig:other_regarding} compares average earnings reported across tax regimes for low other-regarding types and for high other-regarding types.  First, it is clear that reporting of earnings is much higher amongst high other-regarding types compared to low other-regarding types.  For any given tax rate, high other-regarding types have much higher reported average earnings.   But even amongst both groups, we do see evidence of the hypothesised ``tax norm'' effect.  For both types we do see a significant drop in reported earnings between the 20\% and 30\% tax treatments.  For the low other-regarding types we see reporting fall to very small percentages for both the 30\% and 50\% tax treatments -- and as we saw earlier the 40\% treatment is an outlier.  Although average earnings reported are much higher for the other-regarding types, we see nevertheless the same pattern, i.e., a significant drop of in reported earnings between the 20\% and 30\% tax treatments (and again the 40\% tax treatment confounds a strictly linear relationship).   

%\begin{figure}[p]
%\caption{Misreporting of Earnings by Other-regarding Preferences}\label{fig:cheat_orp}
%\centerline{\includegraphics[width=\textwidth]{cheat_orp.pdf}}
%\end{figure}


\begin{figure}[p]
\caption{Misreporting of Earnings by Other-regarding Preferences}\label{fig:other_regarding}
\centerline{\includegraphics[width=\textwidth]{other_regarding.pdf}}
\end{figure}


\clearpage

%\par This result may not be entirely surprising.  In a tax regime in which there is no audit, self-interested actors should misreport earnings.  And this is precisely what we see here -- in fact virtually all subjects misreport regardless of the tax rate.  But amongst other-regarding subjects we are able to detect an impact of tax rates on compliance.  And amongst these subjects its clear that compliance is quite high (on average about 50 percent) at the 10 percent tax rate but declines to only an average of about 20 percent for both the 30 and 50 percent tax rates.  Amongst non-other-regarding subjects there is very little variation -- compliance declines from about 20 percent at the 10 percent tax rate to essentially no compliance at the 50 percent tax rate.  

%\par As part of the online experiment we asked respondents to locate themselves on a Left-Right ideological scale.  Our intuition here is that we might find that Left identifiers would behave similarly to other-regarding types.  This does not appear to be the case.   Figure~\ref{fig:cheat_left} reports average earnings reported across tax regimes for Left, Centre and Right self-identifiers.  The only strong evidence here of an ideological effect is that compliance is very high for Left identifiers in the 10\% tax regime treatment. Other than this there is little evidence of an ideological effect on tax compliance. 

%\begin{figure}[p]
%\caption{Misreporting of Earnings by Left-Right Self-Identification}\label{fig:cheat_left}
%\centerline{\includegraphics[width=\textwidth]{LR.pdf}}
%\end{figure}

%\begin{figure}[p]
%\caption{Misreporting of Earnings by Left-Right Self-Identification}\label{fig:cheat_left}
%\centerline{\includegraphics[width=\textwidth]{cheat_left.pdf}}
%\end{figure}

%\clearpage

%\subsection{Redistribution Treatment}

%\par One of the explanations put forward to explain the puzzle of the median voter's acquiescence to rising levels of inequality (in the U.S. at any rate) is that individuals 1) may not understand the levels of inequality that exist; and 2) they do not perceive clearly the redistributive effects of taxation.  One implication of this argument is that individuals would prefer a tax system that redistributes (from high earners to low earners) over one that does not redistribute and hence tax revenues are distributed equally (ignoring differences in earnings).   Our lab experiment offers an excellent opportunity to test this proposition.  The conjecture implies that when subjects are made aware of the redistributive effects of taxes collected on their earnings they will be more likely to comply and report higher percentages of their actual earnings \citep{Kuziemkoetal2013}.

%\par We included a set of redistribution treatments, sessions 19 to 22, in which the two participants per group with the lower income (each round) receive 35\% of the pooled deductions, while the two with higher income receive 15\%. In case of ties  on the number of additions computed (which determines income) the subject's ``tax bracket'' was decided at random.

%\par  Figure~\ref{fig:treat_redistribute} compares the redistribution treatment with the treatment in which tax revenues are distributed equally to participants in each group.  There is some evidence that, in the redistributive treatment tax, compliance is somewhat less responsive to tax rates and possibly higher for the higher tax regimes.

%\begin{figure}[p]
%\caption{Equal versus Redistributive Allocation of Taxes Collected}\label{fig:treat_redistribute}
%\centerline{\includegraphics[width=\textwidth]{treat_redistribute.pdf}}
%\end{figure}

%\clearpage

%DOES THIS REFLECT INSURANCE CONCERNS?

\par Figure~\ref{fig:other_regarding} suggests that preferences for redistribution taxation are strongly related to other-regarding preferences.  This one individual characteristic seems to dominate explanations for heterogeneity in redistributive preferences.  Figure~\ref{fig:homo_earn} presents the high versus low earner effects and the equality versus the redistributive treatment, controlling for other-regarding types in both cases.  There is evidence that the effects are conditioned to some extent on other-regarding type.  In both cases, there is some evidence that treatment effects are more pronounced for the selfish types.   Note that for the equality versus redistribution treatments, there is a marked difference for the selfish types -- while for the other-regarding types there is essentially no treatment effect.  Figure~\ref{fig:homo_earn} also compares tax compliance for low versus high earners amongst the selfish subjects with low versus high earners amongst the other-regarding subjects.  As we noted earlier there is no tax treatment effect amongst low wage earners (of either type).  And while there is a tax treatment effect for both selfish and other-regarding subjects, its magnitude is much stronger for the selfish subjects.

% represent the most types are by far  indicates, the effect is very much conditCompliance is generally very low for selfish types.  In the redistribution treatment selfish subjects are very responsive to tax rates.  We only see compliance at the 10 percent tax rates -- above this tax rate compliance falls below 5 percent.  For other-regarding subjects, compliance is much higher.  In the redistributive treatment compliance is not very response to tax rates -- its similar across the 10 through 30 tax rates -- it does rise though at the 40 percent tax rate which again reflects this anomalous 40 percent tax rate effect.
 
 
 \begin{figure}[p]
\caption{Earnings Reported by Other-regarding Preferences (Zero Audit)}\label{fig:homo_earn}
\centerline{\subfigure[Equal versus Redistributive Treatment]{\includegraphics[width=.55\textwidth]{redistribute_dictator.pdf}}
\subfigure[High versus Low Earners]{\includegraphics[width=.55\textwidth]{dictator_earn.pdf}}}
\end{figure}

\clearpage


%\begin{figure}[p]
%\caption{Equal versus Redistributive Allocation of Taxes Collected}\label{fig:redistribute_dictator}
%\centerline{\includegraphics[width=\textwidth]{redistribute_dictator.pdf}}
%\end{figure}

%\clearpage

\par We can draw three general conclusions from these these simple presentations of the treatment effects from the tax compliance experiments.  There is some support here for the notion of a tax norm -- tax compliance drops quite precipitously beyond about a 20 percent tax regime.  The tax treatment is much stronger for high, as opposed to, low wage earners -- compliance for the former is much more sensitive to tax rates than it is for the latter.   The redistributive effects of tax revenues does condition compliance -- when taxes have a higher redistributive effect, tax compliance is higher.  Finally, and most impressively, the tax treatments are very much conditioned on other-regarding preferences.  Tax compliance is much lower overall for selfish types.  And in addition, the treatment effects seem to be more pronounced amongst selfish, as opposed to other-regarding, types.

\subsection{Multivariate Results}

\par In an effort to disentangle the relative importance of these different treatments and subject attributes we have estimated a simple subjects' fixed effects model for percentage of declare income.   We estimated a separate equation for each of the three income treatments -- equal, unequal, and redistributive -- and also a equation that combined the three and included a dummy variable representing the three income treatments.  These are all for the zero audit treatment.  Table \ref{table:contribute} presents the results.  Each equation includes a term for the amount subjects donated in the Dictator Game (high values indicate more other-regarding preferences); a measure of how much money each subject earned in the real-effort task they undertook; dummy variables for the tax rates; and tax treatments interacted with a dummy variable for amount given in the Dictator Game (the variable assumes a value of 1 for amounts greater than 200 ECUS -- the median amount given).  


\begin{table}[h]
\begin{footnotesize}
\caption{Fixed Effect Regression Model for Percent of Earnings Reported (Zero Audit)}
\def\sym#1{\ifmmode^{#1}\else\(^{#1}\)\fi}
\begin{tabular}{l*{4}{c}}
\hline\hline
            &\multicolumn{1}{c}{(1)}&\multicolumn{1}{c}{(2)}&\multicolumn{1}{c}{(3)}&\multicolumn{1}{c}{(4)}\\
            &\multicolumn{1}{c}{Equal}&\multicolumn{1}{c}{Unequal}&\multicolumn{1}{c}{Redistribute}&\multicolumn{1}{c}{Total}\\
\hline
  &                     &                     &                     &                     \\
Dictator Game Offer    &    0.000850\sym{*}  &     0.00116\sym{*}  &    0.000841\sym{*}  &    0.000953\sym{***}\\
            &      (2.52)         &      (2.40)         &      (2.45)         &      (4.37)         \\
[1em]
Earnings      &   -0.000225\sym{***}&   -0.000132\sym{***}&   -0.000187\sym{***}&   -0.000182\sym{***}\\
            &    (-12.24)         &     (-6.52)         &     (-8.49)         &    (-15.67)         \\
[1em]
DG Offer X 10\% Tax &      0.0504         &      -0.340         &      -0.323         &      -0.193         \\
            &      (0.29)         &     (-1.39)         &     (-1.72)         &     (-1.66)         \\
[1em]
DG Offer X 20\% Tax &      -0.333         &      -0.268         &     -0.0978         &      -0.234\sym{*}  \\
            &     (-1.95)         &     (-1.25)         &     (-0.56)         &     (-2.18)         \\
[1em]
DG Offer X 30\% Tax &     -0.0804         &      -0.539\sym{**} &     -0.0456         &      -0.196         \\
            &     (-0.53)         &     (-2.59)         &     (-0.26)         &     (-1.93)         \\
[1em]
DG Offer X 40\% Tax &      -0.211         &                     &     -0.0900         &      -0.196         \\
            &     (-1.34)         &                     &     (-0.43)         &     (-1.57)         \\
[1em]
DG Offer X 50\% Tax &      -0.156         &                     &                     &      -0.191         \\
            &     (-0.90)         &                     &                     &     (-1.29)         \\
[1em]
20\% Tax Rate  &       0.104         &      0.0575         &      -0.146         &     0.0028         \\
            &      (0.96)         &      (0.54)         &     (-1.29)         &      (0.04)         \\
[1em]
30\% Tax Rate &     -0.0972         &      0.0928         &      -0.151         &     -0.0502         \\
            &     (-0.98)         &      (0.92)         &     (-1.18)         &     (-0.79)         \\
[1em]
40\% Tax Rate  &      0.0472         &                     &      -0.110         &    -0.0075         \\
            &      (0.44)         &                     &     (-0.79)         &     (-0.09)         \\
[1em]
50\% Tax Rate  &      -0.111         &                     &                     &      -0.146         \\
            &     (-1.02)         &                     &                     &     (-1.44)         \\
[1em]
Unequal Earnings &                     &                     &                     &     -0.0318        \\
            &                     &                     &                     &     (-0.67)         \\
[1em]
Redistributed Earnings &                     &                     &                     &     -0.048        \\
            &                     &                     &                     &     (-1.09)         \\
[1em]
Constant      &       0.523\sym{***}&       0.296\sym{***}&       0.510\sym{***}&       0.477\sym{***}\\
            &      (6.23)         &      (3.54)         &      (5.86)         &      (8.32)         \\
\hline
    &                     &                     &                     &                     \\
Standard Deviation Constant      &      -1.372\sym{***}&      -1.265\sym{***}&      -1.265\sym{***}&      -1.257\sym{***}\\
            &    (-20.14)         &    (-14.49)         &    (-15.80)         &    (-28.43)         \\
\hline
     &                     &                     &                     &                     \\
Standard Deviation Residual      &      -1.708\sym{***}&      -1.821\sym{***}&      -1.765\sym{***}&      -1.753\sym{***}\\
            &    (-79.36)         &    (-65.45)         &    (-68.62)         &   (-123.51)         \\
\hline
N           &        1200         &         720         &         840         &        2760         \\
Log Likelihood          &       165.8         &       165.4         &       150.2         &       461.1         \\
\hline\hline
\multicolumn{5}{l}{\footnotesize \textit{t} statistics in parentheses}\\
\multicolumn{5}{l}{\footnotesize \sym{*} \(p<0.05\), \sym{**} \(p<0.01\), \sym{***} \(p<0.001\)}\\
\end{tabular}
\label{table:contribute}
\end{footnotesize}
\end{table}
\clearpage

\par In each of the three equations, two variables essentially dominate the explanation of tax compliance.  As subjects' earnings in the real-effort tasks rise the actual earnings they report declines.  Subjects appear to be responding to the absolute amounts of earnings that get transferred by the tax regime -- as these increase, the percent of actual earnings reported declines.  When there are no costs to misreporting, the "poor" report greater percentages of their earnings then do the "rich".  The second variable that has an important effect on compliance is other-regarding preferences.   Subjects who contribute more in the Dictator Game report a higher percentage of their actual earnings, i.e., are more likely to comply with prevailing tax rates.  Selfish subjects in the Dictator Game report a lower percentage of their actual earnings.  There is no support in the multivariate estimates for the notion of a tax norm.  Similarly there is no support for the notion that subjects condition their compliance behaviour on the extent to which tax revenues are used for redistributive ends. 
     
\section{Conclusion}

\par This essay reports the results for an ongoing set of experiments designed to help understand individual preferences for redistributive taxation.  Our initial working conjecture is that there exists a taxation norm that is relatively homogeneous within populations and across countries.  This tax norm represents what individuals accept as an appropriate rate at which income should be taxed.  We attempt to identify this tax norm by administering tax compliance experiments in which individuals are assigned to different tax rate treatments.  The results for these first set of experiments suggests that our initial conjecture regarding a tax norm of approximately 20 to 30 percent is plausible although the notion that it might be homogenous within and across populations is less tenable.   Moreover the effect is not significant in the multivariate estimation.

% Hence one of our primary interests is understanding what factors might drive individuals to prefer rates of taxation that are higher (or lower) than this tax norm. Our experiments are designed to help us understand how these preferences for redistribution might be shaped by context and also might vary across identifiable "types" within a population.

\par We included treatments designed to assess whether subjects condition their tax compliance on the extent to which tax revenues are used for redistributive purposes.  There is weak evidence of a redistribution effect.  In particular, we find that high wage earners seem more willing to pay higher taxes when there tax revenues are employed for reducing levels of inequality. Reducing inequality may be a public good for the rich.  But again, these results do not hold up in the multivariate estimation.

\par Two factors stand out in accounting for compliance which we are treating as an indicator of preferences for redistributive taxation.  High wage earners are significantly less compliant than low wage earners.  As earnings increase, subjects are more likely to cheat -- and this is for the zero audit treatment which means subjects incur no cost from cheating.  This result is very robust and holds up in the multivariate analysis.  In addition to playing the tax compliance games, subjects also played a standard Dictator Game.  We find that subjects who are other-regarding when they play the Dictator Game are more likely to report income when audit rates are low.  This is also a very strong result that is confirmed in the multivariate analysis. 

\par These results are preliminary but they do provide some interesting insights into the puzzle that motivated this research project.  On one hand the results seem to confirm the classic representation of the citizen-taxpayer.  As subjects earn more income they are  less willing to voluntarily comply with prevailing tax rates.  This suggests that as one's overall wealth increases one becomes increasingly antagonistic to redistributive taxation.  Now there is some evidence that both the rich and poor become less acceptance of redistributive taxation one rates exceed about 20 percent although the statistical significance of this relationship is weak.  

\par The second strong finding, on the other hand, is not consistent with classic representations of the citizen-taxpayer.  A large segment of the participating subjects exhibit other-regarding preferences -- a finding consistent with a large body of work in behavioural economics.  This characteristic is one of the strongest predictors of the percentage of actual income reported by subjects.   Simply knowing a subject's earning will result in a misleading prediction of their preferences for redistributive taxation -- one obtains a much more accurate prediction if this information is combined with whether the subject is a selfish or other-regarding type.

%   Nevertheless, we continue to see evidence of our tax norm within low and high other-regarding types:  In the case of both groups we see a significant drop in tax compliance as tax rates move from 20\% to 30\%. 

\newpage
\singlespace
\bibliography{dave}

\newpage

\end{document}

\section{Introduction}

\par In most democratic elections voters are faced with a choice between an incumbent coalition, composed of a number of parties, and an expectation that a similar or different group of parties could form a new governing coalition after the election.  But do voters actually engage in the kind of coalition reasoning that would allow them to attribute responsibility to the parties in a multi-party coalition and to anticipate the kinds of coalitions that form after an election?  This essay will focus on the latter aspect of coalition reasoning -- their ability to anticipate the kinds of coalitions that form after an election. 

%Surprisingly, our models of electoral vote choice have, until recently, paid little attention to this central feature of the vote choice.  Most formal models of vote choice build on the classic Downsian model in which the election outcome is a single-party government \citep{Downs1957}.  In contexts in which the election outcome is a multi-party governing coalition rational voters should consider what parties are likely to coalesce, the policies they will collectively adopt, and whether the voter is pivotal in electing any one of these coalitions. 

%\par It makes sense that the point of departure for modeling this vote choice builds on rational choice theory....  Predominant theories of coalition formation build on this... Importance of a single spatial dimension..... And of course by extension the voter should ..... Efforts to model this  

%\par We typically think of voters as making relatively simple calculations when they decide how to vote in a national election.  The simplest characterization is one in which voters support or oppose a party based on performance or policy position.  The 2009 German legislative election posed an interesting dilemma for voters employing this simple calculus because the incumbent consisted of a Grand Coalition of the country's two largest political forces.  This simple calculus was problematic even in the case of the 2010 British Parliamentary election in which, as the election approached, it became highly likely that no party would command a majority of seats in the House of Commons.  This was a relatively unprecedented event and one that made it difficult for voters, accustomed to a two-party majoritarian system, who wanted to simply reward, or punish, an incumbent or opposition. Most voters in democratic contexts make vote choices for parties that were, or may be, in a governing coalition.  And to the extent that we think these vote choices are motivated by concerns for policy or performance then we need to understand how voters take into consideration shared administrative responsibility when they vote.

\par  Recently a number of comparative voting behaviour scholars, in particular \citet{Kedar2009,DuchStevenson2008,Duchetal2010}, have argued for a vote utility function that incorporates the coalition component of the vote calculus.  Common to all these models (some of which have a policy voting orientation while others have a performance voting perspective) is the notion that voters exercise a coalition-directed vote.  Voters are not simply assessing the party in isolation but rather thinking about the party's contribution to an outcome that is taken by a governing coalition, formed after the election and, obviously, made up of multiple parties.

\par And there is empirical support for these models of coalition-directed voting.  \citet{DuchStevenson2008} in fact find that the magnitude of the economic vote for particular parties is conditioned by factors related to its position in the governing coalition (or likely governing coalition).  For example, they predict that certain parties in coalition governments -- perennial prime minister parties in particular -- should get no economic vote.  And \citet{Duchetal2010} find that, based on a large-N comparative study, left-right spatial voting is conditioned by likely coalition outcomes rather than simply the spatial distance between the voter and political parties.

\par But these large N studies do not explore in any detailed fashion the micro-level assumptions underpinning the notion of a coalition-directed vote choice.  One could imagine voters who are uber-sophisticated and perfectly informed in their coalition calculations.  But more likely voters employ heuristics for understanding, and making decisions about, coalition governments.  We make the argument that the sophistication of these heuristics is conditioned on context: coalition reasoning is relatively sophisticated in political context that have a history of complex coalition governance and formation while they are much less well-developed in countries without a history of coalition governance.

%  This essay will identify these coalition reasoning heuristics and examine the extent to which .  Finally, the essay examines whether the vote decisions of coalition-sophisticated voters conform better than the less sophisticated voters to coalition-directed models of vote choice.  In order to  test the micro-foundations of the coalition-directed argument, the study focuses on three distinct institutional contexts: the U.K. which historically has had very little experience with coalition governments prior to 2009;  Denmark which is complex; and Germany which has had stable and quite moderately complex coalition governance.


%  Data from these campaign internet studies are the basis for the analyses presented in this paper.

%two recent elections that provide interesting comparative contexts in which to study the coalition-directed vote: The 2009 German Federal Election and the 2010 British General Election.  In conjunction with these two elections the Centre for Experimental Social Sciences (CESS) at Nuffield College organized a German and British Comparative Cooperative Campaign Project that fielded internet panel surveys that spanned the election campaign for these two elections.  Data from these campaign internet studies are the basis for the analyses presented in this paper.

%\par The contribution of this essay is first to identify voters who in fact conform to this model of coalition-directed voting -- voters who in fact understand the coalition formation process.  They do exist and constitute approximately 20-30 percent of the German voting population.  Secondly, these voters exhibit voting behaviour that is distinct from those who do not exhibit these coalition reasoning skills.  The voting behaviour of savvy coalition reasoning types conforms to two important assumptions of the coalition-directed vote: 1) they are much more likely to anticipate the types of coalitions that form after an election...; 2) they are more likely to incorporate the relative distribution of administrative responsibility within the governing coalition into their vote decision   

\par The essay is organised as follows: First we describe the coalition reasoning that is associated with a coalition-directed vote. We argue that the incidence of coalition reasoning in the population is a function of context, specifically the complexity of coalition governance and formation in the country.  We then propose a strategy for recovering, using experimental vignettes, the heuristics that make up coalition reasoning at the individual voter level.  A subsequent section describes the specific coalition heuristics employed by the British, Danish and German electorates; we establish that in the Danish and German contexts there are large proportions of the population that exhibit relatively sophisticated coalition reasoning heuristics.  We find that the heuristics national electorates employ for understanding coalition politics varies by context.  A final empirical section demonstrates that our measures of coalition reasoning can distinguish individual voters who are more or less likely to exercise a coalition-direct vote.

%\par And I'm not sure this experiment necessarily makes sense so I would very much appreciate any feedback on the overall idea and design.

%\par The paper makes three contributions:  First, it identifies the specific assumptions regarding coalition reasoning on the part of average voters that underlies theories of coalition-directed vote choice.  Second, it reports on the results of a set of internet surveys that are explicitly designed to assess whether voters in fact conform to these theoretical expectations.  And finally, the paper provides some empirical evidence that voting behaviour is in fact shaped by what I call coalition reasoning.

\section{Coalition-directed voting}

%\par This essay builds on theoretical and empirical work suggesting that voters exercise a coalition-directed vote and that voters develop levels of coalition reasoning that are appropriate to the particular institutional context in which they are located.  This section briefly describes the utility function of a coalition directed vote and explains how this results in contextual variation in the sophistication of voters' coalition reasoning.
 

%I will two theories of coalition-directed voting -- the coalition-directed economic vote and ideological vote.  I briefly review the basic mechanics of these models in order to highlight the underlying assumptions about voter knowledge and voter reasoning which will be the subject of the empirical tests in the subsequent section.


\par This essay builds on theoretical work suggesting that voters exercise a coalition-directed vote.  In coalition contexts, coalitions form after elections as a result of bargaining amongst parties over the policies to be enacted by the government \citep{AustenSmithBanks1988,Perssonetal2000}. Policy outcomes in coalition government are expected to reflect the policy preferences of the parties forming the governing coalition weighted by their legislative seats \citep{Indridason2011,Schofieldetal1985}.  In multiparty contexts with coalition governments, \citet{AustenSmithBanks1988} argue, policy voting, directed simply at parties, is not rational.  The implication of the \citet{AustenSmithBanks1988} insight here is that voters anticipate the likely coalition formation negotiations that occur after the election and they condition their vote choices accordingly in order to maximise the likelihood that a coalition government forms that best represents their preferences over government policies or performance.

%\footnote{An alternative, and in our view less plausible, perspective is that the policy outcomes adopted in multiparty contexts reflect the weighted preferences of all parties elected to the legislature \citep{Ortunoortin1997,DeSinopolietal2007}.  This of course significantly reduces the second-order incentives for voters.}

%\par One of the general implications of this theoretical argument is that voters anticipate a disconnect between the election of their preferred party and policy outcomes.  While not specifically focusing on coalition governments, this notion of a disconnect was the focus of considerable attention in the 1980s and 1990s. \citet{Grofman1985}, for example, proposed a modification to the party-directed ideological model that takes into consideration what politicians are actually able to accomplish after an election.  Voters in the \citet{Grofman1985} \textit{discounting} model anticipate that candidates, if elected, will be able to move policy only part way from the status quo position to their bliss point.  And there is a considerable related literature arguing for directional voting by which emotions make voters prefer parties on their side of an issue even though they are spatially more distant than parties on the opposite side of the issue \citep{RabinowitzMacdonald1989,Rabinowitz1978,Adamsetal1999}. 

%\par  \citet{Kedar2005,Kedar2009} more recently has argued that voter reasoning may entail more than simple discounting or voting directionally; they may be reasonably well-informed about post-election coalition formation outcomes and this may condition the ideological vote. She argues that the rational voter focuses on policy outcomes and hence on the issue positions that are ultimately adopted by the coalition government that forms after an election. And she demonstrates that in political systems with coalition governments this leads to ``compensational voting", rather than ideological proximity voting, aimed at minimizing the policy distance between the policy compromises negotiated by the governing coalition and the voters ideal policy position.

%\par \citet{DuchStevenson2008} develop a contextual theory of economic voting in which the amount of administrative responsibility that each party holds conditions the overall magnitude of the economic vote and how it is distributed across parties in the governing coalition but also in the coalitions that are anticipated to form after an election.  A key assumption here is that voters know, or anticipate, the distribution of administrative responsibility which could narrowly be defined as the relative number of portfolios allocated to the parties within the coalition; the particular portfolios held by particular parties; or even administrative responsibility that is exercised by parties not formally in the governing coalition.  Also, voters anticipate the likely coalitions that form after an election and they assess the impact of their vote choice on the likelihood of different coalitions coming to power after an election.  And this information regarding their pivotal role is used by instrumentally rational voters to weight the importance of an economic competency signal in their vote choice function.  Hence, for example, parties that are certain to enter a governing coalition (i.e., perennial coalition partners) should, all things being equal, get no economic vote since a vote for this party has no impact on the coalition that ultimately forms. 

%Both \citet{Kedar2005} and \citet{DuchStevenson2008} go to considerable length to formalize how post-election coalition formation enters into the vote choice function.

%Building on these works, we propose a model of the ideological vote in which voters anticipate the coalitions that form after the election -- what we call the coalition-directed ideological vote.


%\subsection{The coalition-directed economic vote}


%\par \citet{DuchStevenson2008} propose one model of the coalition-directed economic vote that assumes 1) multiple parties; 2) a coalitional executive; 3) opposition policy-making responsibility; and 4) rational beliefs about the competence of alternative governments. Two components of their theory are of particular interest in this essay because they define the coalition-directed aspect of the economic vote: contention and administrative responsibility.  My interest in this essay is understanding what these two features of the theory imply about how voters are informed about coalition governments.  The following equation captures the essence of their argument (details of this result are developed in \citet{DuchStevenson2008} -- here I will focus on the components of the model that relate to contention and administrative responsibility).  It characterizes the voter's utility of voting for party $j$ as opposed to abstaining.  And this particular representation assumes that the status quo distribution of responsibility (which could be thought of as the incumbent coalition) is not in contention, i.e., will not be re-elected.

%\begin{align}
%E[u|v_{j}] - E[u|v_{0}] =Z \sum_{\lambda' \epsilon \Lambda'}  \sum_{\lambda'' \epsilon \Lambda'} \left( P_{j, \lambda'\lambda''} - P_{j, \lambda''\lambda'}\right) \left( \sum_{i \epsilon J} \lambda_{i,t} (\lambda'_{i,t} - \lambda''_{i,t} )  \right) \\ =
%Z \sum_{\lambda' \epsilon \Lambda'}  \sum_{\lambda'' \epsilon \Lambda'} P_{\lambda'\lambda''}(2P(w_{\lambda'}\lambda_{j} - w_{\lambda''} \lambda'_{j} > 0) - 1)  \left( \sum_{i \epsilon J} (\lambda_{i}\lambda'_{i} - \lambda_{i}\lambda''_{i}) \right)
%\end{align}\label{eq:1}

%$Z$ is just the part of the utility difference that does not change across parties e.g. such as perception of retrospective economic performance.



%\paragraph{Administrative Responsibility} The amount of administrative responsibility that each party $i$ holds is captured by $\lambda_{i}$.  And the impact of relative levels of administrative responsibility associated with party on the vote utility is captured by the last part of Equation \ref{eq:1}.  A key assumption here is that voters know the distribution of responsibility across incumbents.  Notice that it still contains the elements of the status quo distribution of authority since it is only through their responsibility for current economic outcomes that voters have any information at all about the competence of the parties in these contending alternatives.  Clearly, if no party in either alternative held responsibility in the status quo distribution, these terms will be zero and there will be no economic voting. Further, if $\lambda_{i}$ and $\lambda''_{i}$  are very similar to one another for most parties this term will be small and economic voting will be muted. So again, we find that distinct alternatives enhance economic voting.

%\par Also, notice that if those parties in the two alternatives that had some experience in the previous distribution of authority, have equal shares of power in the new distribution, this sum will be zero.  For example, suppose that the incumbent distribution of responsibility was a ``Grand Coalition" in which two large parties shared power equally and all other parties had no responsibility at all.  Further, suppose that this coalition breaks up and cannot reform, and that the only contending alternatives are two one-party governments in which one or the other former partners rules alone.  Under these circumstances, the sum in the Equation \ref{eq:1} is (.5*1-.5*0)+ (.5*0-.5*1) = 0.  The intuition of this result is clear:  because the only information the voter has about the competence of the parties comes from economic outcomes under the status quo distribution of responsibility and voters apportion responsibility for those outcomes to the two large parties equally, they must conclude that the two alternatives they must decide between have the same level of competence.  Because of this, they cannot distinguish between them and so there will be no economic voting in this case (since in our example these are the only contending distributions).

%\par This component of the Duch and Stevenson theory assumes that voters are informed about the distribution of administrative responsibility which could be narrowly defined as the relative number of porfolios allocated to the parties within the coalition; the particular porfolios held by particular parties; or even administratrative responsibility that is exercised by parties not formally in the governing coalition.

%\paragraph{Contention.}  Equation \ref{eq:1} includes generalized pivot probabilities for the pairs of potential cabinets that include the incumbent cabinet.  Note that in the Duch and Stevenson theory, $P_{\lambda'\lambda''}$ is the probability that cabinets $\lambda'$ and $\lambda''$ are tied for selection.  $w_{\lambda'}$ is the derivative of selection function for the cabinet $\lambda'$ with respect to electoral support for $\lambda$. Voters do not know $w$ for certain i.e. they are uncertain how changes in electoral support for a given potential cabinet impact its selection. Their belief about these quantities are governed by the probability distributions $f(w_{\lambda}) $and $f(w_{\lambda'})$, respectively.

%\par The important contention intuition here is that the act of voting for a party can only change a voters utility by changing the outcome of the election. The likelihood that a voter's vote changes the outcome is captured by the probability that this vote is decisive. Consequently, the voter's expected utility for voting for a particular party is simply the sum of the ways in which her vote can be decisive multiplied by the utility of the outcome associated with each kind of decisive vote.

%\par This contention component of the Duch and Stevenson model presumes that coalition-directed voters know which distributions of administrative responsibility are most likely to be in contention.  In other words, voters know which governing coalitions are most likely to form after an election and which of these are competitive electorally.

%Other parties competing for seats in the legislature are captured by the $i$ subscript.  The middle term Equation \ref{eq:5} is the pattern of cabinet contention and comes from the fact that $P(X > 0) = 1 - P(X < 0)$



%\subsection{The coalition-directed ideological vote}

%\par A similar challenge faces voters who incorporate party policy positions, rather than performance, into their vote calculus:  administrative responsibility is shared in coalition governments which implies, again, a more complex relationship between party vote choice and expected policy outcomes adopted by coalition governments.

%^  For example, a centrist voter, concerned about policy outcomes and anticipating a likely Tory-Liberal coalition outcome in the recent 2010 British election, might have been more favourable to the Liberal party than would have been the case if the likely outcomes were either a Labour or Conservative majority government.  This suggests that the vote calculus, in context with multi-party governing coalitions, is not simply focused on parties and their ideological proximity to the voter but rather about policy outcomes that result from bargaining amongst party elites after the election takes place \citep{Kedar2009}.


%\par In coalition contexts, coalitions form after elections as a result of bargaining amongst parties over the policies to be enacted by the government \citep{AustenSmithBanks1988,Perssonetal2000}. Policy outcomes in coalition government reflect the policy preferences of the parties forming the governing coalition weighted by their legislative seats \citep{Indridason2007b,DuchStevenson2008,Schofieldetal1985}.\footnote{An alternative, and in our view less plausible, perspective is that the policy outcomes adopted in multiparty contexts reflect the weighted preferences of all parties elected to the legislature \citep{Ortunoortin1997,DeSinopolietal2007}.  This of course significantly reduces the second-order strategic incentives for voters.}  I believe that in coalition contexts voters anticipate these policy outcomes and they use these to condition their ideological vote calculus.\footnote{This anticipation of post-election policy compromises is not restricted to multiparty coalition contexts. \citet{AlesinaRosenthal1995}, for example, suggest that voters in the U.S. context exercise a policy balancing vote, anticipating the policy differences between Congress and the President. \citet{Kedar2006} makes a more general claim suggesting that this occurs in all Presidential regimes.  \citet{Adamsetal2004} analyze individual and aggregate-level data related to U.S. Senate elections and find support for the argument that voters anticipate the moderating effect of the legislative process and hence vote for candidates with more extreme positions.  Although they are careful to point out that their data could not distinguish this discounting argument from a directional voting explanation.}  Rational voters, concerned with final policy outcomes (as opposed to party platforms), condition their vote choices on coalition bargaining outcomes that occur after the election\citep{AustenSmithBanks1988}. In multiparty contexts with coalition governments, \citet{AustenSmithBanks1988} argue, sincere ideological voting is not rational.  

%\par  The coalition-directed ideological vote is an example of how coalition governance complicates the vote utility function \citep{Duchetal2010}. 


\par The \citet{Duchetal2010} formalisation illustrates how the exercise of a coalition-directed vote affects the voter utility calculation.  Voters anticipate the likely coalition formation negotiations that occur after the election and they condition their vote choices accordingly in order to maximise the likelihood that a coalition government forms that best represents their ideological preferences.\footnote{We should be very clear here that this behaviour that we are characterising as strategic only represents one of many dimensions of what is more broadly characterised as strategic voting.  The term strategic ideological vote is simply referring to the conditioning of vote choice on the ideological composition of the likely coalition to form after the election.}  When incumbent governments consist of a single party the standard ideological vote utility is simply expressed in terms of Euclidean distance: $U-(x_{i}-p_{j})^{2}$.  By contrast, \citet{Duchetal2010} argue that a coalition-directed vote utility function should incorporate information about likely coalition outcomes and the distribution of administrative responsibility within those coalitions (this is a slightly simplified version of their argument): 

\begin{footnotesize}
\begin{equation}
u_{i}(j) = \left(
U-\sum_{n=1}^{N_{c_{j}}}(x_{i}-Z_{c_{j_{n}}})^{2}\gamma_{c_{j_{n}}}\right) \end{equation}
\label{eq:20}
\end{footnotesize}


%\begin{footnotesize}
%\begin{equation}
%u_{i}(j) = \lambda\left\{\beta\left(
%U-\sum_{n=1}^{N_{c_{j}}}(x_{i}-Z_{c_{j_{n}}})^{2}\gamma_{c_{j_{n}}}\right) +
%(1-\beta)\left[U-(x_{i}-p_{j})^{2}\right]\right\} + \phi W_{i} \end{equation}
%\label{eq:20}
%\end{footnotesize}


\par Equation \ref{eq:20} represents the utility that voter $i$ derives from party $j$. $\gamma_{c_{j_{n}}}$ is the probability of party $j$ entering into a particular coalition conditional on entering government: ``If Party $j$ were to enter some government coalition, what is the likelihood that it would govern with a particular combination of partners (or on its own)?"  This is a conditional probability such that the sum of these probabilities across all possible coalitions ($c_{j}$), that include $j$, is one.  As a result, each of Party $j$'s likely coalition partners will contribute (either more or less) to the voter's utility function, for party $j$.\footnote{This is important because in our formulation of the $\gamma_{c_{j_{n}}}$ the voter is not making a strategic calculation regarding the likelihood of particular coalitions forming; rather she is simply assessing the likelihood of different coalition partners given that the party does govern in a coalition government (or governing alone).  Note that in this representation of the voter utility calculation, the voter does not weight the particular coalition by its overall likelihood of forming (relative to all coalitions including those of which party $j$ is not potentially a member).} $Z_{c_{j_{n}}}$ is the sum of the seat-weighted ideological positions $p_{k}$ of each party $k$ in the coalition $c_{j_{n}}$, i.e., any possible coalition that includes $j$. Hence, $Z_{c_{j_{n}}}$ is: $\sum_{k\in{c_{j_{n}}}}p_{k}h_{k}$, where $h_{k}$ is the proportion of seats held by party $k$ in coalition $c_{j_{n}}$..

%The first right-hand term in large parentheses in Equation \ref{eq:20} incorporates these coalition-directed components, $\gamma_{c_{j_{n}}}$ and $Z_{c_{j_{n}}}$.

%\par   Hence voters are assumed to be knowledgeable about the electoral strength of parties and how this translates into their shares of portfolios in the cabinets they enter.  Accordingly, the Euclidean distance is between the voter's left-right ideal point and that of the seat-weighted sum of the left-right locations of coalition parties.

%\footnote{ Note that this is a simplification of the vote calculus in that it does not incorporate into the model the coordination dilemma confronting voters; specifically, that voters should not simply anticipate what coalitions are likely to form but also anticipate how other voters will use this information about post-election coalition bargaining.  Voters in these models anticipate how the coalition-directed ideological vote of other voters will affect post-election coalition outcomes and vote accordingly (\citet{McCuenMorton2010} is one of the few efforts to our knowledge that addresses the modelling challenges posed by such behaviour).}


%\par Finally, note that the full coalition-directed component of the model that falls within the large parentheses is weighted by $\beta$ which indicates the importance of coalition-directed considerations and is assumed to vary between 0 and 1.  Equation \ref{eq:20} also includes a conventional party-directed Euclidean distance term which is weighted by $1-\beta$.  As $\beta$ gets large, i.e., voters put more weight on coalition-directed ideological considerations, the party-directed component of the ideological vote gets smaller.
%  Hence voters in this model can give varying weight to ideological considerations that are entirely party-directed which is captured by the standard Euclidean distance term weighted by $1-\beta$.

\par \citet{Duchetal2010} provide evidence that indeed in many contexts the coalition-directed vote, characterised in Equation \ref{eq:20}, is relatively large.  \citet{Kedar2005} finds that voters in contexts with coalition governments engage in compensational voting, i.e., certain voters will vote for more extreme parties with the goal of shifting the policy position of governing coalitions closer to their ideal points.  Recent findings for individual countries suggest that voters do respond in an instrumentally rational fashion to the strategic incentives associated with post-election coalition formation possibilities \citep{BargstedKedar2009,Gschwend2007,Bowleretal2010,Blaisetal2006} or they engage in vote discounting whereby voters support more extreme candidates because they anticipate the moderating impact of the legislative process on policy outcomes \citep{Tomzetal2007,Adamsetal2004,Merrilletal1999,AlesinaRosenthal1995}.

\par An important part of this richer characterisation of the vote calculus is an assumption that voters are reasonably well informed about the coalition formation process after elections.  In particular, voters are informed about $\gamma_{c_{j_{n}}}$ which requires voters to know about the likely coalitions that form after an election.  And they are informed about $Z_{c_{j_{n}}}$ which suggests that voters incorporate, in their expectations about coalition formation, information about the location of parties in the ideological space.  It also presumes voters treat seats in the legislature, or size, as an important factor in determining policy influence in a coalition.

%\par There is a growing literature suggesting that voters, at least in some contexts, are quite knowledgeable about these aspects of coalition formation and governance \citep{Meffertetal2011}.  But of course i

\par Its unlikely that voters exercise the full information calculus implied by Equation \ref{eq:20}.   More likely, voters employ coalition reasoning heuristics that result in choices that closely approximate those implied by Equation \ref{eq:20}. By heuristics we mean strategies that `guide information search and modify problem representations to facilitate solutions' \citep{Goldsteinetal2002}. Heuristics are used when information acquisition is costly and decision making is cognitively challenging \citep{Simon1955}.  Our goal in this essay is to identify the coalition reasoning heuristics employed by voters and characterise their relative incidence in the population.

%\par  The utility function in Equation \ref{eq:20} clearly implies that some voters have well-developed heuristics about coalition formation:  $\gamma_{c_{j_{n}}}$ requires that voters understand that certain coalition combinations are more plausible than others.  $Z_{c_{j_{n}}}$  Surprisingly, though, we have little empirical evidence of precisely what heuristics about coalition bargaining and formation are employed by the typical voter.\fotnote{Although we do have some evidence that voters anticipate the likely coalitions that could form and this in-turn informs a strategic coalition voting calculus \citep{Meffertetal2011}.}  For example, does the typical voter anticipate the formation of minimum winning coalitions which might follow from her understanding of how seats translate into influence in a coalition government?   


\subsection{Context, heuristics and the coalition directed vote}

\par Research on heuristics has made two findings of importance here \citep{Gigerenzeretal1999}.  First, for a wide variety of decision making problems there are heuristics that result in decisions that approximate those of full information rational decision making.  Second, the heuristics individuals adopt are calibrated to the decision making tasks.  Individuals employ heuristics that ``in equilibrium'' result in, or approximate, optimal choices.  Hence if the decision problem is extremely simple, individuals require, and adopt, only very simply heuristics to approximate optimal decisions.  And if the decision problem is informational or cognitively more demanding individuals adopt more complex heuristics that are at the same time less demanding than those assumed by full-information rational models of decision making.  This suggests that the heuristics that voters adopt for making the utility calculation in Equation \ref{eq:20} will depend on the institutional and political context.  The complexity of the $\gamma$ and $Z$ calculations varies quite significantly across our sample of three different national contexts.  In the case of the U.K., prior to the 2009 election, Equation \ref{eq:20} reduces to the simple Euclidean distance calculation noted earlier: $u_{i}(p_{j})=U-(x_{i}-p_{j})^{2}$.  The pre-2009 British voter requires no coalition reasoning heuristics in order to exercise a rational vote choice.  Accordingly, there is no incentive for the British electorate to develop such heuristics and we should find little evidence of them in our British sample.  Equation \ref{eq:20} looks significantly different for the typical German and Danish voter.  The $\gamma$ and $Z$ calculations are moderately complex for the German voter and, comparatively speaking, very complex for the Danish voter.  Clearly, developing coalition reasoning heuristics is valuable for the typical Danish voter. Germany is an intermediate case in which, unlike the U.K., voters have an incentive to develop coalition reasoning heuristics although not as strong an incentive as in Denmark.


\par  Evidence of the contextual variation in the complexity of the utility calculation in Equation \ref{eq:20} is provided by \citet{ArmstrongDuch2010} who have characterised contextual variation in the complexity of coalition governance.  Figure~\ref{fig:coalition_govern} illustrates this variation in two of their key indicators.  On the left of Figure~\ref{fig:coalition_govern} is the effective number of coalitions that is calculated by first determining the proportion of all months that each coalition was in power; these squared proportions are summed; and we then take the inverse of the sum.  This is meant to tap the extent to which there is variance in coalition make-up over time. The number proposed here increases with the frequency of coalition re-shuffles, either due to elections or intra-election cycle bargaining.  Denmark scores a very high 6 on this measure; while Germany is much lower with 2.7; and the U.K. one of the lowest countries with a score of 2.

 
\begin{figure}[p]
\caption{Effective Number of Governing Coalitions and Governing Coalition Parties}\label{fig:coalition_govern}
\centerline{\subfigure{\includegraphics[width=.60\textwidth]{C_c.pdf}}
\subfigure{\includegraphics[width=.60\textwidth]{C_p.pdf}}}
\end{figure}

\clearpage

\par On the right of Figure~\ref{fig:coalition_govern} is the effective number of coalition parties which is calculated by finding the proportion of party-government months that each party was in government; squaring these proportions; summing; and then taking the inverse of the result \citep{ArmstrongDuch2010}.  Rising values of this measure indicate that an increasingly large number of parties historically were likely candidates for entering the governing coalition after the election -- hence suggesting complex coalition formation and governance in the country.  Again, Denmark scores a very high 5.5 on this indicator; Germany scores a lower 3.7; while the U.K. ranks at the bottom with a score of 2. Both the indicators of coalition governance complexity generate the same rank ordering of our three countries: Denmark is highly complex; Germany is moderately complex and the U.K. is not at all complex.



%\par The general model of coalition-directed voting presumes that voters are informed about the composition of governing coalitions, can anticipate coalition bargaining outcomes after an election, and are knowledgeable about the dynamics of coalition governance.  But as we demonstrate above the extent to which such information is of utility to voters will vary quite significantly across national contexts.  Our presumption here is that the information context is commensurate with the coalition reasoning that is expected of the average voter.

%  And our expectation is that the availability of such information will be very context dependent And while this might seem like an exaggerated claim, there are a number of reasons why we might expect voters in fact to display such levels of skill and knowledge.  Most importantly, in contexts in which coalition governments are the norm information about the coalition process is of value to the average voter -- it helps them make informed decisions.

% Our empirical tests will be based on measures of coalition reasoning for three countries: Denmark, Germany and the United Kingdom.  The extent of information about coalition governance varies quite significantly across these three countries: it is very high in Denmark; very low in the United Kingdom; and moderately high in Germany.  The source of this contextual variation is the very different experiences each country has with coalition formation and governance.   

%\par And a number of important factors facilitate voters gathering this information about coalition formation patterns.  One is the relative stability of coalition configurations that typically form in any single country and the fact that these coalitions are not particularly complex in terms of numbers of parties. Most Dutch voters know which parties make up the ``rainbow" coalition and are cognisant that this is the coalition that frequently forms after an election.  \citet{ArmstrongDuch2010} document this stability in their analysis of coalition formation patterns in 30 countries over the period 1960 to the present.  They find that the effective number of parties in a typical coalition government is approximately 3.5 and that the exact same coalitions are returned to power with relatively high frequency.  Hence, the history of coalition formation patterns can be very informative to voters' efforts to anticipate post-election coalition formation outcomes.

%\par The information context contributes to the utility calculation in Equation \ref{eq:20} by providing voters with factual information about the relative electoral strengths of competing parties.  For example, publicly available polling results inform voters about the relative electoral strength of competing parties.  The assumption that public opinion polls are a coordinating device that informs voting behaviour has a rich theoretical foundation \citep{Cox1997,Fey1997}. It also has received convincing support from experimental evidence \citep{Forsytheetal1993,Forsytheetal1996} and from observational data \citep{Cox1997}. In contexts with multi-party governing coalitions, opinion polls also signal the likelihood of different coalitions forming and hence shape the nature of the coalition-directed vote \citep{Bowleretal2010,meffertgschwend2007b}.

%  We will briefly present evidence about the factual knowledge that voters have regarding the relative electoral strengths of different political parties.  But this will not be the central focus of our effort to characterise the coalition reasoning abilities of the voters.

%\citet{Bowleretal2010} present evidence that New Zealand voters condition their vote on the electoral prospects, as reflected in public opinion polls, of different coalition formations.\footnote{Although in their experimental results \citet{meffertgschwend2007b} find that polling information had a weak impact on coalition-directed voting.}  

%\par A second feature of the information context, that is the primary focus of this essay, concerns more general information that educates the voter about the nature of coalition formation and governance.  It is this information that primarily contributes to the development of coalition reasoning heuristics.

%\par  Election campaigns, and in particular the explicit communication efforts by the competing parties, provide voters with information about coalition formation likelihoods ($\gamma_{c_{j}}$).  In some cases the signals are very explicit -- this is the case with pre-electoral coalitions by which parties make explicit commitments, prior to the election, to form a governing coalition \citep{GolderSona2006}.  And parties can also signal to voters that they will not enter into coalitions with particular parties (an ``anti-pact'').  For example, in the recent German Federal elections, the FDP specifically ruled out a ``traffic light" coalition consisting of the SPD, FDP and Greens.  And there is evidence that these ``coalition" cues inform vote choice.  An example is the \citet{meffertgschwend2007b} experiment that documents the strong impact that party cues can have on coalition-directed voting.

\par These results suggest that the full information requirements associated with the exercise of a coalition-directed vote vary quite significantly across contexts.  Our expectation is that the extent, or richness, of the information informing coalition reasoning will be commensurate with the requirements associated with exercising a rational coalition-directed vote.  There will be some contexts in which bargaining to form a coalition is a very important feature of the political landscape -- Norway and Belgium, for example.  Accordingly, voters in these contexts will be heavily exposed to information about coalition bargaining -- this will form an important part of the party rhetoric but also it will make up a significant part of the media coverage of election campaigns. \citet{Strombacketal2008} compare the content of election coverage in Norway and Sweden -- the former experiences multiparty coalitions with considerable bargaining while in the case of Sweden the Social Democrats have dominated government formation.  Their analysis of media coverage of election campaigns suggests quite significant differences in how the campaigns are framed -- in Norway what they call the governing frame, i.e., discussions of the coalition formation process, is the dominant frame while it is much less important in Sweden.  The point here is that in contexts in which coalition bargaining is quite important the electorate will be exposed to considerable information about the process.
%  Hence we expect more well-developed coalition reasoning in contexts in which coalition formation is a salient part of the political conversation.

% The next section presents our efforts to empirical test these arguments.


%[CONTINUE WITH STUFF FROM ONLINE SEARCH -- SHANTO'S E-MAIL]


%\subsection{Empirical implications of context and coalition-directed vote}

%\par Theories of coalition-directed voting suggest that voters acquire specific facts or knowledge about coalition politics -- for example, who the parties are and their relative electoral strengths.  Also voters can develop rather sophisticated coalition reasoning heuristics; although not all will -- for some voters these abilities are useful and in others they are not of particular importance for making a vote decision.  Again, context matters -- so we should be able to show that in some contexts knowledge is high and coalition reasoning is quite sophisticated while in others it is actually quite rudimentary.

%\par We expect that there will be political contexts in which the nature of coalition formation and governance demands that voters develop levels of coalition reasoning in order to exercise an informed vote choice. In these contexts, media and political discourse will devote considerable attention to informing the public about coalition formation processes and politics.  In contexts in which information about coalition formation or coalition governance is quite dense the sophistication of coalition reasoning heuristics should not be strongly correlated with education -- these heuristics are easily absorbed by voters from all walks of life.  On the other hand in contexts in which these coalition reasoning skills are not particularly useful, such information will be costly to obtain, and hence we should see a strong correlation between these skills and education.  The next section presents our efforts to empirical test these arguments.

%\par Accordingly our empirical tests consists of the following: 1) establishing that in fact there is contextual variation in information about coalition governance -- high in some contexts but low in others; 2) we measure the coalition reasoning heuristics that voters employ; 3) we can demonstrate that the sophistication of these heuristics vary according the density of information about coalition governance; and 4) variation in coalition reasoning can have implications for political behaviour -- sophisticates behave differently than the less sophisticated.


%\section{Contextual variation in information about coalition governance} 

%\par Here we present evidence from content analysis of party platforms or from content analysis of the media -- possibly restricted to periods during which there is an election campaign.


%\section{Results}

%\par We begin by describing the data...  Then present results for knowledge.  Then present results for coalition reasoning..

%\subsection{Data}

%\par The primary focus of the empirical analysis is comparing coalition reasoning heuristics across three countries: Britain, Denmark and Germany.  The British data are from a six-wave panel....   A similar four-wave panel conducted during the German 2009 election campaign is the basis for the German results..... And the Danish data are from the iLEE survey conducted in July 2011... 

%\par It is true that voters, in contexts where there is a history of coalition governments, have access to considerable information about the coalition formation process.  Nevertheless, we have little notion of what aspects of the coalition formation process actually informs vote choice; nor do we understand the extent to which this knowledge varies within the voting population.  Our intuition here is that voters employ various heuristics when reasoning about the coalition formation process -- and these will be more or less consistent with rational choice models of voting behaviour.  Moreover these heuristics will, I anticipate, exhibit considerable heterogeneity in the population.  This expectation is in line with other findings regarding the micro-foundations of formal vote choice models.  \citet{Esponda2011}, for example, examine the heuristics employed by subjects in a laboratory strategic voting game.  They find that only about one-third of the subjects, after repeated play of a strategic voting game, exhibit an understanding of the notion of pivotality and condition their voting behaviour on the pivotal event.  The less sophisticated subjects employ other heuristics including conforming behaviour, i.e., voting for the alternative that is voted with highest probability by the computers in their experiment; or simply using only their private information and voting sincerely (ignoring information about pivotality).  Their point, which we share, is that the utility of these models of vote choice is dependent on an understanding of the precise heuristics employed and their heterogeneity in the population.     



%\subsection{Knowledge}


%\par This section is concerned with assessing voters' knowledge about coalitions (either existing or anticipated) and identifying the heuristics that voters employ for anticipating the types of coalition governments that form after an election -- an important assumption that underpins the theories of the coalition-directed vote outlined earlier. I begin by simply describing the levels of knowledge that German voters have regarding coalition government.  This is then followed by an attempt to measure their coalition reasoning skills. Thirdly, I explore explanations for the patterns of heterogeneity in these reasoning skills.  A subsequent section will examine whether reasoning skills help predict voting behaviour.


\par Accordingly, in complex coalition contexts we expect higher overall levels of political knowledge -- voters both require more knowledge to make sense of the political choices available to them and are exposed to more political information.  We explore this with data from the 2010 European Elections Survey which asks a battery of political knowledge questions in 26 countries. The results for the EES knowledge battery are summarised in the left hand graph in Figure~\ref{fig:know_context}.  The ranking of our sample of three countries correlates nicely with the ranking of the coalition complexity summarised in Figure~\ref{fig:coalition_govern} -- Denmark has high levels of political knowledge; the U.K. has low levels; and Germany ranks in the middle.  


\begin{figure}[p]
\caption{Contextual Variation in Political Knowledge: European Election Survey 2010}\label{fig:know_context}
\centerline{\subfigure{\includegraphics[width=.60\textwidth]{polKnow_EES.png}}
\subfigure{\includegraphics[width=.60\textwidth]{partyKnow_CSES.png}}}
\end{figure}

\clearpage

\par The graph on the right side of Figure~\ref{fig:coalition_govern} creates a measure of ability to place political parties on the left-right ideological continuum.\footnote{The measure is constructed by coding for each party asked whether the person knew where to place the party (for each party placed the respondent gets a 1). If they answered ``do not know'' to a party placement question they receive a 0. Refusals are coded as missing.  For each respondent we calculated the total number of parties they were able to place by summing over all party scores for respondent.  For each country we identified the maximum number of parties individuals were able to place.  For each respondent we then estimated the percentage of parties classified. And finally for each country we estimated the mean of the individual percentages.}  The ranking of the three countries in our sample is consistent with our argument.  The Danish 2001 survey average for the ability to place political parties is very high; the British 2005 average ranks very low; and the German 2002 ranks between these two cases.  The anomaly here is the 2005 German study which ranks higher than the 2001 Danish survey average.  At the very least the results are consistent with a distinction between coalition rich contexts -- Germany and Denmark -- having high placement scores and a very low ranking for Britain where there has been no real coalition politics until very recently.

%\par In the next section we will explore our assertion that the complexity of coalition governance and that its representation in the media will shape national levels of coalition reasoning.   But before we progress to this stage of the analysis we explore one final assessment of the extent to which the information contexts of these national contexts vary. The coalition-directed theory we summarised earlier assumes voters prior to an election are informed about the parties that make up the governing coalition; their relative shares of administrative responsibility; and about the parties that are likely to form a governing coalition after the election.  



\par The 2011 Danish and and 2009 German surveys, that we will be discussing in more detail below, included knowledge questions regarding the composition of the governing coalition.  The 2009 German survey included questions regarding the composition of both the out-going and, in the post-election study, the in-coming coalition.\footnote{The exact wording of the German pre-election question is the following: ``To the best of your knowledge, which of the following parties are currently in the federal government of Germany?"; The 2011 Danish pre-election question was worded as follows:``Which, if any, of the following parties do you believe are currently represented in the government? (Please tick all that apply)''}  Most respondents named the CDU (90 percent) and the SPD (80 percent) as members of the out-going Grand Coalition, although a surprisingly small, 60 percent, named the CSU.  And 90 percent of respondents named the FDP and the CDU as members of the new post-election governing coalition.  German voters are clearly knowledgeable about the basic features of their coalition government. The Danes exhibited an even higher level of coalition knowledge -- ninety-five percent or more of the sample were able to identify the Liberal and Conservative People's Parties as making up the incumbent coalition.  

%The graph on the left of Figure~\ref{fig:know_coalition} summarises the responses.  
%  A simple sum of the correct responses to this coalition knowledge question is used to construct a measure of coalition sophistication that will be used in subsequent analyses.


%\begin{figure}[p]
%\caption{Contextual Variation in Coalition Knowledge: Germany 2009; Denmark 2011}\label{fig:know_coalition}
%^\centerline{\subfigure{\includegraphics[width=.60\textwidth]{knowledge_germany.pdf}}
%\subfigure{\includegraphics[width=.60\textwidth]{knowledge_denmark}}}
%\end{figure}

%\clearpage

\par The data in this section provide strong evidence that overall levels of knowledge are conditioned on the complexity of the coalition context -- on most indicators Denmark has the highest levels of political knowledge; Germany tends to fall in the middle; and Britain is consistently at the bottom.  Our expectation is that the sophistication of coalition reasoning heuristics will similarly reflect the contextual complexity of coalition formation and governance.  We now turn to these data.
 

%\begin{figure}[p]
%\caption{Knowledge of out-going and in-coming German Coalitions}\label{fig:know_germany}

%\centerline{\includegraphics[width=.75\textwidth]{knowledge_germany.pdf}}

%\end{figure}


%\begin{figure}[p]
%\caption{Knowledge of governing Denmark Coalitions}\label{fig:know_denmark}
%\centerline{\includegraphics[width=.75\textwidth]{knowledge_denmark}}
%\end{figure}


%\clearpage

%\par In models of coalition-directed voting \citep{Duchetal2010,DuchStevenson2008} voters are assumed to have preferences over distributions of administrative responsibility - for example, it matters to voters what percentage of the cabinet portfolios are controlled by the FDP.  And in some of these models voters are also expected to know whether their vote would be pivotal in electing (or defeating) a particular distribution of administrative responsibility. Hence voters are expected to anticipate the likelihood of different coalitions forming after an election. We can explore whether this is the case with data from the January, August, and September waves of the German DCCAP survey.  In these pre-election waves, respondents were asked to indicate which parties would likely form the post-election governing coalition.\footnote{ The exact wording of the question asked on Waves 1 through 3 is the following: ``Which parties will most likely to enter the governing coalition after the federal election?"} 

%\par Results for the three waves are summarized in Figure~\ref{fig:anticipate_germany}.  In the pre-election September wave of the panel almost 80 percent of the German sample expected the CDU/CSU to be one of the parties in the governing coalition.  And about 50 percent of the sample expected FDP to be one of the governing parties.  On balance these percentages suggest a coalition-informed German electorate;  most understood that the CDU/CSU would be the major party in the coalition that formed after the election and a majority of the electorate expected the FDP to be the junior partner in the coalition.  

%\begin{figure}[p]
%\caption{Correctly Anticipate in-coming German Coalitions}\label{fig:anticipate_germany}

%\centerline{\includegraphics[width=\textwidth]{anticipate.pdf}}

%\end{figure}


%\clearpage




\subsection{Measuring coalition reasoning heuristics with experimental vignettes}

%\par How do voters conclude that a particular coalition of parties is more-or-less feasible?  Is it that voters simply mimic information they obtain from the media and reports of public opinion polls?  Or are voters engaging in reasonably sophisticated reasoning about what parties are feasible coalition partners?  Here we are not referring to factual information, such as which parties made up the incumbent coalition.  Understanding the general dynamics of coalition formation suggests the ability to reason in general terms about how and which coalition governments get formed.  Surprisingly we have little notion of precisely what heuristics voters employ in deciding which coalitions are likely to form.
   
% I focus  Given the importance this has for the coalition-directed vote assumption for the coalition directed models described earlier, is critical for assessing how their vote affects coalition outcomes.  

%\par Recall that a key feature of the coalition-directed utility functions described earlier is that voters are assumed to understand the general dynamics of the coalition formation process.  What precisely does this mean?    

\par  Our expectation is that the heuristics adopted by voters will generally (or ``in expectation'' if you will) result in correct predictions regarding post-election coalition formation.  This will be particularly the case in contexts where this information is important for making rational voting decisions. Theory and the empirical analysis of coalition formation patterns provide a good guide to the heuristics that will, in expectation, result in correct predictions \citep{Martinetal2001,Riker1962}.  We have identified five dimensions that, while not exhaustive, capture aspects of coalition formation that the literature suggests should weigh heavily in the voter's calculations.

\par  First, at the very minimum, a coalition-directed vote requires that voters understand what constitutes a majority governing coalition -- either in terms of a single- or multi-party coalition.  If voters have a difficult time grasping this basic concept then they cannot exercise a coalition-directed vote.  A second dimension of coalitions that likely matters is whether coalitions are over-sized, connected or minimum winning \citep{Martinetal2001,Riker1962}.  The empirical and theoretical literature suggest that a minimum winning heuristic would generally contribute to predicting, correctly, coalition outcomes and hence is a useful voter heuristic.  A third dimension concerns the formateur advantage that is typically accorded the party with the largest number of seats.  Formateur status should increase a party's likelihood of entering a governing coalition \citep{Ansolabehereetal2005}.  A fourth dimension is the left-right spatial representations of electoral competition.  Ideological proximity has been demonstrated to condition the likelihood of parties agreeing to form a coalition.  Accordingly we expect voters to adopt ideological proximity as a heuristic that signals the increasing likelihood of coalition agreement amongst parties.  Finally, our expectation is that voters will recognise that parties from the incumbent coalition have an advantage in post-election coalition formation negotiations.


%\subsection{Description of the Experiment}

%\par This experiment is motivated by a set of results I report in a recent essay using a large number of public opinion surveys from 30 countries and over a 25 year time period \citep{Duchetal2010}.  The essential argument is that voters in certain types of political contexts are able to anticipate the kinds of coalition governments that form after an election and that they use this information, strategically, when they decide which party to vote for in an election.

\par Hence there are five dimensions of coalition reasoning that we attempt to recover with experimental vingettes embedded in internet surveys conducted during the 2008-10 British, 2011 Danish and 2009 German internet surveys: 1) the majority governing status of a coalition; 2) minimum winning coalitions; 3) formateur advantage; 4) ideological proximity; and 5) incumbent advantage.  The experimental treatments described below are designed to determine the extent to which these heuristics are in fact prevalent in the voting population.

%1) voters understand the basic arithmetic of coalition formation -- for example, they understand that two parties each with 30 percent of the legislative seats have sufficient voting strength in the legislature to form a viable coalition government; 2) voters anticipate the formation of minimum winning coalitions -- they think parties prefer coalitions that minimise the number of parties necessary to command a majority in the legislature; 3) voters recognise the fact that formateur status increases a party's likelihood of entering the governing coalition; 4) voters reason in terms of the left-right ideological continuum so they anticipate parties that are proximate on the ideological scale will enter a coalition; 5) voters expect an incumbent advantage in coalition formation.  The treatments described below are designed to determine the extent to which these heuristics are in fact prevalent in the voting population.
%  I expect the political context to play an important role in Germany where there is a history of coalition government and hence where voters are likely to be relatively informed about coalition formation patterns and conform to our expectations regarding the coalition-directed vote. 

%  The BCCAP is a six-wave panel internet study of 10,000 respondents.  The waves are being conducted by Polimetrix YouGov in December, 2008; May and October 2009; and January, May and June of 2010.  
%\par Every attempt was being made to retain these respondents throughout all the waves of the panel.  As with all the cooperative studies, a portion of all the interviews conducted after the baseline wave were reserved for unique data collection designed by participating teams. Every group that participated received data on 1000 respondents interviewed in the baseline, the post-election wave, and at least one campaign wave of the survey. The team data are not contained in the Common Content data release.  The surveys were conducted online and were approximately 20 minutes long. The baseline survey was shorter (about 15 minutes) and contained only Common Content. The first 10 minutes of subsequent waves were common to all respondents (Common Content ? approximately 40 questions) and the last 10 minutes were reserved for participating institutions (Team Content ? approximately 40 questions).

%\par Included in Wave 4 of the BCCAP internet panel survey (Fall 2009) and Wave 2 of the DECCAP survey is an experiment to recover measures of coalition reasoning.  The experiments were very similar.  The British experiments are incentive compatible -- respondents were rewarded if they did well in each of the treatments.  Accordingly, I rewarded respondents who make choices that are consistent with the majority of respondents with bonus YouGov points.  This was not feasible in the German experiments.

\par The experimental vignettes were administered in three online surveys.  The British data is from the 2008-2010 British Cooperative Campaign Project.  This project is a six-wave panel internet study of 10,000 respondents. The survey waves were conducted by Polimetrix YouGov in December, 2008; May and October 2009; and January, May and June of 2010.\footnote{Every attempt is being made to retain these respondents throughout all the waves of the panel.  As with all the cooperative studies, a portion of all the interviews conducted after the baseline wave were reserved for unique data collection designed by participating teams. The surveys were conducted online and were approximately 20 minutes long. The baseline survey was shorter (about 15 minutes) and contained only Common Content.}  The German CCAP study consisted of a baseline survey in June of 2009. Subsequent panel waves took place three months after the first wave (August 2009) before the State elections in Saxony, Saarland, and Thuringia. A third wave was carried a month later (September 2009) before the Federal election of September 27th. And a final post-election wave conducted on October of 2009. A total of 4,301 respondents were interviewed in the 2009 baseline, with an extra 1,904 which entered new in the second wave for a total of 6,205 respondents.  The Danish vignette experiment was run in summer 2011 as part of iLEE4, the fourth wave of the Internet Laboratory for Experimental Economics project at the University of Copenhagen.  iLEE4 is an omnibus experiment (consisting of several independent modules) carried out over the internet with participants drawn from the adult Danish population in collaboration with Denmark Statistics.  Of the 2,291 invited to participate in the survey, 942 logged into the survey and 689 completed the whole survey.  
  
\subsection{Experimental Results}

%\par The experiment consists of four sets of coalition vignettes that describe hypothetical election results for a set of parties -- the results are expressed in seats allocated to each party.  Respondents are presented with a graph that summarises the seat allocations -- in percentages -- for each of the parties.  In addition, the graphs indicate the location of the parties on a left-right ideological continuum.  The respondents are then asked to indicate what majority governing coalition is most likely to form as a result of this election outcome. 


%\subsection{Recovering Coalition Reasoning Heuristics with Experimental Vignettes}

%\par     The two contexts differ though in that the UK historically has virtually no history of coalition governments.  Hence faced with the task of identifying a majority government in situations in which this requires assembling parties that constitute a majority coalition, my expectations are that British respondents would fare quite poorly compared to the Germans.

\par Respondents are exposed to eight hypothetical coalition vignettes in total and in each case are asked to identify the most likely majority governing coalition that would form.  Respondents are presented with a graph that summarises the seat allocations -- in percentages -- for each of the parties.  The question specifically asks respondents to identify the most likely \emph{majority} governing coalition in each case.  In the Danish experiments we added two modifications to the vignettes.  First, we introduced an incumbent cue treatment.  Half of the Danish subjects were randomly assigned to a treatment in which one of the parties was identified as an incumbent from the pre-election coalition government.  The other half received no incumbent cue.  This allows us to assess whether incumbent cues represent an important part of the heuristics voters employ in predicting coalition formation outcomes.  The incumbent cue treatment results in different  coalition choices in only one case: Treatment 3A and 3B which is described in Figure~\ref{fig:treat3A} and  Figure~\ref{fig:treat3B}.  A second treatment varied the left-right spatial location of parties for scenarios that clustered parties on the left or right.  In these cases, half of the subjects see a clustering to the left for a given number of parties, the other half see a clustering to the right.  We introduced this treatment in order to assess whether the left versus right clustering of parties might have affected the parties subjects expected to form a coalition.  There was no difference in the coalition choices of subjects in the left versus right clustering treatments.  A detailed description of the interviewer instructions is provided in the Appendix.  


\par Figure \ref{fig:majority} summarises, for each of the eight vignettes in each of the three countries, the percentage of respondents that in fact select a group of parties that control at least 51 percent of the legislature's seats.  In Treatment 1A and 1B, where there are only two parties, the sampled voters in all three countries perform quite well although only about 70 percent of British respondents select the majority party in Treatment 1B.  


\begin{figure}[p]
\caption{Frequency with which Majority Governments are Selected}\label{fig:majority}
\centerline{\subfigure[UK]{\includegraphics[width=.30\textwidth]{UK_majority.png}}
\subfigure[Germany]{\includegraphics[width=.30\textwidth]{DE_majority.png}}
\subfigure[Denmark]{\includegraphics[width=.30\textwidth]{DK_majority.png}}}
\end{figure}

\clearpage

%^\par In treatments 2A and 2B, any combination of two or three parties will constitute a majority government.  Nevertheless, only about 20 percent of the British are able to correctly perform this task while 80 percent of the Germans succeed. This gaps narrows slightly in the case of treatment 2B but it remains over 3 to 1 in favor of the Germans.

\par Subsequent treatments add more complexity to the choice.  In Treatments 2A and 2B, any combination of two or three parties will constitute a majority government.  Again, in Treatment 3A and 3B, no single party can form a majority government but now there are combinations of parties that would not have sufficient numbers of seats to form a majority governing coalition.  The final Treatment 4A and 4B increments the complexity of the choice by reducing the combination of parties that could form a majority governing coalition. 

% At Treatments 2A and 2B, we see a significant differentiation of the voters in the different countries: for all of the treatments from 2A and 2B on only approximately 20 percent of the British respondents are able to identify a majority coalition; as the treatments get more complex the German respondents drop from about 80 percent to 50 percent being able to identify a majority coalition; and consistently high percentages of the Danish respondents -- in the neighbourhood of 80 percent (with the exception of Treatment 3B) are able to identify majority coalitions.

\par  The relative sophistication of coalition reasoning heuristics in the three sampled populations conform exactly to our expectations:  Note that there are effectively three different patterns of responses to the increasing complexity of the coalition treatments.  The British respondents perform very poorly.  For the simplest of choices -- a single party with a majority of seats -- the British are quite similar to respondents in other countries with about 80 percent selecting the majority party as likely to form the government.  For the three other more complex coalition formation possibilities the British perform poorly with only 20 percent of the respondents able to identify majority coalitions.  The German results suggest a gradual reduction in the ability to identify majority coalitions as the complexity of the choices increases.  In the simplest of treatments, three-quarters of the sample identified a majority coalition while in the more complex treatments with more parties the frequency drops to about 50 percent.  The pattern of results for Denmark are indicative of a context in which voters have high levels of coalition reasoning.  Across all treatments, with one exception, at least 80 percent of the respondents were able to identify a majority coalition.  The three quite distinct patterns of responses to these experimental treatments suggest that national electorates have very distinct coalition reasoning skills.  Moreover the contextual levels of sophistication correspond exactly to our expectations in that they get increasingly sophisticated as the complexity of coalition formation and governance rises.  

\par  Certainly the inability to select a majority governing coalition in the simplest of treatments is the result of low levels of sophistication -- hence about a quarter of the voters in all three countries are not able to reason about the coalition formation process.  It is difficult to say why respondents in the more complex treatments do not select a majority governing coalition.  In the case of some respondents this might reflect a heuristic plausibly consistent with sophistication; possibly the confusion of minority with majority governing coalitions.  For others it may simply indicate that they are unable to cope with more complex coalition formation permutations as the numbers of parties rise.   In the within-subject analyses we will attempt to identify these types with somewhat more precision.  But first we examine each of the scenario results for Denmark and Germany in descriptive detail -- we do not analyse in any detail the British responses to Treatments 2A through 4B because such a small percentage of the voting population exhibit coalition reasoning skills.

%Most rational choice theories of vote choice in coalition government contexts assume that voters are knowledgeable about the types of coalitions that are likely to form after an election. At a very minimum this presumes that voters understand what constitutes a majority governing coalition.  The coalition reasoning experiments were designed to recover this basic level of comprehension in the German voting populations.  Encouragingly, with respect to the very basic reasoning ability -- recognizing which of two competing parties would form a majority government -- German voters exhibit high levels of ability.  By varying the difficulty of identifying majority governing coalitions the experimental treatments allow us to identify what I characterize as coalition sophisticates. In Germany approximately half the population are candidates for being characterized as coalition sophisticate, although this estimate gets refined as we analyze other aspects of the coalition vignette treatments.


% I expected considerable contextual variation in this reasoning ability:  For German voters this is a reasoning capability that is extremely useful for making vote decisions in the German context and its one that receives considerable attention in the German media.  And in the British context, this feature of coalition reasoning has been of little utility for exercising a vote in British Parliamentary elections.

% Inaki: Bar chart for each treatment that compares percent in UK and Germany selecting a majority government -- so each treatment would have a bar for UK and Germany.


%\subsection{Minimum Winning Coalition Reasoning}

%\par There is, of course, an extensive theoretical and empirical literature on why we might expect, or not, to expect to see minimum winning coalitions.  The theoretical results typically are founded on assumptions regarding utility maximizing political political parties and their leaders.  But, are voters predisposed to minimum winning coalitions?  I address this in these experiments.

%\par In the treatments respondents are simply asked to identify the coalitions most likely to occur.  For each  Figure~\ref{fig:minwin} presents the percentage of respondents selecting a particular combination of governing parties -- only respondents correctly identifying a majority party or parties are included in this analysis. In treatment 2A both British and German respondents select one of the connected winning coalitions -- very few select the grand coalition or the unconnected coalitions.  For treatment 2B again the overwhelming majority select a connected coalition, although there is a clear preference, particular amongst the British, for the ideologically proximate BC coalition.


%\begin{figure}[p]
%\caption{The Types of Coalition Governments Selected}\label{fig:minwin}

%\centerline{\includegraphics[width=\textwidth]{minCoal.png}}

%\end{figure}

%\clearpage


%\par This expectation that minimum winning coalitions will form is suggested by Treatment 3A where most British respondents and a majority of German respondents indicate that an AB coalition will form.  Nevertheless, a large number of German respondents predict a BD coalition which is not the minimum winning coalition.  And in Treatment 3B where there is a trade off between minimum winning and ideological proximity, the majority of both country respondents select the unconnected BD coalition.

%\par Treatment 4A and 4B provide interesting insight into how voters anticipate parties will trade off between minimizing the number of parties in the governing coalition and minimizing the number of seats necessary to command a majority in parliament.  In treatment 4A, in which each party is equally distant from its neighboring party on the ideological continuum, most respondents select the CD coalition which minimizes the number of parties but is slightly more costly in terms of legislative seats than ABC -- about 60 percent of respondents in both countries select this CD coalition.  But in Treatment 4B, where respondents receive a distinct spatial cue, respondents shift to preferring the ABC coalition that is the most minimum winning in terms of seats.

%\subsection{Minimal winning coalitions and ideological spatial reasoning}

%\par The experimental vignettes were designed to recover two other dimensions of coalition reasoning: the extent to which respondents favoured minimum winning coalitions and ideological proximate parties in predicting the likely coalitions that would form.  These two dimensions are not currently incorporated in the measure of coalition reasoning that will be employed in the subsequent multi-variate analyses.  Nevertheless, the raw results provide interesting insights into the coalition reasoning of the two electorates and ultimately these dimensions will be added to the measure.

%\par Riker's very early intuition that political parties will form minimum winning coalitions has, over the past five decades, received strong empirical confirmation \citep{Riker1962}.  If, as many theories suggest, voters anticipate the kinds of coalitions that form after an election, voters should anticipate that in general parties will form minimal winning coalitions. The experiments on coalition reasoning summarized in this section clearly support this notion that voters quite overwhelmingly anticipate the formation of minimum winning coalitions.  Interestingly, amongst what would characterize as respondents with a minimum level of coalition reasoning (i.e., they can identify what constitutes a majority government), there is not much difference between the Germans and British in this regard -- both quite overwhelmingly anticipate parties will form minimum winning coalitions.

% Inaki: of those who chose a majority government [drop from consideration those who did not select parties with a majority of seats] Bar chart for each treatment that compares percent in UK and Germany selecting a minimum winning coalition -- so each treatment would have a bar for UK and Germany.


%\subsection{Ideological Spatial Reasoning}

%\par Earlier I pointed out that spatial models of vote choice in coalition government context presume that voters are knowledgeable about how Left-Right ideological considerations condition the type of coalitions that form. Hence voters should anticipate how ideological proximity will affect the composition of coalitions.  The experimental treatments developed for the CCAP project are designed to recover the extent of this spatial reasoning in the British and German voting populations.  In a number of cases respondents see parties with similar seat allocations but located differently on the Left-Right continuum.  Of interest here is whether shifts in these ideological locations affect the coalitions that respondents anticipate will form.

%\par The treatments are designed to calibrate whether respondents favour minimal winning coalitions and whether they anticipate that ideological proximity affects the composition of coalitions.


\par As we pointed out earlier, the coalition scenarios in each of the vignettes were designed to recover a number of dimensions of respondents' reasoning about the coalition formation process.  First we provide a description of the scenarios and of the basic choices made by respondents. We will then provide a set of generalisations about the heuristics respondents appeared to be employing in identifying likely majority governing coalitions.


\par Figure~\ref{fig:treat2A} and Figure~\ref{fig:treat2B} presents the first treatments.  In Treatment 2A and 2B the parties have the same seat allocations -- A has 45 seats; B has 16 seats; and C has 39 seats.  In Treatment 2A the parties are equidistant and straddle the ideological centre.  The respondents' choices are recorded below the figures.  Interestingly, not only do most German and Danish respondents recognise that a coalition is necessary to form a majority government, but they favour the AB majority coalition over BC or ABC. Just under 50 percent of Danish respondents and almost 40 percent of German respondents expect that the party with the largest numbers of seats in the legislature would enter into coalition with B.  The coalition BC was the second-most favoured choice by respondents in both samples.  One interpretation of these results is that voters in both these countries, by favouring AB, recognise formateur advantage in coalition formation.  Moreover these is some evidence that formateur advantage, i.e., AB, trumps a minimum winning heuristics since BC would be the choice associated with such an heuristic.

\par Treatment 2B shifts the parties B and C significantly to the Right extreme of the ideological continuum leaving Party A significantly to the Left of these parties.  Responses to the increase in ideological distance between A and B are as we might expect.  The respondents are now significantly more likely to choose BC (which does not include the largest party in terms of seats) than AB.  Again, one interpretation here is that respondents recognise that the formateur advantage can be out-weighed by party considerations of ideological proximity -- although the ideological treatment in this case is quite extreme.  And of course the BC choice in Treatment 2B has the advantage of being majority, ideological proximate and minimum winning.  There is though a significant difference between the Danes and Germans here.  Over seventy percent of the Danes opt for this choice while only forty-two percent of the Germans do so.  The difference may lie in the greater importance associated with formateur advantage in the German institutional context than in the Danish one. 

%Figure~\ref{fig:treat3A}-Figure~\ref{fig:treat4B} replicate similar spatial treatments and the results are essentially consistent with those in Figure~\ref{fig:treat2A}-Figure~\ref{fig:treat2B}.  My contention here is that respondents who respond to these spatial treatments in a fashion that is consistent with spatial theories of coalition formation exhibit high levels of coalition reasoning.


\clearpage

%\begin{figure}[h]
%\caption{Treatment 2}\label{fig:treat2}
%\centerline{\subfigure[Germany]{\includegraphics[width=.45\textwidth]{DE_majCoal_2.png}}\\
%\subfigure[Denmark]{\includegraphics[width=.45\textwidth]{DK_majCoal_2.png}}}
%\end{figure}

\begin{figure}[h]
\caption{Treatment 2A}\label{fig:treat2A}
\centerline{\includegraphics[width=.5\textwidth]{treat2A.png}}
\end{figure}


\begin{figure}[h]
\caption{Treatment 2B}\label{fig:treat2B}
\centerline{\includegraphics[width=.5\textwidth]{treat2B.png}}
\end{figure}

%\begin{table}[h!]
%%\caption{Results for Treatment 2A}\label{tab:treat2A}
%\begin{center}
%\begin{footnotesize}
%\begin{tabular}{c |c c c c c c c | c}
%\textbf{} & \textbf{A} & \textbf{AB} & \textbf{ABC} & \textbf{AC} & \textbf{B} & \textbf{BC} & \textbf{C} & \textbf{\textit{N}} \\
%\hline\hline
%\textbf{2A} & 17.61 & 37.4 & 1.79 & 11.39 & 4.35 & 23.6 & 3.85 & \textit{4021} \\
%%\textbf{UK} & 54.42 & 10.32 & 0.38 & 3.03 & 6.77 & 8.85 & 16.23 & \textit{2113} \\
%%\hline
%%\end{tabular}
%%\end{footnotesize}
%%\end{center}
%%\end{table}
%%\end{samepage}
%
%
%%\begin{table}[h!]
%%%\caption{Results for Treatment 2B}\label{tab:treat2B}
%%\begin{center}
%%\begin{footnotesize}
%%\begin{tabular}{c |c c c c c c c | c}
%%\textbf{} & \textbf{A} & \textbf{AB} & \textbf{ABC} & \textbf{AC} & \textbf{B} & \textbf{BC} & \textbf{C} & \textbf{\textit{N}} \\
%%\hline\hline
%\textbf{2B} & 19.12 & 20.34 & 1.37 & 9.05 & 3.93 & 41.9 & 4.28 & \textit{4021} \\
%%\textbf{UK} & 41.4 & 2.7 & 0.24 & 1.94 & 7.06 & 22.6 & 24.06 & \textit{2111} \\
%\hline
%\end{tabular}
%\end{footnotesize}
%\end{center}
%\end{table}

\begin{table}[h!]
%\caption{Results for Treatment 2A}\label{tab:treat2A}
\begin{center}
\begin{footnotesize}
\begin{tabular}{c |c c c c c c c | c}
\textbf{} & \textbf{A} & \textbf{AB} & \textbf{ABC} & \textbf{AC} & \textbf{B} & \textbf{BC} & \textbf{C} & \textbf{\textit{N}} \\
\hline\hline
2A &&&&&&&&\\
\textbf{Germany} & 17.61 & 37.4 & 1.79 & 11.39 & 4.35 & 23.6 & 3.85 & \textit{4021} \\
\textbf{Denmark} & 9.51 & 46.54 & 1.44 & 2.74 & 3.6 & 31.99 & 4.18 & \textit{694} \\
\hline\hline
2B &&&&&&&&\\
\textbf{Germany} & 19.12 & 20.34 & 1.37 & 9.05 & 3.93 & 41.9 & 4.28 & \textit{4021} \\
\textbf{Denmark} & 5.76 & 10.66 & 0.29 & 2.45 & 2.31 & 72.91 & 5.62 & \textit{694} \\
\hline
\end{tabular}
\end{footnotesize}
\end{center}
\end{table}
\clearpage



\par In treatment 3A and 3B, parties A and B are the only two parties that can form a minimum winning connected coalition. The results are reported in Figure~\ref{fig:treat3A} and  Figure~\ref{fig:treat3B}.  In treatment 3A the two parties are ideologically equidistant and straddle the ideological centre.  Danish and German respondents are rather different in their choices here.  The Danes are overwhelmingly in agreement on the AB coalition which is both minimum winning connected and includes the party with the most seats -- 65 percent of Danish respondents make this choice.  Germans respond quite differently with only 25 percent selecting the AB minimum winning coalition. The modal response (27 percent) in effect is BC which is just shy of enough seats to form a majority governing coalition.  Somewhat surprisingly, 19 percent chose an unconnected coalition BD.  A common theme though is that most of the German choices include the largest party, B.  The Danes, much more than the Germans, are in agreement on a coalition outcome that is arguably the most likely one to occur. 

\par In treatment 3B, Party A is ideologically isolated on the extreme Left of the continuum while Parties B,C, and D are ideologically cluster on the extreme Right.  Assuming respondents recognise that ideologically distant parties are unlikely to enter a coalition, the percent selecting AB in Treatment 3B should fall dramatically.  This clearly happens in Denmark where the 65 percentage selecting AB in Treatment 3A drops to 4 percent in Treatment 3B.  It is particularly interesting that the effect of the widening of the ideological gulf between A and B increases significantly the extent to which individuals anticipate the inclusion of Party C in the coalition -- one explanation here is that voters anticipate that B (the formateur) will include ideologically proximate C in the oversized coalition in order to balance the policy effect of ideologically distant Party A.  The ideological treatment effect is less pronounced in Germany -- the 24 percent choosing AB in Treatment 3A drops to 10 percent in Treatment 3B; primarily to the benefit of BC which is just shy of constituting a majority governing coalition.  Like Treatment 3A, the results to Treatment 3B suggest little agreement in the voter population and possibly less sophisticated coalition reasoning than we find in the Danish context.  

\par Treatment 3A is the one treatment in which we find a significant difference between the treatments with and without an incumbent cue.  The results for the two treatments are presented in the DK-Incumbent and DK-Non-Incumbent rows in Figure~\ref{fig:treat3A} and  Figure~\ref{fig:treat3B}.  In the incumbent cue treatment Party A is designated as an incumbent party.  And the incumbent cue in Treatment 3A results in fewer subjects predicting that Party A would enter into the governing coalition.  This is not the case in Treatment 3B although it is likely that the incumbent cue here is overwhelmed by the ideological distance heuristic.  To the extent that incumbent cues matter we expected them to increase expectations that a party would enter into the governing coalition.  But in this one case we find that they reduce expected participation in a coalition government.  This is an isolated incumbent cue result -- it does not have an effect in all the other treatments -- and therefore we are not inclined to draw any general conclusions based on this one case. 
 
%\clearpage
%\begin{figure}[h]
%\caption{Treatment 3}\label{fig:treat3}
%\centerline{\subfigure[Germany]{\includegraphics[width=.45\textwidth]{DE_majCoal_3.png}}\\
%\subfigure[Denmark]{\includegraphics[width=.45\textwidth]{DK_majCoal_3.png}}}
%\end{figure}

\clearpage

\begin{figure}[h]
\caption{Treatment 3A}\label{fig:treat3A}
\centerline{\includegraphics[width=.5\textwidth]{treat3A.png}}
\end{figure}

\begin{figure}[h]
\caption{Treatment 3B}\label{fig:treat3B}
\centerline{\includegraphics[width=.5\textwidth]{treat3B.png}}
\end{figure}

\begin{table}[h!]
%\caption{Results for Treatment 3A}\label{tab:treat3A}
\begin{center}
\begin{footnotesize}
\begin{tabular}{c ||c c c c c c c c c | c}
\textbf{} & \textbf{AB} & \textbf{ABC} & \textbf{B} & \textbf{BC} & \textbf{BCD} &  \textbf{BD} & \textbf{C} & \textbf{CD} & \textbf{D} & \textbf{\textit{N}} \\
\hline\hline
Treat 3A &&&&&&&&&\\
\textbf{Germany} & 24.27 & 0.1 & 13.93 & 26.59 & 1.17& 18.9 & 1.64 & 4.4 & 2.64 & \textit{4021} \\
\textbf{Denmark} & 64.55 & 8.93 & 7.78 & 2.31 & 3.03 & 4.76 & 0.29 & 1.3 & 1.01 & \textit{694} \\
\hline
\textbf{DK - Non-Incumbent} & 69.74 & 9.22 & 6.05 &2.31 &1.44 & 4.03 & 0.29 & 1.15 & 0.58 & \textit{347} \\
\textbf{DK - Incumbent} & 59.36  & 8.65 & 9.51 & 2.31 & 4.61& 5.48 & 0.29 & 1.44 & 1.44 & \textit{347} \\
\hline\hline
Treat 3B &&&&&&&&&\\
\textbf{Germany} & 10.07 & 1.02 & 14.52 & 36.98 & 4.53 & 20.49 &1.59 &3.76 & 2.59 & \textit{4021} \\
\textbf{Denmark} & 4.18 & 31.7 & 0.86 & 7.64 & 8.93 & 3.31 & 5.91& 11.1 & 1.87 & \textit{694} \\
\hline
\textbf{DK - Non-Incumbent} & 4.03& 28.82 & 0.29 & 8.36 &10.95 &2.59 & 6.05 & 10.95 & 1.73& \textit{347} \\
\textbf{DK - Incumbent} & 4.32  & 34.58 & 1.44 & 6.92 & 6.92&4.03 & 5.76 & 11.24 & 2.02& \textit{347} \\
\hline\hline
\end{tabular}
\end{footnotesize}
\end{center}
\end{table}
\clearpage

%
%\begin{table}[h!]
%%\caption{Results for Treatment 3A}\label{tab:treat3A}
%\begin{center}
%\begin{footnotesize}
%\begin{tabular}{c ||c c c c c c c c | c}
%\textbf{} & \textbf{AB} & \textbf{B} & \textbf{BC} & \textbf{BCD} & \textbf{BD} & \textbf{C} & \textbf{CD} & \textbf{D} & \textbf{\textit{N}} \\
%\hline\hline
%\textbf{3A} & 24.27 & 13.93 & 26.59 & - & 18.9 & 1.64 & 4.4 & 2.64 & \textit{4021} \\
%%\textbf{UK} & 20.1 & 46.04 & 4.79 & 5.07 & 3.22 & 2.04 & 9.53 & \textit{2109} \\
%%\hline\hline
%%\end{tabular}
%%\end{footnotesize}
%%\end{center}
%%\end{table}
%%\end{samepage}
%
%%\begin{table}[h!]
%%\caption{Results for Treatment 3B}\label{tab:treat3B}
%%\begin{center}
%%\begin{footnotesize}
%%\begin{tabular}{c ||c c c c c c c | c}
%%\textbf{} & \textbf{AB} & \textbf{B} & \textbf{BC} & \textbf{BCD} & \textbf{BD} & \textbf{CD} & \textbf{D} & \textbf{\textit{N}} \\
%%\hline\hline
%\textbf{3B} & 10.07 & 14.52 & 36.98 & 4.53 & 20.49 & - & 3.76 & 2.59 & \textit{4021} \\
%%\textbf{UK} & 2.89 & 44.76 & 11.57 & 6.26 & 11.38 & 3.89 & 10.05 & \textit{2109} \\
%\hline
%\end{tabular}
%\end{footnotesize}
%\end{center}
%\end{table}

%\begin{samepage}
%\end{samepage}



\par Treatments 4A and 4B are summarised in Figure~\ref{fig:treat4A} and Figure~\ref{fig:treat4B}.  In Treatment 4A voters who anticipate minimal (in terms of parties) winning connected coalitions, who recognise the formateur advantage of being the largest party in the legislature, and are sensitive to ideological proximity should anticipate that CD will likely form a governing coalition.  In both Denmark and Germany, 30 percent of the respondents select this as the most likely outcome. But significant numbers of the Danes choose ABC and CDE (24 and 21 percent, respectively) while about 24 percent of the German respondents select a coalition of the two largest, although non-proximate, parties BD.\footnote{One explanation for the German choice of the BD coalition is its similarity to the CDU/SPD Grand Coalition that was in power when the survey was conducted.}  Note that in Treatment 4A the parties are ideological equidistance from each other and straddling the ideological centre. 

\par In Treatment 4B, Parties C and D significantly diverge in terms of their location on the ideological spectrum. The response of the Danes to the ideological treatment recalls the heuristics identified in the treatments 3A and 3B.  First, this formateur heuristics is clearly prevalent in the coalition choices in Treatment 4A -- over 50 percent of the respondents select a majority coalition that includes the largest party.  The response to the ideological treatment speaks to the resilience of this formateur heuristic:  Respondents do not react by anticipating an ABC coalition which is ideologically proximate and minimum winning (in terms of parties and seats).  Rather over 60 percent of the respondents anticipate the formation of a CDE coalition that includes the largest party, suggesting the resilience of this formateur heuristic.  But the inclusion of party E, which is not necessary for a majority coalition, suggests, again, this heuristic of including parties for ideological re-balancing which we saw in Treatments 3A and 3B.  Interestingly, the German response is in some fashion just the opposite to that of the Danes:  In Treatment 4B, the modal choice is ABC which  which is ideologically proximate and minimum winning (in terms of parties and seats) but does not include the party with the largest number of seats.  Unlike the Danes, the Germans abandoned their expectation of formateur influence on coalition formation and embraced the importance of ideological proximity and minimum winningness.  And there is no evidence in the German case of the ideological balance heuristic.

\clearpage
%\begin{figure}[h]
%\caption{Treatment 4}\label{fig:treat4}
%\centerline{\subfigure[Germany]{\includegraphics[width=.45\textwidth]{DE_majCoal_4.png}}\\
%\subfigure[Denmark]{\includegraphics[width=.45\textwidth]{DK_majCoal_4.png}}}
%\end{figure}


\begin{figure}[h]
\caption{Treatment 4A}\label{fig:treat4A}
\centerline{\includegraphics[width=.5\textwidth]{treat4A.png}}
\end{figure}

\begin{figure}[h]
\caption{Treatment 4B}\label{fig:treat4B}
\centerline{\includegraphics[width=.5\textwidth]{treat4B.png}}
\end{figure}

\begin{table}[h!]
%\caption{Results for Treatment 4A}\label{tab:treat4A}
\begin{center}
\begin{footnotesize}
\begin{tabular}{c ||c c c c c c c c | c}
\textbf{} & \textbf{ABC} & \textbf{B} & \textbf{BCD} & \textbf{BD} & \textbf{CD} & \textbf{CDE} & \textbf{D} & \textbf{DE} & \textbf{\textit{N}} \\
\hline\hline
Treat 4A &&&&&&&&&\\
\textbf{Germany} & 6.09 & 2.64 & 3.93 & 24.42 & 30.02 & 7.31 & 12.78 & 3.61 & \textit{4021} \\
\textbf{Denmark} & 24.06 & 1.01 & 4.32 & 3.31 & 32.13 & 21.33 & 5.62 & 1.44 & \textit{694} \\
\hline\hline
Treat 4B &&&&&&&&&\\
\textbf{Germany} & 24.57 & 3.01 & 1.84 & 18.85 & 13.95 & 5.07 & 11.32 & 10.1 & \textit{4021} \\
\textbf{Denmark} & 14.7 & 3.75 & 1.3 & 2.31 & 1.3 & 60.52 & 3.17 & 0.43 & \textit{694} \\
\hline\hline
\end{tabular}
\end{footnotesize}
\end{center}
\end{table}
\clearpage

%\begin{table}[h!]
%%\caption{Results for Treatment 4A}\label{tab:treat4A}
%\begin{center}
%\begin{footnotesize}
%\begin{tabular}{c ||c c c c c c c c | c}
%\textbf{} & \textbf{ABC} & \textbf{B} & \textbf{BCD} & \textbf{BD} & \textbf{CD} & \textbf{CDE} & \textbf{D} & \textbf{DE} & \textbf{\textit{N}} \\
%\hline\hline
%\textbf{4A} & 6.09 & 2.64 & 3.93 & 24.42 & 30.02 & 7.31 & 12.78 & - & \textit{4021} \\
%%\textbf{UK} & 3.13 & 5.93 & 1.47 & 5.31 & 13.47 & 4.55 & 49.29 & \textit{2108} \\
%%\hline
%%\end{tabular}
%%\end{footnotesize}
%%\end{center}
%%\end{table}
%%
%%\end{samepage}
%
%%
%%\begin{table}[h!]
%%%\caption{Results for Treatment 4B}\label{tab:treat4B}
%%\begin{center}
%%\begin{footnotesize}
%%\begin{tabular}{c ||c c c c c c c | c}
%%\textbf{} & \textbf{ABC} & \textbf{B} & \textbf{BCD} & \textbf{BD} & \textbf{CD} & \textbf{CDE} & \textbf{D} & \textbf{DE} & \textbf{\textit{N}} \\
%%\hline\hline
%\textbf{4B} & 24.57 & 3.01 & - & 18.85 & 13.95 & 5.07 & 11.32 & 10.1 & \textit{4021} \\
%%\textbf{UK} & 17.84 & 12.48 & 4.41 & 2.42 & 1.52 & 40.32 & 9.77 & \textit{2108} \\
%\hline
%\end{tabular}
%\end{footnotesize}
%\end{center}
%\end{table}




%\par These preliminary analyses of the coalition formation experimental vignettes suggest, at least to me, that they are very useful for a very basic distinction between those respondents who understand the notion of a majority governing coalition and those who have difficulty with this basic coalition arithmetic.  And this is the basic measure I propose using for calibrating the skill level of respondents' coalition reasoning.  The other dimensions of coalition reasoning -- ideological proximity and minimal winning coalition status -- I think offer some promise in terms of measuring skill levels although these are the subject of future analysis of these data.

%Inaki: Bar chart for those who chose a majority government: Here have two bars (UK/German) for each treatment -- the first bar would show percentage of respondents (UK/Germany) selecting AB in treatment 3A and second bar would show percentage of respondents (UK/Germany) selecting AB in treatment 3b.


% Inaki: of those who chose a majority government [drop from consideration those who did not select parties with a majority of seats] Here have two side-by-side graphs for each country -- in each graph two bars for eachpermutation of parties with a majority of seats in legislature -- one bar for treatment 2A and one bar for treatment 2B.



%\par As was pointed out earlier, we have surprisingly little empirical insight into what, if any, coalition formation heuristics voters rely on when they anticipate, as our models suggest they should, coalition formation possibilities after an election.  

\par Heuristics here are defined as strategies that guide the collection of information regarding coalition formation and summarise the bargaining process so as to facilitate the prediction of coalition formations after an election.  Five distinct coalition reasoning heuristics were recovered from the choices described above:

\begin{enumerate}
\item \emph{Majority Heuristic.} Subjects that consistently choose a group of parties that have a majority of seats in the legislature are considered to employ a majority heuristic.

\item \emph{Connected Heuristic.} Those subjects who select parties that are ideologically connected, i.e., are immediately proximate on the left-right ideological continuum (parties A and B would be connected while A and C would not).
    
\item \emph{Formateur Heuristic.} Subjects who consistently include the party with the largest number of seats in the legislature.
%, recognising the institutional advantage accorded the largest party in the coalition formation process. 

\item \emph{Ideological Proximity Heuristic.} One of the experimental treatments involved significantly increasing the ideological distance between potential coalition partners.  Subjects who responded by shifting to parties that were more ideologically proximate employ an ideological proximity heuristic.

\item \emph{Minimum Winning Coalition.} Subjects who consistently selected the minimum winning coalition outcome.

\item \emph{Largest Party Heuristic.} Evidence of systematic uninformed coalition reasoning would be the selection of the single largest party which, in the experiments described above, would not have sufficient seats to form a majority governing coalition.    
\end{enumerate}


\par  Models of coalition-directed voting typically have rather stylised representations of the coalition vote calculus.  For example, the typical voter is presumed to anticipate that coalitions will more likely include the largest party, will be ideologically proximate and connected.  More likely though voters might simply rely on one, or some reduced set, of such coalition formation heuristics -- or quite likely none.  The incidence of the five coalition heuristics, recovered from the Danish and German experiments described earlier, is summarised in Figure~\ref{fig:heuristics}.  These are within-subject analyses so they indicate the percentage of respondents who consistently employ each of the five heuristics.  The results confirm our contention that the Danish electorate exhibits much more sophisticated coalition reasoning that is the case for the German electorate.  About a quarter of the German sample consistently chose a majority governing coalition; while almost 40 percent of Danish respondents exhibited such behaviour.  Recall that respondents were instructed to identify a likely majority governing coalition so those consistently selecting a majority coalition exhibit an ability to sort out which combinations of parties constitute a majority of seats.

\par The frequencies of the other dimensions of coalition government provide insights into voters' expectations and our contention is that these will be consistent with coalition formation outcomes in these countries.  Note that very few of the respondents consistently selected a minimum winning coalition as a likely majority governing coalition -- less than 10 percent in Denmark and only two percent in Germany.  This is certainly consistent with general empirical findings \citep{Druckmanetal2002,Voldenetal2004} that minimal winning coalitions are not as prevalent as \citet{Riker1962} initially claimed.  Moreover the incidence of minimal winning coalitions in Germany... and Denmark..... 
    
\par While just under 20 percent of the German respondents consistently selected a connected governing coalition, over 60 percent of the Danish sample expected such a coalition type to form.  While about 25 percent of the Danes chose a coalition that included the largest party, this was the case for less than 10 percent of the German respondents.  Approximately 30 percent of the German sample responded consistently to the ideological treatment by modifying their predicted coalition outcome while almost 45 percent of the Danes responded in such a fashion.  Finally a small percentage of both the German and Danish sample consistently selected the single largest party regardless of whether it had a majority of seats, as a likely governing coalition.  A reasonable conclusion here is that at best only a range of 20 to 30 percent of the voting public consistently employs any one of the four informed coalition reasoning heuristics.       

  \begin{figure}[h!]
    \caption{Distribution of Coalition Heuristics}\label{fig:heuristics}
\centerline{
\includegraphics[width=.75\textwidth]{heuristics.pdf}}
\end{figure}



\par Our contention is that the heuristics we have identified here represent dimensions of sophisticated coalition reasoning -- for example, respondents who consistently include the largest party in the coalition exhibit sophisticated coalition reasoning.  To the extent that these different heuristics represent underlying dimensions of coalition reasoning they should be correlated with each other. And they are. Table~\ref{tab:correlation_heuristics} provides the correlations amongst the five different heuristics for both Denmark and Germany.  The correlation between the majority and connected heuristics is .83 in Germany and .48 in Denmark, suggesting that subjects that are consistent in selecting majority coalitions typically favour those in which the parties are ideologically connected. Those that select connected coalitions tend also to predict coalitions will include the formateur party (the correlation is .65 in Germany and .43 in Denmark).  These correlations lend credence to our contention that we have identified the underlying heuristics voters employ in order to understand coalition formation. 

%But the other correlations are significantly smaller suggesting that subjects sensitive to ideological distance and to formateur status may constitute quite distinct sub-populations.  And as we would expect respondents who did not use either one of the four heuristics, majority, connected, formateur, or ideology, had some tendency to favour the large party heuristics -- which is what generates the relatively small negative correlation coefficient between largest and the other four heuristics.  

%We can gain some such insight here by analysing the within-subject choices in the coalition reasoning experiments described in the previous section.  It is not the case that these heuristics are employed by the same people -- in other words there is not a core 20 percent of the voters that consistently employ all of these heuristics.  Rather it appears that German voters typically rely on one or a couple of these heuristics.  

\begin{table}[h!]
\caption{Correlation amongst coalition heuristics}\label{tab:correlation_heuristics}
\begin{center}
\begin{tabular}{ld{3}d{3}d{3}d{3}d{3}d{3}}
 & \mc{Majority} & \mc{Connected} & \mc{Formateur} & \mc{Ideological}  & \mc{Largest}  & \mc{Compact}\\
\hline\hline
\multicolumn{6}{l}{\textbf{Germany}}\\
Majority        & 1   &   &  \\
Connected     &  0.83 &  1 &  &  &   \\
Formateur  &   0.54      & 0.65   & 1 \\
Ideological  &   0.31       & 0.31   &  0.32 & 1 & \\
Largest  &   -0.14      & -0.12   & -0.08 & -0.16 & 1 \\
Compact &  0.19 &  0.23 &   -0.03 &  -0.07 &  -0.03 &   1 \\
%					&          &  & \\
\hline\hline
\multicolumn{6}{l}{\textbf{Denmark}}\\
Majority & 1 &  &  &  & \\
Connected & 0.49 & 1 &  &  & \\
Formateur & 0.22 & 0.43 & 1 &  & \\
Ideological & 0.32 & 0.34 & 0.54 & 1 & \\
Largest & -0.30 & -0.25 & -0.11 & -0.16 & 1\\
Compact & 0.13 &  0.22 & -0.17 & -0.26 & -0.05 &  1\\
\hline\hline
\end{tabular}
\end{center}
\end{table}


\clearpage


\par  The correlations in Table~\ref{tab:correlation_heuristics} suggest that voters employ some combination of these heuristics when they evaluate coalition formation possibilities or outcomes.  Figure~\ref{fig:types} compares the frequency with which Danish and German respondents consistently employ different combinations of these heuristics.  The results confirm our general contention that the Danes exhibit more sophisticated coalition reasoning than the Germans:  Just under 40 percent of Danish respondents consistently selected a majority coalition compared to about 25 percent of the Germans.  About 35 percent of the Danes and 18 percent of Germans consistently selected a majority connected coalition.  The third category are those who consistently anticipated a majority connected coalition that included the largest party: 16 percent of the Danes and 8 percent of the Germans fell in this category.  Finally, about 15 percent of the Danes and 5 percent of the Germans consistently predicted a majority connected coalition that included the formateur party and preferred a coalition with ideologically proximate parties. 
       
%respondents who only consistently anticipated a majority coalition; those consistently anticipating a majority coalition; respondents who consistently anticipated a connected coalition; those who selected, across all treatments, a majority and connected coalition that included the formateur party; and finally those who predicted a majority connected coalition that included the formateur party and preferred a coalition with ideologically proximate parties.  Figure~\ref{fig:types} presents the distribution of these coalition reasoning types for Denmark and Germany

\par The differences between Denmark and Germany summarised in Figure~\ref{fig:types} are actually quite dramatic.  These results are based on panel sampling strategies that differ significantly between the two countries.  The most likely confounding factor related to sampling is education.  Accordingly, Figure~\ref{fig:types_compare} reproduces Figure~\ref{fig:types} controlling for education -- the high education category consists of respondents in the top quartile of the education measure in each country while the low education category consists of those in the bottom quartile.  Controlling for education does not eliminate the German and Danish differences in the sophistication of coalition reasoning heuristics.  The differences in the low education remain very pronounced.  And while the differences are lower in the high education category, the country differences remain quite significant.    

  \begin{figure}[h!]
    \caption{Distribution of Coalition Reasoning Types}\label{fig:types}
\centerline{
\includegraphics[width=.75\textwidth]{heuristicComp.pdf}}
\end{figure}

\clearpage


\par We have created an index of coalition reasoning that consists of the factors scores from a factor analysis of all five heuristics described in Figure~\ref{fig:types} (an alternative strategy of simply adding the values of majority, connected, formateur and ideology and then subtracting largest produces essentially the same result -- the measures have a correlation coefficient of .95).  The distribution of responses suggested a trichotomy of coalition reasoning with factors scores below 0 categorised as ``low coalition"; those between 0 and 1 categorised as ``medium coalition"; and those higher than 1 placed in the ``high coalition" category.  We contend that higher coalition types adopt these heuristics because they enable them to make optimal vote choices, approximating those of a perfectly rational voter.  In the next section we examine whether this is in fact the case. 

\subsection{Heterogeneity in Coalition Reasoning Types}

  \begin{figure}[h!]
    \caption{Distribution of Coalition Reasoning Types}\label{fig:types_compare}
\centering
\subfigure[Low Education]{\includegraphics[width=.45\textwidth]{heuristicComp_LE.pdf}}
\subfigure[High Education]{\includegraphics[width=.45\textwidth]{heuristicComp_HE.pdf}}
\end{figure}


%\par Voters acquire these coalition reasoning skills because in political contexts such as Denmark and Germany they contribute to informed vote decisions.  Moreover our theoretical models presume they have these reasoning skills.  Nevertheless, as was argued earlier, in some contexts the amount of this coalition information is much higher than in others -- contexts in which coalition formation and governance is complex have more extensive media discussions of coalition formation dynamics.  Accordingly, we should see high levels of coalition reasoning for all groups in contexts with high volumes of information because the costs of information acquisition are low.  In contexts where messages about coalition formation and governance are much rarer, information acquisition costs are high and hence we expect higher levels of heterogeneity. To the extent the acquisition of this information is costly it should be more accessible to the better educated and those more engaged in politics.  Hence in low information contexts we would expect socio-economic status to be correlated with these coalition reasoning skills in the sense that education levels, nature of employment, and social status should all facilitate acquisition to this information and the development of these coalition reasoning skills.  Secondly, and highly correlated with socio-economic status, respondents who are knowledgable about politics are more likely to have information about the coalition formation process and hence are more likely to have acquired these coalition reasoning skills.

%\par We explore this argument employing as our dependent variable the trichotomous coalition reasoning variable defined earlier. Table~\ref{tab:germany_coalition} presents the Danish and German estimates for a multi-nomial logit equation that includes a series of socio-economic variables that one might expect to facilitate the acquisition of coalition reasoning skills: gender, education, party identification and political interest.  In addition we have two direct measures of political knowledge, The \emph{Political Knowledge} variable from the German survey is simply a count of the number of correct answers to four political knowledge questions that were asked in the fourth wave of the panel;\footnote{The four German political knowledge questions are the following: 1) For how many years is a Member of Parliament elected? 2)  How are members of the upper house selected? 3) How many L?nder are there in Germany? 4) What is the minimum percentage of votes necessary to be represented in the Federal Parliament?} In Denmark the political knowledge variable consists of a count of correct answers to eight knowledge questions.\footnote{The eight Danish political knowledge questions are the following: 1) Which party does the prime minister represent? (Correct answer: Venstre); 2) Which party does the taxation minister represent? (Correct answer: Venstre); 3) Which party does the finance minister represent? (Correct answer: Venstre); 4) Which party does the minister of foreign affairs represent? (Correct answer: Konservative); 5) Which party does the minister of interior affairs represent? (Correct answer: Venstre); 6) Which party does the justice minister represent? (Correct answer: Konservative); 7) Which party does the education minister represent? (Correct answer: Venstre); 8) Which party does the minister of economic and business affairs represent? (Correct answer: Konservative)} \emph{Coalition Knowledge} indicates whether respondents knew the parties in the governing coalition prior to the election.
% Table~\ref{tab:germany_coalition} presents the results of a regression of coalition reasoning type on these explanatory variables.  The results conform to our expectations. 

%\begin{table}
%\caption{Multi-nomial Logit Model of Coalition Reasoning: Germany (2009) Denmark (2011)}
%\begin{footnotesize}
%\begin{center}
%\begin{tabular}{ld{3}d{3}d{3}}
%\textbf{Variable} &\textbf{Germany} & \textbf{Denmark} \\
%\hline\hline
%\multirow{2}{*}{\textbf{Medium Coalition Reasoning}} &  &  &\\
%\multirow{2}{*}{Political Knowledge} & 0.731 & -0.052 \\
% & (0.152) & (0.047) \\
%\multirow{2}{*}{Coalition Knowledge} & 0.425 & 0.265 \\
% & (0.067) & (0.141) \\
%\multirow{2}{*}{Female} & -0.089 & 0.13 \\
% & (0.098) & (0.194) \\
%\multirow{2}{*}{Education} & 0.034 & 0.023 \\
% & (0.033) & (0.045) \\
%\multirow{2}{*}{LR Self ID} & -0.045 & -0.054 \\
% & (0.022) & (0.042) \\
%\multirow{2}{*}{Constant} & -1.813 & -3.036 \\
% & (0.244) & (1.39) \\
%\hline
%\multirow{2}{*}{\textbf{High Coalition Reasoning}} &  &  &\\
%\multirow{2}{*}{Political Knowledge} & 1.09 & 0.014 \\
% & (0.268) & (0.049) \\
%\multirow{2}{*}{Coalition Knowledge} & 0.387 & 0.352 \\
% & (0.125) & (0.184) \\
%\multirow{2}{*}{Female} & -0.498 & 0.054 \\
% & (0.172) & (0.195) \\
%\multirow{2}{*}{Education} & 0.007 & 0.08 \\
% & (0.055) & (0.047) \\
%\multirow{2}{*}{LR Self ID} & -0.032 & -0.011 \\
% & (0.037) & (0.042) \\
%\multirow{2}{*}{Constant} & -3.132 & -4.877 \\
% & (0.432) & (1.822) \\
%\hline\hline
%N & 2316 & 681 \\
%LogLik & -1948.4679 & -685.48324 \\
%AIC & 3920.9358 & 1394.9665 \\
%\hline\hline
%\end{tabular}
%\end{center}
%\end{footnotesize}
%\label{tab:germany_coalition}
%\end{table}


%\begin{table}
%\caption{OLS Model of Coalition Reasoning: Germany and Denmark 2009}
%\begin{footnotesize}
%\begin{center}
%\begin{tabular}{ld{3}d{3}d{3}}
%\textbf{Variable} &\textbf{Germany} & \textbf{Denmark} \\
%\hline
%\hline
%\multirow{2}{*}{Political Knowledge} & 0.273 & 0.001 \\
% & (0.052) & (0.014) \\
%\multirow{2}{*}{Coalition Knowledge} & 0.097 & 0.08 \\
% & (0.02) & (0.034) \\
%\multirow{2}{*}{Female} & -0.105 & 0.021 \\
% & (0.033) & (0.057) \\
%\multirow{2}{*}{Education} & 0.005 & 0.022 \\
% & (0.011) & (0.013) \\
%\multirow{2}{*}{LR Self ID} & -0.01 & -0.005 \\
% & (0.008) & (0.012) \\
%\multirow{2}{*}{Constant} & -0.258 & -0.922 \\
% & (0.077) & (0.337) \\
%\hline\hline
%N & 2316 & 681 \\
%LogLik & -2640.0194 & -752.64176 \\
%AIC & 5292.0389 & 1517.2835 \\
%\hline
%\hline
%\end{tabular}
%\end{center}
%\end{footnotesize}
%\label{tab:tableOLS}
%\end{table}

%\clearpage

%\par A comparison of the results for the Danish and German samples in Table~\ref{tab:germany_coalition} confirms our hypothesis regarding density of coalition information and heterogeneity.  We expect heterogeneity in the German case which is what we find: Germans with high coalition reasoning skills tend to be highly knowledgeable about politics and partisans.  They also tend to be men. By contrast, the coefficients in the Danish model are, with the exception of the coalition knowledge variable, not statistically significant.  These results represent further evidence that context affects the ease with which citizens acquire coalition reasoning skills: In Germany, where the requisite information is more difficult to acquire, we see evidence of considerable heterogeneity in coalition reasoning skills while in Denmark where such information costs are low we see little evidence of heterogeneity in coalition reasoning.

%  The Danish and German differences in Model 1 results suggest that these two political knowledge variables are strongly correlated with coalition reasoning -- those with more knowledge of the politics and the political process have more well-honed coalition reasoning skills.  Clearly the knowledge variables are correlated with the demographic variables -- Model 2 excludes the knowledge variables and gives some sense of how demographics -- such as gender and education -- have an indirect effect on coalition reasoning via knowledge.  


\section{The Coalition-Directed Vote}

\par Voters, again depending on context, develop coalition reasoning heuristics that allow them to approximate optimal coalition directed voting.  And it is important to point out that in most case they probably develop these heuristics passively -- these heuristics are simply part of the general information context.  But as we have seen not all voters, even in contexts with rich and complex coalition politics, develop these coalition reasoning heuristics.  Voters with underdeveloped coalition reasoning heuristics, regardless of context, are less likely to to approximate an optimal coalition directed vote simply because their coalition reasoning heuristics are not up to the task.  Accordingly, we should find that variations in these coalition reasoning skills affects the extent to which vote choice approximates an optimal coalition directed vote.  Coalition directed voting implies a variety of skills..... So we explore this in a couple of ways.....

\par In \citet{Duchetal2010} the coalition directed vote focuses on ideology ... and coalition reasoning is particularly important....  A critical pre-condition for exercising a coalition-directed ideological vote is the ability to place the parties on the left-right ideological dimension.  One would expect that those who exhibit high levels of coalition reasoning would be better skilled at this particular task.  One way to measure this is simply to examine the degree of variation in left-right placements across levels of coalition reasoning -- our expectation is that lower degrees of reasoning would be associated with much more noise and uncertainty in their placement of parties on this continuum.  And effectively this is what we see in Figure~\ref{fig:variance}

  \begin{figure}[h!]
    \caption{Variance of Left-Right Party Placement by Coalition Reasoning Type}\label{fig:variance}
\centerline{
\includegraphics[width=1\textwidth]{variance_PartyPlacement.pdf}}
\end{figure}

\clearpage

%\par We test this argument by estimating variation, over heuristics types, in the magnitude of the economic vote.  The economic vote is a coalition-directed vote because the vote utility function incorporates information about the administrative responsibility of parties within a governing coalition and how a vote for each party affects the likely coalition governments that will form after an election \citep{DuchStevenson2008}.  This requires that voters understand the types of coalitions that are competitive and which coalitions a party is likely to enter.\footnote{An example of how coalition reasoning is critical to their model of vote choice, is their prediction that voters will not exercise an economic vote for perennial prime ministerial parties because regardless of their electoral results they are certain to participate in the governing coalition.}  It stands to reason that segments of the population with more well-developed heuristics about coalition formation and governance should be more likely to exercise a coalition-directed economic vote.

% The coalition-directed theory of the economic vote presumes that voters, or at least some voters, have certain coalition reasoning skills.  The previous sections demonstrated that certainly there are segments of the electorate that exhibit these skills.  Rational theories of vote choice imply that segments of the population with these reasoning skills should be more likely to exercise a coalition directed economic vote.  This section presents the empirical evidence that tests this proposition.  
  
%\subsection{Coalition-reasoning types anticipate coalition outcomes?}

%\par \emph{Anticipating Coalition Outcomes.}  In models of coalition-directed voting \citep{Duchetal2010,DuchStevenson2008} voters are assumed to have preferences over distributions of administrative responsibility.  They are also expected to know whether their vote would be pivotal in electing (or defeating) a particular distribution of administrative responsibility. Hence these voters are expected to anticipate the different types of coalitions that could form after an election in addition to having a sense of the likelihood of different coalitions forming. 

%  In \citet{Duchetal2010}, the $\gamma$ term in Equation \ref{eq:20} implies that voters know with what probabilities parties are likely to enter a post-election governing coalition.

%\par The measure of coalition reasoning recovered from the experimental vignettes captures what is, I believe, an underlying ability to recognise, in a general fashion, the types of governing coalitions that are feasible.  This is in some sense a necessary, although possibly not a sufficient, condition for being able to make sense of the likely coalition outcomes that can occur in a political context.  Hence those voters who score high on this general coalition reasoning measure should be adept at anticipating particular post-election coalition outcomes.  We can explore this relationship by comparing the ability to anticipate coalition outcomes of those with low versus high coalition reasoning.   In Wave 2 of the German DCCAP survey, which occurred three weeks before the election, respondents were asked to indicate which parties would likely form the post-election governing coalition.  Approximately 35 percent of the respondents correctly identified the CDU/CSU-FDP coalition as the governing coalition that would most likely form after the election.

%\par Table~\ref{tab:germany_anticipate} presents the results of a logit regression of the correct anticipation of the post-election governing coalition on dummy variables for coalition reasoning types plus the socio-economic and knowledge variables we encountered earlier.  Clearly those with Medium or High levels of coalition reasoning are significantly more likely to correctly anticipate the governing coalition that forms after the election.  But the magnitudes of the Medium and High dummy coefficients are not significantly different from each other suggesting that the relevant threshold here, in terms of anticipating post-election governing coalitions, is whether one is in the Low versus Medium or High reasoning categories.  


%\begin{table}[h!]
%\caption{Logit Model of Coalition Anticipation: German CCAP Study 2009}\label{tab:germany_anticipate}
%\begin{footnotesize}
%\begin{center}
%\begin{tabular}{ld{3}d{3}d{3}}
%\textbf{Variable} & \textbf{Model 1} & \textbf{} \\
%\hline\hline
%\multirow{2}{*}{Education} & 0.05 &   \\
% & (0.06) &   \\
%\multirow{2}{*}{Female} & -0.61  &   \\
% & (0.16) &  \\
%\multirow{2}{*}{Income} & 0.08 & \\
% & (0.04) &   \\
%\multirow{2}{*}{Interest} & 0.34 &   \\
% & (0.11) &  \\
%\multirow{2}{*}{Political Knowledge} & .23 &  \\
% & (0.09) &  \\
% \multirow{2}{*}{Medium Sophistication} & 1.04 &  \\
% & (0.19) &  \\
% \multirow{2}{*}{High Sophistication} & 1.26 &  \\
% & (0.23) &  \\
%\multirow{2}{*}{Constant} & -2.92 &   \\
% & (0.40) &   \\
%\hline\hline
%Obs & 2,148 &  \\
%& -3013.304 &  \\
%R-Square & 0.09 &  \\
%& 15 & \\
%AIC & 5163.903 &  \\
%& 5245.715 &  \\
%\hline\hline
%\end{tabular}
%\end{center}
%\end{footnotesize}
%\end{table}

%\clearpage

%\par Note also that the equation includes political knowledge as a control variable -- hence the coalition-reasoning effect is independent of the respondents knowledge about specific aspects of German politics.  Knowledge per se of course is an important contributor to one's ability to anticipate coalition outcomes after an election.  Independent of political knowledge, though, there clearly is a coalition reasoning skill that improves an individual's ability to anticipate the kinds of coalitions that are likely to form.  

%\subsection{Coalition Reasoning Types, Administrative Responsibility and Vote}

%\par There is another aspect of reasoning about coalition formation that I believe is captured by the coalition reasoning measure and one that is very important for exercising a coalition-directed vote:  My intuition is that underlying this reasoning is an understanding, on the part of respondents who score high on the measure, that political parties make policy or rent concessions when they share administrative responsibility with other parties: Subjects in the experiments avoid including ideological distant partners in a coalition (the proximate types identified earlier) because they anticipate they will extract high policy concessions.  For a similar reason these coalition sophisticates avoid ``unconnected'' coalitions.  These voters understand the implications of shared administrative responsibility in a multi-party governing coalition (or in a possible post-election coalition).  In fact, I suspect that the ability of individuals to attribute shared responsibility in a wide variety of every-day decision making contexts -- the senior partner committee in an investment banking firm that decides on the allocation of year-end bonus money, for example -- is in some sense a general reasoning trait.  And it is a cognitive skill that contributes to the coalition reasoning I recover with the experiments described above.\footnote{I suspect in fact that by using experiments we can directly recover the heuristics that individuals employ to attribute responsibility in collective decision making situations such as coalition governments.  This is the subject of a current related experimental project I am conducting.}

%In fact respondents Respondents who score high on this measure...    The coalition reasoning experiments were designed to measure coalition reasoning skills -- the measure distinguishes individuals who are more or less skilled at identifying the combinations of parties more likely to form a coalition government after an election.  Voters who recognize how size and ideological placement shape the likelihood of coalition formations demonstrate an understanding of the bargains that are struck amongst parties in this process. 

\par We test this argument by estimating variation, over heuristics types, in the magnitude of the economic vote.  \citet{DuchStevenson2008} argue that voters must believe that parties are in contention for significant governing responsibility in order for the economy to affect their vote for that party.  Voters cannot have these beliefs unless they recognise how voting for different parties affects the likely composition of a post-election coalition government.  Employing our construct measure of sophistication we can identify those voters who have well-developed coalition reasoning heuristics.  Those heuristics contribute to the voter's ability to anticipate how their party vote will affect the coalition that forms after an election.  Without this heuristic voters cannot exercise a coalition-directed economic vote.  If in fact these heuristics facilitate the exercise of a rational coalition-directed vote then voters with more developed heuristics should have higher levels of economic voting.  We test this argument by estimating vote choice models using the Danish and German surveys. 

% voters assess the impact of their vote choice on the distribution of administrative responsibility in likely post-election coalition governments.  This assessment in turn conditions their economic vote for specific political parties.

%I would add here that these coalition reasoning skills will matter here: those who have a more sophisticated sense of how coalitions form, and which governing coalitions are more likely to form, will be more likely to incorporate the administrative responsibility into their vote preference function.              


%\par In this section I demonstrate that this is in fact the case.  I explore how two different measures of government performance in the vote preference function are conditioned by administrative responsibility.  Economic vote.... And specific policy performance -- crime and health care...    

%But again here it is important to point out that there are quite stylized characterizations of the coalition-directed vote that build on rational choice theories of voting behaviour.  In teasing out the extent to which some voters are more likely to conform to these models of coalition-directed voting I entertain the possibility that some aspects of these stylized characterizations are more salient than others. As I pointed out earlier, the principal feature of coalition-directed voting that will be explored here is how considerations related to the distribution of responsibility enter into the vote choice function and, in particular, how this varies by the levels of coalition reasoning that were recovered from choices in the coalition reasoning experiments. Our expectation of course is that those who demonstrate higher levels of sophisticated coalition reasoning ought also to have vote preferences that exhibit a greater sensitivity to the distribution of administrative responsibility associated with coalition governments.  

\par We begin with a comparison of the coalition-directed economic vote in Germany of those who score high versus those who score low on this measure of coalition reasoning. The 2009 German election occurred after four years of a Grand Coalition consisting of the CDU/CSU and SPD.  The incumbent coalition consisted of these two major parties in which cabinet portfolios were divided equally amongst the two parties (roughly eight ministries for each party).  This deviates from most economic vote situations because the two major parties, that would most certainly dominate the post-election coalition government, were both effectively ``incumbent" parties.  Classic models of the economic vote presume, at a minimum, well-defined incumbent and opposition parties that will either be rewarded or punished as a function of perceived economic performance.  A Grand Coalition does not provide such a clear dichotomy and hence most of these models simply predict a muted economic vote in such contexts \citep{Anderson1995,Powelletal1993}.  In the \citet{DuchStevenson2008} model, for example, if one of the two parties is expected to govern on its own (or effectively on its own) after the election, the equal distribution of responsibility in the incumbent cabinet would lead to neither party receiving an economic vote.  Neither party is ``in contention'' to either re-elect or replace the incumbent distribution of administrative responsibility.  

%  Take the case of the voter who is unhappy with the economy: A vote for neither party for example, could be pivotal in electing a government that had not been responsible for the economy prior to the election. 

%In the \citet{DuchStevenson2008} selection model, the coalition-directed economic vote takes into consideration the distribution of responsibility of the governing (or potentially) governing coalition.  The theory assumes that voters know which parties are in the incumbent cabinet and which parties are in any alternative cabinet they are considering.  Voters are assumed to believe each party's contribution to economic outcomes is proportional to the parties' shares of responsibility.  

%\par Subjects who score high on the coalition reasoning measure recovered from the experiments described above are expected to also demonstrate an ability to reason about the distribution of responsibility of parties within a coalition government and its implications for the incumbent status of particularly parties within that coalition. We explore the extent to which a simple heuristic regarding administrative responsibility guides coalition reasoning on the part of our subjects. 

\par In the 2009 German elections most voters expected one of the two parties to govern in a coalition that excluded the other former grand coalition partner.  Given the equal distribution of responsibility in the incumbent cabinet, this implies that neither party should receive an economic vote.  Lets say a voter was unhappy with the economy, and she was pivotal, a vote against either of the SPD or CDU/CSU would result in a governing coalition that was equally responsible for the economy that made her unhappy.  Voters with well-developed coalition heuristics should recognise this.  Voters who do not employ such heuristics would not recognise this and might in fact cast an economic vote for parties making up the governing coalition.  The empirical analysis should shed light on whether the heuristics we have recovered experimentally account for variations in the economic vote.

%\item \emph{Sophisticated}: A plausible coalition outcome would have been either 1) the CDU/CSU and FDP in coalition (which is the actual coalition that formed); or 2) the SPD in coalition with the Greens and the Linke parties.  The Duch and Stevenson model would predict the following: 1) coalition directed voters who were happy with the economy should vote for the CDU/CSU or FDP parties -- the likely resulting coalition would result in the ``incumbent" CDU/CSU having about 85 percent of the distribution of responsibility; 2) coalition directed voters who were unhappy with the economy should vote for the SPD, Greens or Linke parties (in effect, whichever party vote they thought would maximize the likelihood of this coalition forming) -- this coalition would have about half of the responsibility accounted for by ``opposition" parties that were not in the Grand Coalition (the Greens obtained about 10 percent of the vote; the Linke 12 percent; and the SPD about 23 percent).\footnote{Another other plausible coalition outcome was another Grand Coalition but this was one that was heavily discounted by most observers and by the voters.  In this case the economic vote for both the CDU/CSU and SPD would be large, similarly signed and indistinguishable in magnitude.}
%\end{itemize}

%\noindent Hence the theoretical expectation here is that coalition-oriented voters should reward the CDU/CSU and FDP parties for good perceived economic outcomes but should prefer the SPD, Green, and Linke parties if they perceive the economy as doing poorly.  This is precisely what happens in the 2009 German election.

\par Table~\ref{tab:germany_economy} first presents the multinomial logit results for a classic economic vote model based on the 2009 German and 2011 Danish studies.  The dependent variable is vote preference expressed in surveys that occurred prior to the elections and the independent variables are retrospective evaluation of the national economy (coded such that a high score indicates a negative evaluation of the retrospective economy) along with a standard set of control variables (not reported here but available from the authors).  We have estimated the model for each of the three levels of coalition reasoning that were described earlier. Our expectation that those with the most well-developed coalition heuristics would exercise no economic vote for either of the two incumbent parties is confirmed in the third column of Table~\ref{tab:germany_economy}.  For those with the most well-developed heuristics the coefficients on economic evaluations, across the party choices, are for the most part statistically insignificant. By contrast, the coefficients for the low and moderately sophisticated are typically statistically significant.  The contrast is more clearly illustrated in the left-hand panel of Figure~\ref{fig:coalition_ev} that simulates the economic evaluation effects (the simulation is the change in probability of voting for each party associated with a unit deterioration in subjective evaluations of the retrospective economy).  Note that for the CDU/CSU and Linke party choices, the respondents with the least well-developed heuristics have estimated economic votes that are statistically significant.  On the other hand, those with the most well-developed heuristics respondents have no statistically significant economic votes across all of the party choices.  A reasonable conclusion is that the economy played virtually no role in the vote preference of those employing more sophisticated coalition heuristics.

\par Voters who have acquired limited coalition reasoning heuristics seem to be treating the CDU/CSU as the incumbent while ignoring the incumbent status of the SPD.  Our interpretation is that these voters with limited heuristics have difficulty conceptualising collective decision making by parties within a multi-party coalition.  Hence their heuristic is to identify a single, prime ministerial party, for example, as the incumbent and ignore other aspects of shared administrative responsibility within a governing coalition. This suggests that acquiring coalition reasoning heuristics enables voters to more closely approximate a rational economic vote.

\par The 2011 Danish election presented a much more familiar situation for the economic voter. The Danish incumbent minority coalition government consisted of the Liberal Party and the Conservative People's Party.  The minority government was supported by the Danish People's Party.  And it was widely expected that the Social Democrats were in a strong position to form a new coalition government.  Rational Danish voters in this case would need to exercise a coalition-directed vote that incorporates information about the parties that made up the incumbent coalition and the parties likely to enter the governing coalition after the election.  In the case of the Danish economic vote we expect the Conservative People's Party (KF) and the Liberal Party (V) to receive the bulk of the economic blame (hence a negative economic vote) while the most likely opposition party to form the government, Social Democrats (SDP), to receive a positive economic vote.  And while overall the Danes have very well-developed coalition reasoning heuristics, this does vary in the population and our expectation is that those who have acquired these heuristics will exercise a more rational coalition-directed economic vote.  Those with the most well-developed heuristics exhibit the most consistent economic voting behaviour -- clearly rewarding Social Democrats for negative perceptions of economic performance and punishing the Liberal and Conservative People's Party for negative economic assessments.  The economic vote of those with less-developed heuristics is for the most part not significant.  And the medium sophisticates exhibit moderate levels of economic voting -- in particular with respect to the Social Democrats.  Again, here there is some support here for the notion that acquiring these coalition reasoning heuristics facilitates the exercise of a more rational economic vote.  

%\citet{DuchStevenson2008} argue that as administrative responsibility is more diffusely, and evenly, shared amongst incumbent coalition partners (something voters are expected to know), the impact, on vote choice, of the incumbent's economic performance should decline \citep{DuchStevenson2008}. 

%\par The results for respondents with the least developed coalition reasoning skills, and for those falling in the intermediate category of coalition reasoning, are quite different.  Note that in Table~\ref{tab:germany_economy} the reference party is the CDU/CSU, hence in effect the ``prime ministerial party" because of course this was the party of Chancellor Merkel.  The multi-nomial coefficients for each of other parties are large, statistically significant and are positively signed.  This suggests that as economic evaluations deteriorate (note that a high score on national economic evaluations indicates a worsening assessment) voters are significantly more likely prefer the SPD, and any one of the other parties, over the CDU/CSU.  Low and moderate coalition reasoning types seem to be treating the CDU/CSU as the incumbent while considering the SPD as an ``opposition'' party.  One interpretation here is that low and medium coalition reasoning types have a limited ability to conceptualize collective decision making by parties within a multi-party coalition.  Hence their heuristic is to identify a single, prime ministerial, party as the incumbent and ignore the shared administrative responsibility of other parties in the coalition. This heterogeneity in heuristics regarding the incumbent status of parties in multi-party governing coalitions suggests that the nature of the economic vote in these coalition contexts is quite distinct across groups within the population.


%\begin{table}
%\caption{Retrospective National Economic Evaluations and Coalition Reasoning}
%\begin{tabular}{cccc}
%\toprule
%\hline
%National Economic Evaluations \\
%\hline
%Party & Low Sophistication & Mid & High Sophistication \\
%\hline\hline
%\multirow{2}{*}{SPD} & 0.186 & 0.181 & 0.222 \\
% & (0.042) & (0.053) & (0.118) \\
%\multirow{2}{*}{CDU/CSU} & 0.226 & 0.197 & 0.229 \\
% & (0.042) & (0.05) & (0.099) \\
%\multirow{2}{*}{FDP} & 0.197 & 0.166 & 0.178 \\
% & (0.042) & (0.048) & (0.104) \\
%\multirow{2}{*}{Green} & 0.104 & 0.159 & 0.129 \\
% & (0.034) & (0.05) & (0.096) \\
%\multirow{2}{*}{Linke} & 0.197 & 0.237 & 0.201 \\
% & (0.036) & (0.052) & (0.09) \\
%\multirow{2}{*}{Other} & 0.09 & 0.061 & 0.041 \\
% & (0.031) & (0.036) & (0.035) \\
%\hline\hline
%Party & Low Knowledge & Mid & High Knowledge \\
%\hline
%\multirow{2}{*}{SPD} & 0.188 & 0.18 & 0.214 \\
% & (0.043) & (0.053) & (0.119) \\
%\multirow{2}{*}{CDU/CSU} & 0.225 & 0.202 & 0.229 \\
% & (0.043) & (0.051) & (0.103) \\
%\multirow{2}{*}{FDP} & 0.198 & 0.165 & 0.18 \\
% & (0.043) & (0.05) & (0.106) \\
%\multirow{2}{*}{Green} & 0.104 & 0.155 & 0.132 \\
% & (0.034) & (0.051) & (0.098) \\
%\multirow{2}{*}{Linke} & 0.194 & 0.238 & 0.202 \\
% & (0.037) & (0.053) & (0.089) \\
%\multirow{2}{*}{Other} & 0.09 & 0.06 & 0.044 \\
% & (0.032) & (0.036) & (0.043) \\
%\hline\hline
%\bottomrule
%\end{tabular}
%\label{tab:germany_economy}
%\end{table}

\begin{table}
\caption{Multi-nomial Logit Estimates for Vote Choice: Denmark (2011) and Germany (2009)}
\begin{tabular}{cccc}
\toprule
\hline
National Economic Evaluations \\
\hline
Party & Low Sophistication & Mid & High Sophistication \\
\hline\hline
\textbf{Germany}\\
\multirow{2}{*}{SPD} & 0.45 & 0.36 & 0.15 \\
 & (0.12) & (0.52) & (0.26) \\
\multirow{2}{*}{FDP} & 0.33 & 0.22 & 0.23 \\
 & (0.1) & (0.37) & (0.25) \\
\multirow{2}{*}{Green} & 0.44 & 0.44 & 0.05 \\
 & (0.14) & (0.48) & (0.28) \\
\multirow{2}{*}{Linke} & 0.95 & 1.3 & 0.57 \\
 & (0.14) & (0.53) & (0.29) \\
\multirow{2}{*}{Others} & 0.8 & 2.87 & 0.89 \\
 & (0.15) & (1.22) & (0.39) \\
%
%\multirow{2}{*}{SPD} & 0.49 & 0.24 & -0.01\\
% & (0.13) & (0.19) & (0.4) \\
%\multirow{2}{*}{FDP} & 0.24 & 0.51 & -0.02\\
% & (0.11) & (0.18) & (0.38) \\
%\multirow{2}{*}{Green} & 0.51 & 0.28 & -0.27\\
% & (0.16) & (0.2) & (0.45) \\
%\multirow{2}{*}{Linke} & 0.98 & 0.92 & -0.04\\
% & (0.16) & (0.22) & (0.46) \\
%\multirow{2}{*}{Others} & 0.83 & 0.87 & 1.3\\
% & (0.17) & (0.28) & (0.79) \\
\footnotesize Base category: CDU/CSU \\ 
\hline\hline
\textbf{Denmark}\\
%\multirow{2}{*}{SD} & 0.62 & 1.26 & 0.96\\
% & (0.22) & (0.43) & (0.36) \\
%\multirow{2}{*}{DRV} & 0.37 & 0.95 & 0.42\\
% & (0.23) & (0.47) & (0.35) \\
%\multirow{2}{*}{K} & 0.01 & 0.05 & 0.27\\
% & (0.26) & (0.46) & (0.4) \\
%\multirow{2}{*}{SF} & 0.37 & 0.86 & 0.63\\
% & (0.25) & (0.45) & (0.42) \\
%\multirow{2}{*}{Others} & 0.4 & 0.85 & 0.55\\
% & (0.19) & (0.38) & (0.32) \\
\multirow{2}{*}{SD} & 0.72 & 0.76 & 0.85 \\
 & (0.36) & (0.31) & (0.52) \\
\multirow{2}{*}{DRV} & -0.17 & 0.68 & 0.64 \\
 & (0.48) & (0.3) & (0.5) \\
\multirow{2}{*}{K} & 0.14 & -0.09 & 0.52 \\
 & (0.37) & (0.37) & (0.58) \\
\multirow{2}{*}{SF} & 0.43 & 0.73 & 0.15 \\
 & (0.44) & (0.33) & (0.6) \\
\multirow{2}{*}{Others} & 0.54 & 0.62 & 0.1 \\
 & (0.29) & (0.26) & (0.5) \\
\footnotesize Base category: VE \\  
\hline\hline
\bottomrule
\end{tabular}
\label{tab:germany_economy}
\end{table}


\clearpage


\par Heuristics regarding coalition formation and governance are distinct from political knowledge.  Our contention is that acquiring well-developed coalition heuristics facilitates voting decisions that approximate the predictions of rational voting.  By contrast simply being knowledgable about politics -- in the sense of having factual knowledge such as knowing the names of parties and ministers or knowing facts about the political institutions -- does not significantly contribute to understanding the dynamics of coalition formation and governance.  Accordingly, it should \emph{not} be the case that the strength of the economic vote varies across political knowledge levels in fashion similar to the case for coalition heuristics.  The right hand graphs in Figure~\ref{fig:coalition_ev} present the simulated economic vote for three knowledge types: Low, Medium and High. There is very little resemblance to the coalition heuristic results.  In the German case, for the Low knowledge types, we see no evidence of a statistically significant effect of economic evaluations -- compare this to the Low coalition reasoning types who have a preponderance of significant economic votes.  By contrast, the High knowledge types have economic votes that are significant for CDU/CSU and close to significance for FDP and the Greens -- compare this to those with well-developed coalition heuristics who have no significant economic vote. And in Denmark, political knowledge appears to have little systematic influence on the economic vote.  The important conclusion here is that high levels of political knowledge do not result in coalition-directed voting, as was the case with our measure of coalition reasoning heuristics.   

%There are much less stark differences amongst the high, medium and low knowledge types.

\begin{figure}[h!]
\caption{Coalition Reasoning and the Economic Vote}\label{fig:coalition_ev}
\centering
\subfigure[Denmark: Coalition Reasoning]{\includegraphics[height=3.10in]{retnat_ev_DE.png}}
\subfigure[Denmark: Political Knowledge]{\includegraphics[height=3.10in]{retnat_pk_DE.png}}\\
\subfigure[Germany: Coalition Reasoning]{\includegraphics[height=3.10in]{retnat_ev_DK.png}}
\subfigure[Germany: Political Knowledge]{\includegraphics[height=3.10in]{retnat_pk_DK.png}}
\end{figure}

\clearpage

\par Theories of coalition directed voting at a minimum presume voters employ simple coalition reasoning heuristics, such as the ones identified in this essay.  Our expectation is that those who have acquired extensive coalition reasoning heuristics, as recovered by the experimental vignettes, would understand how to condition their economic vote on coalition characteristics in order to approximate a rational coalition-directed economic vote.  This seems to be the case for Danish and German voters.  German respondents with well-developed coalition reasoning heuristics were less likely than other types to exercise an economic vote for any of the major parties -- consistent with our expectation given that the incumbent was a Grand Coalition.  Those with similarly well-developed coalition heuristics in Denmark were more likely to exercise an economic vote for the two principal parties in the incumbent governing coalition and for the opposition party that would most likely form a post-election coalition government.      

%\par Given the high saliency of economic issues in the 2009 Federal election, the DCCAP asked a number of questions concerning the government's handling of the economic crisis.\fotnote{The specific wording of the question was: ......}  Again, consistent with the administrative responsibility argument developed above, we would expect the importance of these evaluations in vote choice to vary by coalition reasoning types.  Hence, the importance of this evaluation in the vote choice for either the SPD or CDU/CSU should be significantly lower for those with more sophisticated coalition reasoning.  A vote choice model similar to the standard economic vote model discussed above was estimated for the ``handling of the economy'' question.  The results are reported in  Table~\ref{tab:germany_economy} and Figure~\ref{fig:coalition_ev} and they essentially confirm the pattern we was for retrospective economic evaluations.  In the case of High coalition reasoning types there are no parties for which their estimated vote choice probability is affected by assessment of the government' handling of the economic crisis.  And in the case of the Low coalition reasoning types, vote choice for the CDU/CSU and the Linke Party is in the direction one might expect if the CDU/CSU was treated as the incumbent with Linke as the opposition.  The Low coalition type also have a positive coefficient for the SPD which is close to conventional levels of significance and is in the  

%\par In order to evaluate these propositions I have estimated multinomial logit regression models of party voter preference on wave 2 (when???) for low ( ) and high ( ) coalition types.  There are separate models for each of the policy evaluations.    


\section{Conclusion}

%\par Interestingly, until recently most empirical models of vote choice in coalition government contexts assumed voters essentially ignored the fact that election outcomes were then subject to negotiations amongst the parties in order to form a governing coalition.  Formal models of voting behaviour, of course, suggest that they should not ignore this information.  Recently there has appeared a number of promising efforts to incorporate post-election coalition bargaining into the theoretical and empirical vote utility function. But both the formal and empirical efforts in this regard have lacked an understanding of precisely how voters reason, or think, about the coalition formation process.  In particular, what heuristics do the typical voters employ to anticipate the coalitions that form post-election?  This essay provides what we believe are novel insights into this puzzle.

\par This essay proposes an experimental method for recovering the heuristics that voters employ in order to anticipate coalitions that are likely to form as a result of an election outcome.  We examine two contexts where multi-party coalition governments are the norm - Denmark and Germany -- and a third where they are decidedly not the norm -- the U.K.  A large sample of respondents in internet panel surveys are asked to identify likely coalition formations given hypothetical election outcomes that result in different hypothetical seat allocations to parties that are located along different positions on the left-right ideological continuum. 

\par We contend, first, that there are identifiable coalition formation heuristics employed by voters in contexts, such as Denmark and Germany, where such reasoning is useful for exercising a rational vote decision.  And there are contexts, such as the U.K., where, until recently, they have been of little value for making rational voting decisions.  The complexity of coalition formation and governance varies by contexts which in turn affects the volume of information about coalition governance to which the electorate is exposed.  As a result we argue that coalition reasoning skills should be particularly high in Denmark where coalition governance is complex; moderately high in Germany where coalition outcomes are relatively stable and uncomplicated; and very low in the UK where coalition governance, until very recently has not existed.

\par Both of these contentions are supported by the data.  First, we identify a number of heuristics that respondents seem to employ when asked to anticipate likely coalition outcomes.  Second, we establish that very high proportions of the Danish electorate employ such heuristics; there is evidence of moderately high levels of such coalition reasoning in Germany; and the UK respondents give little evidence of any aptitude for coalition reasoning.  A very basic heuristic is the ability to understand the notion of a majority coalition:  In Germany about one-quarter of the respondents consistently choose majority coalitions in the experimental vignettes while almost 40 percent of the Danish respondent exhibit such consistent behaviour.  Just under 40 percent of German respondents, but over 60 percent of Danish respondents, consistently selected a connected coalition.  Ideological proximity was a consistent heuristic for over 40 percent of the Danes and for about 30 percent of the Germans. Formateur status consistently informed the choices of about 25 percent of Danes and less than 10 percent of Germans.  Surprisingly, the incumbent status of parties does not seem to be an heuristic employed in anticipating coalition formations.

% Subjects also appeared employ a ``connected'' heuristic, i.e., favouring coalition partners that were ideologically connected -- just under 20 percent of the respondents employed this heuristics consistently in all their choices.  Similarly, just under 20 percent of the respondents consistently predicted a coalition would form that included the formateur party.  Experimental treatments were implemented in order to identify whether ideological distance affected the coalitions respondents' expected coalition formation -- this heuristic consistently shaped the predictions of about 30 percent of the respondents.  Hence, between 20 and 30 percent of the German respondents employ one or more of these heuristics when anticipating the type of coalition likely to form after an election.

%\par An analysis of the choices made by the respondents in the different treatments suggests three coalition types and provides some insight into their distribution in the population.  Approximately 60 percent of the German population has what I call low coalition reasoning -- at best they employ the simplest of heuristics, i.e., they recognize the concept of a majority coalition.  But they are consistently incapable of applying any of the other heuristics I identified in order to predict coalition outcomes. This leaves forty percent of the German population that exhibits moderate to high levels of coalition reasoning: just over 30 percent fall in the high coalition reasoning category while just under 10 can be characterized as having high coalition reasoning skills.  Those in the high reasoning category essentially apply all of the coalition reasoning heuristics across the treatments to which they were exposed.  This provides some sense of what portion of the population -- or the German population at least -- are likely to made choices consistent with coalition-directed models of voting behaviour.      

\par Within both the Danish and German populations there is considerable heterogeneity with respect to coalition heuristics:  some respondents exhibit very well-developed coalition reasoning heuristics while others exhibit essentially none.  A final empirical analysis examines whether those with more well-developed heuristics are more likely than those with less-developed heuristics to exercise a coalition-directed vote.  German respondents with well-developed heuristics were less likely than other types to exercise an economic vote for any of the major parties -- consistent with our expectation given that the incumbent was a Grand Coalition.  As we expected, in Denmark those with well-developed heuristics were more likely to exercise an economic vote for the two principal parties in the incumbent governing coalition and for the opposition party that would most likely form a post-election coalition government.    

\par Hence we find relatively high levels of sophisticated heuristics regarding coalition formation and governance in multi-party coalition governments contexts where they are useful for exercising a rational voting decision.  And in single-party government contexts we find little evidence of sophisticated coalition reasoning heuristics.  This provides an important empirical micro-foundation for models of coalition-directed voting in which vote choice is conditioned on the likely coalition outcomes after an election

%\par When the election results were announced after both the 2009 German and 2010 British elections, the voters did not know what government was elected - in both cases they had to wait weeks before a governing coalition was agreed upon.  These immediate vote tallies of course are important but rational choice theories suggest that voters are more concerned with the actual governing coalition that forms after the election results are announced.  I believe this is in fact the case for many voters in contexts that have a history of coalition government.  This implies that large numbers of voters exercise a coalition-directed vote rather than a sincere party-directed vote.  And to the extent that this is the case our current theoretical and empirical approaches to explaining vote choice need re-thinking.

%\par Most importantly, if voters are actually conditioning their vote choice on their expectations of what happens in post-election coalition bargaining then most of the vote choice models in the literature are misspecified. In this essay I briefly review recent attempts to specify a coalition-directed economic vote model \citep{DuchStevenson2008} and a coalition-directed ideology vote model \citep{Duchetal2010}.  And there are others; most notably \citep{Kedar2009}.

%\par A second important challenge here is developing estimation strategies that distinguish sincere from coalition-directed voting.  As \citet{Duchetal2010} point out, both considerations are likely to shape vote choice simultaneously -- the challenge is to identify, for any individual or population, the independent contribution, and hence the relative importance, of one versus the other.  In this essay I briefly describe our efforts in this regard using large-N observational studies \citep{Duchetal2010}.

%\par The focus of this essay has been on a third challenge associated with the coalition-directed vote: Theories of coalition-directed voting suggest that voters have reasonably well developed levels of reasoning and knowledge about coalitions and the coalition formation process.  But do they? This essay has proposed a strategy for recovering, using incentive-compatible experimental vignettes, coalition reasoning skills. Additionally, it proposes fairly standard measures of knowledge of coalition composition and secondly of administrative responsibility.  All three of these dimensions define savy versus non-savy coalition reasoning.  As we might expect, given their histories of coalition government, the Germans score significantly higher on these measures than the British.  Of particular theoretical concern here is whether, as the coalition-directed voting theories would predict, voters who exercise a coalition-directed vote display more well-developed coalition reasoning skills.  The preliminary evidence presented here suggests in fact that this is the case, providing further micro-evidence for the notion that there is in fact a coalition-directed vote.

\par These findings provide the micro-foundations for the coalition-directed vote.  Like many other situations in which individuals are confronted with choice problems that entail high information costs and challenging cognitive processing, voters in contexts with coalition governments employ heuristics that result in decisions that approximate those of full information rational voting.  The first empirical challenge that we addressed in this essay is precisely identifying the heuristics voters employ to anticipate coalition formation after an election.  We test two subsequent empirical implications that are implied by a heuristic model of coalition-directed voting.  Voters should invest in, and develop, coalition heuristics that are appropriate to the political context which is what we find.  In political contexts with relatively complex coalition bargaining and formations we find highly developed coalition heuristics and much less developed ones in contexts where coalition politics is less complex.  Finally, voters invest in heuristics that allow them to approximate full information rational voting decisions.  This implies that the voters who exhibit well-developed levels of the coalition heuristics, as recovered in our experiments, should make voting decisions that approximate rational coalition-directed voting.  Our empirical results confirm that those with well-developed heuristics are more likely to exercise a rational coalition-directed economic vote than is the case for those with less-developed heuristics.



\newpage
\singlespace
\bibliography{dave}

\newpage

\end{document}

%\section{Danish Post-Script}

%\par There is a comparative post-script to this discussion of coalition reasoning in the German context: we are currently conducting similar experiments in other countries.  A similar experiment was conducted in the UK and we are currently conducting one in Denmark.  The Danish results are from an initial pre-test of 100 respondents but they are of particular interest because they are being conducted a context with a history of multiparty coalition governments, like Germany, although with a strong history of minority governments which is not the case in Germany.  This post-script presents both the results from a similar experiment in the UK and those for the Danish pre-test.  These are exactly the same coalition scenario treatments that were presented earlier for the German case.  We will briefly compare here the Danish results with those from Germany.

%\par It should be pointed out that the Danish results are from a pre-test with subjects who scored high on educational and other socio-economic measures.  The results confirm that citizens in Denmark conform surprisingly well to the coalition sophisticates we identified in Germany; undoubtedly the high socio-economic status of the pretest sample is at least partially an explanation although it will be interesting to see whether these same patterns hold up when the study is conducted with a full representative sample of the Danish public.  As you can see from Figure~\ref{fig:majority_dk} to Figure~\ref{fig:treat4B_dk}, the Danish respondents consistently select the coalition outcomes that are consistent with the high coalition reasoning types we identified in the German case:  they exhibited higher rates of consistently selecting majority coalitions;  more likely to employ a ``connected'' heuristic, i.e., favouring coalition partners that were ideologically connected; predicted, with higher frequency, that a coalition would form that included the formateur party; and conditioned, as expected (and more frequently than the German respondents), their predicted coalition formation on ideological distance.  

%\par These Danish preliminary results provide support for two very tentative conclusions: First they seem to validate the general heuristics that were identified in the German case -- these do seem to be identifiable coalition reasoning heuristics and they appear to be employed in diverse national contexts.  Second, the types of individuals who we would expect to exhibit sophisticated coalition reasoning in fact do -- this Danish pretest subject pool, that exhibited such highly sophisticated coalition reasoning, had very highs scores on educational and socio-economic measures. 

%
%\begin{figure}[p]
%\caption{Frequency with which Majority Governments Selected}
%\centering
%\subfig[Germany]{\includegraphics[width=.5\textwidth]{DE_majority.png}} \\
%\subfig[Denmark]{\includegraphics[width=.5\textwidth]{DK_majority.png}}
%\label{fig:majority_dk}
%\end{figure}
%
%\clearpage
%
%\begin{samepage}
%\begin{figure}[h]
%\caption{Treatment 2A}\label{fig:treat2A_dk}
%\centerline{\scalebox{0.75}{\includegraphics[width=\textwidth]{images/treat2A.png}}}
%\end{figure}
%
%\begin{table}[h!]
%%\caption{Results for Treatment 2A}\label{tab:treat2A}
%\begin{center}
%\begin{footnotesize}
%\begin{tabular}{c |c c c c c c c | c}
%\textbf{} & \textbf{A} & \textbf{AB} & \textbf{ABC} & \textbf{AC} & \textbf{B} & \textbf{BC} & \textbf{C} & \textbf{\textit{N}} \\
%\hline\hline
%\textbf{Germany} & 17.61 & 37.4 & 1.79 & 11.39 & 4.35 & 23.6 & 3.85 & \textit{4021} \\
%%\textbf{UK} & 54.42 & 10.32 & 0.38 & 3.03 & 6.77 & 8.85 & 16.23 & \textit{2113} \\
%\textbf{Denmark} & 7.08 & 53.1 & 1.77 & 1.77 & 2.65 & 30.97 & 2.65 & \textit{113} \\
%\hline
%\end{tabular}
%\end{footnotesize}
%\end{center}
%\end{table}
%\end{samepage}
%
%\begin{samepage}
%\begin{figure}[h]
%\caption{Treatment 2B}\label{fig:treat2B_dk}
%\centerline{\scalebox{0.75}{\includegraphics[width=\textwidth]{images/treat2B.png}}}
%\end{figure}
%
%\begin{table}[h!]
%%\caption{Results for Treatment 2B}\label{tab:treat2B}
%\begin{center}
%\begin{footnotesize}
%\begin{tabular}{c |c c c c c c c | c}
%\textbf{} & \textbf{A} & \textbf{AB} & \textbf{ABC} & \textbf{AC} & \textbf{B} & \textbf{BC} & \textbf{C} & \textbf{\textit{N}} \\
%\hline\hline
%\textbf{Germany} & 19.12 & 20.34 & 1.37 & 9.05 & 3.93 & 41.9 & 4.28 & \textit{4021} \\
%%\textbf{UK} & 41.4 & 2.7 & 0.24 & 1.94 & 7.06 & 22.6 & 24.06 & \textit{2111} \\
%\textbf{Denmark} & 5.31 & 10.62 & 0 & 0 & 0.88 & 79.65 & 3.54 & \textit{113} \\
%\hline
%\end{tabular}
%\end{footnotesize}
%\end{center}
%\end{table}
%\end{samepage}
%
%\clearpage
%
%
%\begin{samepage}
%\begin{figure}[h]
%\caption{Treatment 3A}\label{fig:treat3A_dk}
%\centerline{\scalebox{0.75}{\includegraphics[width=\textwidth]{images/treat3A.png}}}
%\end{figure}
%
%\begin{table}[h!]
%%\caption{Results for Treatment 3A}\label{tab:treat3A}
%\begin{center}
%\begin{footnotesize}
%\begin{tabular}{c ||c c c c c c c c | c}
%\textbf{} & \textbf{AB} & \textbf{ABC} & \textbf{B} & \textbf{BC} & \textbf{BD} & \textbf{C} & \textbf{CD} & \textbf{D} & \textbf{\textit{N}} \\
%\hline\hline
%\textbf{Germany} & 24.27 & 0.1 & 13.93 & 26.59 & 18.9 & 1.64 & 4.4 & 2.64 & \textit{4021} \\
%%\textbf{UK} & 20.1 & 0.19 & 46.04 & 4.79 & 5.07 & 3.22 & 2.04 & 9.53 & \textit{2109} \\
%\textbf{Denmark} & 73.45 & 10.62 & 5.31 & 2.65 & 2.65 & 0 & 0 & 0.88 & \textit{113} \\
%\hline\hline
%\end{tabular}
%\end{footnotesize}
%\end{center}
%\end{table}
%\end{samepage}
%
%\begin{samepage}
%\begin{figure}[h]
%\caption{Treatment 3B}\label{fig:treat3B_dk}
%\centerline{\scalebox{0.75}{\includegraphics[width=\textwidth]{images/treat3B.png}}}
%\end{figure}
%
%\begin{table}[h!]
%%\caption{Results for Treatment 3B}\label{tab:treat3B}
%\begin{center}
%\begin{footnotesize}
%\begin{tabular}{c ||c c c c c c c c | c}
%\textbf{} & \textbf{AB} & \textbf{ABC} & \textbf{B} & \textbf{BC} & \textbf{BCD} & \textbf{BD} & \textbf{CD} & \textbf{D} & \textbf{\textit{N}} \\
%\hline\hline
%\textbf{Germany} & 10.07 & 1.02 & 14.52 & 36.98 & 4.53 & 20.49 & 3.76 & 2.59 & \textit{4021} \\
%%\textbf{UK} & 2.89 & 0.43 & 44.76 & 11.57 & 6.26 & 11.38 & 3.89 & 10.05 & \textit{2109} \\
%\textbf{Denmark} & 9.73 & 2.65 & 3.54 & 8.85 & 37.17 & 31.86 & 0 & 1.77 & \textit{113} \\
%\hline\hline
%\end{tabular}
%\end{footnotesize}
%\end{center}
%\end{table}
%\end{samepage}
%
%
%\clearpage
%
%\begin{samepage}
%\begin{figure}[h]
%\caption{Treatment 4A}\label{fig:treat4A_dk}
%\centerline{\scalebox{0.75}{\includegraphics[width=\textwidth]{images/treat4A.png}}}
%\end{figure}
%
%\begin{table}[h!]
%%\caption{Results for Treatment 4A}\label{tab:treat4A}
%\begin{center}
%\begin{footnotesize}
%\begin{tabular}{c ||c c c c c c c | c}
%\textbf{} & \textbf{ABC} & \textbf{B} & \textbf{BCD} & \textbf{BD} & \textbf{CD} & \textbf{CDE} & \textbf{D} & \textbf{\textit{N}} \\
%\hline\hline
%\textbf{Germany} & 6.09 & 2.64 & 3.93 & 24.42 & 30.02 & 7.31 & 12.78 & \textit{4021} \\
%%\textbf{UK} & 3.13 & 5.93 & 1.47 & 5.31 & 13.47 & 4.55 & 49.29 & \textit{2108} \\
%\textbf{Denmark} & 22.12 & 0.88 & 2.65 & 3.54 & 49.56 & 16.81 & 2.65 & \textit{113} \\
%\hline\hline
%\end{tabular}
%\end{footnotesize}
%\end{center}
%\end{table}
%
%\end{samepage}
%
%\begin{samepage}
%\begin{figure}[h]
%\caption{Treatment 4B}\label{fig:treat4B_dk}
%\centerline{\scalebox{0.75}{\includegraphics[width=\textwidth]{images/treat4B.png}}}
%\end{figure}
%
%\begin{table}[h!]
%%\caption{Results for Treatment 4B}\label{tab:treat4B}
%\begin{center}
%\begin{footnotesize}
%\begin{tabular}{c ||c c c c c c c | c}
%\textbf{} & \textbf{ABC} & \textbf{B} & \textbf{BD} & \textbf{CD} & \textbf{CDE} & \textbf{D} & \textbf{DE} & \textbf{\textit{N}} \\
%\hline\hline
%\textbf{Germany} & 24.57 & 3.01 & 18.85 & 13.95 & 5.07 & 11.32 & 10.1 & \textit{4021} \\
%%\textbf{UK} & 17.84 & 12.48 & 4.41 & 2.42 & 1.52 & 40.32 & 9.77 & \textit{2108} \\
%\textbf{Denmark} & 76.11 & 0.88 & 1.77 & 6.19 & 7.08 & 0.88 & 1.77 & \textit{113} \\
%\hline\hline
%\end{tabular}
%\end{footnotesize}
%\end{center}
%\end{table}
%
%\end{samepage}

\end{document}





  
\subsection{Coalition Reasoning Types}

\par The results of the reasoning experiments suggests that a large proportion of the German electorate, and a quite small one in the UK, are comfortable engaging in this coalition reasoning.  This suggests first that we ought to be able to recover a coalition reasoning measure using this, or similar, embedded survey experiment. Secondly, we should expect considerable variation within the electorate.
%And finally the analysis of this variation should help validate the measure.

\par The embedded experiments allow us to recover a measure of the individual's level of sophistication employed in reasoning about governing coalitions. Table~\ref{tab:coalition_types} summarizes how I defined the three levels of sophistication based on the responses to the experimental scenarios.  Low sophisticates couldn't identify a majority coalition in any of the simple treatments (1A or 1B). If they identified any one of these simple majority coalitions, but did not meet the criteria for a High sophisticate, then they were assigned to the Medium sophisticate category.  High sophistication identified at least 2 out of the 6 minimum winning majority coalitions (2A, 2B, 3A, 3B, 4A, 4B).  And Figure~\ref{fig:types_distribution} summarizes the distribution of coalition reasoning types in Germany and the UK.  Not surprisingly, given the previous discussion, about 45 percent of the German respondents are coalition reasoning sophisticates while only 22 percent fall in this category in the UK.  The UK has a higher percent falling in the low and medium reasoning categories.


\begin{table}[h!]
\caption{Levels of Coalition Reasoning Defined}\label{tab:coalition_types}
\begin{center}
\begin{tabular}{ld{3}d{3}d{3}}
 & \mc{Low} & \mc{Medium} \mc{High} \\
\hline\hline
Identify majority coalition -- two parties        &  no   &  yes & yes \\
                                    &    &   & \\
Identify $<$ 3 majority coalitions -- multiple parties     &  no  &  yes & yes \\
                                    &          &  &  \\
Identify $>$ 2 majority coalition -- multiple parties  &  no        &  no & yes \\
                                    &          &  & \\
\hline\hline
\end{tabular}
\end{center}
\end{table}

  \begin{figure}[h!]
    \caption{Coalition Reasoning Types: Germany (2009) and UK (2010)}\label{fig:types_distribution}
\centerline{
\includegraphics[width=7.0in]{coalition_types.pdf}}
\end{figure}

\newpage

\par I contend that in some sense the distribution of these three coalition reasoning types in the population define coalition-directed voting in the electorate.  But is this capturing anything unique about the voter population?  A reasonable suposition here is that coalition reasoning simply reflects what is typically characterized as voter sophistication, i.e., knowledge about politics.  If this were the case, then the measure of coalition reasoning should correlate highly with standard measures of voter sophisitication.  We can think of voter sophistication as being capture by four conventional measures of socio-economic resources and interest: education, political interest, income and gender.  And I also directly measure political knowledge with four knowledge questions that were administered in Wave 4 of the survey.\footnote{The questions were as follows:   .}

\par Recall that the three coalition-reasoning types in Figure~\ref{fig:types_distribution} are based on a measure that ranges in value from 0 to 5 with high values indicating high coalition reasoning. To the extent that coalition-reasoning simply reflects sophistication, or being knowledgeable, then we should expect a high correlation between these socio-economic variables, knowledge and coalition reasoning.  Table~\ref{tab:germany_sophistication} reports the results for the OLS regressions of these variables on coalition reasoning. The results in Model 1 are as we would expect: coalition reasoning is positively correlated with education, being male, interest, and income.  But the model explains about 8 percent of the variation in coalition reasoning, suggesting these are correlates but hardly overwhelmingly strong correlates.  Model 2 adds the political knowledge variable which is highly correlated with coalition reasoning but even with this variable in the model the variance explained is about .16 which is strong but does not suggest the two variables are measuring the same underlying concept.

\begin{table}[h!]
\caption{OLS Mode of Coalition Reasoning: German CCAP Study 2009}\label{tab:germany_sophistication}
\begin{footnotesize}
\begin{center}
\begin{tabular}{ld{3}d{3}d{3}}
\textbf{Variable} & \textbf{Model 1} & \textbf{Model 2} \\
\hline\hline
\multirow{2}{*}{Education} & 0.04 & 0.03  \\
 & (0.02) & (0.03)  \\
\multirow{2}{*}{Female} & -0.31  & -0.29  \\
 & (0.06) & (0.09) \\
\multirow{2}{*}{Income} & 0.06 & 0.02 \\
 & (0.01) & (0.02)  \\
\multirow{2}{*}{Interest} & 0.25 & 0.21  \\
 & (0.04) & (0.06)  \\
\multirow{2}{*}{Political Knowledge} & & 0.34 \\
 & & (0.04) \\
\multirow{2}{*}{Constant} & 1.47 & 0.63  \\
 & (0.11) & (0.96)  \\
\hline\hline
Obs & 2,148 & 887 \\
& -3013.304 & -1217.038 \\
R-Square & 0.09 & 0.16 \\
& 15 & 25 \\
AIC & 5163.903 & 2097.101 \\
& 5245.715 & 2210.843 \\
\hline\hline
\end{tabular}
\end{center}
\end{footnotesize}
\end{table}

\clearpage


\par Figure~\ref{fig:types_distribution} suggests that there are three distinct coalition reasoning types in the population: those who essentially do not have a clue; those who are comfortable with the basic arthimetic associated with the coalition formation process; and finally, those who exhibit a sophisticated level of coalition reasoning.  In fact these three types have quite distinct characteristics. These are illustrated in the multi-nomial logit estimatation results in Table~\ref{tab:germany_coalition}.  The dependent variable in these models is the three category (High, Medium, Low) coalition reasoning variable with Medium as the referent category.  Model 1 includes the four socio-economic variables employed in Table~\ref{tab:germany_sophistication}.  The results quite clearly suggest that basic socio-economic resources (education, interet, gender, and income) are a necessary condition for Medium and High levels of coalition reasoning.  And while they clearly distinguish Medium and High from Low coalition reasoning, they do not distinguish Medium from High coalition reasoning. It would seem that there is some threshold of socio-economic resources that facilitates Medium/High coalition reasoning but that beyond this threshold these variables do not then condition whether one is a Medium versus High coalition reasoner.

\par Model 2 in Table~\ref{tab:germany_coalition} adds political knowledge to the MNL equation.  Not surprisingly political knowledge disintiguishes the Low coalition reasoning types from the Medium/High coalition types, as was the case with the other socio-economic variables in Model 1.  But political knowledge is also statistically significant for High coalition reasoners.  This suggests that while basic socio-economic resources does not distinghish High from Medium coalition reasoning, political knowledge does.  Hence there is some evidence here to suggest that the highest level of coalition reasoning is conditional on a reasonably high level of political knowledge.



\begin{table}[h!]
\caption{Multi-nomial Logit Model of Coalition Reasoning: German CCAP Study 2009}\label{tab:germany_coalition}
\begin{footnotesize}
\begin{center}
\begin{tabular}{ld{3}d{3}d{3}}
\textbf{Variable} & \textbf{Model 1} & \textbf{Model 2} \\
\hline\hline
\multirow{2}{*}{\textbf{Low Coalition Reasoning}} &  &  \\
\multirow{2}{*}{Education} & -0.14 & -0.16  \\
 & (0.04) & (0.06)  \\
\multirow{2}{*}{Female} & 0.58  & 0.34  \\
 & (0.11) & (0.18) \\
\multirow{2}{*}{Income} & -0.09 & -0.05 \\
 & (0.03) & (0.04)  \\
\multirow{2}{*}{Interest} & -0.54 & -0.47  \\
 & (0.07) & (0.12)  \\
\multirow{2}{*}{Political Knowledge} & & -0.47 \\
 & & (0.09) \\
\multirow{2}{*}{Constant} & 1.39 & 2.71  \\
 & (0.22) & (0.40)  \\
\hline
\multirow{2}{*}{\textbf{High Coalition Reasoning}} &  &  \\
\multirow{2}{*}{Education} & -0.03 & -0.05 \\
 & (0.04) & (0.07) \\
\multirow{2}{*}{Female} & -0.07 & -0.17 \\
 & (0.13) & (0.21) \\
\multirow{2}{*}{Income} & 0.05 & -0.003 \\
 &
 (0.03) & (0.04) \\
\multirow{2}{*}{Interest} & 0.01 & -0.02 \\
 & (0.08) & (0.14) \\
 \multirow{2}{*}{Political Knowledge} &  & 0.34 \\
 & - & (0.12) \\
\multirow{2}{*}{Constant} & -0.94 & -1.73 \\
 & (0.26) & (0.54) \\
\hline\hline
Obs & 2,148 & 886 \\
& -3013.304 & -1217.038 \\
ll(model) & -2566.951 & -1023.551 \\
& 15 & 25 \\
AIC & 5163.903 & 2097.101 \\
& 5245.715 & 2210.843 \\
\hline\hline
\end{tabular}
\end{center}
\end{footnotesize}
\end{table}

\clearpage


\subsection{Coalition-reasoning types anticipate coalition outcomes?}


\par Again in either of these coalition-directed vote utility functions voters need to anticipate the different types of coalitions that could form after an election in addition to having a sense of the likelihood of different coalitions forming. This is the notion from \citet{DuchStevenson2008} that voters have preferences over distributions of administrative responsibility -- and the notion that their vote would be pivotal in electing (or defeating) a particular distribution of administrative responsibility.  In \citet{Duchetal2010}, the $\gamma$ term in Equation \ref{eq:20} implies that voters know with what probabilities parties are likely to enter a post-election governing coalition.

\par Voters who are adept at anticipating the likely coalition formations in the experiment described earlier are precisely those who I would expect to anticipate post-election coalition outcomes.  In Wave 2 of the German survey which occurred three weeks before the election, respondents were asked to indicate which parties would likely form the post-election governing coalition.  Approximately 35 percent of the respondents correctly identified the CDU/CSU-FDP coalition as the governing coalition that would most likely form after the election.  Table~\ref{tab:germany_anticipate} presents  he results of a logit regression of the correct anticipation of the post-election governing coalition on dummy variables for coalition reasoning types plus the socio-economic and knowledge variables we encountered earlier.  Clearly those with Medium or High levels of coalition reasoning are significantly more likely to correctly anticipate the governing coalition that forms after the election.  But the magnitude of the Medium and High dummy coefficients are not significantly different from each other suggesting that the relevant threshold here, in terms of anticipating post-election governing coalitions is whether one is in the Low versus Medium or High reasoning categories.


\begin{table}[h!]
\caption{Logit Model of Coalition Anticipation: German CCAP Study 2009}\label{tab:germany_anticipate}
\begin{footnotesize}
\begin{center}
\begin{tabular}{ld{3}d{3}d{3}}
\textbf{Variable} & \textbf{Model 1} & \textbf{Model 2} \\
\hline\hline
\multirow{2}{*}{Education} & 0.05 &   \\
 & (0.06) &   \\
\multirow{2}{*}{Female} & -0.61  &   \\
 & (0.16) &  \\
\multirow{2}{*}{Income} & 0.08 & \\
 & (0.04) &   \\
\multirow{2}{*}{Interest} & 0.34 &   \\
 & (0.11) &  \\
\multirow{2}{*}{Political Knowledge} & .23 &  \\
 & (0.09) &  \\
 \multirow{2}{*}{Medium Sophistication} & 1.04 &  \\
 & (0.19) &  \\
 \multirow{2}{*}{High Sophistication} & 1.26 &  \\
 & (0.23) &  \\
\multirow{2}{*}{Constant} & -2.92 &   \\
 & (0.40) &   \\
\hline\hline
Obs & 2,148 & 887 \\
& -3013.304 & -1217.038 \\
R-Square & 0.09 & 0.16 \\
& 15 & 25 \\
AIC & 5163.903 & 2097.101 \\
& 5245.715 & 2210.843 \\
\hline\hline
\end{tabular}
\end{center}
\end{footnotesize}
\end{table}

\clearpage


%\subsection{Measuring knowledge of coalition composition}

%\par Again in either of these coalition-directed vote utility functions voters need to know which parties constitute the governing coalition and to anticipate the different types of coalitions that could form after an election in addition to having a sense of the likelihood of different coalitions forming. This is the notion from \citet{DuchStevenson2008} that voters have preferences over distributions of administrative responsibility -- and the notion that their vote would be pivotal in electing (or defeating) a particular distribution of administrative responsibility.  In \citet{Duchetal2010} its the $\gamma$ term in Equation \ref{eq:20} that assumes voters know what parties make up the incumbent governing coalition and which are likely to enter a post-election governing coalition.

%\par The German DCCAP survey included two sets of knowledge questions regarding the composition of both the out-going and, in the post-election study, the in-coming coalition.  Figure~\ref{fig:know_germany} summarizes the responses.  Most repondents named the CDU (90 percent) and the SPD (80 percent) as members of the out-going Grand Coalition, although a surprisingly small, 60 percent, named the CSU.  And 90 percent of respondents named the FDP and the CDU as members of the new post-election governing coalition.  Responses to this coalition knowledge question will be used in constructing the final measure of coalition sophistication.

%\begin{figure}[p]
%\caption{Knowledge of out-going and in-coming German Coalitions}\label{fig:know_germany}

%\centerline{\includegraphics[width=\textwidth]{knowledge_germany.pdf}}

%\end{figure}

%\clearpage


%\subsection{Measuring administrative responsibility reasoning}

%\par The final and probably most difficult to measure assumption regarding the coalition-direct vote is this notion that voters have preferences over distributions of administrative responsibility (which is what determines policy outcomes) rather than over parties.  This is the $\lambda$ term in \citet{DuchStevenson2008} and the $Z$ term in \citet{Duchetal2010}.  For this particular analysis we will not be incorporating this measure in the analysis.

%\subsection{Coalition reasoning and knowledge}

%\par What is the relationship between these two coalitions skill sets -- knowledge and reasoning?

%\par We begin with the U.K.  Is this a reasonable strategy for recovering underlying sophistication regarding governing coalitions (which parties are likely to coalesce, longevity, etc.)?  Table~\ref{tab:types_validity} provides some insight into the validity of the reasoning metric.  The High Sophisticates are clearly disproportionately better educated; better able to anticipate the coalition government that formed after the 2010 British Election; and better informed about politics generally.

%\begin{table}[h!]
%\caption{Levels of Coalition Reasoning Defined}\label{tab:types_validity}
%\begin{center}
%\begin{tabular}{ld{3}d{3}d{3}}
% & \mc{Low} & \mc{Medium} \mc{High}  \\
%\hline\hline
% High Education        &  28\%   & 34\%  & 40\% \\
%                                    &    &   & \\
%Correctly Predict Coalition Government     &    &   &  \\
%                                    &          &  &  \\
%High Political Knowledge  &         &  &  \\
%                                    &          &  & \\
%\hline\hline
%\end{tabular}
%\end{center}
%\end{table}




\section{Micro-foundations of the coalition-directed vote}

\par At the outset of this essay I laid out the theoretical foundations for a coalition-directed vote.  I provided a description of the micro-foundations of the coalition-directed economic and ideological vote and pointed out that both models assume voters engage in reasonably sophisticated coalition reasoning.  The previous sections suggests that, yes, there are voters that have this profile although there clearly are many who do not.  To the extent that voters vary significantly in terms of the sophistication of their coalition knowledge and reasoning, I would expect these sophistication measures to distinguish those who in fact are exercising a coalition-directed vote from those who are not.  This section will test this argument.  The coalition-savy measure is simply the sum of the respondent scores on the coalition knowledge measure (which ranges in values from 0 to 3) and the coalition reasoning measure (which ranges from 1 to 3).

\subsection{The Coalition-directed Economic Vote in Germany and the UK}

\par I begin with a comparison of the coalition-directed economic vote of those who score high versus those who score low on this measure of savy coalition reasoning.  The 2009 German Federal election provides a particularly interesting context in which to explore whether my measure of coalition reasoning has any empirical foundation.  Because of the unique outcome of the 2010 British election, it also provides some insight into coalition reasoning and the economic vote.

\par The 2009 German election represents a theoretical anomaly for students of the economic vote because the incumbent government in that election consisted of the two parties making up the Grand Coalition, the CDU/CSU and the SPD.  Classic moral hazard models of the economic vote presume, at a minimum, well-defined incumbent parties that will either be rewarded or punished as a function of perceived economic performance.  A Grand Coalition does not provide such a clear dichotomy and hence at best these models simply predict a muted economic vote in such contexts \citep{Anderson1995,Powelletal1993}.


%It is reasonable to characterize the SPD as the junior partner in this coalition -- the Prime Ministerial party, and hence senior partner in the coalition, was represented by the CDU/CSU.  In these situations, coalition-directed voters in the \citet{DuchStevenson2008} model identify the likely alternative cabinets to the incumbent cabinet.  And parties can be members of both alternative and incumbent cabinets.  So voters could consider their vote as being pivotal in re-electing the incumbent cabinet or electing an alternative cabinet.  So in the case of a Grand Coalition the voters need to decide what is an alternative coalition -- would a CDU/CSU + FDP coalition be considered an alternative coalition or an incumbent coalition?  Would a SPD + Linke coalition be considerated an alternative or an incumbent coalition?

\par One feature of the \citet{DuchStevenson2008} model is particularly appropriate for modeling a ``Grand Coalition" economic vote because it assumes that voters have preferences over distributions of responsibilities rather than parties per se (which is the weakness of the typical moral hazard model).  The theory assumes that voters know which parties are in the incumbent cabinet and which parties are in any alternative cabinet they are considering.  Voters are assumed to believe each party's contribution to economic outcomes is proportional to the parties' shares of responsibility.  Hence increased cabinet responsibility should enhance a party's economic vote. This implication stems directly from the \citet{DuchStevenson2008} assumption that voters believe the impact of a vote for party $j$ on the selection of a distribution of responsibility $\lambda$ is proportional to party $j$'s share of responsibility in $\lambda$ and that they want to vote rationally to maximize their impact."

\par The analysis of the economic vote in the 2009 German election, that occurred after four years of a Grand Coalition consisting of the CDU/CSU and SPD, allows a direct test of the \citet{DuchStevenson2008} hypothesis regarding this ``distribution of responsibility" reasoning.  First, the incumbent coalition consisted of the the two major parties in which cabinet portfolios were divided equally amongst the two parties (roughly eight ministries for each party).  Second, Duch and Stevenson predicted economic vote would be contingent on voter expectations regarding the coalitions that would form after the election:

\begin{itemize}
\item \emph{Naive}: A naive expectation would be that one of the two parties would govern on its own (or effectively on its own) -- in this case, given the equal distribution of responsibility in the incumbent cabinet, neither party should receive an economic vote.
\item \emph{Sophisticated}: A plausible coalition outcome would have been either 1) the CDU/CSU and FDP in coalition (which is the actual coalition that formed); or 2) the SPD in coalition with the Greens and the Linke parties.  The Duch and Stevenson model would predict the following: 1) coalition directed voters who were happy with the economy should vote for the CDU/CSU or FDP parties -- the likely resulting coalition would result in the ``incumbent" CDU/CSU having about 85 percent of the distribution of responsibility; 2) coalition directed voters who were unhappy with the economy should vote for the SPD, Greens or Linke parties (in effect, whichever party vote they thought would maximize the likelihood of this coalition forming) -- this coalition would have about half of the responsibility accounted for by ``opposition" parties that were not in the Grand Coalition (the Greens obtained about 10 percent of the vote; the Linke 12 percent; and the SPD about 23 percent).\footnote{Another other plausible coalition outcome was another Grand Coalition but this was one that was heavily discounted by most observers and by the voters.  In this case the economic vote for both the CDU/CSU and SPD would be large, similarly signed and indistinguishable in magnitude.}
\end{itemize}

\noindent Hence the theoretical expectation here is that coalition-oriented voters should reward the CDU/CSU and FDP parties for good perceived economic outcomes but should prefer the SPD, Green, and Linke parties if they perceive the economy as doing poorly.  This is precisely what happens in the 2009 German election.

\par Table~\ref{tab:germany_economy} summarizes the estimation of a simple multi-nomial logit economic voting model.  In Model 1, the dependent variable is the vote choice of German CCAP respondents in the second wave of the panel (July, 2009) with national retrospective evaluations as the independent variable and controlling simply for left-right self identification.  The economic evaluation variable (which is coded in such a fashion that high values indicate dissatisfaction with the retrospective economy) has a positive and significant coefficient for all of the party choices, including the SPD.  Note that the CDU/CSU vote option is the referent category in the estimation.  The results are consistent with the Duch and Stevenson model which predicts that voters unhappy with the economy would cast a vote that would maximize the chances of the election of a coalition -- or a distribution of responsibility -- that had an ``opposition" distribution of responsibility (i.e., parties that were not in the incumbent governing coalition).  As was pointed out above, the most plausible ``opposition" distribution of responsibility was the SPD, Green and Linke -- hence we would expect a positive coefficient on the economic evaluations for these parties.

\par A stronger test of the argument would be to explore the nature of the economic vote for those I characterized earlier as having relatively savy coalition reasoning skills versus those who did not.  Savy coalition reasoning skills here is defined as the sum of the coalition knowledge variable and the coalition reasoning variable.  This variable is interacted with the economic evaluation variable.  Model 2 reports the estimates of a model that includes the coalition variable and its interaction with economic evaluations.  The results are reported in Table~\ref{tab:germany_economy}.  Again, these results are precisely what the Duch and Stevenson model predicts.  Note that for the SPD choice, the coefficient on economic evaluations is statistically insignificant while the coalition-savvy interaction term has a statistically significant coefficient.  This suggests that the coalition-savvy respondents are behaving according to the Duch and Stevenson model by favouring parties that will likely help form an ``opposition" distribution of responsibility.  But of course the Duch and Stevenson model assumes coalition reasoning skills and knowledge about the likely coalition compositions after the election.  For those without those skills, one expectation might be the naive one desscribed above and treat each party as likely to form a separate government on its own -- in which case the expectation is that there would be no economic vote for either party which is precisely what we see in Model 2 -- for the non-sophisticates the coefficient on economic evaluations is effectively zero.

\begin{table}[h!]
\caption{Coalition-directed economic vote: German CCAP Study 2009}\label{tab:germany_economy}
\begin{footnotesize}
\begin{center}
\begin{tabular}{ld{3}d{3}d{3}}
\textbf{Variable} & \textbf{Model 1} & \textbf{Model 2} \\
\hline\hline
\multirow{2}{*}{\textbf{SPD}} &  &  \\
\multirow{2}{*}{Prosp. Nat'l} & 0.323 & -0.031 \\
 & (0.092) & (0.179) \\
\multirow{2}{*}{LR place} & -0.682 & -0.669 \\
 & (0.05) & (0.08) \\
\multirow{2}{*}{ev savy} & - & 0.115 \\
 & - & (0.037) \\
\multirow{2}{*}{savy} & - & -0.149 \\
 & - & (0.171) \\
\multirow{2}{*}{Constant} & 3.001 & 2.993 \\
 & (0.406) & (0.961) \\
\hline
\multirow{2}{*}{\textbf{FDP}} &  &  \\
\multirow{2}{*}{Prosp. Nat'l} & 0.3 & 0.154 \\
 & (0.085) & (0.169) \\
\multirow{2}{*}{LR place} & -0.075 & -0.095 \\
 & (0.041) & (0.069) \\
\multirow{2}{*}{ev savy} & - & 0.012 \\
 & - & (0.034) \\
\multirow{2}{*}{savy} & - & 0.165 \\
 & - & (0.152) \\
\multirow{2}{*}{Constant} & -0.498 & -0.967 \\
 & (0.386) & (0.905) \\
\hline
\multirow{2}{*}{\textbf{Green}} &  &  \\
\multirow{2}{*}{Prosp. Nat'l} & 0.427 & -0.061 \\
 & (0.103) & (0.2) \\
\multirow{2}{*}{LR place} & -0.742 & -0.787 \\
 & (0.056) & (0.09) \\
\multirow{2}{*}{ev savy} & - & 0.092 \\
 & - & (0.041) \\
\multirow{2}{*}{savy} & - & -0.038 \\
 & - & (0.195) \\
\multirow{2}{*}{Constant} & 2.543 & 3.059 \\
 & (0.447) & (1.083) \\
\hline
\multirow{2}{*}{\textbf{Linke}} &  &  \\
\multirow{2}{*}{Prosp. Nat'l} & 1.012 & 0.448 \\
 & (0.102) & (0.198) \\
\multirow{2}{*}{LR place} & -1.048 & -1.099 \\
 & (0.057) & (0.094) \\
\multirow{2}{*}{ev savy} & - & 0.13 \\
 & - & (0.041) \\
\multirow{2}{*}{savy} & - & -0.203 \\
 & - & (0.204) \\
\multirow{2}{*}{Constant} & 2.293 & 3.161 \\
 & (0.441) & (1.094) \\
\hline\hline
Obs & 1727 & 699 \\
& -3013.304 & -1217.038 \\
ll(model) & -2566.951 & -1023.551 \\
& 15 & 25 \\
AIC & 5163.903 & 2097.101 \\
& 5245.715 & 2210.843 \\
\hline\hline
\end{tabular}
\end{center}
\end{footnotesize}
\end{table}


%\multirow{2}{*}{\textbf{other}} &  &  \\
%\multirow{2}{*}{Prosp. Nat'l} & 0.951 & 0.192 \\
%& (0.114) & (0.229) \\
%\multirow{2}{*}{LR place} & -0.311 & -0.359 \\
%& (0.057) & (0.099) \\
%\multirow{2}{*}{ev savy} & - & 0.167 \\
%& - & (0.049) \\
%\multirow{2}{*}{savy} & - & -0.596 \\
%& - & (0.231) \\
%\multirow{2}{*}{Constant} & -1.967 & 0.85 \\
%& (0.527) & (1.217) \\

\clearpage


\subsection{The Coalition-directed ideological vote in Germany and the UK}


%there were roughly two Accordingly the distribution of incumbent performance responsibility for each party was equally shared.  The Duch and Stevenson predicted vote for each party seems to follow: 1) voters gave a high probability to the formation of a CDU/CSU + FDP coalition; 2) they also gave a higher probability to a SPD + Linke coalition as opposed to another Grand Coalition of CDU/CSU + SPD;  3) hence a vote for SPD was more likely to be pivotal in electing an SPD+Linke coalition than the incumbent coalition; 4) the SPD+Linke distribution of responsibility would have a large enough alternative ``distribution of responsibility" in it (because the Linke did relatively well electorally) to counterbalance, in their expected utility calculation, the relatively small probability of their vote re-electing a Grand Coalition.  Duch and Stevenson predict that when economy is doing poorly, and only plausible alternatives are parties in incumbent coalition, voters will vote for party in incumbent coalition that had smallest distribution of responsibility, i.e., SPD (see page 220 in pdf manuscript). [ALSO WORK THROUGH EQUATION 8.40 IN PDF MANUSCRIPT -- essential message here is that CDU/CSU had more administrative responsibility in Grand Coalition]

%Evidence that voters are knowledgeable about these probabilities is available for a number of national contexts \citep{BargstedKedar2009,DuchStevenson2008,GalenHolsteyn2003,Blaisetal2006}. A number of important factors contribute to voter information levels regarding coalition formation patterns.  One is simply the relative stability of coalition configurations that typically form in any single country and the fact that these coalitions are not particularly complex in terms of numbers of parties. \citep{ArmstrongDuch2010} document this stability in their analysis of coalition formation patterns in 30 countries over the period 1960 to the present.  They find that the effective number of parties in a typical coalition government is approximately 3.5 and that the exact same coalitions are returned to power with relatively high frequency.  Hence, the history of coalition formation patterns can be very informative to voters' efforts to anticipate post-election coalition formation outcomes.  A second factor is publicly available polling results that inform voters about the relative electoral strength of competing parties.  The assumption that public opinion polls are a coordinating device that informs strategic voting behaviour has a rich theoretical foundation \citep{Cox1997,Fey1997}. It also has received convincing support from experimental evidence \citep{Forsytheetal1993,Forsytheetal1996} and from observational data \citep{Cox1997}.\footnote{Although in their experimental results \citet{meffertgschwend2007b} find that polling information had a weak impact on coalition-directed voting}  A third factor concerns explicit signals that voters receive from competing parties. An example is \citet{meffertgschwend2007b} who document the strong impact on coalition-directed voting resulting from party cues.

%\par  \citep{Duchetal2010} have made a strong empirical case for a strategic ideological vote that is based on the analysis of large-N voter preference studies. There have nonetheless be relatively few efforts to directly calibrate the extent to which voters are knowledgeable about the coalition government composition and how they form \footnote{Some recent exceptions are  \citet{Barnesetal2010}.}  This essay reports efforts to recover indicators of coalition reasoning -- specifically the extent to which individuals have sophisticated versus non-sophisticated levels of coalition reasoning -- employing experimental techniques.

%\par I will focus on three micro-assumptions about coalition reasoning that are common to all of the theories described above.  First, voters are assumed to have a very basic understanding of what constitutes a parliamentary majority.  And of course this understanding of parliamentary majority will differ in a contexts with single party majority governments versus multi-party coalition governments.  Second, theories of strategic vote choice in coalition contexts assume that voters are able to anticipate the likely coalitions that will form.  Theories of coalition formation, and the accompanying empirical evidence, suggest parties favor entering minimum winning coalitions.  Voter should understand this and anticipate that typically minimum winning coalitions form after an election.  Third, theory and empirics suggest that ideological distance plays an important role in determining which parties join in a coalition government. Again informed voters should anticipate that ideological distance will condition the kinds of coalitions that form after an election.  The experiment described in this essay is designed to identify the extent to which voters conform to each of these three dimensions of sophisticated coalition reasoning.


%\par Voters develop these coalition reasoning capabilities because, in a coalition government context, they ensure that vote decisions maximize the likelihood of post-election coalition outcomes that are proximate to their policy preferences \citep{Duchetal2010}. If this is the case then we should see contextual variation in overall levels of sophisticated coalition reasoning.  There are obviously electoral contexts in which this reasoning ability is useful, i.e., countries with multi-party coalition governments.  And there are countries where it is essentially useless -- historically, countries like the U.S., U.K., and Canada where coalition governments are extremely exotic.  I test this argument by conducting the coalition reasoning experiments in a context in which coalition government are the norm, i.e., Germany, and in the U.K. where they historically have been extremely rare.



\par Empirical tests of the \citet{Duchetal2010} theoretical claims about how ideology shapes vote choice require observations that vary over $x_{i}$, $p_{k}$, $\gamma_{c_{j_{n}}}$, and $h_{k}$. The first two requirements are quite standard: there needs to be variation in the self-placement of voters on the ideological continuum and parties need to vary along this same continuum.  The other requirements are somewhat more demanding: parties need to vary considerably in terms of their probability of participating in a governing coalition; and there needs to be variation across parties and over time in the allocation of cabinet portfolios to different parties in the governing coalition. Germany certainly meets these criteria. Further, the functional form of the empirical model has to be specified such that it generates estimates for the parameter $\beta$.\footnote{This initial analysis does not estimate the $\lambda$ coefficient.}

\par The challenge here is to leverage the individual-level vote choice data in the panel surveys so that we can calibrate the magnitude of $\beta$.  Note that even for those cases in which there are opportunities to exercise a coalition-directed ideological vote, the predictions from a model in which $\beta=1$ versus a model in which $\beta=0$ will be identical for a large number of voters.  This frequently happens because, given the ideological self-placement of voters, the optimal coalition-directed vote choice, taking into consideration post-election coalition compromises, is the same as one that simply considered the ideological proximity of parties.  This makes it difficult to assess the independent contribution of the coalition- and party-directed components of Equation \ref{eq:20} with an empirical model that includes both terms.


\paragraph{Conventional multi-nomial logit estimation.} Here I estimate a set of preliminary vote choice models that provide insight into the relatively magnitude of $\beta$.  The multi-nomial logit models include two different sets of left-right distant variables.  One set consists of what I have referred to as the sincere ideological distance variables ($[U-(x_{i}-p_{j})^{2}]$ from Equation \ref{eq:20}).  Euclidean distance values are calculated for each of the major parties: CDU, CSU, SPD, FDP, Greens, Linke -- these are based on individual left-right self placements ($x_{i}$) and the respondent's placement of each $j$ party on the left-right continuum ($p_{j}$).  A second set of left-right distance variables constitute coalition-directed ideological distance calculations that are attributed to the voters ($\left(
U-\sum_{n=1}^{N_{c_{j}}}(x_{i}-Z_{c_{j_{n}}})^{2}\gamma_{c_{j_{n}}}\right)$ from from Equation \ref{eq:20}).  In this initial analysis, which is a much simplified version of \citet{Duchetal2010}, I include two coalition distance terms: a Right coalition distance term: in this case $Z_{c_{j_{n}}}$ is based on the respondent's left-right placement of the CDU/CSU and the FDP weighted by the party's share of the seats in the legislature won by the parties in the Right coalition.  A similar calculation is made for the Left coalition which I defined as the SPD, Greens and the Linke parties.

\par In this preliminary analysis my primary interest is to 1) determine whether the coalition-directed calculations, theoretically captured by the $\beta$ term, are guiding the vote choice of some of the German voters; and 2) test the hypothesis that this incidence of coalition-directed voting is exaggerated amongst those who score high on coalition reasoning and coalition knowledge.  Accordingly, I estimate conventional multi-nomial vote choice models with a standard set of controls in addition to the sincere and coalition-direct ideological distance terms.

\par The estimates of these models are presented in Table~\ref{tab:germany_ideology} and Table~\ref{tab:germany_ideology_2}. Again, these are preliminary but they provide initial support for the argument that voters engage in coalition-directed ideological voting.  Model 1 is a standard vote choice model with Euclidean left-right distance terms for each of the major political parties (I do not report the results for the controls). Note that the distance terms for the samller parties, FDP, Green and Linke, are significant in the model while those for the two Grand Coalition partners, the CDU/CSU and SPD typically are not statistically significant.  Model 2 includes only the two coalition-directed distance terms.  In this case, both the Left and Right coalition-directed distance terms are statistically significant, and the in the expected direction, for the SPD, Green Party and Linke.  Including both the coalition-directed and party-directed distance terms -- Model 3 -- suggests that coalition-directed distance terms continue to perform well.  But given that for many voters the two distance terms should be associated with the same vote choice the results in Model 3 are probably not that informative.  And in fact we see a number of curious anomalies, such as the negative coefficient on the Right distance terms for the SPD and Linke vote.  My major point here though is that preliminary evidence suggests that modeling vote choice with a coalition-directed measure of ideological distance provides added value above and beyond models that exclusively incorporate sincere party ideological distance.

\begin{table}[h!]
\caption{Coalition-directed ideological vote: Germany}\label{tab:germany_ideology}
\begin{center}
\begin{tabular}{ld{3}d{3}d{3}d{3}d{3}}
\textbf{Variable} & \textbf{Model 1} & \textbf{Model 2} & \textbf{Model 3} & \textbf{Model 4} & \textbf{Model 5}  \\
\hline\hline
\multirow{2}{*}{\textbf{SPD}} &  &  &  &  &  \\
\multirow{2}{*}{sqrd dist to Left Coal} & - & -0.078 & -0.046 & -0.398 & -0.03 \\
 & - & (0.014) & (0.071) & (0.321) & (0.074) \\
\multirow{2}{*}{sqrd dist to Right Coal} & - & 0.136 & -0.434 & -0.915 & -0.405 \\
 & - & (0.02) & (0.171) & (0.649) & (0.204) \\
\multirow{2}{*}{sqrd dist to CDU} & 0.035 & - & 0.198 & 0.36 & 0.2 \\
 & (0.036) & - & (0.079) & (0.289) & (0.098) \\
\multirow{2}{*}{sqrd dist to CSU} & 0.044 & - & 0.174 & 0.312 & 0.167 \\
 & (0.029) & - & (0.057) & (0.19) & (0.07) \\
\multirow{2}{*}{sqrd dist to SPD} & -0.01 & - & 0.018 & 0.165 & 0.024 \\
 & (0.021) & - & (0.047) & (0.232) & (0.047) \\
\multirow{2}{*}{sqrd dist to Green Party} & -0.058 & - & -0.048 & 0.11 & -0.066 \\
 & (0.021) & - & (0.03) & (0.09) & (0.036) \\
\multirow{2}{*}{sqrd dist to Linke} & -0.015 & - & -0.01 & 0.025 & -0.014 \\
 & (0.01) & - & (0.016) & (0.064) & (0.019) \\
\multirow{2}{*}{sqrd dist to FDP} & 0.064 & - & 0.191 & 0.489 & 0.129 \\
 & (0.02) & - & (0.056) & (0.235) & (0.063) \\
\multirow{2}{*}{constant} & -1.139 & -1.148 & -1.022 & -1.07 & -1.68 \\
 & (0.922) & (0.906) & (0.926) & (2.732) & (1.164) \\
\hline
\multirow{2}{*}{\textbf{FDP}} &  &  &  &  &  \\
\multirow{2}{*}{sqrd dist to Left Coal} & - & 0.002 & -0.013 & 0.24 & 0.055 \\
 & - & (0.007) & (0.07) & (0.274) & (0.082) \\
\multirow{2}{*}{sqrd dist to Right Coal} & - & -0.025 & 0.068 & 0.256 & 0.162 \\
 & - & (0.025) & (0.176) & (0.583) & (0.227) \\
\multirow{2}{*}{sqrd dist to CDU} & 0.04 & - & 0.021 & -0.085 & 0.011 \\
 & (0.041) & - & (0.084) & (0.255) & (0.112) \\
\multirow{2}{*}{sqrd dist to CSU} & -0.021 & - & -0.037 & -0.056 & -0.082 \\
 & (0.033) & - & (0.059) & (0.166) & (0.08) \\
\multirow{2}{*}{sqrd dist to SPD} & 0.021 & - & 0.032 & -0.187 & -0.002 \\
 & (0.014) & - & (0.044) & (0.175) & (0.05) \\
\multirow{2}{*}{sqrd dist to Green Party} & -0.005 & - & -0.002 & 0.025 & -0.04 \\
 & (0.013) & - & (0.023) & (0.069) & (0.029) \\
\multirow{2}{*}{sqrd dist to Linke} & -0.008 & - & -0.006 & -0.057 & -0.015 \\
 & (0.009) & - & (0.016) & (0.058) & (0.019) \\
\multirow{2}{*}{sqrd dist to FDP} & -0.071 & - & -0.1 & -0.117 & -0.145 \\
 & (0.027) & - & (0.061) & (0.208) & (0.077) \\
\multirow{2}{*}{constant} & 0.038 & -0.089 & 0.053 & -2.348 & -0.783 \\
 & (0.826) & (0.819) & (0.831) & (2.312) & (1.082) \\
\hline\hline
\end{tabular}
\end{center}
\end{table}



\begin{table}[h!]
\caption{Coalition-directed ideological vote: Germany}\label{tab:germany_ideology_2}
\begin{center}
\begin{tabular}{ld{3}d{3}d{3}d{3}d{3}}
\textbf{} & \textbf{} & \textbf{} & \textbf{} & \textbf{} & \textbf{}  \\
\hline\hline
\multirow{2}{*}{\textbf{Green}} &  &  &  &  &  \\
\multirow{2}{*}{sqrd dist to Left Coal} & - & -0.111 & -0.116 & -0.571 & -0.017 \\
 & - & (0.022) & (0.078) & (0.332) & (0.092) \\
\multirow{2}{*}{sqrd dist to Right Coal} & - & 0.14 & -0.69 & -0.816 & -0.703 \\
 & - & (0.02) & (0.177) & (0.666) & (0.211) \\
\multirow{2}{*}{sqrd dist to CDU} & 0.023 & - & 0.275 & 0.367 & 0.262 \\
 & (0.038) & - & (0.081) & (0.294) & (0.1) \\
\multirow{2}{*}{sqrd dist to CSU} & 0.058 & - & 0.273 & 0.257 & 0.301 \\
 & (0.03) & - & (0.06) & (0.197) & (0.074) \\
\multirow{2}{*}{sqrd dist to SPD} & 0.027 & - & 0.09 & 0.318 & 0.042 \\
 & (0.023) & - & (0.048) & (0.233) & (0.057) \\
\multirow{2}{*}{sqrd dist to Green Party} & -0.153 & - & -0.131 & 0.08 & -0.199 \\
 & (0.034) & - & (0.042) & (0.11) & (0.058) \\
\multirow{2}{*}{sqrd dist to Linke} & -0.02 & - & -0.007 & 0.006 & -0.01 \\
 & (0.012) & - & (0.018) & (0.066) & (0.021) \\
\multirow{2}{*}{sqrd dist to FDP} & 0.063 & - & 0.267 & 0.431 & 0.228 \\
 & (0.021) & - & (0.058) & (0.24) & (0.066) \\
\multirow{2}{*}{constant} & -1.457 & -1.345 & -1.322 & -0.181 & -2.247 \\
 & (1.017) & (0.991) & (1.038) & (2.94) & (1.335) \\
\hline
\multirow{2}{*}{\textbf{Linke}} &  &  &  &  &  \\
\multirow{2}{*}{sqrd dist to Left Coal} & - & -0.074 & -0.204 & -0.662 & -0.174 \\
 & - & (0.015) & (0.07) & (0.327) & (0.074) \\
\multirow{2}{*}{sqrd dist to Right Coal} & - & 0.164 & -0.259 & -0.337 & -0.367 \\
 & - & (0.02) & (0.187) & (0.685) & (0.22) \\
\multirow{2}{*}{sqrd dist to CDU} & 0.081 & - & 0.182 & 0.253 & 0.214 \\
 & (0.037) & - & (0.083) & (0.299) & (0.102) \\
\multirow{2}{*}{sqrd dist to CSU} & 0.026 & - & 0.1 & 0.075 & 0.152 \\
 & (0.029) & - & (0.063) & (0.204) & (0.077) \\
\multirow{2}{*}{sqrd dist to SPD} & 0.037 & - & 0.127 & 0.345 & 0.108 \\
 & (0.02) & - & (0.044) & (0.231) & (0.044) \\
\multirow{2}{*}{sqrd dist to Green Party} & -0.015 & - & 0.034 & 0.268 & 0.014 \\
 & (0.023) & - & (0.032) & (0.096) & (0.036) \\
\multirow{2}{*}{sqrd dist to Linke} & -0.085 & - & -0.041 & -0.045 & -0.017 \\
 & (0.014) & - & (0.019) & (0.065) & (0.022) \\
\multirow{2}{*}{sqrd dist to FDP} & 0.041 & - & 0.11 & 0.246 & 0.109 \\
 & (0.021) & - & (0.062) & (0.245) & (0.069) \\
\multirow{2}{*}{constant} & -2.449 & -2.836 & -2.509 & -3.953 & -2.67 \\
 & (1.057) & (1.024) & (1.076) & (3.079) & (1.367) \\
\hline
Obs & 877 & 877 & 877 & 226 & 535 \\
ll(null) & -1476.005 & -1476.005 & -1476.005 & -375.096 & -904.022 \\
ll(model) & -1141.792 & -1086.076 & -1064.463 & -208.895 & -666.614 \\
df & 50 & 70 & 80 & 80 & 80 \\
AIC & 2383.585 & 2312.152 & 2288.926 & 577.791 & 1493.227 \\
BIC & 2622.41 & 2646.507 & 2671.046 & 851.434 & 1835.808 \\
\hline\hline
\end{tabular}
\end{center}
\end{table}

\clearpage

\par The results for Model 1 and Model 2 suggests that simply modeling vote choice as a function of spatial distance from likely coalition outcomes may not perform appreciably worse (or better mind you) than a model with spatial distance from parties.  This is confirmed when we compare predicted vote choices with actual vote choices. The model with party-directed ideological measures correctly predicts vote choice in 23 percent of the cases whilie the model with coalition-directed ideological measures generates correct preditions in 19 percent of the cases.  My suspicion, implied by Equation \ref{eq:20}, is that there are groups of voters for whom the coalition-directed elements in this model better describes vote choice while for others the party-directed consideration is more important.

\par The magnitude of the estimated ideology distance effect is another metric for simply assessing whether the coalition-directed term adds value in modeling the vote utility function.  Figure~\ref{fig:change_probs} presents the simulated effects -- averaged overall all respondents in the sample -- associated with a two-unit negative shift (i.e., more Left-wing) on the left-right self placement measure. In the case of the CDU and FDP the probabilities decline by about .1 which is an effect that is consistent with my expectations \citep{DuchStevenson2008}.  The results for the SPD and the Greens are particularly interesting.  Clearly, in the models that include the sincere distance terms, the simulated affect of ideology is effectively zero.  In the models that include the coalition-directed distance terms the simulated effects are in the expected direction and appear to be higher although even here their confidence intervals just barely include zero.  Finally, the simulated effects of ideology are quite high in magnitude for Linke, although here the effect for the coalition-directed model is lower.  Its certainly plausible that many voters discounted the likelihood that Linke would enter a governing coalition in which case the predominant ideology effect may have been a sincere one. On balance these results suggest that many voters are exercizing a coalition-directed ideological vote.  But this estimation strategy does not effectively tease out the relative importance of a coalition-directed as opposed to strategic ideological vote.

\begin{figure}[p]
\caption{Simulated change in vote probabilities associated with ideology}\label{fig:change_probs}

\centerline{\includegraphics[width=\textwidth]{PrProbs.png}}

\end{figure}

\clearpage


\par It stands to reason, as I just pointed out, that this coalition-directed voting should be more prominent amongst voters with higher levels of coalition reasoning and knowledge.  Model 4 estimates Model 3 for those who score low on the coalition savy metric and Model 5 presents the results for the high sophisticates.  Model 5 results are more consistent with a coalition-direct ideological vote.  Note in particular that in this model the Left coalition distance terms for the Green and Linke vote choices are negative and statistically significant and their magnitudes in this model a significantly larger than those for the non-savy types -- also, there is no longer an anomalous coefficient on the Right coalition for the SPD.  On balance the sophisticates give more evidence of engaging in coalition-directed voting than is the case for the non-sophisticates.



\section{Conclusion}

\par When the election results were announced after both the 2009 German and 2010 British elections, the voters did not know what government was elected - in both cases they had to wait weeks before a governing coalition was agreed upon.  These immediate vote tallies of course are important but rational choice theories suggest that voters are more concerned with the actual governing coalition that forms after the election results are announced.  I believe this is in fact the case for many voters in contexts that have a history of coalition government.  This implies that large numbers of voters exercise a coalition-directed vote rather than a sincere party-directed vote.  And to the extent that this is the case our current theoretical and empirical approaches to explaining vote choice need re-thinking.

\par Most importantly, if voters are actually conditioning their vote choice on their expectations of what happens in post-election coalition bargaining then most of the vote choice models in the literature are misspecified. In this essay I briefly review recent attempts to specify a coalition-directed economic vote model \citep{DuchStevenson2008} and a coalition-directed ideology vote model \citep{Duchetal2010}.  And there are others; most notably \citep{Kedar2009}.

\par A second important challenge here is developing estimation strategies that distinguish sincere from coalition-directed voting.  As \citet{Duchetal2010} point out, both considerations are likely to shape vote choice simultaneously -- the challenge is to identify, for any individual or population, the independent contribution, and hence the relative importance, of one versus the other.  In this essay I briefly describe our efforts in this regard using large-N observational studies \citep{Duchetal2010}.

\par The focus of this essay has been on a third challenge associated with the coalition-directed vote: Theories of coalition-directed voting suggest that voters have reasonably well developed levels of reasoning and knowledge about coalitions and the coalition formation process.  But do they? This essay has proposed a strategy for recovering, using incentive-compatible experimental vignettes, coalition reasoning skills. Additionally, it proposes fairly standard measures of knowledge of coalition composition and secondly of administrative responsibility.  All three of these dimensions define savy versus non-savy coalition reasoning.  As we might expect, given their histories of coalition government, the Germans score significantly higher on these measures than the British.  Of particular theoretical concern here is whether, as the coalition-directed voting theories would predict, voters who exercise a coalition-directed vote display more well-developed coalition reasoning skills.  The preliminary evidence presented here suggests in fact that this is the case, providing further micro-evidence for the notion that there is in fact a coalition-directed vote.


\newpage

\section{Appendix}



\subsection{Random Ordering of Treatments}

\par In the two surveys respondents were administered each of eight treatments (Treatments 1A through 4B) -- the treatments were randomly ordered for each respondent.

%\par A subset of the respondents (20 percent) will randomly be assigned treatments (5A through 8B) in which the incumbent parties are indicated in red.


\subsection{Description to Participants}

\par (UK) Participants were given the following set of instructions (the German instructions were similar with one important caveat -- there Germans were not rewarded for "correct" answers):

\par We will now like you to play a game. You will be asked to make some choices.  Other participants in this survey will be asked to make the same choices.  Each time you make a choice that is the same as the choice made by the majority of respondents in this survey you will be awarded bonus YouGov points.

\par We are going to describe a number of hypothetical political situations to you and for each different situation you will need to make a choice that you think most other participants in this survey will agree on.  First, all of the questions are going to use a Left versus Right political scale like the one below.  At one end of the scale you find the most Left-wing position (Position 0).  At the other end of the scale you find the most Right-wing position (Position 10).  And right in the middle you find those with Centrist positions (Position 5).

\par We are going to add two pieces of information to this scale.  First we are going to locate political parties on this scale -- Party A, Party B, etc.  Secondly, we are going to report the results of an election -- so we will tell you how many seats in the Parliament each party won as a result of the election.  Based on this information you will need to answer some questions.  Remember, if your answers are the same as the majority of respondents in this sample you earn additional YouGov points.


\newpage

\subsection{Treatment 1A}
\vspace{.1in}

\begin{figure}[h!]
\caption{Frequency with which Majority Governments Selected}\label{fig:treat1A}
\centerline{\scalebox{0.75}{\includegraphics[width=\textwidth]{treat1A.png}}}
\end{figure}



Two parties win Parliamentary seats in this election.  Which party or parties most likely will form a majority government in Parliament?

\begin{enumerate}
\item Party A is at .40 and Party B is at .60
\item Party A: 51 percent and Party B: 49 percent
\end{enumerate}


\subsection{Treatment 1B}
\vspace{.1in}
\begin{figure}[h!]
\caption{Frequency with which Majority Governments Selected}\label{fig:treat1B}
\centerline{\scalebox{0.75}{\includegraphics[width=\textwidth]{treat1B.png}}}
\end{figure}

Two parties win Parliamentary seats in this election.  Which party or parties most likely will form a majority government in Parliament?

\begin{enumerate}
\item Party A is at .10 and Party B is at .50
\item Party A: 51 percent and Party B: 49 percent
\end{enumerate}

\newpage
\subsection{Treatment 2A}
\vspace{.1in}
\begin{figure}[h!]
\caption{Frequency with which Majority Governments Selected}\label{fig:treat2A}
\centerline{\scalebox{0.75}{\includegraphics[width=\textwidth]{treat2A.png}}}
\end{figure}

Three parties win Parliamentary seats in this election.  Which party or parties most likely will form a majority government in Parliament?

\begin{enumerate}
\item Party A is at .40; Party B is at .50; and Party C is at .60
\item Party A: 45 percent; Party B: 16 percent; and Party C: 39 percent
\end{enumerate}


\subsection{Treatment 2B}
\vspace{.1in}
\begin{figure}[h!]
\caption{Frequency with which Majority Governments Selected}\label{fig:treat2B}
\centerline{\scalebox{0.75}{\includegraphics[width=\textwidth]{treat2B.png}}}
\end{figure}

Three parties win Parliamentary seats in this election.  Which party or parties most likely will form a majority government in Parliament?

\begin{enumerate}
\item Party A is at .40; Party B is at .80; and Party C is at .90
\item Party A: 45 percent; Party B: 16 percent; and Party C: 39 percent
\end{enumerate}

\newpage
\subsection{Treatment 3A}
\vspace{.1in}
\begin{figure}[h!]
\caption{Frequency with which Majority Governments Selected}\label{fig:treat3A}
\centerline{\scalebox{0.75}{\includegraphics[width=\textwidth]{treat3A.png}}}
\end{figure}


Four parties win Parliamentary seats in this election.  Which party or parties most likely will form a majority government in Parliament?

\begin{enumerate}
\item Party A is at .20; Party B is at .40; Party C is at .60; Party D is at .80
\item Party A: 13 percent; Party B: 39 percent; Party C: 10 percent; and Party D: 38 percent
\end{enumerate}

\subsection{Treatment 3B}
\vspace{.1in}
\begin{figure}[h!]
\caption{Frequency with which Majority Governments Selected}\label{fig:treat3B}
\centerline{\scalebox{0.75}{\includegraphics[width=\textwidth]{treat3B.png}}}
\end{figure}

Four parties win Parliamentary seats in this election.  Which party or parties most likely will form a majority government in Parliament?

\begin{enumerate}
\item Party A is at .20; Party B is at .80; Party C is at .90; Party D is at 1.0
\item Party A: 13 percent; Party B: 39 percent; Party C: 10 percent; and Party D: 38 percent
\end{enumerate}
\newpage
\subsection{Treatment 4A}

\vspace{.1in}
\begin{figure}[h!]
\caption{Frequency with which Majority Governments Selected}\label{fig:treat4A}
\centerline{\scalebox{0.75}{\includegraphics[width=\textwidth]{treat4A.png}}}
\end{figure}

Five parties win Parliamentary seats in this election.  Which party or parties most likely will form a majority government in Parliament?

\begin{enumerate}
\item Party A is at .10; Party B is at .30; Party C is at .50; Party D is at .70; Party E is at .90
\item Party A: 10 percent; Party B: 28 percent; Party C: 13 percent; Party D: 39 percent; and Party E: 10 percent
\end{enumerate}

\subsection{Treatment 4B}

\vspace{.1in}
\begin{figure}[h!]
\caption{Frequency with which Majority Governments Selected}\label{fig:treat4B}
\centerline{\scalebox{0.75}{\includegraphics[width=\textwidth]{treat4B.png}}}
\end{figure}

Five parties win Parliamentary seats in this election.  Which party or parties most likely will form a majority government in Parliament?

\begin{enumerate}
\item Party A is at 0; Party B is at .10; Party C is at .20; Party D is at .50; Party E is at .60
\item Party A: 10 percent; Party B: 28 percent; Party C: 13 percent; Party D: 39 percent; and Party E: 10 percent
\end{enumerate}

\newpage

\newpage


\newpage
\singlespace
\bibliography{dave}

\newpage

\end{document}


\subsection{Random Assignment of Incumbent Treatment}

\newpage

\subsection{Treatment 5A}
\vspace{.1in}
\centerline{
\scalebox{1} % Change this value to rescale the drawing.
{
\begin{pspicture}(-0.5,0)(10.5,3)
\definecolor{color0b}{rgb}{0.6,0.6,0.6}
\usefont{T1}{ptm}{m}{n}
\rput(4,0.5){A}
\usefont{T1}{ptm}{m}{n}
\rput(6,0.5){B}
\psframe[linewidth=0.00,dimen=outer,fillstyle=solid,fillcolor=color0b](5.75,1)(6.25,2.63)
\psframe[linewidth=0.00,dimen=outer,fillstyle=solid,fillcolor=color0b](3.75,1)(4.25,2.7)
\psline[linewidth=0.05cm,arrowsize=0.05291667cm 2.0,arrowlength=1.4,arrowinset=0.4]{<->}(-0.5,1)(10.5,1)
\usefont{T1}{ptm}{m}{n}
\rput(4,0.1){(51\%)}
\usefont{T1}{ptm}{m}{n}
\rput(6,0.1){(49\%)}
\end{pspicture}
}
}




Two parties win Parliamentary seats in this election.  Which party or parties most likely will form a majority government in Parliament?

\begin{enumerate}
\item Party B is incumbent (in Red)
\item Party A is at .40 and Party B is at .60
\item Party A: 51 percent and Party B: 49 percent
\end{enumerate}


\subsection{Treatment 5B}
\vspace{.1in}
\centerline{
\scalebox{1} % Change this value to rescale the drawing.
{
\begin{pspicture}(-0.5,0)(10.5,3)
\usefont{T1}{ptm}{m}{n}
\rput(1,0.5){A}
\usefont{T1}{ptm}{m}{n}
\rput(5,0.5){B}
\psframe[linewidth=0.00,dimen=outer,fillstyle=solid,fillcolor=color0b](0.75,1)(1.25,2.7)
\psframe[linewidth=0.00,dimen=outer,fillstyle=solid,fillcolor=color0b](4.75,1)(5.25,2.63)
\psline[linewidth=0.05cm,arrowsize=0.05291667cm 2.0,arrowlength=1.4,arrowinset=0.4]{<->}(-0.5,1)(10.5,1)
%\psline[linewidth=0.05cm](1.0,1.25)(1.0,0.75)
%\psline[linewidth=0.05cm](5.0,1.25)(5.0,0.75)
\usefont{T1}{ptm}{m}{n}
\rput(1,0.1){(51\%)}
\usefont{T1}{ptm}{m}{n}
\rput(5,0.1){(49\%)}
\end{pspicture}
}
}

Two parties win Parliamentary seats in this election.  Which party or parties most likely will form a majority government in Parliament?

\begin{enumerate}
\item Party B is incumbent (in Red)
\item Party A is at .10 and Party B is at .50
\item Party A: 51 percent and Party B: 49 percent
\end{enumerate}

\newpage
\subsection{Treatment 6A}
\vspace{.1in}
\centerline{
\scalebox{1} % Change this value to rescale the drawing.
{
\begin{pspicture}(-0.5,0)(10.5,3)
\usefont{T1}{ptm}{m}{n}
\rput(4,0.5){A}
\usefont{T1}{ptm}{m}{n}
\rput(5,0.5){B}
\usefont{T1}{ptm}{m}{n}
\rput(6,0.5){C}
\psframe[linewidth=0.00,dimen=outer,fillstyle=solid,fillcolor=color0b](3.75,1)(4.25,2.5)
\psframe[linewidth=0.00,dimen=outer,fillstyle=solid,fillcolor=color0b](4.75,1)(5.25,1.53)
\psframe[linewidth=0.00,dimen=outer,fillstyle=solid,fillcolor=color0b](5.75,1)(6.25,2.3)
\psline[linewidth=0.05cm,arrowsize=0.05291667cm 2.0,arrowlength=1.4,arrowinset=0.4]{<->}(-0.5,1)(10.5,1)
%\psline[linewidth=0.05cm](4.0,1.25)(4.0,0.75)
%\psline[linewidth=0.05cm](5.0,1.25)(5.0,0.75)
%\psline[linewidth=0.05cm](6.0,1.25)(6.0,0.75)
\usefont{T1}{ptm}{m}{n}
\rput(4,0.1){(45\%)}
\usefont{T1}{ptm}{m}{n}
\rput(5,0.1){(16\%)}
\usefont{T1}{ptm}{m}{n}
\rput(6,0.1){(39\%)}
\end{pspicture}
}
}

Three parties win Parliamentary seats in this election.  Which party or parties most likely will form a majority government in Parliament?

\begin{enumerate}
\item Party C is incumbent (in Red)
\item Party A is at .40; Party B is at .50; and Party C is at .60
\item Party A: 45 percent; Party B: 16 percent; and Party C: 39 percent
\end{enumerate}


\subsection{Treatment 6B}
\vspace{.1in}
\centerline{
\scalebox{1} % Change this value to rescale the drawing.
{
\begin{pspicture}(-0.5,0)(10.5,3)
\usefont{T1}{ptm}{m}{n}
\rput(4,0.5){A}
\usefont{T1}{ptm}{m}{n}
\rput(8,0.5){B}
\usefont{T1}{ptm}{m}{n}
\rput(9,0.5){C}
\psframe[linewidth=0.00,dimen=outer,fillstyle=solid,fillcolor=color0b](3.75,1)(4.25,2.5)
\psframe[linewidth=0.00,dimen=outer,fillstyle=solid,fillcolor=color0b](7.75,1)(8.25,1.53)
\psframe[linewidth=0.00,dimen=outer,fillstyle=solid,fillcolor=color0b](8.75,1)(9.25,2.3)
\psline[linewidth=0.05cm,arrowsize=0.05291667cm 2.0,arrowlength=1.4,arrowinset=0.4]{<->}(-0.5,1)(10.5,1)
%\psline[linewidth=0.05cm](4.0,1.25)(4.0,0.75)
%\psline[linewidth=0.05cm](8.0,1.25)(8.0,0.75)
%\psline[linewidth=0.05cm](9.0,1.25)(9.0,0.75)
\usefont{T1}{ptm}{m}{n}
\rput(4,0.1){(45\%)}
\usefont{T1}{ptm}{m}{n}
\rput(8,0.1){(16\%)}
\usefont{T1}{ptm}{m}{n}
\rput(9,0.1){(39\%)}
\end{pspicture}
}
}

Three parties win Parliamentary seats in this election.  Which party or parties most likely will form a majority government in Parliament?

\begin{enumerate}
\item Party C is incumbent (in Red)
\item Party A is at .40; Party B is at .80; and Party C is at .90
\item Party A: 45 percent; Party B: 16 percent; and Party C: 39 percent
\end{enumerate}

\newpage
\subsection{Treatment 7A}
\vspace{.1in}
\centerline{
\scalebox{1} % Change this value to rescale the drawing.
{
\begin{pspicture}(-0.5,0)(10.5,3)
\usefont{T1}{ptm}{m}{n}
\rput(2,0.5){A}
\usefont{T1}{ptm}{m}{n}
\rput(4,0.5){B}
\usefont{T1}{ptm}{m}{n}
\rput(6,0.5){C}
\usefont{T1}{ptm}{m}{n}
\rput(8,0.5){D}
\psframe[linewidth=0.00,dimen=outer,fillstyle=solid,fillcolor=color0b](1.75,1)(2.25,1.43)
\psframe[linewidth=0.00,dimen=outer,fillstyle=solid,fillcolor=color0b](3.75,1)(4.25,2.3)
\psframe[linewidth=0.00,dimen=outer,fillstyle=solid,fillcolor=color0b](5.75,1)(6.25,1.33)
\psframe[linewidth=0.00,dimen=outer,fillstyle=solid,fillcolor=color0b](7.75,1)(8.25,2.27)
\psline[linewidth=0.05cm,arrowsize=0.05291667cm 2.0,arrowlength=1.4,arrowinset=0.4]{<->}(-0.5,1)(10.5,1)
%\psline[linewidth=0.05cm](2.0,1.25)(2.0,0.75)
%\psline[linewidth=0.05cm](4.0,1.25)(4.0,0.75)
%\psline[linewidth=0.05cm](6.0,1.25)(6.0,0.75)
%\psline[linewidth=0.05cm](8.0,1.25)(8.0,0.75)
\usefont{T1}{ptm}{m}{n}
\rput(2,0.1){(13\%)}
\usefont{T1}{ptm}{m}{n}
\rput(4,0.1){(39\%)}
\usefont{T1}{ptm}{m}{n}
\rput(6,0.1){(10\%)}
\usefont{T1}{ptm}{m}{n}
\rput(8,0.1){(38\%)}
\end{pspicture}
}
}

Four parties win Parliamentary seats in this election.  Which party or parties most likely will form a majority government in Parliament?

\begin{enumerate}
\item Party A is incumbent (in Red)
\item Party A is at .20; Party B is at .40; Party C is at .60; Party D is at .80
\item Party A: 13 percent; Party B: 39 percent; Party C: 10 percent; and Party D: 38 percent
\end{enumerate}

\subsection{Treatment 7B}
\vspace{.1in}
\centerline{
\scalebox{1} % Change this value to rescale the drawing.
{
\begin{pspicture}(-0.5,0)(10.5,3)
\usefont{T1}{ptm}{m}{n}
\rput(2,0.5){A}
\usefont{T1}{ptm}{m}{n}
\rput(8,0.5){B}
\usefont{T1}{ptm}{m}{n}
\rput(9,0.5){C}
\usefont{T1}{ptm}{m}{n}
\rput(10,0.5){D}
\psframe[linewidth=0.00,dimen=outer,fillstyle=solid,fillcolor=color0b](1.75,1)(2.25,1.43)
\psframe[linewidth=0.00,dimen=outer,fillstyle=solid,fillcolor=color0b](7.75,1)(8.25,2.3)
\psframe[linewidth=0.00,dimen=outer,fillstyle=solid,fillcolor=color0b](8.75,1)(9.25,1.33)
\psframe[linewidth=0.00,dimen=outer,fillstyle=solid,fillcolor=color0b](9.75,1)(10.25,2.27)
\psline[linewidth=0.05cm,arrowsize=0.05291667cm 2.0,arrowlength=1.4,arrowinset=0.4]{<->}(-0.5,1)(10.5,1)
%\psline[linewidth=0.05cm](2.0,1.25)(2.0,0.75)
%\psline[linewidth=0.05cm](8.0,1.25)(8.0,0.75)
%\psline[linewidth=0.05cm](9.0,1.25)(9.0,0.75)
%\psline[linewidth=0.05cm](10,1.25)(10,0.75)
\usefont{T1}{ptm}{m}{n}
\rput(2,0.1){(13\%)}
\usefont{T1}{ptm}{m}{n}
\rput(8,0.1){(39\%)}
\usefont{T1}{ptm}{m}{n}
\rput(9,0.1){(10\%)}
\usefont{T1}{ptm}{m}{n}
\rput(10,0.1){(38\%)}
\end{pspicture}
}
}

Four parties win Parliamentary seats in this election.  Which party or parties most likely will form a majority government in Parliament?

\begin{enumerate}
\item Party A is incumbent (in Red)
\item Party A is at .20; Party B is at .80; Party C is at .90; Party D is at 1.0
\item Party A: 13 percent; Party B: 39 percent; Party C: 10 percent; and Party D: 38 percent
\end{enumerate}
\newpage
\subsection{Treatment 8A}

\vspace{.1in}
\centerline{
\scalebox{1} % Change this value to rescale the drawing.
{
\begin{pspicture}(-0.5,0)(10.5,3)
\usefont{T1}{ptm}{m}{n}
\rput(1,0.5){A}
\usefont{T1}{ptm}{m}{n}
\rput(3,0.5){B}
\usefont{T1}{ptm}{m}{n}
\rput(5,0.5){C}
\usefont{T1}{ptm}{m}{n}
\rput(7,0.5){D}
\usefont{T1}{ptm}{m}{n}
\rput(9,0.5){E}
\psframe[linewidth=0.00,dimen=outer,fillstyle=solid,fillcolor=color0b](0.75,1)(1.25,1.33)
\psframe[linewidth=0.00,dimen=outer,fillstyle=solid,fillcolor=color0b](2.75,1)(3.25,1.93)
\psframe[linewidth=0.00,dimen=outer,fillstyle=solid,fillcolor=color0b](4.75,1)(5.25,1.43)
\psframe[linewidth=0.00,dimen=outer,fillstyle=solid,fillcolor=color0b](6.75,1)(7.25,2.3)
\psframe[linewidth=0.00,dimen=outer,fillstyle=solid,fillcolor=color0b](8.75,1)(9.25,1.33)
\psline[linewidth=0.05cm,arrowsize=0.05291667cm 2.0,arrowlength=1.4,arrowinset=0.4]{<->}(-0.5,1)(10.5,1)
%\psline[linewidth=0.05cm](1.0,1.25)(1.0,0.75)
%\psline[linewidth=0.05cm](3.0,1.25)(3.0,0.75)
%\psline[linewidth=0.05cm](5.0,1.25)(5.0,0.75)
%\psline[linewidth=0.05cm](7,1.25)(7,0.75)
%\psline[linewidth=0.05cm](9,1.25)(9,0.75)
\usefont{T1}{ptm}{m}{n}
\rput(1,0.1){(10\%)}
\usefont{T1}{ptm}{m}{n}
\rput(3,0.1){(28\%)}
\usefont{T1}{ptm}{m}{n}
\rput(5,0.1){(13\%)}
\usefont{T1}{ptm}{m}{n}
\rput(7,0.1){(39\%)}
\usefont{T1}{ptm}{m}{n}
\rput(9,0.1){(10\%)}
\end{pspicture}
}
}

Five parties win Parliamentary seats in this election.  Which party or parties most likely will form a majority government in Parliament?

\begin{enumerate}
\item Party B is incumbent (in Red)
\item Party A is at .10; Party B is at .30; Party C is at .50; Party D is at .70; Party E is at .90
\item Party A: 10 percent; Party B: 28 percent; Party C: 13 percent; Party D: 39 percent; and Party E: 10 percent
\end{enumerate}

\subsection{Treatment 8B}

\vspace{.1in}
\centerline{
\scalebox{1} % Change this value to rescale the drawing.
{
\begin{pspicture}(-0.5,0)(10.5,3)
\usefont{T1}{ptm}{m}{n}
\rput(0,0.5){A}
\usefont{T1}{ptm}{m}{n}
\rput(1,0.5){B}
\usefont{T1}{ptm}{m}{n}
\rput(2,0.5){C}
\usefont{T1}{ptm}{m}{n}
\rput(5,0.5){D}
\usefont{T1}{ptm}{m}{n}
\rput(6,0.5){E}
\psframe[linewidth=0.00,dimen=outer,fillstyle=solid,fillcolor=color0b](-0.25,1)(0.25,1.33)
\psframe[linewidth=0.00,dimen=outer,fillstyle=solid,fillcolor=color0b](0.75,1)(1.25,1.93)
\psframe[linewidth=0.00,dimen=outer,fillstyle=solid,fillcolor=color0b](1.75,1)(2.25,1.43)
\psframe[linewidth=0.00,dimen=outer,fillstyle=solid,fillcolor=color0b](4.75,1)(5.25,2.3)
\psframe[linewidth=0.00,dimen=outer,fillstyle=solid,fillcolor=color0b](5.75,1)(6.25,1.33)
\psline[linewidth=0.05cm,arrowsize=0.05291667cm 2.0,arrowlength=1.4,arrowinset=0.4]{<->}(-0.5,1)(10.5,1)
%\psline[linewidth=0.05cm](0.0,1.25)(0.0,0.75)
%\psline[linewidth=0.05cm](1.0,1.25)(1.0,0.75)
%\psline[linewidth=0.05cm](2.0,1.25)(2.0,0.75)
%\psline[linewidth=0.05cm](5,1.25)(5,0.75)
%\psline[linewidth=0.05cm](6,1.25)(6,0.75)
\usefont{T1}{ptm}{m}{n}
\rput(0,0.1){(10\%)}
\usefont{T1}{ptm}{m}{n}
\rput(1,0.1){(28\%)}
\usefont{T1}{ptm}{m}{n}
\rput(2,0.1){(13\%)}
\usefont{T1}{ptm}{m}{n}
\rput(5,0.1){(39\%)}
\usefont{T1}{ptm}{m}{n}
\rput(6,0.1){(10\%)}
\end{pspicture}
}
}
Five parties win Parliamentary seats in this election.  Which party or parties most likely will form a majority government in Parliament?

\begin{enumerate}
\item Party B is incumbent (in Red)
\item Party A is at 0; Party B is at .10; Party C is at .20; Party D is at .50; Party E is at .60
\item Party A: 10 percent; Party B: 28 percent; Party C: 13 percent; Party D: 39 percent; and Party E: 10 percent
\end{enumerate}





\end{document}


\section{Theory}


\subsection{Spatial Voting}

\par First, vote choice conforms to a variant of the classic Downsian model \citep{Downs1957} in which voters locate themselves and candidates in a salient issue space and make choices based on their proximity to the issue positions of competing candidates \citep{EnelowHinich1994}.  And second, the left-right ideological continuum is arguably the most important policy dimension shaping vote choice.  These observations build on a literature suggesting that ideology plays a central role in contemporary democratic politics.  There is overwhelming evidence that the left-right continuum shapes party competition \citep{LaverBudge1993,BudgeRobertson1987,Huberetal1995,Knutsen1998,Adametal2004}; that it determines legislative voting \citep{PooleRosenthal1997} and government spending priorities \citep{Blaisetal1993}; and that it affects coalition outcomes \citep{Warwick1992,Warwick2005}.  Most importantly we know that the ideological vote is important in certain countries \citep{Kedar2005,Adamsetal2005,Merrilletal1999,Blaisetal2001a,Westholm1997,Inglehartetal1976}.

%ideological composition of the electorate varies systematically cross-nationally and over time \citep{Stevenson2001,KimFording1998}; that the

\par One of the most influential theoretical developments in the study of vote choice is the notion that vote utility is informed by spatial distance. Certainly the most influential early use of spatial distance in a theory of voting is Anthony Downs' \textit{An Economic Theory of Democracy}.  Downs \citeyearpar{Downs1957} argued that individuals make vote choices based on their comparison of expected utilities for each of the competing parties.  In Downs' model, voters are instrumentally rational which implies that they are motivated to select parties that are ideologically proximate. This translates into the the conventional characterization of the ideological vote in terms of Euclidean distance,

 \begin{equation}
 u(j_{i})=-(x_{i}-p_{j})^{2}\label{eq:iv}
 \end{equation}

\noindent where $x_{i}$ represents the ideological position of voter $i$ and $p_{j}$ represents the ideological position of party $j$. A smaller Euclidean distance translates into more utility and hence contributes to the likelihood that a voter would vote for that party. This is what we characterize as sincere ideological voting.  If all voters adopt this proximate ideological voting decision rule, we would find homogeneity in the importance of ideology in explaining vote choice across all democratic contexts.

\par Over the last 50 years since Downs' work appeared, the basic Euclidean expression in Equation~\ref{eq:iv} has undergone extensive elaboration and revision.  In particular, many have explored how this Euclidean reasoning works in contexts with multi-party coalition governments.  \citet[][146]{Downs1957} himself points out that one of the factors complicating the voter's decision calculus is a political context with coalition governments. Downs \citeyearpar{Downs1957} in fact was less than sanguine about the average voters ability to undertake these calculations \citep[][256]{Downs1957}. In these coalition contexts, coalitions form after elections as a result of bargain amongst parties over the policies to be enacted by the government \citep{AustenSmithBanks1988,Perssonetal2000}. Policy outcomes in coalition government reflect the policy preferences of the parties forming the governing coalition weighted by their legislative seats \citep{Indridason2007b,DuchStevenson2008,Schofieldetal1985}.\footnote{An alternative, and in our view less plausible, perspective is that the policy outcomes adopted in multiparty contexts reflect the weighted preferences of all parties elected to the legislature \citep{Ortunoortin1997,DeSinopolietal2007}.  This of course significantly reduces the second-order strategic incentives for voters.}  We believe that in coalition contexts voters anticipate these policy outcomes and they use these to condition their
ideological vote calculus represented in Equation~\ref{eq:iv}.\footnote{This anticipation of post-election policy compromises is not restricted to multiparty coalition contexts. \citet{AlesinaRosenthal1995}, for example, suggest that voters in the U.S. context exercise a policy balancing vote, anticipating the policy differences between Congress and the President. \citet{Kedar2006} makes a more general claim suggesting that this occurs in all Presidential regimes.  \citet{Adamsetal2004} analyze individual and aggregate-level data related to U.S. Senate elections and find support for the argument that voters anticipate the moderating effect of the
legislative process and hence vote for candidates with more extreme positions.  Although they are careful to point out that their data could not distinguish this discounting argument from a directional voting explanation.}  Strategic voters, concerned with final policy outcomes (as opposed to party platforms), condition their vote choices on coalition bargaining outcomes that occur after the election\citep{AustenSmithBanks1988}. In multiparty contexts with coalition governments, \citet{AustenSmithBanks1988} argue, sincere ideological voting is not rational.  The implication of the \citet{AustenSmithBanks1988} insight here is that the link between ideology and vote choice is conditioned by rational voters engaging in strategic voting. Voters anticipate the likely coalition formation negotiations that occur after the election and they condition their vote choices accordingly in order to maximize the likelihood that a coalition government forms that best represents their policy preferences.


%\par A related line of reasoning regarding the vote calculus suggests that voters focus on the direction of policy movement.  Voters in these \textit{directional} models of ideological voting implicitly understand that there is a status quo bias in the post election policy making process.  Hence, as \citet{Matthews1979} argues, voters prefer candidates who move policy from the status quo toward their ideal point.  In a uni-dimensional policy world where left-right self identification is the only policy dimension determining vote choice, the candidate's location relative to the status quo point is the only consideration that matters to voters -- intensity does not come into play.



%\citet{RabinowitzMacdonald1989} explicitly add intensity to their directional model of vote choice.  The voter utility function is a scalar or dot product of the vectors representing the policy positions of voters ($\mathbf{V}$) and candidates ($\mathbf{C}$): $U(\mathbf{V,C})=\mathbf{V}\cdot\mathbf{C}=\sum^{n}_{i=1}v_{i}c_{i}$.  If we assume that vote choice is determined by a single left-right ideology dimension then the vote utility is simply the product of the voter and candidate's ideal points, both calculated relative to the neutral point.  Take the case where there are two conservative parties located to the right of the neutral point on a left-right continuum.  Voters to the right of the neutral point will give all of their votes to the conservative party with the most extreme location to the right of the neutral point -- the other conservative party would receive none of the votes of voters to the right of the neutral point.

%\par \citet{Adamsetal2005} and \citet{Merrilletal1999} convincingly argue that voters employ mixed strategies of discounted and directional voting that likely vary by context.  Clearly voters are conditioning their ideological vote on their expectations regarding post-election policy outcomes.  But the nature of voter expectations in both the directional and discounting models resembles a relatively naive heuristic:  Voters anticipate political and institutional resistance to changing the status quo and therefore vote for parties that are "directionally proximate" but have more extreme ideal points.


%\par Secondly, the spatial term incorporates strategic voting considerations....

%, typically to make it more sensitive to different political contexts.

%\par These formal statements that link coalition outcomes and vote choice present a challenge: How do we precisely characterize this voter calculus that anticipates coalition outcomes after the election?  \citet{Grofman1985} proposed a modification to the proximate ideological model that takes into consideration what politicians are actually able to accomplish after an election.  Voters in the \citet{Grofman1985} \textit{discounting} model anticipate that candidates, if elected, will be able to move policy only part way from the status quo position to their bliss point.  This intermediate distance between the candidates ideal point and the status quo is determined by a common discounting factor shared by all voters. Hence, rather than the voters assessing the Euclidean distance between their ideal point and $p_{j}$ in Equation \ref{eq:iv}, they employ a discounted version of  $p_{j}$, i.e.,  $p_{j}*d$ where $d$ varies between $0$ and $1$.  When $d=1$ we have a simple proximate ideological model and when $d$ approaches 0, Euclidean distance does not matter.


\par This argument suggests that voters are reasonably well-informed about post-election coalition formation outcomes and this conditions the ideological vote. \citet{Kedar2005} argues that the rational voter focuses on policy outcomes and hence on the issue positions that are ultimately adopted by the coalition government that forms after an election. And she demonstrates that in political systems with coalition governments this leads to ``compensational voting", rather than ideological proximity voting, aimed at minimizing the policy distance between the policy compromises negotiated by the governing coalition and the voters ideal policy position.

%\par \citet{DuchStevenson2008} develop a contextual theory of economic voting in which voters anticipate the likely coalitions that form after an election and they assess the impact of their vote choice on the likelihood of different coalitions coming to power after an election.  And this information is used by instrumentally rational voters to weight the importance of an economic competency signal in their vote choice function.  Hence, parties that are certain to enter a governing coalition (i.e., perennial coalition partners) should, all things being equal, get no economic vote since a vote for this party has no impact on the coalition that ultimately forms.  Both \citet{Kedar2005} and \citet{DuchStevenson2008} go to considerable length to formalize how post-election coalition formation enters into the vote choice function.  Building on these works, we propose a model of the ideological vote in which voters anticipate the coalitions that form after the election -- what we call the strategic ideological vote.



%\par This recent literature on the strategic element of the spatial ideological vote makes one important assumption: Voters are able to anticipate the likely coalition formation after the election.


%\par All of this rich literature on ideological spatial voting has in common the notion that in some form or other voters engage in spatial reasoning that compares their own left-right spatial location to those of competing political parties \textit{and} that this informs the vote decision.  This essay focus on this rather general proposition although we recognize that there is additional information that enters into the spatial calculation and in fact elsewhere we explore this in great detail \citep{Duchetal2008}.  In fact this added complexity to the ideological vote implies contextual variation in the ideological vote.  As \citet{DuchStevenson2008} point out in a recent work, once a theory of vote choice suggests contextual variation in the importance of variables in the vote utility function, one cannot characterize the importance of those effects with a small number of individual-level studies.  One cannot, for example, make very reliable statements about the importance of the ideological vote based on a sample of election studies from 10 or even 20 countries.  Since we expect the importance of the ideological vote to vary by political and institutional context, with a small sample we cannot be sure whether a weak, or strong, result, is a function of a unique set of institutional or political contexts.  A reliable estimate of the importance of any variable in the vote choice function that is conditioned on context must be based on a large-N sample of voter preferences surveys that reflects the contextual diversity of the population to which one expects to generalize.  Hence, our contribution is simply to establish that in fact there is an ideological vote and to provide a global mapping of it.  This, we would argue, is a precursor to future research that explores how context enters into the spatial calculations of voters.


%\subsection{The Left-Right Ideological Continuum}


%\par But what considerations enter into this Euclidean distance term?  The left-right ideological continuum is arguably the most important spatial policy dimension shaping vote choice.  These observations build on a literature suggesting that ideology plays a central role in contemporary democratic politics.  There is overwhelming evidence that the left-right continuum shapes party competition \citep{LaverBudge1993,BudgeRobertson1987,Huberetal1995,Knutsen1998,Adametal2004}; that it determines legislative voting \citep{PooleRosenthal1997} and government spending priorities \citep{Blaisetal1993}; and that it affects coalition outcomes \citep{Warwick1992,Warwick2005}.  Most importantly we know that the ideological vote is important in certain countries \citep{Kedar2005,Adamsetal2005,Merrilletal1999,Blaisetal2001a,Westholm1997,Inglehartetal1976}.

%\par Our claim is simply that the left-right ideological continuum represents a meaningful spatial dimension on which voters, in virtually all political systems, are able to organize both themselves and competing political parties.  We do not make any claims as to how voters map policies onto this spatial dimension -- there may, or may not, be cross-national similarities.  Three empirical conditions need to hold, in our opinion, in order to make the claim that the right-left ideological continuum represents a meaningful spatial dimension for the world's voters.  First, there needs to be empirical evidence that when asked to locate themselves on the left-right ideological continuum voters understand the question.  Second, in a similar fashion, we need empirical evidence that voters understand questions asking them to place political parties on the left-right continuum.  Third, for any country we should expect to voter self-placements and voter placements of parties on this continuum to be relatively stable over time.  In this essay we provide large-N cross-national evidence for all three of these conditions.


\subsection{Voters Anticipate the Likely Coalitions that Form After an Election}


\par \citet{Duchetal2009b} argue that voter's utility for a particular party will be determined by the likely coalition government the party will enter.  This is the case because what matters to the voter is the policy outcome that will be implemented by the government that is formed after the election.  Hence the relevant spatial distance in the vote utility function is the distance between the respondents Left-Right self placement and the Left-Right location of the coalition that forms with this particular party.  Hence in this model voters are assumed to be knowledgeable about the likelihoods of different coalition permutations forming after an election.  They are also able to assess the Left-Right policy compromise that is conditioned on the Left-Right ideal points of each coalition party.

\par This theory of a strategic ideological vote developed by \citet{Duchetal2009b}, and those developed by some of the others mentioned earlier, implies that voters are informed about certain features of coalition "landscape".  And this is unlikely to be the case if the costs associated with gathering this information are prohibitive for the average voter. In this essay we analyze historical coalition formation data indicating that the costs of informing themselves about the likelihood of coalition outcomes is not high.  We focus on two types of information regarding coalition outcomes that are available to the voter.

\par First, voters in these models use historical information regarding participation in governing coalitions to establish the probabilities that parties will participate in the government after an election.  These theories assume that voters are knowledgeable about the likelihoods of different combinations of parties making up the governing coalition that forms after an election. And there is evidence to suggest in fact that they are quite knowledgeable about these probabilities \citep{BargstedKedar2007,DuchStevenson2008,GalenHolsteyn2003}.  We will argue here that patterns of coalition formation in most national contexts are quite stable which facilitates the voter's ability to anticipate the kinds of coalitions that are most likely to form after an election.

\par Second, voters are expected to know the ideological spatial locations of parties and resulting coalition compromises amongst parties. For example, voters are expected to understand that the addition of the French Communist Party to a Socialist Party governing coalition will shift the ideological policy orientations in a left direction. We will argue that the coalitions that typically form in advanced democracies are ideological very stable, they typically remain reasonably faithful to the location of the median voter.

%\par One strategy for ensuring appropriate variation is through experimental treatments.  \citet{meffertgschwend2007a}, for example, employ experiments to demonstrate that voters are capable of making strategic voting decisions that anticipate post-election coalition formations and the relative policy weights of parties in these coalitions. \citet{Tomzetal2007} implement an online experiment demonstrating that voters can make sophisticated policy balancing decisions as part of their vote choice and that this is particularly the case with centrist voters. \citet{Goodinetal2007} report experimental results suggesting that a subset of their subjects -- those assigned a party leader role -- exercise a strategic ideological vote when they are informed about pre-election coalition agreements.\footnote{Other experimental advances in this regard include \citet{Claassen2007,Lacy2005}.}




%\begin{footnotesize}
%\begin{equation}
%u(j_{i}) = \lambda\left\{\beta\left[\left(
%U-\sum_{c_{j}=1}^{N_{c_{j}}}(x_{i}-Z_{c_{j}})^{2}\gamma_{c_{j}}\right) -
%U\left(1-\sum_{c_{j}}^{N_{c_{j}}}\gamma_{c_{j}}\right) \right]+
%(1-\beta)\left[U-(x_{i}-p_{j})^{2}\right]\right\} + \phi \bm{W}_{i} \label{eq:20}
%\end{equation}
%\end{footnotesize}

%\par Equation \ref{eq:20} represents the utility that voter $i$ derives from party $j$. The first right-hand term in large parentheses in Equation \ref{eq:20} incorporates these two strategic components, $\gamma_{c_{j}}$ and $Z_{c_{j}}$. For any party there will be $N_{c_{j}}$ coalition permutations that could include party $j$ as a member where $c_{j}$ describes the possible coalitions including party $j$. And $\gamma_{c_{j}}$ is simply the probability associated with each of these coalitions forming after the election.  Given that the $\gamma_{c_{j}}$ terms exhaustively partition the government experience of party $j$,

%\begin{equation*}
%\sum_{c_{j}}^{N_{c_{j}}}\gamma_{c_{j}} = \left\{
%  \begin{array}{ll}
%    1, & \hbox{if $j$ has ever been in government;} \\
%    0, & \hbox{if $j$ has never been in government.}
%  \end{array}
%\right.
%\end{equation*}


%\par It is the $\gamma_{c_{j}}$ and $h_{jc_{j}}$ (through the $Z_{c_{j}}$ term) in Equation \ref{eq:20} that distinguish strategic from sincere representations of the ideological vote.  The $\gamma_{c_{j}}$ represents the voter's assessment of the likelihood of all possible coalition permutations in which party $j$ could participate.  We employ an historical approach to measuring voter beliefs about the likelihood of participating in each possible coalition permutation. Specifically, we calculate, for each party (at the time of the survey), the months since 1960 that the party has been in the cabinet and discounted versions of this measure that gives more weight to more recent experience.

%\par Each party $j$ is involved in some number of coalitions $N_{c_{j}}$, where $c_{j} \in \{1_{j}, 2_{j}, \ldots, N_{c_{j}j}\}$, if $N_{c_{j}} > 0$.  Each survey is taken at month $m$. ${t_{c_{j}}}$ assumes a value of 1 for the first month in which party $j$ participated in a particular coalition $c_{j}$, a value of 2 for the second month, and $M_{c_{j}}$ represents the month in which the coalition terminated, so $t_{c_{j}} = \{1, 2, \ldots, M_{c_{j}}\}$.

%\begin{equation}
%\gamma_{c_{j}} = \left(\sum_{c_{j}}^{N_{c_{j}}}\sum_{t_{c_{j}}}^{M_{c_{j}}} 0.98^{m_{e} -
%t_{c_{j}}} \right)^{-1}\left(\sum_{t_{c_{j}}}^{M_{c_{j}}} 0.98^{m_{e} - t_{c_{j}}}\right)\label{eq:22}
%\end{equation}

%\noindent We discount past observations using an exponential weighting function, $\delta^{m_{e}}$, where $m_{e}$ is the number of months between the current election survey month (for which a score is being calculated) and the month in the past that is under consideration.  We choose $\delta=.98$ , which means that service five years in the past is discounted by about one half and ten years in the past by about a third.

%\par The strategic ideological term in Equation \ref{eq:20} replaces each party's location on the ideological continuum with $Z_{c_{j}}$ which is the left-right position of each party in the coalition weighted by its historical share of the cabinet portfolios in the cabinet ($h_{jc_{j}}$).  These data contain information on the number of ministries held by each party in each month from 1960 to the present (which is defined as the date of the survey) and they identify the party of the prime minister and all of the parties in the governing coalition.  These data on cabinet portfolios were compiled by the authors and a detailed description of the sources is available at www.raymondduch.com/ideologicalvote.

%\par To calculate a party $j$'s seat shares in a particular coalition $c_{j}$, we simply sum the cabinet portfolios held each month ($t_{c_{j}}$) by that party for all of the months the coalition was in power $M_{c_{j}}$ which results in $s_{jc_{j}}$.  We then divide this sum by the sum of all portfolios, again over all of the months of the coalition, held by all coalition partners in that particular coalition $c_{j}$. This, of course, gets repeated for all parties over all possible coalitions that occurred from 1960 until the date of the survey.



\section{Assessing Theoretical Assumptions about Voter Coalition Information}

\par The claims we make in this theory presume that certain regularities in the coalition formation process register with the voter and they use this information in assessing coalition formation patterns after an election.  What are these regularities?  We will analyze five: coalition permutations; effective number of coalition parties; effective number of PM parties; re-election probabilities; and the ideological composition of governing coalitions.  Before moving to these we comment on the data and methods.

\subsection{Data \& Methods}

\par First a brief discussion of the data.  To answer the questions posed above, we employ data from 489 voter preference surveys from 56 countries over the years 1980-2008.  The figure in Appendix shows the country-years for which we have data.\footnote{Some countries have multiple datasets in a single year.  This is especially true of the US, where there are a number of studies existing for the most recent election year; however, this is also true for European countries which have, for example, both EES and WVS surveys completed in the same year.}  The data come from a number of different sources, including ().\footnote{See XX for a full discussion of the data employed here.}  Each dataset contains, at a minimum, vote choice, left-right self-placement and a minimal set of controls necessary for properly specifying the vote function.  Some datasets also contain a more full set of controls and as discussed below, we utilize those as they are available.



\subsection{Coalition Permutations}

\par We begin with Figure~\ref{fig:coalhist} that simply indicates the frequency with which countries in our sample have been governed by a multi-party coalition. Note that with the exception of three countries, coalition governments are the norm and in fact the median frequency of coalition governments is about 65 percent of the time.

\begin{figure}[p]
\caption{Frequency with which Coalition Governments Occur}\label{fig:coalhist}

\centerline{\includegraphics[width=\textwidth]{coalhist.pdf}}

\end{figure}

\clearpage


We first present a simple frequency of coalition permutations in each country where we find multi-party coalition governments. In the second figure we only include coalition parties that obtained XX percent of the vote.  Note that the modal frequency across all countries.....

\subsection{Effective Number of Coalition Parties}

\par One metric is the number of "governing" parties in a political system.  At one end of the continuum is a two-party system in which two parties alternate in government.  Voters can anticipate post-election government formation with certainty -- the party with the majority of seats in the legislature will form the government -- typically the case in Canada or the UK, for example.  But of course in coalition government contexts the permutations of possible governing coalitions that occur after an election are greater than 2. And as these possible permutations rise in number we might expect it to become increasingly difficult for voters to anticipate the likely coalition negotiations that occur after an election.  We propose a metric that we believe captures this complexity.  It is the effective number of coalition parties ($C$) calculated as follows,


\begin{equation}
C=\frac{1}{\sum_{i=1}^{n}P^{2}_{i}}\label{eq:22}
\end{equation}

\noindent where $P_{i}$ is calculated by summing for each party the total $m_{i}$ months during which it occupied a coalition governing position divided by the sum of all party coalition months ${\sum_{i=1}^{n}m_{i}}$ for the country.

\par The formation of a governing coalition consists of two particularly important outcomes: the designation of the parties receiving portfolios in the newly formed governing coalition; and the designation of the Prime Minster Party, the party from which the Prime Minister is chosen.  We begin by characterizing the effective number of Prime Ministerial parties in each of the countries in our sample, $C_{pm}$.  One piece of information that the voter can reconstruct from historical coalition formation outcomes is the effective number of Prime Ministerial parties.  A $C_{pm}$ of 2 would indicate that two parties effectively alternate in controlling the Prime Ministership in a governing coalition.  As this value rises about two voters face an increasingly more complex task of assessing which party would provide the Prime Minister in post-election coalition bargaining. Figure~\ref{fig:numpp} provides a summary of the $C_{pm}$ for our sample of countries.

\begin{figure}[p]
\caption{Effective Number of Prime Ministerial Parties}\label{fig:numpp}

\centerline{\includegraphics[width=\textwidth]{numpp.pdf}}

\end{figure}

\clearpage

\par What is interesting here is that there is considerable stability in terms of the party serving as the Prime Minister in coalition governments.  Note that in some countries the $C_{pm}$ value is less than 2 suggesting that one party is highly likely to be selected as the Prime Minister party in virtually all of the coalitions that form after an election.  This is the case, for example in Netherlands where the CDA is virtually always designated as the PM party and hence the country has a $C_{pm}$ score of 1.8.  Japan is at the extreme low end with a value of 1 since the same party has held the Prime Ministership over the entire time period covered by our data.  Hence in some countries this particular element of the post-election negotiation process is know almost with certainty, not unlike the situation in two-party systems where the post-election designation of the governing party is known with certainty.  Note that the median $C_{pm}$ value for all of the countries in our sample is 2. But note also that a number of countries have quite high values, greater than 3.  These tend to be the relatively new democracies of Eastern Europe where we find much less stability in designation of the Prime Minister party.

\par A second metric that characterizes the complexity of the post-election coalition bargaining outcomes is the effective number of coalition parties, $C_{p}$.  The calculation of this value involves the percentage of all months in the period 1960 to the present that each party was included in a governing coalition.  Rising values of this measure indicate an increasingly large number of parties historically have been likely candidates for entering the governing coalition after the election.  As Figure ??? illustrates, our sample of countries vary in terms of this effective coalition party metric.  Overall, the median number of effective coalition parties is ???.... And this suggests.....  But note that the results suggest interesting clusters....


\subsection{Re-election Probabilities}

\par As we pointed out earlier, one of the defining characteristics of coalition government contexts is that election results, while certainly affecting post-election coalition negotiations, are not determinative.  Hence, this raises the possibility of either exaggerated or moderated volatility in government turnover as a result of elections.  In this section we explore whether the volatility of government turnover in coalition government contexts significantly increases the difficulty for voters trying to anticipate likely post-election bargaining.  We will explore a number of different measures of volatility in order to establish this degree of post-election bargaining complexity.

\par First, probably one of the most restricted measures of volatility is the degree to which an incumbent coalition is returned to power with exactly the same party composition.  Our sample includes ?? elections over the ?? countries and the ?? time period.  Figure~\ref{fig:coalition_same} presents, for each country, the frequency with which coalitions are returned to power intact after an election.  As one might expect, countries that typically have single-party governments (Japan, Canada, Germany, Australia, and the UK) stand out as having the highest re-election frequency.  What is interesting though is that even in countries with a long history of multi-party coalitions (Belgium, Italy, Norway and the Netherlands to name some) we find that around 40 percent of the time the coalitions that form after an election look exactly like the incumbent coalition government.  The newly democratized countries have very low frequencies of re-electing the precise same coalition although this is simply likely to the fact that we have very few observations (i.e., elections) for these cases.

\begin{figure}[p]
\caption{Frequency with which Exact same Coalition is Re-elected}\label{fig:coalition_same}

\centerline{\includegraphics[width=\textwidth]{samecoal2.pdf}}

\end{figure}

\clearpage

\par A second, somewhat less restrictive, measure of volatility is the frequency with which the Prime Ministerial party in the incumbent governing coalition is returned to power after an election.  Figure~\ref{fig:coalition_pm} presents the frequency with which the PM party is returned.  Note that for the majority of countries in our sample, the probability of re-electing the PM party is greater than 0.5 -- in fact the median country has a probability of 0.6.  And many of the countries at or above the median are those with a long history of multi-party coalition governments.  In Belgium the PM party has an 80 percent chance of re-election; 75 percent in the Netherlands; and 65 percent in Italy.

\begin{figure}[p]
\caption{Frequency with which PM Party is Re-elected}\label{fig:coalition_pm}

\centerline{\includegraphics[width=\textwidth]{samepm.pdf}}

\end{figure}

\clearpage


\par  Figure~\ref{fig:samemem} presents the frequency with which at least one party is returned.


\begin{figure}[p]
\caption{Frequency with which at least one Party Re-elected}\label{fig:samemem}

\centerline{\includegraphics[width=\textwidth]{samemem.pdf}}

\end{figure}

\clearpage

\par  Figure~\ref{fig:nosamemem}  presents the frequency with which no party from the incumbent coalition is returned to the post-election governing coalition....

\begin{figure}[p]
\caption{Frequency with which at least one Party Re-elected}\label{fig:nosamemem}

\centerline{\includegraphics[width=\textwidth]{nosamemem.pdf}}

\end{figure}

\clearpage


%\subsection{Ideological Composition of Governing Coalitions}

%\par A third element of post-election stability concerns the ideological composition of post-election governing coalitions.  Does the location in the ideological space of parties that join coalition governments or the ideological characteristics of the governing coalition help voters anticipate the likely coalitions that form after an election.  This section explores this theme.


%\par First we explore whether the parties that enter into governing coalitions exhibit stable ideological characteristics.  Are parties that enter into the governing coalition more likely to be proximate to the median voter?  We explore this in Figure ?.  For the 1200 political parties in our data set we have simply plotted the frequency with each party was a member of a governing coalition (in months) against their absolute distance from the median voter [FOR WHAT PERIOD?]......  The results are quite striking.....  This suggests that voters.....


%\par We have suggested that voters make a critical calculation in anticipating the likely coalitions that form after an election -- that is that they understand the weighted ideological location of different coalitions that might form after an election.   One factor in aiding the voter in anticipating which coalitions form is the fact that the coalitions that do form are typically ideologically proximate to the median voter.  In other words the weighted ideological location of most coalitions is predicably proximate to the median voter.  We explore this by calculating the weighted ideological location of each coalition that formed in each of our ?? countries during the period ?? to ??.  Figure ?? presents a frequency histogram of absolute values of these distance measures.  Note that the frequency declines as this magnitude increases.




\section{Conclusion}




\newpage
\singlespace
\bibliography{dave}
\end{document}

Result 1:
First I made a variable called "incoal" which is a dummy that takes value 1 if the party is in a coalition government at the given time and zero otherwise.  Then, I calculated the P_i as the mean of the incoal variable for each party.  Finally, I calculated the effective number of coalition parties as simply the sum of p_i^2, because its inverse is undefined for a number of countries and even for some where it is defined, the number is very large (e.g., >2000).  The median here is 0.73.  The figure is sum_pi2.pdf

Result 2:
I created a new dataset that was subset on pm = 1.  Then I calculated the percentage of the prime-ministerships held by each party.  Then I squared and summed these entries and took the inverse of the sum.  The figure for this result is numpp.pdf

Result 3:
I'm not sure how what you want in the paragraph at the bottom of page 11 (starting "A second metric") is different from what I calculated for the first result.

Result 4:
I used the election dates in the Manifesto data and for "robustness" (i.e., coalitions not changing exactly on the election), I looked at whether the governing coalition 2 months prior to the election was the same as the governing coalition 2 months following the election.  Same with prime minister's party, and some member of the coalition.  The probability of no member of the coalition being re-elected is one minus the probability that some member of the coalition is re-elected.  These are in figures samecoal.pdf, samepm.pdf, samemem.pdf and nosamemem.pdf, respectively.

I suspect that I won't be done the ideological placement analysis before you get on the plane, so I thought I'd start you off with this and work on that so you'll have it tomorrow. I also didn't get a chance to do the monthly version of the analysis for results 1 and 2 yet, but I suspect that will be less interesting.  Since each time-point simply adds one observation and re-calculates the result, there will be strong dependence over time, but deterministically, not stochastically, so I'm not sure we can do much with that, but I'll work on it after I've generated all the other results.  I've also attached the data I used to generate all of the plots so you can actually see and manipulate the numbers if need be.

Best,
Dave.






%\par The models discussed below will estimate the underlying utility of respondents for parties by estimating a conditional logit with vote preference over competing political parties as the dependent variable.  The vote preference question in the surveys we analyze is typically of the form, ``if an election were held today which party would you vote for?".  We did not for example use surveys that only asked respondents to which party they felt closest  (e.g., the early Italian Eurobarometers) or that asked respondents if they would ever vote for a certain party (e.g., work by \citet{vandereijketal2007}).

%\par The vote choice questions we use differed in two ways: (1) in their relationship to the election for which the vote applied and (2) in their treatment of non-voting.  With respect to the first issue, surveys conducted directly after elections simply ask respondents to report their vote choice in the preceding election.  In contrast, those surveys that were conducted just before an election ask respondents for whom they intend to vote for in the upcoming election.  Finally, surveys that were not proximate to an election (e.g., many of the Eurobarometer surveys) ask the voter about a hypothetical election: ``If there were a general election tomorrow, which party would you support?"

%\par The second difference in the vote choice variable concerns the treatment of non-voting.  All thesurveys we used allow the voter to express whether they did not vote or do not intend to vote.  Further, most allow the voter to indicate if she cast (or intends to cast) a blank ballot.  Where these studies differ is in how they elicit the information that the respondent does not intend to vote. In many surveys the option of non-voting is simply included along with the other parties in the vote choice question.  In others, however, a two-question format is used in which the respondent is first asked whether she voted (or intends to vote) and only then for whom she voted (or intended to vote).  While this is a readily apparent difference in the way the vote choice question is asked in different surveys, it is unlikely to be consequential in our analysis, since (for other reasons) we decided to ignore non-voters in our analysis.  The important point is that in all our studies voters where allowed some way to express that they either did not vote or did not intend to vote.

%\par The Euclidean distance term which captures the proximity of voters to parties requires a measure of the respondent's self-placement on a the left-right ideological continuum.  The left-right self-placement questions we employ typically had wording similar to: ``In political matters, people talk of `the left' and `the right'.  How would you place your views on this scale? 1=Left 10=Right."  The left-right scales were of different ranges across the surveys (some were 10-point, others 7-point, etc.) but were all standardized to have mean zero and unit variance to facilitate comparisons across surveys.\footnote{A detailed description of the surveys and question
wording of items used in the analysis is available on the author's web site: www.raymondduch.com/ideology.}

\par Empirically, the more problematic element in the Euclidean term is the measure of party
placements.  We measure party placement with the mean of the left-right placement of the voters
for each party.  The attraction of this measurement strategy is that it enables us to estimate
ideological distance for all of the voter preference studies in our sample.  It also has the
advantage of avoiding endogeneity bias that some claim is associated with measures of party
placement that are based on respondents locating parties on the left-right
ideological continuum \citep{Macdonaldetal2007}.\footnote{\citet{Rehm2007} provides a detailed
discussion of the merits and disadvantages of employing the constituency-based strategy for
estimating party positions, including comparisons with the other methods for locating parties in
policy space.}

\par As mentioned in the theoretical section, a goal of the estimation for each survey is to produce
consistent estimates of the impact of ideological self-placement on vote choice.  We argued that to
achieve consistency, however, we cannot simply include ideological self-placement and the vote in
the statistical model, but must include other variables that impact the vote choice.  Only if we
have accounted for all the important influences on the vote will we be confident that our estimates
reflect the true relationship between ideological self-placement and vote choice in the population
to which the relevant survey applies.  With this in mind, we build our statistical models for each
survey by including variables that are known to be important in voting in the particular country
and time.  We identify those variables from the
literature on comparative voting behavior and on the country specific literatures on voting in each
country.  In theory, this strategy can lead to very different model specifications in different
countries, but in practice this is not the case.  First, there is a great deal of agreement across
countries about the basic factors that drive vote choice.  This means that the literatures in the
different countries usually point to the same kinds of variables as important determinants of the
vote.  Second, since the scholars who have written the voting literatures in each country are
usually the same people who design the surveys, measures of these basic factors are usually
included in election studies.\footnote{Details on these model specifications is available on the
author's web site: www.raymondduch.com/ideology.}

\par To estimate the extent to which vote choice depends on ideological proximity, we model voice choice with a conditional logit \citep{maddala1983}.  Since the control variables available to us across different countries, elections and data-sources vary widely, we adopt a two-stage strategy.  In the first stage, we estimate the conventional conditional logistic regression where:

\begin{equation}
\Pr(Y_{i} = m) = \frac{e^{\beta^{\prime}X_{im} + \alpha_{m}^{\prime}Z_{i}}}{\sum_{m=1}^{M}e^{\beta^{\prime}X_{im} + \alpha_{m}^{\prime}Z_{i}}}
\end{equation}

\noindent Where $X_{im}$ is an $N\times M$ matrix of choice-specific (i.e., party-specific) characteristics, such as the distance from each individual to each party, $Z_{i}$ is a $N\times K$ matrix of choice-invariant characteristics (e.g., respondent age or education).  $\beta$ is a vector of  coefficients relating the choice-specific characteristics to the choice probabilities.  Notice, these do not vary by choice; that is to say, the same coefficient relates each choice specific variables to the response.  $\alpha_{m}$ are choice-specific coefficients relating the choice-invariant characteristics to choice probabilities.  For identification, we set $\alpha_{1} = 0$.

\par The second stage model attempts to take results from the first stage and model them as a function of either party- or survey-specific (i.e., country-year specific) characteristics.  As such, we need to generate estimates of both quantities.  The survey-specific measure is relatively easy to obtain.  For this, we can simply use the coefficient on the party-distance term from each model.\footnote{Given the automatic scaling of logit coefficients to meet the assumed error variance, this strategy may not be perfect, however, we see few better alternatives and thus will proceed with this strategy until a better one reveals itself.}  Given that the distance coefficients do not vary by party, we need to consider the party-specific implications for those non-varying coefficients.  The marginal effect of distance will be quite different depending on $\alpha_{m}$ and $Z_{i}$.  To capture this difference, we simulate a change in each voter's left-right placement.  Specifically, we do the following:

\begin{enumerate}

\item Calculate the baseline number of predicted votes for each party $n_{m}$. Then, for each party in turn,

\begin{enumerate}

\item Move voters, $v_{i}$, one-unit further away from the party of interest $p_{m}$ on the left-right dimension to $v_{i}^{\prime}$.

\begin{equation}
v_{i}^{\prime} = \left\{
               \begin{array}{ll}
                 v_{i}+1, & \hbox{if $v_{i}\geq p_{m}$} \\
                 v_{i}-1, & \hbox{if $v_{i} < p_{m}$.}
               \end{array}
             \right.
\end{equation}

\item Re-calculate party distances $|v_{i}-p_{m}|$ for each $m = \{1, \ldots, M\}$.

\item Re-calculate the number of votes predicted for each party, $n_{m}^{\prime}$.

\end{enumerate}

\item Calculate the percentage of votes lost due to the simulated one unit change:

\begin{equation}
\Delta n_{m} = \frac{n_{m} - n_{m}^{\prime}}{n_{m}}
\end{equation}

\end{enumerate}

\noindent As ideology becomes more important, parties should lose more of their votes to competing parties as voters move away.  If the voters' utilities are dominated by other factors (e.g., age, education, gender, income), the marginal change in probabilities induced by the simulation should be quite small and parties should not lose many votes.

\section{Results}

\par Figure~\ref{fig:coefs} shows dot-plots of the coefficients on the ideological proximity variables by country.  The dots represent the median in each country and the width of the bars represent the range of the coefficients in each country.\footnote{The results presented here are obviously the results from nearly 500 independent models.  While we do not have space to present all of the model results, the model results will be made available as will replication code on the author's website}



\subsection{Left-Right Ideology: Cross-national Validity}


\begin{figure}[p]
\caption{Missing Values on Left-Right Ideology Question}\label{fig:miss_dot}

\centerline{\includegraphics[width=\textwidth]{miss_dot.pdf}}

\end{figure}

\clearpage



\begin{figure}[p]
\caption{Missing Values on Left-Right Ideology Question over Time}\label{fig:miss_time}

\centerline{\includegraphics[width=\textwidth]{missing_time.pdf}}

\end{figure}

\clearpage



\begin{figure}[p]
\caption{Missing Values on Left-Right Ideology Question Compared}\label{fig:miss_eu}

\centerline{\includegraphics[width=\textwidth]{missing.pdf}}

\end{figure}

\clearpage

\subsection{Left-Right Ideology: Cross-national Variation}

\par With OLS, of the 12 countries with 10-point which have at least 8 time-points, 50 percent (6) have negative and significant coefficients on time while only 2 (Greece and Italy) have positive and significant coefficients on time.  Looking across the 12 countries without regard for significance, 9 of the 12 countries have negative coefficients on time.  Of the 8 countries with 11-point scales which have at least 8 time-points, one (Uruguay) has a significant, negative coefficient on time and none have significant positive coefficients on time. Looking across all 8 countries, regardless of significance, 7 of the coefficients on time are negative.


\begin{table}[h!]
\caption{Latent Growth Model Results}
\begin{center}
\begin{tabular}{ld{3}d{3}d{3}d{3}}
& \multicolumn{2}{c}{10-Point} & \multicolumn{2}{c}{11-Point}\\
& \multicolumn{2}{c}{Scale} & \multicolumn{2}{c}{Scale}\\
& \multicolumn{1}{c}{All} & \multicolumn{1}{c}{$t \geq 8$} & \multicolumn{1}{c}{All} & \multicolumn{1}{c}{$t \geq 8$}\\
\hline
Year      & -0.010  & -0.010  & -0.018    &  -0.024   \\
          & (0.002) & (0.002) & (0.007)   &  (0.016)  \\
Intercept & 26.091  & 27.101  & 42.211    &  53.551   \\
          & (5.221) & (5.426) & (15.948)  &  (33.111) \\
\hline
\# Groups               &  \multicolumn{1}{c}{53}  &   \multicolumn{1}{c}{23}  &  \multicolumn{1}{c}{44}  &  \multicolumn{1}{c}{8}  \\
\# Obs                  &  \multicolumn{1}{c}{299} &   \multicolumn{1}{c}{189} &  \multicolumn{1}{c}{196} &  \multicolumn{1}{c}{73} \\
DF                      &  \multicolumn{1}{c}{245} &   \multicolumn{1}{c}{176} &  \multicolumn{1}{c}{151} &  \multicolumn{1}{c}{64} \\
\hline
$\sigma_{\alpha}$       & \multicolumn{1}{c}{0.492} & \multicolumn{1}{c}{0.349} &\multicolumn{1}{c}{0.380} &\multicolumn{1}{c}{0.121} \\
$\sigma_{\beta}$        & \multicolumn{1}{c}{2e-7}  & \multicolumn{1}{c}{5e-8}  &\multicolumn{1}{c}{7e-7}  &\multicolumn{1}{c}{2e-4}  \\
$\sigma_{\varepsilon}$  & \multicolumn{1}{c}{0.243} & \multicolumn{1}{c}{0.225} &\multicolumn{1}{c}{0.464} &\multicolumn{1}{c}{0.473} \\
\hline
\end{tabular}
\end{center}
\end{table}


\begin{figure}[h!]
\caption{Cross-National Variation in Left-Right Self Placement}\label{fig:trends_one}

\centerline{\subfigure[Seven Point Scale]{\includegraphics[height=3.5in]{lr_dot_7.pdf}}
\subfigure[Ten Point Scale]{\includegraphics[height=3.5in]{lr_dot_10.pdf}}}
\vspace{-.18in}
\centerline{\subfigure[Eleven Point Scale]{\includegraphics[height=3.5in]{lr_dot_11.pdf}}}


\vspace{.15in}

\end{figure}


\begin{figure}[h!]
\caption{Trends in Left-Right Self Placement}\label{fig:trends_one}

\centerline{\subfigure[Ten Point Scale]{\includegraphics[height=4.5in]{lr_time10.pdf}}
\subfigure[Sub-set of Ten Point Scale]{\includegraphics[height=4.5in]{lr_time10_sub.pdf}}}

\vspace{.15in}

\end{figure}

\begin{figure}[p]
\caption{Trends in Left-Right Self Placement}\label{fig:trends1}

\centerline{\includegraphics[width=\textwidth]{lr_time10.pdf}}

\end{figure}

\begin{figure}[p]
\caption{Trends in Left-Right Self Placement}\label{fig:trends2}

\centerline{\includegraphics[width=\textwidth]{lr_time10_sub.pdf}}

\end{figure}


\begin{figure}[h!]
\caption{Trends in Left-Right Self Placement}\label{fig:trends_two}

\centerline{\subfigure[Ten Point Scale]{\includegraphics[height=4.5in]{lr_time11.pdf}}
\subfigure[Sub-set of Ten Point Scale]{\includegraphics[height=4.5in]{lr_time11_sub.pdf}}}

\vspace{.15in}

\end{figure}

\begin{figure}[p]
\caption{Trends in Left-Right Self Placement}\label{fig:trends3}

\centerline{\includegraphics[width=\textwidth]{lr_time11.pdf}}

\end{figure}

\begin{figure}[p]
\caption{Trends in Left-Right Self Placement}\label{fig:trends4}

\centerline{\includegraphics[width=\textwidth]{lr_time11_sub.pdf}}

\end{figure}

\subsection{The Ideological Vote}

\begin{figure}[p]
\caption{Coefficients on Ideological Proximity from the Conditional Logit Models}\label{fig:coefs}

\centerline{\includegraphics[width=\textwidth]{dist_coefs.pdf}}

\end{figure}


\par Figure~\ref{fig:lose} shows the median percentage of voters lost for a simulated one-unit change in left-right ideology.



\begin{figure}[p]
\caption{Median Percentage of Voters Lost for a One-unit change in Left-right Self-placement}\label{fig:lose}

\centerline{\includegraphics[width=\textwidth]{medlose.pdf}}

\end{figure}


\par We can also consider trends over time in ideological voting.  Figures~\ref{fig:time_coef} and \ref{fig:time_lose} show the temporal trends in each country for the ideological proximity coefficient and the median percentage of voters lost for each country.   There are no overwhelmingly obvious time-trends in either context and simple linear latent growth models suggest statistically insignificant temporal trends on average with little interesting variation across countries.\footnote{These models are not presented here, but are available from the author upon request.}


\begin{figure}[p]
\caption{Trends in Proximity Coefficients}\label{fig:time_coef}

\centerline{\includegraphics[width=\textwidth]{coef_time.pdf}}

\end{figure}


\begin{figure}[p]
\caption{Trends in Median Percentage of Lost Votes}\label{fig:time_lose}

\centerline{\includegraphics[width=\textwidth]{lose_med_time.pdf}}

\end{figure}

\clearpage



\begin{table}
 \caption{Strength of Ideological Proximity Coefficient as a Function of LR Placement Missingness}
 \begin{center}
 \begin{tabular}{ld{3}d{3}}

&        \multicolumn{1}{c}{With} & \multicolumn{1}{c}{Without}\\
&        \multicolumn{1}{c}{Outliers} & \multicolumn{1}{c}{Outliers}\\
 \hline
  Intercept   &    -1.783   &    -1.697 \\
    &   ( 0.053)   &   ( 0.042) \\
\% Missing on LR Self Placement   &    3.773*   &    3.135* \\
    &   ( 0.389)   &   ( 0.309) \\
 \hline
N   &    \multicolumn{1}{c}{435}   &    \multicolumn{1}{c}{415} \\
 $R^2$       & 0.179       & 0.2  \\

\end{tabular}
\end{center}
\end{table}

\newpage
\singlespace
\bibliography{dave}
\end{document}

\subsection{Modeling the Results}

\par In second-stage models as suggested above, we can model the ideological distance coefficients and the percentage of votes lost.  These are both presented in Table~\ref{tab:2stage}.  Data on electoral system come from Golder XX.  Data on party size are taken from each individual voter preference survey.  We operationalize party size as the number of voters voting for the party in each survey.


\begin{table}[h!]
\caption{Second Stage Estimates}\label{tab:2stage}
\centerline{
\subfloat[Ideological Proximity]{\begin{tabular}{ld{3}}
 & \multicolumn{1}{c}{Estimate} \\
 & \multicolumn{1}{c}{(SE)}\\
\hline
(Intercept)  & -1.197  \\
             & (0.082) \\
Proportional & -0.269  \\
             & (0.103) \\
Multi        & -0.592  \\
             & (0.103) \\
Mixed        & -0.003  \\
             & (0.125) \\
\hline
$R^{2}$ & 0.120 \\
N & 336 \\
\end{tabular}}
\quad \quad\subfloat[\% Lost Votes]{
\begin{tabular}{ld{3}}
 & \multicolumn{1}{c}{Estimate} \\
 & \multicolumn{1}{c}{(SE)}\\
\hline
(Intercept)     &  0.602  \\
                & (0.052) \\
Log(Party Size) & -0.056  \\
                & (0.010) \\
\hline
$R^{2}$ & 0.018 \\
N & 1531 \\
\end{tabular}}}
$^{*}$ Main entries are OLS coefficients, standard errors are in parentheses.
\end{table}








\clearpage



\newpage

\section*{Appendix 1}

\begin{figure}[h!]
\caption{Country-years with Existing Data}
\centerline{\includegraphics[width=\textwidth]{data_years.pdf}}

\end{figure}



\subsection{Overall Does the Ideological Vote Matter?}

\begin{enumerate}
\item is ideology important?
\item distribution of $\theta$ within and across countries
\item distribution of $\phi$ for key control variables
\end{enumerate}


\par We begin by presenting empirical results that address our first contention, that the ideological vote is universal.  Here we rely entirely on the Model 1 sincere ideological vote specification.  Recall that we estimate fully specified conditional logit vote choice models for 400 voter preference surveys conducted in 57 countries.  Hence we have 400 parameter estimates for the sincere ideological distance term.\footnote{There are only 245 of these 400 surveys for which we have data on coalition portfolios, thus for Models 2 and 3, we only use that subset of the data presented in Figure~\ref{fig:coefdot}.  However, given the large number of data sets compiled, we thought it would be beneficial to present all of the information across all surveys here.}  Figure~\ref{fig:coefdot}(a) shows a dot-plot of these coefficients with a 95\% confidence interval imposed and with the lowest coefficient magnitudes at the top of the figure and highest at the bottom.  Most countries have votes that are cast on ideological grounds.  Note that the precision and statistical significance of coefficients varies quite considerably.  Recall that for any country we will have multiple estimates of the ideology coefficient because we have on average four studies per country, although the average for the more developed democracies is actually closer to seven (you can see the number of studies associated with each country in Panel (c)). Panel (b) shows the range of the estimated coefficients for each country - the minimum and maximum coefficients estimated in the conditional logit models in each country. The modal distance coefficient is approximately -0.5 and in fact the country distribution of modal coefficient values is actually quite skewed with most countries clustering around this value; only a handful of countries exceed this -0.5 value (i.e., have weaker ideological voting); but a reasonably large number of countries registering quite high ideological voting with modal coefficient values less than -1.0.

\par Finally, panel (c) shows the proportion of coefficients in each country that are {\em not} statistically different from zero.  Three countries were certain to have insignificant ideology coefficients -- Peru, Panama and Guatemala.  And nineteen other countries had probabilities exceeding 0.10 of having insignificant ideology coefficients.  Thirty-five countries, or 60 percent of the total, were certain \textbf{not} to have insignificant coefficients.  All of the non-significant countries with three notable exceptions can be considered transition democracies. Ireland is one of the exceptions and this is not unexpected given the importance of other more salient cleavages in the country \citep{Inglehartetal1976}.  Our results for the United States, the second exception, supports \citet{Fiorina2006}'s recent observation that the American voters 'are not very well-informed about politics, do not hold many of their views very strongly, and are not ideological' (p. 19). The other exception is Luxembourg.

\par Ideological voting occurs, at least with some frequency, in virtually all countries in the world: there are only three countries where we never see a significant ideological vote; about 15 countries where we frequently see no ideological voting; and almost two-thirds of the countries in our sample always have ideological voting.  There are some potentially interesting differences between those countries that always have an ideological vote and those where it is more variable, if not absent.  Transition democracies, but more particularly Latin American countries, are most likely to have insignificant ideological distance coefficients.  Of the Latin American countries, only Chile, El Salvador and Uruguay fall in the category of countries always having a significant ideological distance coefficient.  The other 13 Latin American countries either have no ideological vote or one that varies between significance and insignificance. This tendency for less-mature democracies to exhibit lower ideological votes confirms a similar pattern documented by \citet{Huberetal1995}. All of the Western European countries (with the exception of Luxembourg) always have a significant ideological vote.  Interestingly, virtually all of the transition democracies from East and Central Europe also fall in this category.\footnote{Given that the data come from various different sources (such as the Euro-Barometer, Latino-Barometer, CSES, etc.) we considered the possibility that some data sources might generate, on average, different inferences for the same country-years.  Of the 54 country-years that had surveys in the field from multiple organizations, only 12 showed inconsistency in the inferences made about the distance coefficient; that is, one study showed a significant distance coefficient and another did not.  The p-value for the $\chi^{2}$ statistic on the table of data sources and inconsistencies is 0.09 suggesting that there is no statistically significant relationship between the data source and inconsistency of results.}

\begin{figure}[h!]
\caption{Distance Coefficients in Conditional Logit Models$^{*}$}\label{fig:coefdot}

\centerline{\subfigure[]{\includegraphics[height=3in]{distcoefs.pdf}} \quad
\subfigure[]{\includegraphics[height=3in]{coefdot.pdf}}}
\centerline{\subfigure[]{\includegraphics[height=3in]{prinsig.pdf}}} \vspace{.2in} $^{*}$ Panel (a) shows the 95\% confidence interval of the distance coefficients from each of 461 conditional logit models (we use all available datasets here, not only the ones for which we have data on cabinet seats as below).  In panel (b), the bars represent the range of the estimated distance coefficients for each country.  For panel (c), the points represent the proportion of distance coefficients for a given country that were not significantly different from zero.  The number on the right gives the total number of surveys for each country.
$\end{figure}

% NOTE THIS COMMENT: As I am doing the revisions of the APSR manuscript, the question of how to present the new results has come up.  I am assuming that all of the empirical results concerning sincere ideological voting get dropped - basically pages 28 through 31 in Section 4.  Hence Figure 3 gets dropped.  Since are now estimating beta, the comparison between the sincere and strategic models that make up the rest of empirical results section + Figure 4 should also be dropped.  Hence Figure 4 I think should now be a presentation of the magnitudes of the beta coefficients across the various studies (first part of Figure) and then a summary measure of beta size/significance for each country.  Does this make sense.  If so I'll edit the sections accordingly in anticipation of the beta results.


\par We expect to see contextual variation in the ideological vote and this is confirmed by Panels (a) and (b) in Figure~\ref{fig:coefdot}.  Some countries appear to have dramatically more ideological voting than others -- this appears to be the case for Greece, France, the Czech Republic and Iceland, for example.  Panels (a) and (b) also make clear that the magnitude of this ideological vote varies quite significantly within any particular country, and hence from one time point to the next.  Some of this variation results because in certain contexts our $\lambda$ term is low -- considerations other than ideology dominate the vote choice decision.  But some of this observed variation in the sincere ideological vote results because voters are engaging in strategic ideological voting.  If voters are conditioning their ideological vote on post-election strategic considerations then the distance coefficient on sincere Euclidean distance should perform better in some contexts than others which it does. We now attempt to tease out of these data an indication of the importance of strategic ideological voting.

\par Again there are two strategic models: The strong test of our theory is embodied in Model 3 where we weight party locations according to the strategic term in Equation~\ref{eq:20}.  Our theory would be less convincing if we were to find that our strategic result was dominated by the Model 2 specification -- i.e., simply dropping parties that had no history of serving in government.  One test of the strategic model is to determine whether it generates significantly stronger (more negative) distance coefficients than those estimated for the sincere model.\footnote{Since the we're moving points around, the variance of the distance variable is likely to change potentially changing the coefficient without having really changed inferences. To prevent finding a spurious relationship like this, we normalize both the distances to have unit variance.}

\par In Figure~\ref{fig:difcoef} we compare the magnitudes of the distance coefficients for Models 1-3. Panel (a) presents the results of subtracting the distance coefficient generated by Model 2 from the distance coefficient estimated for Model 1 -- these are generated for all 245 voter preference studies along with their 95\% confidence intervals.\footnote{For distance coefficients $b_{n}$ and $b_{t}$ from the sincere and strategic models, respectively; \begin{equation*} \hat{\sigma}_{b_{n} + b_{t}}=\sqrt{\hat{\sigma}_{b_{t}}^{2} + \hat{\sigma}_{b_{n}}^{2}} \end{equation*}} Negative values here favour the strategic Model 2 over the sincere Model 1.  On balance most of the significant differences are decidedly in favour of the sincere model (93) as opposed to the strategic Model 2 (9).  Hence simply constraining the vote choice to parties that are likely to participate in the post-election governing coalition does not improve on a sincere model of the ideological vote that includes these parties.  To the extent there is a strategic ideological vote it is not simply the abandoning of parties with $\gamma_{c_{j}}$ terms that are all zeros.


\par Model 3 incorporates the richer component of the strategic ideological vote identified in Equation~\ref{eq:20}: Voters' re-position parties in the ideological space consistent with their expectations as to how these parties will influence post-election coalition policy compromises.  Panel (b) in Figure~\ref{fig:difcoef} presents the difference in coefficient magnitudes between this Model 3 specification and the Model 1 sincere ideological vote specification. Model 3 has significantly more negative distance coefficient than Model 1 and in only 18 cases do we see the reverse. These results are confirmed in Panels (c) and (d) that provide a tally of countries where the strategic model has a significantly more negative coefficient than the sincere model. This is the number of surveys within each country that had a significantly more negative coefficient for distance divided by the total number of surveys in that country.  Hence, a value of 1 in Panel (c) indicates that all of Model 2's predictions were significantly more negative than those of the sincere Model 1; a value of 0 indicates that none of Model 2's predictions were more negative than Model 1.  In Panel (c) only 5 of 30 countries have a significantly more negative distance coefficient in Model 2 as compared to Model 1, confirming the results in Panel (a).  On the other hand, in Panel (d) around half of the countries have some probability of having a significantly more negative distance coefficient in Model 3 as compared to Model 1. This indicates that there is a stronger relationship between ideological distance and vote choice in vote models that incorporate a richer specification of strategic ideological incentives than is the case with the standard Euclidean ideological distance term that is typically found in vote choice models.

\begin{figure}[h!]
\caption{Differences between Strategic and Sincere Coefficients}\label{fig:difcoef}

\centerline{\subfigure[]{\includegraphics[height=3in]{distcoef12.pdf}} \quad
\subfigure[]{\includegraphics[height=3in]{distcoef31.pdf}}}

\centerline{\subfigure[]{\includegraphics[height=3in]{prsig12.pdf}} \quad
\subfigure[]{\includegraphics[height=3in]{prsig13.pdf}}}
\end{figure}


\par The crux of our theoretical claim is that the $\beta$ term in Equation~\ref{eq:20} is non-zero.  We do not claim that $\beta$ is universally non-zero. First, there are a large number of contexts in which the strategic ideological component of Equation~\ref{eq:20} is zero by construction, i.e., there is no history of coalition governments.  Second, we entertain the possibility that there is an expressive component to the ideological vote, i.e., $1-\beta$ is non-zero.  Accordingly, the empirical test of our theory consists of demonstrating that there clearly are contexts in which, and parties for which, the strategic formulation of the ideological vote does a better job than the sincere model of explaining vote choice.  We believe the empirical evidence presented in this section clearly establish this is the case.\footnote{This raises a number of interesting questions regarding cross-national characterizations of the left-right ideological space that we do not address here but will in future research. One might expect, for example, that party placements are more uncertain in contexts where there is significant deviations between our weighted ideological placement of parties and sincere ideological placements.}

\subsection{Predicting Vote Choice}

\par The previous section established that our strategic ideological model frequently performs better with respect to the coefficient on the ideological distance term.  A more demanding test of our theory is whether, over a large number of cases, Model 3 in fact does a better job of predicting actual vote preferences.  This is a demanding test because even in contexts in which the $\beta$ is non-zero, the strategic ideological model will generate vote predictions that are identical to those of the sincere ideological model. Nevertheless we expect that in many contexts the strategic ideological will generate more accurate vote predictions for competing political parties.  Hence in this section we present the frequency with the strategic ideological model generates more accurate predictions; and describe the contexts in which strategic ideological voting appears to be more common.

\par Predicted votes associated with each of the three models are based on simulations that included 182 parties from 30 countries and 245 distinct data sets - a total of 892 predicted party votes.\footnote{We only used data sets where we had data on the partisan breakdown of cabinet or ministry level positions and where at least one government formed in the past had a coalition.  Given that the results of the strategic and sincere models would be identical for those countries without coalition history, this seems reasonable.} The performance of each of the three models is assessed by the difference in correctly predicted votes between either Model 2 or Model 3 and Model 1.  On average there are more negative values than positive ones (favoring the sincere model).  Over all of the cases in our data, the sincere voting model outperforms Model 2 for 394 parties (44 percent of the cases) and Model 3 for 368 parties (41 percent) -- these are cases that have differences less than -5\% (344 and 315 significantly different from zero, respectively). The sincere model outperforms both Model 2 and Model 3 in about 60 percent of the cases.  Note that parties with no government experience will very likely show significant differences in favor of the sincere model because the strategic model will predict 0 correct votes by construction.  If the sincere model correctly predicts on average even a few of those parties' votes, there will be significant positive differences.

\par  In roughly forty percent of the cases Model 2 or Model 3 outperform the sincere ideological model.  Some of this deviation from sincere ideological voting is consistent with our theory.  Recall that Model 2 identifies that aspect of the strategic ideological vote decision that simply results from voters abandoning parties with no history of governing.  To the extent that this is the only incremental contribution of our model to explaining party vote choice then our contribution to vote choice theory would be relatively limited.  Model 3 on the other hand captures the more novel aspect of the theory whereby voters reposition parties in ideological space conditional on their likely participation in the governing coalition.  The strong test of this novel component of our theory is the extent to which Model 3 generates more accurate predictions of a party's vote.  Our simulated predictions indicate that in about half of the cases, predicted to deviate from sincere ideological voting, Model 3 performs better than Model 2.  Model 2 generates differences great than 5\% for 206 parties (23 percent) while Model 3 generates differences greater than 5\% for 197 parties (22 percent), with around 140 significantly different from zero in both cases.

\par Our second goal here is to identify the contexts in which our model of the strategic ideological vote (Model 3) outperforms other representations of the ideological vote.  Within any particular survey context some parties are more sensitive to this particular component of the strategic ideological calculations than others.  To illustrate we examine parties whose difference between Models 2 and 3 is significant at the 95\% level.\footnote{This means 95\% of the bootstrap replicate percent correctly predicted figures for Model 3 would be higher (or lower) than those from Model 2.} Table~\ref{tab:m2m3} shows these results where Models 2 and 3 differ significantly and indicates which parties gain and lose predicted votes.

\par There are systematic patterns in these Model 2 and Model 3 differences that are certainly worth further exploration.  Note, for example, in France the strategic Model 3 generates significantly better predictions for the three of the four major parties in the countries: the Socialists and the Union pour la D\'{e}mocratie Francaise (UDF) are always predicted better by Model 3 when there is a significant difference in predictions; and the Rassemblement pour la R\'{e}publique (RPR) are better predicted by Model 3 twice as often as they are better predicted by Model 2.  The Communists (PCF) are the sole exception in that they are always better predicted by Model 2.  In Germany, like France, the two largest parties are favored by strategic Model 3: the Christian Democratic Union (CDU) is always predicted better by Model 3; and the Social Democratic Party (SDP) is better predicted by Model 3 twice as often as it is better predicted by Model 2. The FDP is always better predicted by Model 2.\footnote{The Greens are always better predicted by Model 3 but this is based on one observation}  Finally, in the Netherlands, the PvdA and VVD are about evenly split between being better predicted by Models 2 and 3; however the CDA is always predicted worse by Model 3; and D66 is always better predicted by Model 3.

\begin{table}[p]
\caption{Significant Differences between Model 2 and Model 3}\label{tab:m2m3}
\begin{center}
\begin{footnotesize}
\begin{tabular}{lcccccc}
 & $M3 > M2$ & $M2 > M3$ & & & $M3 > M2$ & $M2 > M3$\\
\hline\hline
Albania/PDS             & 1  & 0& & Ireland/FG              & 4  & 0\\
Albania/PSS             & 1  & 0& & Ireland/Lab             & 0  & 7\\
Australia/Lib           & 0  & 2& & Israel/Labour           & 1  & 0\\
Australia/Nat           & 0  & 1& & Israel/Merez            & 2  & 0\\
Belgium/CVP             & 0  & 5& & Italy/exright           & 1  & 0\\
Belgium/Green           & 1  & 0& & Italy/FI                & 0  & 1\\
Belgium/PS              & 7  & 2& & Italy/LN                & 0  & 1\\
Belgium/PVV-VLD         & 2  & 0& & Italy/PDS               & 1  & 1\\
CzechRepublic/CSSD      & 4  & 0& & Lithuania/LCS           & 0  & 1\\
CzechRepublic/KDU-CSL   & 3  & 0& & Lithuania/TS            & 0  & 1\\
Denmark/CD              & 1  & 0& & Netherlands/CDA         & 0  &11\\
Denmark/KF              & 1  & 6& & Netherlands/D66         & 5  & 0\\
Denmark/RV              & 0  & 1& & Netherlands/PvdA        &13  &10\\
Denmark/SD              &11  & 0& & Netherlands/VVD         & 2  & 2\\
Denmark/V               & 2  & 2& & NewZealand/Nat          & 0  & 1\\
Estonia/KESK            & 1  & 0& & Norway/DNA              & 0  & 1\\
Estonia/Mood-SDE        & 0  & 1& & Norway/H                & 4  & 0\\
Estonia/Reform          & 1  & 0& & Norway/KRF              & 0  & 1\\
Finland/KES             & 0  & 1& & Poland/PSL              & 0  & 2\\
Finland/KOK             & 2  & 0& & Poland/SLD              & 1  & 2\\
Finland/SFP             & 0  & 1& & Portugal/PSD            & 1  & 0\\
Finland/SSDP            & 2  & 0& & Portugal/PSP            & 0  & 1\\
Finland/VAS             & 1  & 1& & Slovakia/APR/DEUS       & 0  & 1\\
Finland/VIHR            & 0  & 2& & Slovakia/HZDS           & 1  & 3\\
France/PCF              & 0  & 4& & Slovakia/KDH            & 2  & 0\\
France/PS               & 7  & 0& & Slovakia/SDL            & 0  & 4\\
France/RPR              & 8  & 4& & Slovakia/SKDU           & 0  & 1\\
France/UDF              & 2  & 0& & Slovakia/SMK            & 0  & 1\\
Germany/CDU-CSU         & 3  & 0& & Slovenia/LDS            & 0  & 1\\
Germany/FDP             & 0  & 3& & Slovenia/SDSS           & 1  & 1\\
Germany/Green           & 1  & 0& & Slovenia/ZLSD           & 0  & 1\\
Germany/SDP             & 2  & 1& & Sweden/CD               & 2  & 0\\
Greece/KKE              & 0  & 4& & Sweden/CEN              & 1  & 0\\
Greece/Left Alliance    & 0  & 1& & Sweden/MOD              & 0  & 3\\
Greece/ND               & 0  & 5& & Sweden/PP               & 5  & 0\\
Hungary/FKGP            & 1  & 0& & Sweden/SD               & 2  & 2\\
Hungary/MDF             & 1  & 0& & & & \\
Hungary/MSZP            & 0  & 2& & & & \\
\hline\hline
\end{tabular}
\end{footnotesize}
\end{center}
\end{table}


\par As we pointed out earlier, the strategic and sincere components of Equation~\ref{eq:20} often generate very similar, if not identical, predicted party votes. Nevertheless, as Table~\ref{tab:m2m3} indicates, in some countries and for some political parties the strategic ideological model always outperforms the sincere model.  And there is also considerable evidence that the strategic and sincere models alternate in terms of generating more accurate vote predictions.  We do not pretend to have a general explanation here as to when, or in which context, one model will outperform the other (aside from the obvious that strategic ideological voting only occurs in coalition contexts). When and where strategic ideological considerations come into play is of course difficult to anticipate since their importance are contingent on the both voter utilities, party placements and coalition histories.  Hence, the conservative modeling strategy is to always incorporate strategic ideological terms in vote choice models in coalition contexts.

%\par The comparative statics presented earlier in Figure \ref{fig:plottwo}, suggested two possible relationships between party characteristics in their strategic ideological vote:  First we suggested that parties with particularly low and high levels of administrative responsibility within a coalition will not be the recipients of a strategic ideological vote while the strategic ideological vote will be quite pronounced for those parties with moderate levels of administrative responsibility.  Hence, we expect a quadratic between administrative responsibility and strategic ideological voting.\footnote{We drop from consideration the parties that never govern although leaving them in can produce roughly the same result, but requires another degree of freedom, basically a dummy variable representing these parties.}  We do not expect  Figure \ref{fig:plottwo} also suggested that in general we should expect a positive, linear, relationship between the probabilities of a party entering into a coalition and its strategic ideological vote.

%\par While it is difficult to generalize strictly based on our model as to when strategic considerations will dominate sincere ideological voting, our large data set of ideological votes enables us to draw out some empirical regularities.  Our interest here is determining how the strategic ideological vote for particular parties varies as a function of the two key parameters in our model: $\gamma_{c_{j}}$ (probabilities of entering a coalition) and $h_{jc_{j}}$ (share of administrative responsibility).  Figure~\ref{fig:mod1} presents the predictions from a Generalized Additive Model \citep[see,][]{Wood2006} that fits the relationship between the strategic vote with percentage of ministries held and coalition probability as a smoothed surface.  The strategic vote being summarized here is the difference between Model 3 (our strategic voting model) and Model 1 (the sincere model). We are presenting these results simply as a means of a smoothed description of strategic voting by these two variables (i.e., not as a predictive model from which we will make inferences about the magnitude and significance of coefficients). However, this model (that uses roughly 10 degrees of freedom) explains about 30 percent of the variance in strategic voting.  Two patterns seem clear in these data: First strategic ideological voting is particular high for parties that have probabilities of around 50 percent of participating in coalitions.  If their $\gamma$ term gets too high or too low they become a less likely recipient of a strategic ideological vote.  Second, for those parties that do have around 50 percent probability of participating in a coalition government, the relationship between administrative responsibility and the size of their strategic ideological vote is decidedly negative.  For these parties, as their administrative responsibility rises their strategic ideological vote declines.  Clearly there are coalition contexts that trigger high levels of strategic ideological voting -- vote choice models that do not incorporate this aspect of the voting calculus will be misspecified.


%However, this is not a perfect world as we do not have one measure of strategic voting for each
%model, but rather $n=100$ bootstrap replicates. We use the method described by
%\citet{Armstrongetal2007} to incorporate the first-stage estimation uncertainty into the model.
%Briefly, this method uses the method described by \citet{Rubin1987} and \citet{Schafer1997} to
%propagate multiple imputation uncertainty through statistical models to generate the appropriate
%coefficient and standard error estimates. This requires the estimation of 100 different models and
%the procedure basically averages across them. GAMs are not appropriate here because their
%flexibility allows the knot locations to move to accommodate the data.  To generate consistent
%results from the method used here, we use a natural spline on the percentage of ministries held
%with six knots evenly spaced over the interval [0.1,0.9] to characterize the
%relationship.\footnote{We use six knots as that was essentially the advice obtained from running a
%GAM of the average strategic vote on the percentage of ministries held.}  Figure~\ref{fig:mod1},
%panel(a), shows the predictions of Model 2 over Model 1 with 95\% confidence intervals across the
%levels of percent of ministries.  The strategic model tends to do better amongst observations with
%low administrative responsibility within the coalition although the inflexion point is relatively
%low, ie. around .25.  Nevertheless, the strategic model does particularly poorly for parties with
%high levels of administrative responsibility (around .8 or .9).  Then, as expected, the strategic
%vote has essentially no effect on parties who hold all of the ministries (i.e., single party
%governors).  Panel (b) presents the comparison of Model 3 with Model 1 predictions across levels of
%administrative responsibility.  These results confirm the relatively low inflexion point of about
%.25 at which the strategic model begins to be outperformed by the sincere model.  And the parties
%with around 0.6 or 0.7 administrative responsibility to part poorly by the strategic model and
%those beyond .7 do better.

%\begin{figure}[h!]
%\caption{Smoothed Surface of Strategic Vote over Coalition Probabilities and Administrative
%Responsibility}\label{fig:mod1}

%\centerline{\subfigure[Perspective Plot]{\includegraphics[height=3in]{perspgam.pdf}}
%\subfigure[Image Plot]{\includegraphics[height=3in]{imagepred.pdf}}}\vspace{.2in} *The lighter shading in the image plot in Panel (b) represents higher levels of %strategic ideological voting.
%\end{figure}

%\subsection{Comparing Alternative Representations of the Ideological Vote}

%\par Our conceptualization of the strategic ideological vote builds on existing spatial voting models in which voters anticipate a post-election political process that moderates in some fashion how elected parties impact policy outcomes \citep{Adamsetal2005,Merrilletal1999}. We believe our added value here is to provide a richer account of how voters incorporate information regarding post-election coalition outcomes into their expectations about how candidates affect the ideological positioning of coalition governments.  In this section we assess whether in fact our model of the strategic ideological vote provides added insight into how ideology shapes vote choice in contexts with multi-party governing coalitions.  Our benchmark is the unified voting model proposed by \citet{Adamsetal2005} which includes two of the most widely employed alternatives to the pure proximity model: the directional and the discounting models of spatial voting.  Accordingly, we estimate a directional and discounting model employing the same data used for Model 3.

%\par Equation~\ref{eq:26} represents the discounting model that we estimate.  There are three coefficients to be estimated here: The $d$ term represents the extent to which voters discount the distance between the candidate's ideal and theirs in recognition of the moderating effect of the policy process.  The $\lambda$ term indicates the relative importance of the discounted ideological distance term in the voter utility function.  Finally, $\beta$ represents the importance of other factors in the vote calculus.\footnote{Note that in both models the ideological measure is standardized to have mean zero so that the neutral point for candidates and voters is zero.}

%\begin{equation}
%u(j_{i}) = \lambda[-(x_{i}-(1-d)p_{j})] + \beta \Psi_{i} \label{eq:26}
%\end{equation}

%\par The directional model that we estimate is summarized in Equation~\ref{eq:27}.  Again, three coefficients are estimated: The $(1-\alpha)$  term represents the importance of the directional component of voter's ideological calculus (the product of voter and candidate distance from a neutral point).  And $\alpha$ captures the extent to which the directional effect is moderated by a conventional ideological proximity term.  The $\lambda$ term indicates the relative importance of the directional ideological term in the voter utility function.  Finally, $\beta$ represents the importance of other factors in the vote calculus.
%\begin{equation}
%u(j_{i}) = \lambda[2(1-\alpha)[(x_{i})(p_{j}) - \alpha(x_{i}-p_{j})^{2}] + \beta \Psi_{i} \label{eq:27}
%\end{equation}

%\par Following \citep{Adamsetal2005} we generate MLE estimates of the $\lambda$, $\alpha$ and $\beta$ terms in these two unified voter utility models, in addition to the standard proximity model.\footnote{The $R$ code is available on the authors' web site www.raymondduch.com/ideologicalvote}  For comparative purposes we employ this same estimation strategy for the $\lambda$ and $\beta$ terms in our Model 3 that was described in Equation~\ref{eq:20}.  From the 245 studies considered, all four utility models converge in around half (120).\footnote{These results were estimated in R version 2.6.2 using the \texttt{nlm} non-linear maximization routine on computing resources provided by the Oxford Supercomputing Centre.  We used \texttt{Rmpi} version 0.5-5 to distribute the jobs to 16 nodes each of which was equipped with 2$\times$ dual core Intel Xeon 2.6 GHz (Woodcrest) processors and 4GB DDR2 RAM, for a total of 64 processors (of which one was the primary and sixty three were the subordinates). The total run time of the estimation was roughly eight hours ($8\times 64=512$ hours of computing time).  Models were considered to have converged if the estimation procedure returned a convergence code of 1 [``relative gradient is close to zero, current iteration is probably solution''] or 2 [``successive iterates within tolerance, current iterate is probably solution''] and was considered not to have converged if code 3 [``last global step failed to located a point lower than `estimate'.  Either `estimate' is an approximate local minimum of the function or `steptol' is too small] was returned. The code used in the estimation is available on the authors' web site. } The results below are from considering these 120 models.  Our contention is that the strategic model set forth above is ``better'' (i.e., provides a better fit to the data) than the sincere model in at least some circumstances and is not worse than other strategic representations of a voter's utility.  The data bear out our expectations.

%\par First, we consider the BIC (Bayesian or Schwarz Information Criterion).  Since these models are not technically nested within each other, the BIC is a natural choice for model comparison.  Raftery suggests positive support for a model if it's BIC is more than two points smaller than an alternative model \citep{Raftery1996}.  First, we look at the cases where the BIC for any one model is ``significantly'' better than all of the three other models.  These results are in Table~\ref{tab:BIC1}.  As you can see from the table, there are only 4 studies where our strategic ideological model dominates all others; and there are 36 studies in which the sincere model is significantly better.  However, there are no instances where either the directional or discounting model dominates all others. These results suggest, first, that in the vast majority of cases (70 percent) there is no significant difference amongst the models in terms of model fit. Hence, in most cases incorporating a richer specification of the voter's spatial calculus is defensible in terms of model fit.  Secondly, these results indicate that our strategic ideological model is never worse than the other modified proximity models (discounting and directional) and in fact in a handful of cases actually generates a better fit statistic.  Third, there clearly are cases in which the sincere ideological model provides a better fit statistic.  This, of course, is consistent with our earlier findings that show variation in the importance of strategic versus sincere ideological voting across political contexts.\footnote{Some of the higher fit statistics for the sincere model also may result from the fact that the BIC statistic tends to favour parsimony in model specification and the sincere model is the most parsimonious of the three.}

%\begin{table}[h!]
%\caption{BIC Model Comparisons 1}\label{tab:BIC1}

%\begin{center}
%\begin{tabular}{ll}
%Sincere & Strategic \\
%\hline
% Belgium 1988, 1993                 & Netherlands 1981, 1985, 1986\\
% Bulgaria 1991, 1992, 1997, 1999    & Italy 1996   \\
% Denmark 1981, 1998                 &   \\
% Finland 1990                       &   \\
% France 1991                        &   \\
% Germany 1984, 1991, 1994           &   \\
% Greece 1990, 1991, 1994            &   \\
% Hungary 2001                       &   \\
% Iceland 1990                       &   \\
% Israel 1981, 1984, 1988            &   \\
% Italy 2006                         &   \\
% Luxembourg 2004                    &   \\
% Netherlands 1989                   &   \\
% Poland 1997, 2000                  &   \\
% Portugal 1988, 1990, 1994, 2002    &   \\
% Slovenia 1999, 2004                &   \\
% Sweden 1985, 1988, 1991            &   \\
%\hline
%\end{tabular}
%\end{center}
%\end{table}

%\par The next set of results are a little less conservative.  Here, we compare the percent of vote preferences correctly predicted by each model.  Table~\ref{tab:PCP1} shows the studies in which one model generates a higher percent of correct predictions than all of the other models (however, these differences could be, and often are, quite small).  Again, there are a number of cases where our strategic ideological model predicts better than all of its competitors (including the sincere model).  However, there are a number of situations where the sincere model out-predicts all of the others.  As in the previous table, neither the directional nor the discounting models are ever ``better'' than all the other models.

%\begin{table}[h!]
%\caption{Comparison of Percent Correctly Predicted}\label{tab:PCP1}

%\begin{center}
%\begin{tabular}{ll}
%Sincere & Strategic \\
%\hline
%Australia 2001   & Australia 1987\\
%Belgium 1983   & Belgium 1988\\
%Denmark 1983   & Denmark 1990, 1998\\
%Finland 1990   & France 1983, 1986\\
%Germany 1984, 1988   & Germany 1983, 1993\\
%Hungary 2001   & Iceland 1984\\
%Italy 1990, 1991, 1993, 1994   & Israel 2001\\
%Netherlands 1988, 1994   & Netherlands 1982, 1985, 1986, 1989, 1994\\
%Norway 1991, 1993   & Poland 2001\\
%Poland 1997   & Romania 1998\\
%Portugal 1987, 1988, 1990, 1991, 1993, 1994   & Slovenia 1995\\
%Slovenia 1999   & Slovakia 1994\\
%   & Sweden 1988, 1991\\
%\hline
%\end{tabular}
%\end{center}
%\end{table}






%\centerline{\subfigure[Model 2 vs Model 1]{\includegraphics[height=3in]{mod1.pdf}} \subfigure[Model
%3 vs Model 1]{\includegraphics[height=3in]{mod1b.pdf}}}
%\end{figure}


%\par Another implication of the theoretical model is that parties with more experience in coalitions (bigger $\gamma$ terms) should benefit more from strategic votes.  We operationalize this concept as the number of months spent in a coalition government divided by the total number of months up until the survey (since the later of 1960 or the party's creation). The results here are not quite as convincing.  The same procedure used to generate the results above was used to generate these results.  Figure~\ref{fig:mod2} shows these results.  Panels (a) and (b) show the results considering coalition membership as a percentage of time until the survey.  Neither case confirms our predictions to any reasonable degree. There are some places where negative strategic voters are related to coalition frequency, but not otherwise.  Panels (c) and (d) show the results for coalition governing experience as a percentage of total governing experience.  These results are roughly equally convincing.  In both cases, though more pronounced in panel (d), parties with relatively little coalition experience have positive strategic votes, but the effect rapidly goes to zero as the proportion of coalition governing goes above 0.3.

%\begin{figure}[h!]
%\caption{Regression Predictions of Strategic Vote on Percentage of Times in
%Coalition}\label{fig:mod2}

%\centerline{\subfigure[Model 2 vs Model 1]{\includegraphics[height=3in]{mod2.pdf}} \subfigure[Model
%3 vs Model 1]{\includegraphics[height=3in]{mod2b.pdf}}}

%\centerline{\subfigure[Model 2 vs Model 1]{\includegraphics[height=3in]{mod3.pdf}} \subfigure[Model
%3 vs Model 1]{\includegraphics[height=3in]{mod3b.pdf}}}
%\end{figure}

%\clearpage

\section{Conclusion}

% Our goal in this paper is to propose and test a model that accurately accounts for the manner in which voters use ideology in their vote decision.  We contend that ... .  Getting the functional form of the ideological calculus in individual-level voting models is a critical step in exploring contextual variation in the ideological vote.  Ultimately, we would like to model variation in $\lambda$ or $\beta$ -- when is the strategic ideological vote important or when is ideology important versus other factors.  This presupposes a well-specified individual-level model.


\par The point of departure for this essay is the analysis of 400 voter preference surveys, conducted in 57 countries, over a 25 year time period that demonstrates quite definitely that the ideological vote is pervasive in both mature and transition democracies.  There is, as we expected, considerable contextual variation in the ideological vote.  Some countries, particularly transition democracies from Latin America, have quite low levels of ideological voting while in Western, Eastern and Central Europe it is quite high. The fact that the conventional measures of ideological voting are both pervasive and variable across contexts is an important necessary condition for the theory of strategic ideological voting that we develop in this essay.

\par We argue that rational voters should condition their ideological vote on the likely coalitions that form after an election since these agreements determine the ideological orientation of government policy.  In this essay we develop a model of the ideological vote that incorporates in the voter's utility for each party a term consisting of the distance between the voter's left-right self-identification and the expected left-right ideological composition of each coalition that a party might join.

\par Accordingly, a fully specified empirical model of the vote choice includes a conventional expressive ideological distance term; our strategic ideological distance term; and controls for the other factors that typically also affect voting behavior.  We assess the independent contribution of the strategic ideological distance term to the vote decision based on 245 voter preference surveys from 30 countries.  Even though the strategic and sincere ideological terms often generate predictions that are similar, and frequently exactly the same, we are able to demonstrate empirically that the strategic ideological distance component of our theoretical model has an important independent affect on vote choice.

\par Our representation of the voter's strategic ideological calculus builds on a body of research, not dissimilar to ours, that aims to better capture the voter's anticipation of the ideological compromises negotiated after an election in contexts with coalition governments.  We compare the explanatory and predictive power of our representation of how voters use ideology in their vote decision with that of these other models.  On balance our strategic ideological model performs just as well as these other models; for a number of cases it performs better; and never performs worse. A reasonable conclusion to draw from these results is that our richer specification of the voter's strategic ideological vote calculation represents theoretical value added to our understanding of the vote decision in coalition government contexts.

\par Our specification of the strategic ideological theoretical model, and the empirical results summarized in this essay, highlight an important feature of, and a challenge associated with, the study of electoral behavior.  There are four terms in our theoretical model of the voter utility -- $\lambda$ captures the overall importance of ideology; $1-\lambda$ indicates the relative importance of other non-ideological factors in the vote utility function; and $1-\beta$ and $\beta$ indicate the relative importance, respectively, of sincere versus strategic ideological voting.  The relative magnitudes of these coefficients will vary systematically across countries but also within countries from one time period to the next. Some countries should never have a strategic ideological vote and, even for those that typically do, there may be occasions when $\lambda$ or $\beta$ are near zero and hence $1-\lambda$ is uncharacteristically high.  If we believe the theoretical model and the empirical results presented here, then we can only learn about the voter utility function with repetitive observations from varied political contexts.  Hence, as has been pointed out elsewhere \citep{DuchStevenson2008}, we learn very little, and possibly can be seriously mislead, by testing our theories on a small number of voter preference studies.  The challenge presented by these results is explaining variation in the strategic ideological vote in contexts with multi-party coalition governments.  What contextual factors result in variations in the $\lambda$, $\beta$, $\gamma$ terms in our model?  A challenge we hope to take up in future research.

%\par These findings have important implications for our understanding of how parties represent public opinion.  First, parties that benefit from a strategic ideological vote, i.e., those that are participating in post-election coalition formation negotiations, will likely be less responsive to shifts in left-right sentiment in the electorate: a) because the shift in public sentiment concerns the ideological ideal point of the coalition government and hence the implications for any particular party likely to participate in the governing coalition will be moderated; b) parties make left-right movements anticipating what this implies for their chances of participating in a post-election coalition government and hence not directly responsive to public opinion.  Second parties that do not benefit from these second-order strategic vote will be much more responsive to shifts in public opinion.  And note this is the finding of \citet{Adametal2004} -- they find that extreme parties who tend not participate in governing coalitions are the only parties who's left-right positioning responds to shifts in the left-right shifts in the electorate.  And this may also be consistent with \citet{Budge1994} who finds considerable over time stability in party ideological positioning.

\newpage
\singlespace
\bibliography{dave}
\end{document}

######################################################################################################################


\section{The Data and Method}

\par One testimony to the importance that voting scholars give to the ideological vote is that it is, outside of basic demographic variables, probably the most widely employed independent variable in vote choice models in both developing and developed democracies. Accordingly there are a very large number of voter preferences studies that include this variable along with a vote choice variable -- we in fact were able to identify 495 studies that fit these criteria.  Moreover these voter preference studies were conducted in very diverse political contexts which will allow us first to establish whether in fact the ideological vote varies significantly across diverse political contexts and secondly whether our theory provides a convincing explanation for this variation.  We will employ the two-stage strategy popularized by \citet{DuchStevenson2005,DuchStevenson2008} for estimating the contextual effects hypothesized in the previous section. This will involve first estimating the ideological vote for each major political party in each of the 495 studies in our sample.  This ideological vote will then be the dependent variable in a second stage analysis with the contextual variables as independent variables.  We first describe how we measure the key concepts in this analysis: vote choice, left-right self-placement, placement of the political parties in the left-right issue space.  We then describe the first and second stage estimation strategies and report the results of the second stage analyses that directly concern the hypotheses developed in the earlier section.

\subsubsection{Measuring Vote Choice}
\par The three kinds of vote questions we use in our sample of surveys differed in two ways: (1) in their temporal relationship to the election for which the vote applied and (2) in their treatment of non-voting.  With respect to the first issue, surveys conducted directly after elections simply asked respondents to report their vote choice in the preceding election.  In contrast, those surveys that were conducted just before an election asked respondents for their vote intention in the upcoming election. Finally, surveys that were not proximate to an election asked the voter about a hypothetical election: ``If there were a general election tomorrow which party would you support?" \footnote{There is a large literature on the strengths and weaknesses of these different kinds of questions in measuring vote choice; however, the key question for our analysis, is whether these differences introduce systematic biases into our estimates of the strength of ideological self-placement - our initial analysis suggests the use of these different question introduces no systematic biases.  A more extensive analysis of whether question wording of the dependent variable introduces systematic bias in the results can be found in \citet{DuchStevenson2008}.}

%We need to check on post-election surveys -- these might inflate strategic ideological vote.

%\par The second difference in the vote choice variable concerns the treatment of non-voting.  All the surveys we used allow the voter to express whether they did not vote or did not intend to vote. Further, most allow the voter to indicate if she cast (or intended to cast) a blank ballot.  These studies differ in how they determine whether the respondent did not intend to vote. In many surveys the option of non-voting is simply included along with the other parties in the vote choice question. In others, however, a two-question format is used in which the respondent is first asked whether she voted (or intends to vote) and only then for whom she voted (or intended to vote).  While this is a readily apparent difference in the way the vote choice question is asked in different surveys, it is unlikely to be consequential, since we decided to ignore non-voters in our analysis.  The important point is that in all our studies voters were allowed some way to express that they either did not vote or did not intend to vote.

\subsection{Measuring Ideological Placement}

\par We have defined the ideological vote in Euclidean terms and hence the ideological vote is high when voters vote for the party closest to them on the left-right continuum.  By contrast it is low when parties that are not ideologically proximate receive an individual's vote.  We faced considerable constraints in attempting to construct a left-right distance measure for each of the 495 surveys in our sample. Some have argued that the ideal party distance measure is one where voters are asked to place both themselves and the political parties on a left-right continuum \citep{Westholm1997,Kedar2005}.\footnote{\citet{Westholm1997} provides individual-level evidence to this effect.  His results indicate that the use of mean placement of parties tend to reduce the variance explained and favor a directional over a proximity model of vote choice.}  These typically are time consuming questions on a public opinion survey and hence appear in a relatively small number of the surveys in our sample. Hence for a relatively small subset of our sample we will construct distance variables that consist of the squared distance between the respondents self-placement and her placement of each of the major competing parties in the country.

\par An alternative approach, that is also championed by a number of students of issue voting, consists of placing parties on the left-right continuum based on the mean placement of their constituents in the sample \citep{Macdonaldetal2007}.\footnote{\citet{Rehm2007} provides a detailed discussion of the merits and disadvantages of employing the constituency-based strategy for estimating party positions, including comparisons with the other methods for locating parties in policy space.} One of the arguments in its favor is that it reduces endogeneity bias associated with calculating voter-party distances based on each individual's placement of the political parties.  The attraction for us of using mean left-right placements is that it allows us to expand significantly the number of surveys in our analysis. This is a strategy that has been adopted by \citet{Huber1989} and also \citet{ThomassenSchmitt1997}. This method uses the mean placement of party identifiers on the left-right continuum in each survey as a measure of party placement for that survey. The distance variable then is just the squared distance between the respondent and each of these mean left-right identification scores for party identifiers of each of the political parties.  But even adopting this strategy restricts the analysis to only 120 of the 495 surveys because party identification questions are not asked in a large number of voter preference surveys.  But we can further expand the number of surveys in our analysis by adopting a more loose definition of party partisan.  In a number of the surveys (that do not have an explicit party identification question) there is a generic question as to whether the respondent feels close to a particular political party. We have defined those who respond that they feel very close to a political party as partisans of the party for which they voted. The mean left-right placement of these ``party partisans" are then used as a proxy for the location of each party on the left-right continuum.  We can expand the set of studies to the entire number by placing parties at the mean of its voters' self-placements, whether their party attachments were strong or not.

\par We measure left-right self placement with a relatively standardized question.\footnote{The actual left-right placement question wording employed by these different surveys is available in the technical appendix at www.raymondduch.com/ideologicalvote.} The U.S. is the only country in which we use a significantly different worded question.  Here we use the Liberal-Conservative item employed in the American National Election Studies.  In most voter preference surveys the left-right placement scale has 10 or 11 categories - although there are some with 7 and one with a 6 point scale.  Again we have explored the extent to which the results reported here are sensitive to which of the scales are employed and have found no significant difference. These different left-right placement questions have been standardized to have a mean of 0 and standard deviation of 1.

%edit this once we add early Israel elections.

%Put this stuff in an online appendix.

%\begin{table}[h!]
%\caption{Question wording for ideological self-placement scales}
%\begin{center}
%\begin{tabular}{l|c}
%Question &  Scale \\
%\hline
%In politics, people normally speak of `left' \& `right.' On a scale & 11pt \\
%where 0 is left and 10 is right, where would you place yourself? &        \\[.1in]
%In political matters, people talk of `the left' and `the right.' &  10pt \\
%How would you place your views on this scale? &  \\[.1in]
%Generally speaking, where would you put your views on a scale,& 7pt\\
%where 1 is most left and 7 is most right? & \\[.1in]
%We hear a lot of talk these days about liberals and conservatives.& 7pt\\
%Here is a 7 point scale on which the political views that people &     \\
%might hold are arranged from extremely liberal to extremely      &     \\
%conservative. Where would you place yourself on this scale, or   &     \\
%haven't you thought much about this?                             &     \\[.1in]
%Many people use the terms `left' and `right' when talking about & 7pt \\
%politics. Here is a scale running from left to right. Thinking about& \\
%your own political views, please indicate where you place yourself &  \\
%on this scale.& \\[.1in]
%In political matters, people sometimes talk of left, center left, & 7pt \\
%center right and right. On this scale 1 means left and 7 means right.&  \\
%Where would you place your views on this scale?                    &    \\[.1in]
%Regarding your political views, if you had to place yourself on a & 6pt\\
%`political scale' like this one [INTERVIEWER: SHOW CARD], at       &    \\
%which number would you place yourself? 1=radical/left-wing;        &    \\
%6=conservative/right-wing                                          &    \\
%\hline
%\end{tabular}
%\end{center}
%\end{table}


%\begin{table}[h!]
%\caption{Question wording for ideological self-placement scales}

%\centering
%\includegraphics[width=7in]{LRSELFwording.pdf}
%\end{table}

\clearpage


\subsection{The Two-Stage Estimation Strategy}


\par We adopt a ``two-stage" strategy for estimating the impact of our contextual variables on the ideological vote.\footnote{In cases in which they can be compared, two-stage approaches produce substantively identical results to one-stage multi-level random-coefficients models \citep{DuchStevenson2005}. This is not surprising given the large number of respondents in each of our 495 surveys.} First we identify a large number of election surveys that ask individuals about their vote choices and their left-right self placement and that cover a wide selection of electoral contexts. Next, we use the information in the surveys to estimate conditional logit models of vote choice, which are appropriate for multiparty systems with choice specific variables (in our case the respondent's distance from each of the major parties in the survey).\footnote{There is of course a debate regarding the appropriate estimation strategy for vote choice models for multi-party contexts. \citet{AlvarezNagler1998} argue for employing multinomial probit in this case because the estimates are robust when iid is violated.  While we appreciate these contributions, we do not employ MNP primarily for practical computational reasons.  Multinomial probit is computational intractable with such a large number of individual surveys.  The maximum likelihood versions of these models are quite slow and exist in slightly less flexible computing environments.  The Bayesian versions exist in more flexible computing environments (e.g., R), but as \citet{Gill2007} suggests, we must be confident that all parameters of all models have converged before we can be confident of our inferences.  If data augmentation is being used, as is the case with the MNP package in R, there are at least $N+k*(j-1)$ parameters in each model, where $N$ is the number of observations, $k$ is the number of linear equation parameters and $j$ is the number of choices.  This will generally be in excess of 1000 which would lead to roughly half a million parameters that would need to be checked.  This is computationally beyond the scope of this project.}  The strategy requires that each survey include the two similarly measured variables discussed above: a vote choice question and a left-right self placement question. We estimate party placement on the left-right scale using the crudest of measures discussed above, i.e., party left-right placement measured by the mean left-right placement of respondents who voted for the party in the sample.  Replicate of the analyses reported below using the more refined distance measures described above generates results that are very similar -- these results are reported on the web site: www.raymondduch.com/ideodlogy.

\subsection{Estimating a Strategic Ideological Vote: The First Stage}

\par In order to test our arguments we have assembled and analyzed 495 individual-level election surveys from 64 developing and developed democracies for the period 1980-2007.\footnote{Here we provide a brief sketch of the data analysis methods and strategies employed as part of this project.  A much more extensive discussion of the method is provided in \citep{DuchStevenson2005} and more in-depth description of the individual survey results is available on www.raymondduch.com/ideologicalvote.}  Recall from Equation \ref{eq:20} that there are three components to the vote utility function: strategic ideological vote; sincere ideological vote; and a set of control variables that represent other factors that enter into an individual's evaluations of political parties.  Ultimately, our primary interest is getting a good estimate of the strategic ideological vote.  In order to do this we first estimate, using conditional logit, a voter preference equation, for each of the 495 voter preferences surveys, that assumes proximity voting along with the appropriate control variables for non-ideological factors that are important to vote choice.

\begin{equation}
Pr(V_{i}=j)=\frac{e^{\hat{\beta} (I_{ij}) + \Sigma_{k=1}^{K} \hat{\phi_{j}} Z_{jik}}}{1+e^{\hat{\beta} (I_{ij}) + \Sigma_{k=1}^{K} \hat{\phi_{j}} Z_{jik}}} \label{eq:14}
\end{equation}

\noindent where $(I_{ij})$ is the ideological distance between each individual $i$ and party $j$; and $Z_{jik}$ are the $K$ control variables in the model that measure other individual-level characteristics and attitudes that enter into the vote function.\footnote{Details of the estimation strategy and the R code employed can be found at
www.raymondduch.com/ideologicalvote.}

\par In this first part of our estimation strategy we generate for each individual (in each of the voter preference surveys) $Pr(V_{i}=j)$, i.e., her probability of voting for each of the major parties in the political system.  This prediction reflects the impact of the voter's sincere ideological preference and also the other control variables $(Z_{jik})$ that are typically considered as important in shaping vote choice in that particular political system (some of which will be relatively unique to a particular country; for example, race in the U.S. or language in Canada).  A vote is characterized as being a strategic ideological vote if the observed vote choice deviates from the predicted vote choice based on the model in Equation \ref{eq:14}.

 \begin{equation}
SV_{ij}=1 \text{ if } Pr(V_{i}=j) <  Pr(V_{i} = j^{\prime})\text{ where } V_{i} = j \text{ and } j^{\prime} \neq j \label{eq:15}
\end{equation}

\noindent Accordingly, for each voter preference survey we calculate each political party's strategic ideological vote as follows:  For each respondent in the sample we determine whether their preferred vote choice corresponds to our predicted vote choice. Deviations from these correct predictions represent our estimate of the strategic ideological vote for each party. We then determine for each party in the sample the percentage of their expressed vote preferences that were incorrectly predicted by the model. The greater the importance of non-strategic considerations -- ideological proximity and the control variables -- to the vote probabilities of a particular party, the greater this percentage should be over all respondents in the survey.

\par The incorrectly predicted votes for each party, averaged over the sample of voters, are our estimates of the ideological vote for each party.  As these are estimates, they should not be treated as observed data in the second-stage model.  Just calculating standard errors would not be sufficient as it is not necessary (or even likely) that the change in percent correctly predicted will be normally distributed - as a result simply making draws out of a normal distribution with known mean and variance is unhelpful.  We follow the suggestion of \citet{Armstrongetal2007} that uses a technique similar to the one used to in accounting for uncertainty in multiple imputation.  The process starts with a parametric bootstrap of the
multinomial logit coefficient matrix similar to the procedure set forth in \citet{Kingetal2000} and \citet{Tomzetal2003}. For each survey, we take $D=100$ draws from the multivariate normal sampling distribution of the coefficients.  We generated predicted probabilities of voting for each party for each observation for each of 100 draws of the coefficient vector:
$P_{im} = \widehat{\Pr(y_{id}=m|\mathbf{X}, \hat{\betabf})}$, where $\mathbf{X}$ is the data matrix, $\hat{\betabf}$ is the vector of coefficients for the individual characteristics of the voters including their distance from each of the parties in the model. Then, we calculate the percentage of the actual expressed vote preferences for each party that were correctly predicted: $PCP_{jd} = \frac{\sum_{y_{i}=j}^{m} \hat{V}_{id} = j}{n_{j}}$ this gives a matrix that is $m \times 100$ where $m$ is the number of parties in the voter survey.  We then calculate the strategic ideological vote for each $j$ party as follows:

 \begin{equation}
SV_{jd}= 1- PCP_{jd} \label{eq:16}
\end{equation}

\noindent This provides us with 100 draws out of the distribution of percent incorrectly predicted, which can then be employed in the strategy set forth by \citet{Armstrongetal2007}.

\par We illustrate this measure with two examples from our data set: Ireland and Greece.  Figure \ref{fig:examples} presents actual data and results from the Greek and Irish cases from 1991.  The first frame of Figure \ref{fig:examples} illustrates the Greek case.  The plotted lines represent our estimated probabilities of voting for each of the three parties, based on a simple sincere ideological proximate model, for voters with left-right self-placements corresponding to those on the horizontal axis.  Take the case of the Communist Party of Greece (KKE): As voters move from the extreme left to the center of the left-right political spectrum, their associated probability of voting for KKE declines such that for those in the center of the left-right political spectrum, the probability of voting for KKE is close to zero.  Our predicted sincere ideological vote at any point along the horizontal axis simply corresponds to the party with the highest probability curve.  Hence any voter to the left of the intersection of the KKE and the Panhellenic Socialist Movement (PASOK) lines is predicted to vote for KKE and those to the right of this intersection but to the left of the intersection of the PASOK and New Democracy curves are predicted to vote for PASOK.  Not all of these sampled respondents will record a preference corresponding to these sincere ideological predictions.  Those who do not are classified by us as strategic ideological voters.  The bar plots below the figure indicate the incidence of these sincere and strategic voters for each of the three Greek parties in our example.  The red bars indicate the incidence of sincere ideological vote preferences expressed for KKE while the grey bars on the KKE line indicate the incidence of strategic ideological voters -- in the case of KKE they are dispersed over the left-right space to the right of the intersection of the KKE and PASOK probability curves.

\begin{figure}[h!]
\caption{Examples of Ideological Vote Measure}\label{fig:examples}

\centerline{\subfigure[Greece]{\includegraphics[height=3in]{gre1new.pdf}}
\subfigure[Ireland]{\includegraphics[height=3in]{ire1new.pdf}}}
\end{figure}

\clearpage

\par The results from Greece are certainly plausible because ideology matters in this context.  But as we pointed out earlier, in order to estimate the strategic ideological vote it is critical to distinguish between contexts in which ideology actually structures vote choice and contexts in which it simply does not enter the voter preference function. This is highlighted in the second frame of Figure \ref{fig:examples} that summarizes the effect of left-right identification on the vote probabilities for the three major Irish parties.  Note, that movement along the left-right dimension essentially has no impact on vote probabilities.  Regardless of one's location on the left-right spectrum, the vote probability is always highest for Fianna Fail.  Hence the sincere ideological model in the second panel predicts a Fianna Fail (FF) vote for all respondents in the sample.  Hence, the bar charts below the figure essentially indicate the incidence of voters, at different points along the left-right spectrum, who did not vote for Fianna Fail.  As a result, all of Fine Gael (FG) and Progressive Democrat voters will be classified as strategic ideological voters -- note their bar charts are all shaded in grey.  And all respondents indicating a vote preference for Fianna Fail will be categorized as sincere.  Overall this would result in a rather high degree of strategic ideological voting in Ireland which would be misleading.  In fact, since their is essentially no ideological voting (of a sincere or strategic kind) in this context, there can be no strategic ideological voting.  And as we pointed out earlier, one of the estimation challenges is identifying these ``non-ideological" contexts which in turn should be excluded from our models explaining the strategic ideological vote.



%\par At this point, we have a measure of ideological vote that makes sense.  There may be contexts in which our models of vote choice that work well in Western European and
%North American contexts, for example, simply are inappropriate for explaining vote choice.  One of the caveats mentioned above is that this method works only if we can specify
%good first-stage model.  For some of these studies, there is relatively little we can gain from knowing the standard socio-demographic factors
%and ideology that are typically included in vote choice models.  Thus, during the simulation, we calculate the expected percent correctly predicted (ePCP) suggested by
%\citet{Herron1999} for both the null and reduced models and then calculate the expected proportional reduction in error. We
%use this information to subset the data so countries with relatively low proportional reductions in error do not get
%included in the model estimation.

%This is a bit unclear -- null versus reduced models



\subsection{Contextual Variation in the Strategic Ideological Vote}

\par We have argued that, assuming voters are instrumentally rational, the importance of ideology in the vote calculus should 1) represent a significant factor in the vote calculus and 2) vary systematically across institutional contexts.  And as we pointed out there is virtually no evidence in the comparative literature regarding these two hypotheses.
The top part of Figure~\ref{fig:boxcountry} presents box plots of the strategic ideological votes estimated for each of the countries in our sample. Note that the countries are ranked according to the median value of our measure of the strategic ideological vote. The median ideological vote across all countries in the sample is about 0.4. This suggests on average about 40 percent of a party's predicted votes, based on sincere ideological preferences plus control variables, are incorrect.  And we are attributing these deviations to strategic ideological voting.  But there clearly are significant numbers of countries in this sample that obtain quite high median levels of strategic ideological voting -- some in the neighbourhood of 60 percent.  Similarly the median value in some countries falls below 20 percent.  It is quite evident from this figure that in some countries parties clearly are the recipient of second-order strategic ideological voting while in others they are not. The notion of a homogeneous ideological vote -- strategic or sincere -- across countries and political parties clear is untenable.

\begin{figure}[h!]
\caption{Box Plot of Ideology Vote}\label{fig:boxcountry}

\centerline{\subfigure[By Country]{\includegraphics[height=4in]{boxbycountry.pdf}}}

\centerline{\subfigure[By Party]{\includegraphics[height=4in]{boxbyparty.pdf}}}

\end{figure}

\clearpage

\par The top part of Figure~\ref{fig:boxcountry} hints at the fact that within each country there is considerable variation -- note the breadth of the whiskers within each box plot.  Some of this variation results from the ideological vote of a particular parties varying from one time point to the next -- hence within-party variation in the ideological vote.  We can get a better flavor of this variation in the bottom graph in Figure~\ref{fig:boxcountry} which simply organizes all of the political parties in our sample along the horizontal axis and presents a box plot of each party's estimated ideological vote.  Parties clearly differ quite dramatically in terms of their median ideological vote.  But also what is quite remarkable here is the extent to which any single party's ideological vote will vary from one time point to the next.  Clearly its not simply static features of the institutional context that condition the ideological vote.  Voters are conditioning their ideological vote for a party differently from one time point to the next.  Hence, it is clear here that dynamic features of the political context are shaping the manner in which ideological proximity enters the vote function.



%\begin{figure}[h!]
%\caption{Box Plot of Ideology Vote for some Countries}\label{fig:20country}
%\centering
%\includegraphics[height=7in]{boxxy1.pdf}
%\end{figure}
%\clearpage

\subsection{Exploring the Data}


%We characterize the variation in Figure~\ref{fig:boxcountry} and Figure~\ref{fig:20country} as variation in second-order strategic ideological voting.  As we pointed out earlier it is important to recognize that some of this variation likely results from first-order strategic considerations related to rational voters wanting to avoid wasting their vote.  Nevertheless, we are reasonably confident that much of this variation is linked to second-order strategic considerations.  First, we would contend that first-order incentives for strategic voting do not vary significantly over time and hence could not be driving the temporal variation we saw in Figure~\ref{fig:20country}.
%Second, and even more convincing is the fact, illustrated in Figure~\ref{fig:boxmajor}, that the median strategic
%ideological vote in majoritarian electoral contexts is not much different from that of non-majoritarian contexts.  Hence
%something other than first-order strategic voting incentives seems to be driving this variation in the ideological vote. We
%now turn to explaining this variation.

\par We characterize the variation in Figure \ref{fig:examples} as variation in second-order strategic ideological voting.  Although we pointed out that some of this variation is likely to result from first-order strategic incentives.  A closer look at some of the countries in our sample helps illustrate the estimation issues here.  Recall that our first hypothesis suggested that contexts with single party governments should register the lowest levels of second-order strategic ideological voting because voters do not face any strategic incentives associated with post-election coalition formations.  This absence of second order strategic incentives, we hypothesize, should translate into high levels of sincere proximity ideological voting for most political parties in such contexts.  Complicating this prediction, as we pointed out above, is the presence of first-order strategic incentives in these contexts.  In particular voters who have sincere preferences for third party candidates with little chance of forming the government have incentives to vote strategically. Figure~\ref{fig:single_party} presents the ideological vote for the three parliamentary cases in our sample (Australia, Canada and the UK) that have consistently had single party governments and hence should conform to our single party government hypothesis.

\begin{figure}[h!]
\caption{Ideological Vote in Single-Party Parliamentary Governments}\label{fig:single_party}

\centerline{ \subfigure[UK]{\includegraphics[height=3.5in]{UKideology.pdf}}
\subfigure[Canada]{\includegraphics[height=3.5in]{canadaideology.pdf}}}

\centerline{ \subfigure[Australia]{\includegraphics[height=3.5in]{australiaideology.pdf}}}

\end{figure}

\clearpage

\par While Figure~\ref{fig:single_party} confirms part of our argument regarding strategic ideological voting in single party government contexts, it also illustrates how this hypothesized effect is confounded with first-order strategic behaviour on the part of voters.  First, note that the ideology model does a very good job of predicting vote choice for all parties that are major contenders for forming a single party government: Labour and the Conservatives in the UK; Labour and the Liberals in Australia; and the Liberals in Canada.  The ideology model, with controls, typically correctly predicts between 80 and 90 percent of actual vote preferences registered for these parties.  The exception that actually proves the rule here is the Canadian Progressive Conservative Party -- they have a very low median sincere ideological vote but it varies quite dramatically.  This reflects the fact that in the early surveys the Progressive Conservatives were a serious contender for forming a government and hence received a high proximate ideological vote.  After their disastrous performance in 1993 they were no longer a credible government party and hence their sincere ideological vote declined dramatically.  In all three of these countries third party candidates receive very low ideological votes: typically the ideology model with controls is able to predict between 10 and 20 percent of actual vote choice.  Our theory of second order strategic voting predicts that these third parties in single party government contexts should get no second order strategic ideological vote.  But they should get a first-order strategic vote which we would argue accounts for the low sincere ideological voting accorded to these smaller third party candidates that are not in contention to participate in government.

\par The interesting challenge here is to empirically differentiate these first-order incentives to abandon third-party candidates from second order incentives to affect the policy composition of the governing coalition that forms after an election.  Both types of strategic behaviour can generate low ideological votes, i.e., vote preferences for a party that are poorly predicted by the ideology model with controls.  But we believe we can tease out these different incentives with our data.  Figure~\ref{fig:multi_party} presents results for three countries that have distinctively lower incentives than the majoritarian single party government cases for first-order strategic voting.

\par In the Italian graph we presents results for the Christian Democrats (who disappear in 1994), the Italian Socialist Party (PSI), the Italian Communist Party (PCI) and their reincarnation, the PSD.  The Christian Democrats, until 1992, dominated government coalitions -- they by far had the largest amount of administrative responsibility and they were the PM party in virtually every government prior to 1992.  Given their dominance of administrative responsibility, Hypothesis 5 from above predicts they will have a very high sincere proximity ideological vote which is precisely what we see in Figure~\ref{fig:multi_party}\:  The DC's ideological vote is in the range of .8 to .9.  In the post-1981 period covered by our data, the PSI was a frequent governing coalition partner (it had participated in four governments in 1987-1994 period) but it never had a disproportionately high level of administrative responsibility.  Our theoretical expectation here is that the party should be the recipient of significantly high levels of strategic ideological voting.  And this is the case -- it receives an average ideological vote falling between .3 and .4 which is comparable to similar parties in other countries.  The PCI and its replacement, the PSD, are particularly interesting because these are parties for whom the second-order strategic voting incentives suggest a sincere ideological vote (the have essentially been excluded from Italian governing coalitions) but who do not face, to the same extent as in majoritarian systems, first-order strategic incentives that might motivate voters to abandon their sincere ideological preferences.  Hence here we have a test of our theory that is less likely to be confounded with first-order strategic incentives.  And the results are supportive -- the PCI and PSD both get comparatively high levels of sincere ideological voting.

\begin{figure}[h!]
\caption{Ideological Vote in Parliamentary Systems with Multi-Party Governments}\label{fig:multi_party}

\centerline{\subfigure[Italy]{\includegraphics[height=3.5in]{Italyideology.pdf}}
\subfigure[Germany]{\includegraphics[height=3.5in]{Germanyideology.pdf}}}
\centerline{\subfigure[The Netherlands]{\includegraphics[height=3.5in]{Netherlandsideology.pdf}}}

\end{figure}

\clearpage

\par  The results for Germany, summarized in the second frame of Figure~\ref{fig:multi_party}, are also encouragingly consistent with our theoretical expectations.  First, note that the two largest German parties, that typically dominate any governing coalition, the Christian Democratic Union/Christian Social Union (CDU/CSU) and Social Democratic Party (SPD), have high levels of proximity voting.  This follows from our hypotheses that parties that typically have high levels of administrative responsibility within the governing coalition receive high levels of proximity voting.  And this is pretty constant over the 20 year period in our sample.  On the other hand parties that are perennial coalition partners should receive a significant amount of strategic ideological voting and hence a low proximity vote and this is borne out in  Figure~\ref{fig:multi_party} with the case of the Free Democratic Party (FDP) -- one of the most perennial coalition partners in Europe.  We have three estimated ideological votes for the FDP and clearly the party receives a low proximity ideological vote -- its virtually zero, suggesting that they are the recipient of considerable amounts of strategic ideological voting.  What is particularly interesting in the German case is the estimated ideological vote for the Greens.  Our theory suggests that during the early period of the Green Party's existence, when it had no history of participating in government, it should have had a high proximity vote.  As it became increasingly likely that the Green Party might enter a governing coalition its proximity ideologic vote should decline as voters increasingly view the party as a target for strategic ideological voting.  The data in Figure~\ref{fig:multi_party} is not overwhelmingly supportive: the party's ideological vote during the early period of our sample is only moderately high -- in the range of .4 -- and it declines rather precipitously but starting in 1990 which is earlier than our theory would predict.  We would have expected the measure to drop in the post-1998 period after the party entered the so-called Red-Green Alliance governing coalition.

\par The Netherlands, in the third frame of Figure~\ref{fig:multi_party}, is an interesting case because its electoral laws should generate, comparatively speaking, low incentives for first-order strategic voting.  Our theory predicts that the Christian Democratic Appeal (CDA), which, prior to 1994, accounted for a large proportion of administrative responsibility within all Dutch governing coalitions -- should receive a high sincere ideological vote.  This is essentially what we see and there is maybe some weak evidence here that this sincere ideological vote drops in 1994 when the party is no longer in a governing coalition.  The Labour Party (PvdA) between 1960 and 1989 had little government experience.  Being excluded from participation in governing coalition our theory predicts it should have a high sincere ideological vote which is essentially what we see here.  And in 1989 when PvdA became a much more regular governing party -- it was in most governments from this point until end of our data series -- we see its sincere ideological vote drop pretty significantly, from roughly .8 to range of .5 to .6.  D66, although only founded in 1967, represented a relatively credible coalition partner over most of our sampled time period -- the party participated in governments in 1972-76, in 1981, and again from 1994 to the end of our time period.  Hence our theory suggests the party should receive a relatively low sincere ideological vote which is effectively what we see:  On average the party receives an ideological vote of less than .2. The People's Party for Freedom and Democracy (VVD) is clearly a perennial coalition party -- in our time period it was only in opposition for a short period of time around 1989.  And historically its received a moderate amount of administrative responsibility within the governing coalition.  Both of these factors, according to our theory, suggest a relatively small sincere ideological vote.  And this is roughly the case -- the VVD receives an ideological vote of around .4 which is comparatively small.


%\begin{figure}[h!]
%\caption{Boxplots of Ideological Vote By Electoral System}\label{fig:boxmajor}
%
%\centerline{\includegraphics[height=3in]{major.pdf}}
%\end{figure}


%\par Recall from equation ?? that contextual variation in the ideological vote may result from variations in strategic
%considerations but it can also simply reflect the fact that in some contexts ideological proximity may not matter in the
%vote choice function -- it might be overwhelmed by other factors.  For instance, some may suggest that the electoral system
%may explain much of the variation in ideological vote.  Figure~\ref{fig:boxmajor} shows that this is not the case.
%Majoritarian and non-majoritarian systems both see considerable variation of ideological vote.


%\par Our measure does a good job of identifying when ideology is important in vote choice for a particular party. But our interest is more specific than this: We hope to tease out of the data strategic considerations that moderate the impact of ideology on party vote choice. Since the impact of ideology may be low because ideology generally does not matter we need to be able to identify these contexts and control for their effect on contextual variation in the ideological vote. Table \ref{tab:estimate} illustrates the measurement issue. Our theory regarding the strategic ideological vote addresses the differences between Country A and Country B.  Note that in the case of both Country A and Country B we have estimated a fully specified model of vote choice. In both countries we correctly predict the following percentage of votes for each party: 25 percent of Party 1's actual votes are correctly predicted; 10 percent of Party 2's votes; and 30 percent of Party 3's votes.  The second column reports the results of reducing the left-right placement of each sample respondent by 1 unit. The Country 1 example illustrates a context in which proximity voting is relatively strong -- predictions change, respectively, 15, 5 and 23 percentage points. In Country 2 the changes are much smaller suggesting strategic ideological considerations may be in play here -- respectively the changes in predictions are 2, 1 and 2 percentage points.  The last column of Table \ref{tab:estimate} suggests that these differences in the size of the ideological vote are not simply a function of ideology not having any influence on vote choice in Country B.  Note that in the reduced ideological model in both countries there is a significant percentage of votes correctly predicted -- particularly relative to the correct predictions presented in the full model in column 1. Country C illustrates those cases in which a weak ideological vote is likely the result of ideology simply not mattering the vote utility function -- i.e., ?? assuming a value of zero in equation ??.  The results for all three parties in the first column of Table \ref{tab:estimate} are exactly the same as the other two country examples.  And the change in correct predictions registered in column 2 is exactly the same as for Country B, hence suggesting that the presence of strategic ideological voting that might moderate the impact of proximity on vote choice.  But what leads us to suspect that the column 2 changes reflect the simply unimportance of ideology in the vote function is the last column that is quite different from Country B.  In this column the reduced model for ideology results in few party votes being correctly predicted suggesting that ideology is simply not important in this particularly country.

%\begin{table}[h!]
%\caption{Estimation of Strategic Ideological Vote}\label{tab:estimate}

%\centering
%\includegraphics[width=7in]{estimation_examples.pdf}
%\end{table}


%\par So it is clearly possible that countries that score low on our ideological vote measure in Figure~\ref{fig:boxcountry} are simply
%countries where ideology does not matter -- rather than countries in which strategic considerations moderate the proximity
%relationship.  And as our hypothetical example in Table \ref{tab:estimate} illustrated, we can get some sense of which
%countries these might be by estimating a reduced ideological vote model that simply includes ideological proximity as an
%explanatory variable.\footnote{This reduced MNL model had vote choice as the dependent variable and left-right distance of
%the voter from the major parties and left-right self identification as the independent variables.}  We then determine the
%correctly predicted votes employing this reduced model -- i.e., the last column in Table \ref{tab:estimate}.  Countries for
%which this model does a very poor job of predicting vote choice are likely contexts in which ideological proximity simply is
%of little importance in the vote function -- crowded out by other factors.  Hence countries for which this model performs
%poorly at predicting vote choice AND countries that score low on our ideological vote measure are likely countries in which
%ideology simply doesn't matter and hence in which the relationships proposed in Hypotheses ? through? are not likely to
%hold.  Figure~\ref{fig:change_reduced} plots the country median values of our ideological vote measure against the median
%percentage of votes correctly predicted by the reduced ideological model.  Those countries falling on the lower left
%quadrant -- low ideological vote score and low reduced model predictions -- are countries where we suspect that ideology
%simply is not important in the vote function.  These would certainly include Canada, Colombia, Venezuela, Peru, South
%Africa.  But we would also probably add India, Panama, Guatemala, Honduras, and Costa Rica. Some general comments -- Latin
%American countries -- relatively fragile/new democracies.  And also countries with very significant ethno-linquistic
%cleavages -- Canada and South Africa.


%\begin{figure}[h!]
%\caption{Ideological Vote and Reduced MNL Correct Predictions}\label{fig:change_reduced}

%\centering
%\includegraphics[width=5in]{redchange}
%\end{figure}

%\noindent Hence our priors here are that the hypothesized relationships of the previous section will be much weaker, if not
%simply insignificant, in those countries falling in the lower left hand quadrant of Figure~\ref{fig:change_reduced}.
%However, after confronting the data, we find that though the trend we expect to find is exists -- relationships increase in
%magnitude when as we remove observations where left-right proximity explains very little -- these are not statistically
%different from the full-sample results.  Thus, we present the full-sample results, but the results of our investigation are
%available from the authors upon request.


\par We present these selected data and brief narratives in order to provide some insight into the nature of variation in strategic ideological voting within some illustrative national contexts.  Visual inspection of these graphs we think makes a strong case for the notion that there is significant and interesting variation in the ideological vote and, secondly, that at least some of this variation can be accounted for by our theory of the strategic ideological vote.  Before proceeding to a more systematic modeling of this variation in the ideological vote we describe the measurement of the contextual variables that we will employ as explanatory variables.

\subsubsection{Second Stage Contextual Variables}

\par Our hypotheses suggest second stage contextual variables that measure a party's history of participation in governing coalitions will explain the level of ideological voting.  We assume that each party's prospects of participation in a governing coalition at the time of a voter preference survey is a function of its previous record of office holding.  For
example, we hypothesize that voters will not consider parties with short histories of cabinet participation likely members of the governing coalition that will form after the next election and so their vote will be dominated by ``ideological" or proximity voting. We also hypothesize that parties that are certain to form single member governing coalitions should
receive a high ideological vote. Again, we use the history of each party's participation in a coalition government to measure this concept. The probability that a party will form a single party governing coalition is simply the frequency (in months) over the period 1960 to the time of the voter preference study that the party served in a single party government
divided by the total months from 1960 to that time point that the party served in government of any kind (either as a single-party government or part of a multi-party coalition).

\par All our measures of the history of office holding are calculated from monthly data between January 1960 and the date of the survey corresponding to the data point.  Thus, the average voter's ``memory" in our surveys extends from 20 to 40 years into the past and always includes most of her adult political life (the average voter is about 40 years old at the time the surveys). There are four key independent variables employed in the second stage analysis that characterize the office holding history of each political party in the data set.

\par \textit{Not in Government:} First, we created a variable ({\em nogov})that equals one when the party is not incumbent and has no ministries and zero otherwise.  The measure then for each $t_{s}$ (the month/year) in the dataset is:

\begin{equation*}
\frac{\sum nogov_{t<t_{s}}}{\text{Number of Months}}
\end{equation*}

The first frame of Figure~\ref{fig:political_context} presents the distribution of this measure over our sample.  Note that the modal value for this variable is 1.0 suggesting that many parties in our sample have no experience governing. On the other hand, the distribution of governing experience is fairly uniform over all other values of this variable.  Ignoring the extreme category of 1, we see that slightly more than half the parties serve in government less than 50 percent of the time and slightly less than 50 percent of them served in government more than 50 percent of this time period.

\textit{Single-Party Government:} First, we created a variable ({\em spgov}) that equals one when the party is an incumbent and holds 100\% of the ministries and zero otherwise.  The
measure then for each $t_{s}$ (the month/year) in the dataset is:

\begin{equation*}
\frac{\sum spgov_{t<t_{s}}}{\text{Number of Months in Government}}
\end{equation*}

For the parties in our sample, the second frame of Figure~\ref{fig:political_context} graphs the values of this variable.  Note there is a tendency for parties to either primarily  participate in single party governments or primarily participate in multi-party governing coalitions.

\textit{Perennial Coalition Partner:} First, we created a variable ({\em coal}) that equals one when the party is an incumbent and holds $< 100$\% of the ministries and zero otherwise.  The measure then for each $t_{s}$ (the month/year) in the dataset is:

\begin{equation*}
\frac{\sum coal_{t<t_{s}}}{\text{Number of Months}_{t<t_{s}}}
\end{equation*}

The third frame of Figure~\ref{fig:political_context} indicates the frequency with which parties enter a coalition government.  Note again that most parties have little or no experience in a governing coalition.  Nevertheless, there are clearly parties who are virtually certain to enter any governing coalition that forms -- we identifies approximately 40 parties that have served in at least 80 percent of all governing coalitions that form.

\textit{Average \% Ministries:} Our measure of each party's share of administrative responsibility is calculated for each party over each $t_{s}$, the month/year in the dataset:

\begin{equation*}
\frac{\sum \text{\% Ministries}_{t<t_{s}}}{\text{Number of Months in
Government}}
\end{equation*}


\begin{figure}[h!]
\caption{Measures of Political Context}\label{fig:political_context}

\centerline{\subfigure[Probability of Not Being in Government]{\includegraphics[height=3in]{nogovhist.pdf}}
\subfigure[Probability of Forming Single Party Government]{\includegraphics[height=3in]{spgovhist.pdf}}
}\centerline{\subfigure[Probability of Being a Coalition Partner]{\includegraphics[height=3in]{coalhist.pdf}}
\subfigure[Distribution of Administrative Responsibility]{\includegraphics[height=3in]{aveperminhist.pdf}}}
\end{figure}

\clearpage


\par \textit{Institutional context}.  Our sample of countries consists of democracies, both developing and developed, for which we were able to obtain voter preference surveys.  Our expectation is that second-order strategic incentives will be quite different in parliamentary versus presidential systems.  Accordingly we distinguish between three types of regimes: parliamentary regimes in which the government serves as long as it can maintain the confidence of the legislature; popularly elected presidential regimes in which the president determines the composition and tenure of the government; semi-presidential regimes are systems in which the Prime Minister is responsive to both an elected parliament and an elected president.  These three measures are taken from \citet{Golder2005}.\footnote{Political systems with presidents that are not popularly elected are not distinguished from parliamentary regimes in this analysis}

\par Earlier in Table \ref{tab:tabnew} we identified first-order electoral strategic incentives that could confound our hypothesized second-order effects.  In the subsequent analyses we control for two potentially confounding features of the electoral system.  First, we control for district magnitude.  We use the \citet{Golder2005} measure which is the median district magnitude in the lowest electoral tier. This is the district magnitude associated with the median legislator in the lowest tier. The median legislator is determined by finding the number of legislators elected in the lower tier and dividing by two.  For further details on this variable see Amorim Neto and Cox (1997) and Golder (2003).  Second we use a dichotomous indicator of whether the electoral system is proportional (as opposed to majoritarian)\citet{Golder2005}.


\section{Second-Stage Multi-variate Estimation Results}

\par We begin by presenting the two-stage results for each of the five hypotheses noted earlier.

\subsection{Hypothesis 1: Contextual Variation in the Ideological Vote}

\par We have argued that, assuming voters are instrumentally rational, the importance of ideology in the vote calculus should 1) represent a significant factor in the vote calculus and 2) vary systematically across institutional contexts.  And as we pointed out there is virtually no evidence in the comparative literature regarding these two hypotheses. Earlier, we presented Figures~\ref{fig:boxcountry} as evidence that ideological voting varies systematic across electoral contexts.  What is not entirely clear from these graphical displays is precisely how this variation is apportioned: How large is the temporal variation relative to the cross-national variation relative to the party variation?  And is there any significant residual variation that is survey-specific?  Understanding these relative magnitudes is important because they provide a guide to the theoretical variables that are likely to have the highest explanatory pay-offs.  One convenient way to do this is to estimate a variance components model in which we partition the ideological vote for any particular case into five additive parts, an overall mean, a country specific effect, a time specific effect, a party effect and a residual effect that is particular to the specific case.  By treating the last four of these effects as random draws from appropriate zero-mean distributions, we can estimate the respective variances of these distributions and so get an indication of how much of the total variation in the ideological vote is accounted for by differences in country means, differences in temporal means, differences in party means and factors associated with specific surveys.  This is useful because it will immediately tell us which explanatory variables are most likely to account for the total variation in ideological voting that we observe.

\par Our estimate of the total variation in the ideological vote across all cases is .51.  16 percent is accounted for by variation in country means, 4 percent by variation in yearly means, 32 percent by variation in party means, and 47 percent by survey specific variation. To put it another way, if we could identify variables to account for 100 percent of the variance in ideological voting, then we would expect almost one half of the explanatory power to come from variables that vary from survey to survey and about a third of the explanatory power to come from variables that vary across parties.  Significantly less of the explanatory power would be associated with factors varying across countries -- less than a fifth.  And very little of the variation is associated with temporal effects shared by all the countries (a right-wing shift in public opinion that is shared by all countries, for example).  This confirms the importance of modeling the ideological vote in terms of strategic considerations that vary from one period to the next and also can differ for one party versus another within the same country.  Hence, for example, this calls in question modeling variation in the ideological vote as a simple function of electoral system characteristics -- majoritarian versus proportional, for example.


\subsection{Hypothesis 2: Positive correlation between second-order strategic ideological voting and a country's history of coalition governments}

\par We hypothesized that strategic ideological voting will be more prevalent in contexts where there coalition government is more common.  Hence, there should be a positive correlation between the historical frequency of coalition governments and strategic ideological voting.  One empirical implication of this, that has been explored by others (Kedar paper), is that these second-order strategic incentives should be stronger in some institutional contexts than others.  The expectation is that contexts with majoritarian electoral laws will have the lowest level of strategic ideological voting. Strategic ideological voting would also be relatively low in presidential regimes which tend to offer little opportunity for coalition government.
 %These would include the four countries in our sample that have only experienced single-party governments: Canada, the U.K., the U.S. and Greece.
  %note i have not included Latin America presidential systems here
%A second group consists of systems with single-member majoritarian electoral rules that have a history of coalition governments -- one example here is Australia.

A third grouping consists of PR systems with comparatively small district sizes.  And finally a group consisting of PR systems with relatively large district sizes.  Our expectation is that the size of the strategic ideological vote gets increasingly large as we move from single-party government contexts to contexts governed by PR rules with large district sizes.  This is actually a quite conservative test of our argument because it ignores the potentially confounding effect of first-order strategic incentives.  First-order strategic incentives imply exactly the opposite relationship -- low levels of sincere voting in single-member majoritarian systems and high levels of sincere voting in highly proportional electoral systems.

\par Figure presents the results...

\par This is actually an indirect test of our theory since the actual prediction is that the strategic ideological vote would be positively correlated with a country's history of coalition governments -- as coalition governments represent a greater percentage of the governments formed in a country, the median strategic vote should rise.  The second frame of Figure ? presents a plot of a country's median strategic ideological vote against the percent of the 1960-2007 period that the government was made of a coalition of parties.  Note that we have identified the institutional category of each country by different colored circles.  This result is consistent with the notion that it is the coalition governments that result from these institutional arrangements that generate second order strategic voting incentives.

\par We can demonstrate the importance of the coalition history incentive by controlling for institutional type.  We do this in the third frame that looks at the relationship between median strategic ideological vote and history of coalition governments only in proportional representation systems.


\subsection{Hypothesis 3: The strategic ideological vote should be positively correlated with the probability of participating in a governing coalition}

\par Hypothesis 3 suggests that there are no second-order strategic incentives associated with voting for a party that will surely not be in government which, all things being equal, should result in high proximity voting for the party.  Of course, at least in some institutional contexts, first-order strategic incentives will have exactly the opposite effect on parties that have no chance of governing -- ideological proximity will not be a good predictor of vote choice because voters will strategically abandon these proximate parties for ideologically ``distant" parties that have a better chance of governing.  Since first-order strategic incentives may be confounding the hypothesized second-order effects here, we propose to to incorporate the potential interaction between first- and second-order effects by examining the relationship between the history of participation in coalition governments and the strategic ideological vote in three institutional contexts: coalition government contexts with proportional representation and large district magnitudes; coalition government  contexts with proportional representation and small district magnitudes or with majoritarian electoral rules; and finally single party government contexts.  Figure ?? summarizes relationship we expect to see, in each context, between history of coalition participation and the strategic ideological vote: Coalition contexts with PR and large district magnitudes are ones in which first-order strategic considerations are minimized and hence we should see a positive linear relationship between coalition experience and strategic ideological vote. In coalition contexts with either small PR districts or with majoritarian electoral rules, first- and second-order strategic incentives are likely to produce a convex curvilinear relationship because parties with little or no governing experience are likely to get a big strategic ideological vote for first-order strategic reasons (they are a wasted vote) but parties with a very extensive record of participation in coalition governments will also get a large strategic ideological vote, although in this case for second-order strategic reasons.  Although it might be case here that the two effects simply cancel each other out and we get no significant relationship.  Finally, in single-party governing contexts there are no second-order strategic incentives but there are important first-order strategic incentives.  Hence in this context we should get a linear relationship between history of participation in government that is the party has had no history of participation in government it should get a high strategic ideological vote and this should decline as the history of participation increases.

\par The strong test of this hypothesis would suggest that $\bar{\mu}_{never} > \bar{\mu}_{!never}$.  There are 197 instances of parties that have never been in government and the remaining 851 have seen some government experience.  The difference here is significant, but in the wrong direction.  Parties with no governmental experience get {\em larger} strategic ideological votes on average than parties that have been in government, a difference of 0.07.  If we look at parties that have never been in a coalition (either because of never being in government or only being single-party governors), the difference in means is not statistically significant (with 499 parties having some coalition experience and 549 having no coalition experience).  The difference here is only 0.021. This is confirmed by the bivariate regression result in Column 1 of Table~\ref{tab:simple} which shows the effect of the probability of not being in government on the ideological vote.  The result here suggests that parties not in government will see higher levels of strategic ideological voting although the relationship is not statistically significant.  This is not the expected direction, but it might be that first-order strategic considerations are in play here.  For instance, voters in majoritarian systems may not want to ``waste'' their vote on small parties with little chance of winning even one seat.  Hence, the relationship we expect may hold in places where small parties are not disadvantaged by the electoral rules.  Column 1 of table~\ref{tab:multiple} shows the results of the probability of not being in government multiplied by district magnitude.  Figure~\ref{fig:condcoefs}a shows the conditional coefficients and confidence bounds. When district magnitudes are quite low, strategic ideological voting gets higher as the probability of not being in government goes up but the effect is not statistically significant. When district magnitudes are high (e.g., in proportional systems), the coefficients on the probability of not being in government are negative but again are not statistically significant.

\begin{table}[h!]
\caption{Bivariate Regression Results with Adjusted Standard Errors}\label{tab:simple}
\begin{center}
\begin{tabular}{ld{3}d{3}d{3}d{3}}
 & \multicolumn{1}{r}{Model 1} & \multicolumn{1}{r}{Model 2} & \multicolumn{1}{r}{Model 3} & \multicolumn{1}{r}{Model 4} \\
\hline\hline
Pr(Not in Government)       & 0.023   &         &         &         \\
                            & (0.019) &         &         &         \\
Pr(Single Party Government) &           & -0.141* &         &         \\
                            &         & (0.013) &         &         \\
Pr(Coalition Partner)       &         &         & 0.2*   &         \\
                            &         &         & (0.028) &         \\
Average \% Ministries       &        &         &         & -0.178* \\
                            &        &         &         & (0.023) \\
Intercept                   & 0.389*   & 0.474*  & 0.376*  & 0.444*  \\
                            &  (0.013) & (0.009) & (0.007) & (0.009) \\
\hline\\[-.15in]
$\bar{R}^{2}$               & 0.001   & 0.125   & 0.061  &  0.053  \\
\hline
\multicolumn{5}{l}{Main entries are OLS coefficients} \\
\multicolumn{5}{l}{Adj. SE's in parentheses, see Armstrong et al (2007)}\\
\multicolumn{5}{l}{N=1027 for each regression}\\
\multicolumn{5}{l}{*$p < $0.05}\\
\hline\hline
\end{tabular}
\end{center}
\end{table}

\clearpage



\begin{table}[h!]
\caption{Multiple Regression Results with Adjusted Standard Errors}\label{tab:multiple}
\begin{center}
\begin{tabular}{ld{3}d{3}d{3}d{3}d{3}}
 & \multicolumn{1}{r}{Model 5} & \multicolumn{1}{r}{Model 6} & \multicolumn{1}{r}{Model 7} & \multicolumn{1}{r}{Model 8}  & \multicolumn{1}{r}{Model 9}\\
\hline\hline

Median District Magnitude                       & -0.026* &  -0.03* &  -0.007  &   -0.027* &           \\
                                                & (0.007) &  (0.01) &  ( 0.013)&   (0.009) &           \\
Pr(Coalition Partner)                           & 0.227*  &         &          &           &           \\
                                                & (0.034) &         &          &           &           \\
Pr(Coalition Partner) $\times$ Dist. Mag.       & 0.002   &         &          &           &           \\
                                                & (0.031) &         &          &           &           \\
Pr(Single Party Government)                     &         & -0.145* &          &           &           \\
                                                &         & (0.015) &          &           &           \\
Pr(Single Party Government) $\times$ Dist. Mag. &         & 0.006   &          &           &           \\
                                                &         & (0.013) &          &           &           \\
Pr(Not in Government)                           &         &         & 0.017    &           &           \\
                                                &         &         & (0.023)  &           &           \\
Pr(Not in Government) $\times$ Dist. Mag.       &         &         & -0.02    &           &           \\
                                                &         &         & (0.017)  &           &           \\
Average \% Ministries                           &         &         &          & -0.174*   &  -0.151* \\
                                                &         &         &          & (0.025)   &  (0.025) \\
Average \% Ministries $\times$ Dist.Mag.        &         &         &          & 0.014     &          \\
                                                &         &         &          & (0.019)   &          \\
Average \% Ministries $\times$ Pr(Coalition Gov)&         &         &          &           &  -0.148  \\
                                                &         &         &          &           &  (0.163) \\
Intercept                                       & 0.36*   & 0.461*  & 0.379*   &   0.428*  &   0.413* \\
                                                & (0.008) & (0.011) & (0.016)  &   (0.01)  &   (0.01) \\
\hline\\[-.15in]
$\bar{R}^{2}$               & 0.090   & 0.153   & 0.015   & 0.065 & 0.102\\
\hline
\multicolumn{6}{l}{Main entries are OLS coefficients} \\
\multicolumn{6}{l}{Adj. SE's in parentheses, see Armstrong et al (2007)}\\
\multicolumn{6}{l}{N=1027 for each regression}\\
\multicolumn{6}{l}{*$p < $0.05}\\
\hline\hline
\end{tabular}
\end{center}
\end{table}


\begin{figure}[h!]
\caption{Conditional Coefficients by District Magnitude}\label{fig:condcoefs}

\centerline{\subfigure[$\beta_{no gov} \mid$ Dist Magnitude]{\includegraphics[height=3in]{nogovint.pdf}}
\subfigure[$\beta_{sp gov} \mid$ Dist Magnitude]{\includegraphics[height=3in]{spgovint.pdf}}}

\centerline{\subfigure[$\beta_{coal gov} \mid$ Dist Magnitude]{\includegraphics[height=3in]{coalint.pdf}}}
\end{figure}


\par This hypothesis suggests that a party that is certain to enter a governing coalition should be the recipient of lots of second-order strategic votes.  First, this particular strategic voting calculation is reserved for coalition contexts.  Hence we exclude from the analysis cases that have always had single-party governments -- essentially the four countries noted above: Canada, Greece, the U.K. and the U.S.  Secondly, it is possible, although not really likely, that first-order strategic incentives might confound this perennial coalition member effect and hence in our graphs and regression equations we control for first-order strategic incentives.

\par Evidence from the previous section is suggestive that this fourth hypothesis will likely hold - namely that perennial coalition partners are likely to see low levels of proximity voting. Here, there is a much more even distribution of parties with perennial coalition partners.  190 parties have {\em only} been in coalition governments when they govern.  661 parties have not always been in coalition governments.  The strong test of this hypothesis is that parties that are always in coalitions should see less proximate ideological voting.  There are 14 parties (party-elections) that have {\em always} been in coalition governments.  The mean ideological vote for these parties is .16 compared to .36 for the other 1034 parties.  The difference is huge in relative terms and is statistically significant with a $p$-value $\approx$ 0.  We might also be interested not only in parties that have always been in coalitions, but parties that have always been in coalitions {\em when they were in government}. The mean of ideological voting for perennial coalition partners here is .30 and for other parties is 0.39, a difference of roughly 0.09 with a $p$-value $\approx$ 0.  In both cases, the expected relationship holds.

\par It is less clear how these alternative strategic considerations enter the calculus here, but for the sake of completeness, we present not only simple model results, but those including interaction terms with both district magnitude and electoral system.  In the simple model, the expected relationship holds.  There is a negative relationship between the probability of being in a coalition government and the strategic ideological vote.  This is shown in column 3 of table~\ref{tab:simple}.  Columns 1
table~\ref{tab:multiple} shows the result when probability of being in the coalition is interacted with district magnitude.  And Figure~\ref{fig:condcoefs}c shows the conditional coefficients of coalition governments implied by column 1.  It shows that the magnitude of the coalition coefficient is insensitive to district magnitudes.


\subsection{Hypothesis 4: Parties with more experience governing alone will have a lower strategic ideological vote}

\par The fourth hypothesis suggests that parties likely to form a single party government should experience a low level of strategic ideological voting.  Again we want to evaluate the hypothesis in such a fashion as to minimize confounding first- and second-order strategic effects.  Accordingly, we divide our cases into three different institutional contexts: The first category consists of countries where the government consists of a single-party and, as it happens, the electoral institutions are always majoritarian.  There are only a handful of countries in our sample that meet these criteria: Canada, Greece, the U.K. and the U.S.  In these countries, a party is either certain to enter government as a single governing party or is certain to have no governing experience.  Those parties with no history of participating in government -- the Canadian NDP or the British Liberals, for example -- will likely contribute confounding support to the hypothesis.  First- and second-order strategic considerations will likely affect their vote in contradictory fashions: majoritarian incentives will reduce their proximate ideological vote (its a wasted vote) while second-order considerations should increase the sincere ideological vote (they have no chance of affecting the policy adopted by a governing coalition).  The second category consists of countries where coalition governments are the norm (we define these as cases in which there have never been single party governments for more than 12 months).  In these contexts the probability of a any party entering government as a single party is zero.  But included here will be parties with no chance of entering a governing coalition as well as those with coalition governing experience.  Ignoring the difference between these two cases biases the results in favor of our hypothesis since we expect the former to get a high strategic vote because of first-order strategic incentives (they represent a wasted vote) while only second-order strategic considerations should result in the latter group of parties getting a strategic ideological vote.  A third category consists of countries that experience both single-party and multi-party coalition governments -- there are a total of 19 countries that meet this criterion. \footnote{These countries are: Denmark, Ireland, Italy, Spain, Norway, New Zealand, Sweden, Australia, Portugal, Bulgaria, Czech Republic, Lithuania, Romania, Poland, Croatia, France, Austria, Finland and Japan.} This group of countries is a particular good sample on which to base the empirical test of Hypothesis 4 since there is real variation in single-party versus coalition governments.  But even in this group those parties with no history of participating in government pose a problem since first-order strategic incentives will generate a high strategic vote but second-order strategic considerations suggest a low strategic ideological vote.  Accordingly, the first set of graphs explore the relationship between history of government participation and the strategic ideological vote for the entire sample of cases; we then restrict the analysis to those parties with a non-zero probability of participating in a governing coalition (on the assumption that this reduces much of the confounding first-order strategic incentives); we examine the relationship for those contexts in which there is actually variation between single-party and coalition governments; and finally we examine this relationship controlling for parties with a zero probability of entering a governing coalition.


\par There are 68 instances of parties who have always formed a single party government and 370 instances of parties that have not {\em always} formed a single party government.  The strong test suggests that those parties that always form a single party government will have a higher mean proximity voting than other parties.  Again, this test does not show a statistically significant result, though the difference is in the hypothesized direction ($\mu_{spgov}-\mu_{mpgov}$ = -0.006, p-value =0.722).  This may be due to our  operationalization of single-party government. We use a measure that codes parties who have had single-party government experience {\em when in government}.  This would suggest that a party with but a few years of government experience as a single-party government and the remainder of their experience out of government would receive a high score.  If we restrict the parties in the set to those likely to govern as well (probability of being in government $>$ .5), we get the expected result with a difference of -0.07 in favor of single-party governors with a $p$-value of 0.0004.


\par Column 2 of table~\ref{tab:simple} shows simple OLS results of strategic ideological voting regressed on the probability of being in a single party government.  This shows that strategic ideological voting decreases with the probability of governing as a single party.  This is statistically significant and in the expected direction, though it is possible that alternative strategic considerations are conditioning the relationship we see.  Columns 2 of table~\ref{tab:multiple} shows the interaction between institutional context and the probability of single-party government.  Figure~\ref{fig:condcoefs}b shows the conditional coefficient for single-party government given district magnitude.  Both of these suggest that this single-party government result holds up in spite of first-order strategic voting incentives.


\subsection{Hypothesis 5: Administrative Responsibility and the Strategic Ideological Vote}


\par Finally, we hypothesize a convex curvilinear relationship between administrative responsibility and second-order strategic voting.  Parties at the extremes of the administrative responsibility scale are hypothesized to get a low strategic ideological vote.  Hence a party that holds very few seats in the cabinet will not be the subject of strategic ideological voting; neither will a party that controls virtually all the portfolios in the cabinet.  It is parties that control around half of the portfolios who are expected to be the largest recipients of second-order strategic voting.  First-order strategic incentives might confound this relationship although in some respects they make it more difficult for us to find the hypothesized curvilinear relationship.  It is true that first-order strategic arguments would predict that large parties should receive little strategic voting and we are predicting that parties with a large administrative responsibility should get little strategic ideological voting -- since these will also typically be larger parties both strategic incentives are suggesting similar results.  First-order strategic incentives predict that small parties will be the subject of relatively high levels of strategic voting -- it is for these parties that the wasted vote argument is most relevant.  Second-order strategic incentives, on the other hand, predict that these parties will get relatively little strategic ideological voting since their likely weight in the governing coalition will be minimal.  In the graphical and regression analyses that follow we control for these potentially confounding first-order incentives.

\par The simple OLS results from regressing the strategic ideological vote on a party's average percentage of ministries supports the hypothesis: the coefficient is negative and statistically significant.  Model 8 of Table~\ref{tab:multiple} presents the results when we include an interaction term for district magnitude.  The interaction term is not statistically significant and Figure~\ref{fig:coalcond} confirms this is the case: the wide confidence bounds suggest that the interactive model is probably not better than the one without the interaction.  Further, the fact that a flat line would fit within the confidence bounds strengthens this conclusion.



\begin{figure}[h!]
\caption{Conditional Coefficient for \% Ministries Interaction Model}\label{fig:coalcond}

\centerline{\subfigure[Model 8]{\includegraphics[height=3in]{aveperminint.pdf}}\subfigure[Model
9]{\includegraphics[height=3in]{aveperminint2.pdf}}} \
\end{figure}

\clearpage


\section{Summary}
\par This paper presents the preliminary findings of a rather ambitious effort to characterize the ideological vote in all of the contemporary democracies -- both developed and developing.  Our primary objective here is to present the preliminary results of our data collection from 385 voter preference surveys world wide, from 64 countries, and over a 25 year period.  We confirm that ideology is an important factor in the vote calculus for citizens in democracies of all varieties.  Nevertheless, we also establish that there is considerable variation in the importance of ideology in the vote choice decision.  Most importantly there clearly are parties that seem to get reasonably large ideological votes while for others there is essentially no ideological vote.  This significant variation in the ideological vote is not consistent with the notion that the ideological vote is a simple expressive proximity vote.

\par Our second goal in this paper to begin sketching out a rudimentary model of contextual variation in the ideological vote.  This model builds on the early work of \citet{AustenSmithBanks1988} and more recently \citet{Kedar2005} who argue that voters are concerned with policy outcomes and hence their voting decisions are conditioned by the coalition policy agreements that are arrived at after the election outcome. To the extent this is the case we sketch out how these considerations should shape the instrumentally rational vote choice for parties competing in elections.  And while the model is admittedly very preliminary it provides some guidance in our initial analyses of the contextual variation that we clearly see in the data.

\par Our intuition reflected in our model of the ideological vote is that it is strategic -- that voters anticipate the policy compromises that result from coalition governments and vote accordingly.  We believe even this preliminary evidence is on balance convincing.  Our prediction of relatively high proximity voting, and hence a low strategic ideological vote, for parties certain not to enter a governing coalition proved wrong -- the relationship was in exactly the opposite direction. Clearly there are a large number of uncompetitive parties for whom ideology is simply not salient -- certainly worth investigation as we refine this analysis. There was on the other hand some initial support for the other hypotheses.  We hypothesized that parties that are certain to form a single party government should have low levels of strategic ideological voting -- in fact what we see is that as a party's probability of forming a single party government increase so does its strategic ideological vote declines. This supports our contention that instrumentally rational voters have no incentive to vote in an ideological strategic fashion for parties that are almost certain to form a government without coalition partners.  Our expectation is exactly the opposite for parties that are perennial coalition partners -- they should receive relatively little sincere proximity voting and should be parties most favoured by strategic ideological voting.  And this is the case -- we find the mean proximity vote is about half of what it is for other parties.  Finally, our model suggests that strategic ideological voting should be low for party's with a relatively large share of coalition portfolios.  Again, we find empirical support for this contention, although we did not explore here the full hypothesis that suggests the relationship should in fact be curvilinear.

\newpage
\singlespace
\bibliography{dave}
\end{document}



\newpage
\noindent \textbf{\large{References}}


\hi \hspace{-.25in}Alesina A. and H. Rosenthal.  1995. \textit{Partisan Politics, Divided Government, and the Economy}. Cambridge: Cambridge University Press.

\hi \hspace{-.25in} Alvarez, R Michael and Jonathan Nagler.  1998. "When Politics and Models Collide: Estimating Models of Multiparty Elections." \textit{American Journal of Political Science} 42: 55-96.

\hi \hspace{-.25in}Alvarez, R Michael, Jonathan Nagler and Jennifer
R. Willette.  2000. ``Measuring the Relative Impact of Issues and
the Economy in Democratic Elections."  \textit{Electoral Studies} 19: 237-53.

\hi \hspace{-.25in}Austen-Smith, David and Jeffrey Banks. (1988). Elections, Coalitions,
and Legislative Outcomes. \textit{American Political Science Review} 82: 405-422.

\hi \hspace{-.25in}Bargsted, Matias A. and Orit Kedar. 2007. "Voting for Coalitions: Strategic Voting under Proportional Representation". MIT, manuscript, April.

\hi \hspace{-.25in}Blais, Andre, John Aldrich, Indridi Indridason, Renan Levine. Voting for a Coalition. manuscript. University of Montreal.

\hi \hspace{-.25in}Blais, Andre, Donald Blake, and Stephane Dion.
1993. ``Do Parties Make a Difference?: Parties and the Size of
Government in Liberal Democracies." \textit{American Journal of Political
Science} 37: 40-62.

\hi \hspace{-.25in}Blais, Andre, Mathieu Turgeon, Elisabeth
Gidengil, Neil Nevitte and Richard Nadeau.  2004.  ``Which Matters
Most? Comparing the Impact of Issues and the Economy in American,
British and Canadian Elections." {British Journal of Political
Science}  34: 555-564.

\hi \hspace{-.25in}Bowler, Shaun, Todd Donovan, and Jeffrey A. Karp. 2006. "Strategic Voting Over Coalition Governments: The Case of New Zealand."
    Paper presented for presentation at the Annual Meeting of the Western Political Science Association, Alburquerque, N.M.

\hi \hspace{-.25in}Downs, Anthony. 1957. {An Economic Theory of Democracy}. New York: Harper and Row.


\hi\hspace{-.25in}Duch, Raymond M. and Randy Stevenson. 2008. Voting in Context:
How Political and Economic Institutions Condition the Economic Vote. Manuscript. www.raymondduch.com/economicvoting.


\hi \hspace{-.25in}Enelow, James M., and Melvin J. Hinich. 1994. ?A Test of the Predictive Dimensions Model in Spatial Voting Theory.? \textit{Public Choice} 78:155?169.



\hi \hspace{-.25in}Franklin, Mark, Tom Mackie, Henry Valen, et al. 1992. \textit{Electoral Change: Responses to Evolving Social and Attitudinal Structures in Western Countries}. Cambridge: Cambridge University Press.

\hi \hspace{-.25in}Gabel, Matthew and John D. Huber. 2000. "Putting Parties in Their Place: Inferring Party Left-Right Ideological Positions from Party Manifestos Data".  \textit{American Journal of Political Science}. 44(1) 94-103.

\hi \hspace{-.25in}Gschwend, Thomas. 2007. "Ticket-Splitting and Strategic Voting under Mixed Electoral Rules: Evidence from Germany." {European Journal of Political Research} 46:1-23.

\hi \hspace{-.25in}Indridason, Indridi H. 2006. Coalitions and Clientelism:
 Explaining Cross-National Variation in Patterns of Coalition Formation.
 Manuscript http://users.ox.ac.uk/~polf0144/CoalitionsClientelism.pdf

\hi \hspace{-.25in}Irwin, Galen A. and Joop J.M. van Holsteyn. 2003. "They Say it can't be Done? Strategic Voting in Multi-Party Proportional Systems: The Case of the Netherlands." Paper presented at the Annual Meeting of the American Political Science Association, Philadelphia, PA.

\hi \hspace{-.25in}Kedar, Orit. 2005. ``When Moderate Voters Prefer
Extreme Parties: Policy Balancing in Parliamentary Elections."
\textit{American Political Science Review}. 99(2) May: 185-199.


\hi \hspace{-.25in}Kim, HeeMin, and Richard C. Fording. 1998.
``Voter Ideology in Western Democracies, 1946-1989." \textit{European
Journal of Political Research} 33: 73-97.



\hi \hspace{-.25in}King, Gary, Michael Tomz and Jason Wittenberg.
2000.  ``Making the Most of Statistical Analysis: Improving
Interpretation and Presentation.''  \textit{American Journal of
Political Science}.  44(2): 347-361.


\hi \hspace{-.25in}McCarty, Nolan, Keith T. Poole and Howard Rosenthal. 2006. \textit{Polarized America: The Dance of Ideology and
Unequal Riches}. Cambridge, MA: MIT Press.

\hi \hspace{-.25in} McCuen, Brian and Rebecca Morton. (2000). Tactical Coalition Voting. Working Paper.

\hi \hspace{-.25in}McKelvey, Richard and Peter Ordeshook. 1972. ``A
General Theory of the Calculus of Voting." In \textit{Mathematical
Applications in Political Science}, vol. 6, ed. J.F. Herndon and J.L
Bernd. Charlottesville: University Press of Virginia.

\hi \hspace{-.25in}Stevenson, Randolph T. 2001. ``The Economy and
Policy Mood: A Fundamental Dynamic of Democratic Politics?? \textit{American
Journal of Political Science} 45: 620-633.

\hi \hspace{-.25in}Tomz, Michael, Jason Wittenberg and Gary King.
2003.  CLARIFY: Software for Interpreting and Presenting Statistical
Results.  Version 2.1.  Stanford University, University of Wisconsin
and Harvard University.  January 5.  Available at:
http://gking.harvard.edu/.


\hi \hspace{-.25in}Warwick, Paul. 1992. ``Ideological Diversity and
Government Survival in Western European Parliamentary Democracies."
\textit{Comparative Political Studies} 25: 332-361.


\hi
\hspace{-.25in}Westholm, Anders. 1997. ``Distance Versus Direction: The Illusory Defeat of the Proximity Theory of Electoral Choice.? \textit{American Political Science Review} 91(4):865?885.



\section{Estimation of Ideological Vote}

\par Rather this is what i think i want to estimate:  I want to know how adding L/R distance to the voter's utility function for a particular party improves my ability to predict the voter's selection of this party "choice" -- so the estimate of a party's "ideological" vote can be generated by first estimating a fully specified model (with choice specific variable) -- using the coefficients to predict each respondents vote choice -- a zero or one for each party -- calculate correct predictions for each party -- then set L/R coefficient to zero -- generate predicted vote choices -- calculate correct predictions for each party -- parties for whom there is a significant deterioration in predicted vote choice would be ones that get a "high" ideological vote -- parties for whom there is little different would get a "low" ideological vote.  I think my reasoning here is that the estimate of the choice specific coefficient is generated in such a fashion that it is based on those parties for whom there is the most systematic relationship between ideology and vote choice -- and least sensitive to those parties for whom the relationship is most "noisy" -- hence the parties that have the most "ideological" voting will be those for whom the distance variable improves correct predictions the most.

\par Multinomial logit will estimate a coefficient for the party choice specific variable "Left-Right Distance" which is the distance, on this L/R scale, between each individual in the sample and each of the political parties in the data set.  Our interest is understanding how the estimated "Left-Right Distance" coefficient (which is common across parties) affects the accuracy of predicted vote choice for each party in the data set.  In fact our hypotheses concern 1) which parties should be better predicted than others; 2) whether the distance coefficient should result in over-prediction or under-prediction of vote for a particular party; 3) institutional contexts that might interact with party characteristics and affect over/under-prediction for a party.  Take the following example: A party might be extreme with no chance of entering any coalition government.  If there is no strategic voting in this context then a) we should do a good job of predicting each respondent's vote for this party; b) there should be no systematic pattern in bad predictions; c) without the distance variable in the equation we should do a bad job of predicting each respondent's vote for this party.  If there is strategic voting in this context then 1) we should do a bad job of predicting each respondent's vote for this party; b) the bad predictions should all consist of 0s rather than 1s -- in other words bad predictions are respondents who are close to party but chose strategically to vote for other party -- bad predictions should not be the result of too many 1s relative to 0s; c) dropping the distance variable should have no affect on how well we predict votes for this party.




\begin{figure}[h!]
\caption{Fitted Surface for Interaction Model$^{*}$}\label{fig:interact}

\centerline{\includegraphics[height=4in]{interact.pdf}}

$^{*}$ The red regions on the graph represent the areas of highest (greater than 50$^{th}$ percentile) bivariate density
between single party governments and probability of being in government.  The unshaded area represents the entire surface.


\end{figure}


\begin{figure}[h!]
\caption{Conditional Coefficients for Single Party Government Model}\label{fig:cond}

\centerline{ \subfigure[$\hat{\beta}_{spgov} \mid$ Pr(In Gov)]{\includegraphics[height=3in]{cond1.pdf}}
\subfigure[$\hat{\beta}_{ingov} \mid$ Pr(Single-Party Gov)]{\includegraphics[height=3in]{cond2.pdf}}}

\end{figure}



\subsection{Hypothesis 5: Proximity voting is positively correlated with the party's share of coalition portfolios}

\begin{table}[h!]
\caption{Results from Statistical Models 2}\label{tab:results2}
\begin{center}
\begin{tabular}{ld{3}d{3}}
 & \mc{1} & \mc{2} \\
\hline\hline
District Magnitude                  &  0.043   &  0.043* \\
                                    &  (0.01)  &  (0.01) \\
Average \% of Ministries            &  0.205*  &  0.158*  \\
                                    &  (0.042) &  (0.046) \\
Pr(Coalition Government)            &          &  -0.2*   \\
                                    &          &  (0.07)  \\
\%Ministries $\times$ Pr(Coalition) &          &  0.735*  \\
                                    &          &  (0.356) \\
Intercept                           &  0.271*  &  0.34*   \\
                                    &  (0.018) &  (0.016) \\
\hline\hline
\end{tabular}
\end{center}
\end{table}
